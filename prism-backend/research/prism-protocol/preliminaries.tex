\section{Preliminaries}
\label{sec:preliminaries}

\subsection{Cryptographic Building Blocks}
\label{ssec:bbs}

\todo{Briefly describe algorithms they use.}

\paragraph{Append-Only Bulletin Boards.} %
%\todo{For now, simplified notion as in \cite{acc+20}.}

\paragraph{Digital Signatures.} %
%Maybe build on \IdealFSig from \cite{canetti03}.

\paragraph{Merkle Trees.}

\paragraph{Public Key Infrastructures.} %
% Maybe build on \IdealFCA from \cite{canetti03}.

\subsection{Self-Sovereign Identities}
\label{ssec:ssi}

\paragraph{Decentralised Identifiers (DIDs).} %
Much of the SSI community is built on top of DIDs, which is a W3C Recommendation
since 2022\footnote{\url{https://www.w3.org/TR/did-core}. Last access, 3rd
  November, 2022.}. Essentially, the DID specification defines how to associate
public keys and additional metadata to unique identifiers, leveraging verifiable
data registries (VDR). Whereas there is no concrete requirement on these VDRs,
the typical choice are blockchains, which leads to the ``Decentralised''
adjective for DIDs. The DID specification describes the syntax that identifiers
must follow, as well as the operations (CRUD: Create, Read, Update, Deactivate)
that any DID-compatible mechanism must implement.

That is, a mechanism following the DID specification must enable any user to be
able to register a unique identifier, associate public key material to it, and
manange it. Thus, it is natural to see such a mechamism as an equivalent to
traditional Public Key Infrastructures (PKIs), where different instantiations of
such a mechanism -- e.g., with different approaches to implement the CRUD
methods required by the DID specification, or based on different VDRs -- can be
seen as a different type of DID-based PKI.

\todo{Maybe expand. Also, describe the difference between DID Document, and DID.
  Also describe the concept of DID controller, essential to prevent manipulation
  of DIDs and their documents.}

\paragraph{Verifiable Credentials (VCs).} %
\todo{Describe: Introduction to VCs. Relating to DIDs.}

\subsection{Overview of Atala PRISM}
\label{ssec:overview-prism}

In a nutshell, Atala PRISM builds VCs and DIDs on top of the Cardano blockchain.
More concretely, any entity can create (and manage) a DID, and issuers can issue
VCs associated to a previously existing DID, where Cardano is the chosen VDR.
Users can later leverage VCs to authenticate against any potential verifier, in
a manner that all the information that the verifier needs to (cryptographically)
authenticate the user is contained in the Cardano blockchain.

\todo{Ellaborate more.}


%%% Local Variables:
%%% mode: latex
%%% TeX-master: "prism-protocol"
%%% End:
