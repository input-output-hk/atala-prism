\section{Atala PRISM's \RealPKIDIDAtala: A DID-based PKI}
\label{sec:did-atala}

In this section we first describe Atala PRISM DID method \RealPKIDIDAtala, and
prove that it realizes \IdealGPKIDID.
%
We begin by describing the types of keys supported in Atala PRISM, and then
specify how sets of such keys can be managed through the previously introduced
dictionaries in \IdealGPKIDID.

\paragraph{Key Types.} %
All operations in Atala PRISM are authenticated, meaning that a cryptographic
key needs to be involved in proving that the operation comes from the intended
party. Atala specifies four key types, that are thus supported in Atala's DID
method. Namely: master keys (\MasterKey), for operating on DIDs; issuing keys,
for issuing VCs (\IssueKey); communication keys (\CommKey), for key-exchange
protocols; and authentication keys, to associate to VCs (\AuthKey).

\paragraph{Custom Notation.} %
Keys in Atala are identified by labels. In the following description of the
\RealPKIDIDAtala construction, given a set of keys, we frequently refer to the
``master key with lowest label'' (using any order relation, like lexicographic
order).  For brevity, given a set \st, we use $\LMKL(\st)$ to denote such label.

\paragraph{$\P_{Atala}$ policy.} %
Before proceeding with the specififying the DID operations in Atala, we define
its policy on the keys' labels and types.
\todo{This needs to take into account the caller, and the previous state of the
  PKI list.}

\begin{figure}[ht]
  \centering
  \procedure{$\P_{Atala}(\lbrace (\lbl_i,\typ_i) \rbrace_{i\in[n]})$}{    
    \pcif \exists i\in[n]~\suchthat~\typ_i \notin [\MasterKey,\IssueKey,
    \CommKey,\AuthKey]: \pcreturn 0~\pccomment{All types must be known} \\   
    \pcif \not\exists i\in[n]~\suchthat~\typ_i = \MasterKey: \pcreturn 0~
    \pccomment{There must be at least one master key} \\
    \pcif \exists i, j \in[n], i \neq j~\suchthat~\lbl_i = \lbl_j: \pcreturn 0~
    \pccomment{No duplicate labels} \\
    \pcreturn 1
  }
  \caption{Policy $\P_{Atala}$ that all DID documents must meet in Atala.}
\end{figure}

\paragraph{DID Operations.} %
\todo{For now, this ignores \emph{unpublished} DIDs.}
\figref{fig:atalapkidid1} and \figref{fig:atalapkidid2} contain the
specification of Atala PRISM's DID method,
which we describe next. Looking at the \texttt{Create} and \texttt{Update}
operations, it is clear that $\P(\lbrace (i, \lbl_i) \rbrace_{i\in[n]})$ in
Atala requires that all key labels are of a known type (\MasterKey, \IssueKey,
\CommKey, or \AuthKey) and that at least one of the keys in the DID Document
is of type \MasterKey.

\begin{description}
\item{\uccmd{Create}.} %
  To create a DID, the user has to specify the key pairs to create, along with
  fresh identifiers (labels) and their type. For ease of exposition, we assume
  that the used identifiers just follow a sequence. The predicate \P in Atala
  simply requires that at least one key pair of \MasterKey type is created. The
  \uccmd{Create} algorithm creates as many keys of each type as requested,
  and keeps a list with the identifier, type and public key of each key pair.
  This list is then hashed into a value \texttt{h}, and the created DID is
  ``\texttt{\did:prism:h}''. A transaction containing the created list, the
  DID, and the \texttt{CreateDID} instruction is sent to the blockchain,
  signed by one of the master keys -- w.l.o.g., we simply use the first one
  in the list. The DID owner stores locally the list, along with the
  corresponding private keys.
\item[\uccmd{Read}.] %
  Getting the contents of a DID simply requires processing the associated
  transactions in the blockchain. Namely, transactions $(\tx,\sig)$ where $\tx$
  is either $(\texttt{CreateDID},\did,\cdot)$ or $(\texttt{UpdateDID},\did,
  \cdot)$. The associated DID Document is the result of applying in an ordered
  manner (from oldest to newest) all such validly signed transactions.
\item[\uccmd{Update}.] %
  To update a DID, we first assume for ease of exposition that the calling party
  $P$ has an updated version of the associated DID Document (which included
  a list of $n$ keys) -- if this is not the case, it suffices to make first a
  call to \uccmd{Read} for the given DID. \todo{This seems relevant enough to
    be highlighted in some more visible place, no?}
  Then, updating a DID requires to (optionally) specify a list of previously
  existing key identifiers to remove, and (optionally) specify a list of new key
  identifiers, and their types, to add. To simplify notation, we just adopt the
  (equivalent) convention that the caller indicates a list of $\lbrace (i,
  \lbl_i) \rbrace_{i\in\tilde{N}\subseteq[\tilde{n}]}$ pairs, where $\tilde{n}$
  may be smaller or larger than $n$, depending on the desired update. This
  list is then interpreted as follows. If $i$ is an already existing
  identifier, the label is ignored and the corresponding key is not removed. If
  $i$ is a new identifier, then a new key pair of the given type is created.
  Omitting an existing identifier is then interpreted as the desire to remove
  the corresponding key pair. Otherwise, the same policy as for \uccmd{Create}
  applies here. Key pairs for the new keys are generated, and a transaction is
  sent to the blockchain, signed with one of the master keys in the DID
  \emph{before applying the update}. This transaction contains a list of the
  identifiers of the keys to remove, and a list of the identifiers and types
  of the keys to add.
\item[\uccmd{Deactivate}.] %
  A deactivate operation is simply an update operation with no new key
  identifiers, and that only specifies the existing identifiers that will
  \emph{not} be deactivated.
\end{description}

\begin{figure}[ht!]
  \begin{framed}

    \scalebox{0.9}{
      \begin{minipage}[t]{\textwidth}
        \textrm{Party $P$ running operation \cmd with arguments \arg, in
          session \sid.}
      \end{minipage}
    }
    \vspace*{0.5em}
    
    \scalebox{0.9}{
      \begin{minipage}[t]{0.55\textwidth}
        \procedure[linenumbering=on]{$\pcif \cmd = \uccmd{Create} \land \arg =
          \lbrace (\lbl_i,\typ_i) \rbrace_{i\in[n]}$}{
          \pcif \P_{Atala}(\lbrace (\lbl_i,\typ_i) \rbrace_{i\in[n]}) = 0: abort \\
          \st \gets \emptyset \\
          \pcfor i \in [n]: \\
          \pcind \sid' \gets ((P,\lbl_i),\sid) \\
          \pcind \ucsend~(\uccmd{KeyGen},\sid')~\text{to}~\IdealFSig \\
          \pcind \ucrecv~(\uccmd{VerKey},\sid',\pk_i)~\text{from}~
          \IdealFSig \\
          \pcind \st \gets \st \cup \lbrace (\lbl_i,\typ_i,\pk_i) \rbrace \\
          \ucsend~(\uccmd{Hash},\st)~\text{to}~\IdealGRO \\
          \ucrecv~(\uccmd{Hashed},\st,h)~\text{from}~\IdealGRO \\
          \did \gets ``did:prism:''||~h \\
          \tx \gets (\texttt{CreateDID}, \did, \st) \\
          \lbl \gets \LMKL(\st); \sid' \gets ((P,\lbl),\sid) \\
          \ucsend~(\uccmd{Sign},\sid',\tx)~\text{to}~\IdealFSig \\
          \ucrecv~(\uccmd{Signed},\sid',\tx,\sig)~\text{from}~\IdealFSig \\
          \ucsend~(\uccmd{Append},(\tx,\sig))~\text{to}~\IdealGdledger \\
          \text{Locally store}~
          (\did, \lbrace (\lbl_i,\typ_i,\pk_i) \rbrace_{i\in[n]}) \\
          \ucio{Output}~(\uccmd{Created},\sid,\did,\lbrace \pk_i
          \rbrace_{i\in[n]})
        }
      \end{minipage}
    }
    \scalebox{0.9}{
      \begin{minipage}[t]{0.5\textwidth}
        \procedure[linenumbering=on]{$\pcif \cmd = \uccmd{Read} \land
          \arg = did$}{
          \ucsend~(\uccmd{Retrieve})~\texttt{to}~\IdealGdledger \\
          \ucrecv~(\uccmd{Retrieve},L)~\texttt{from}~\IdealGdledger \\
          \st \gets \emptyset \\
          L_{\did} \gets (\ast,\did,\ast) \subseteq \DID \\
          \pcfor (\tx=(\texttt{op},\did,\cdot),\sig) \in L_{\did}: \\
          \pcind \pk \gets \emptyset \\
          \pcind \pcif \texttt{op} = \texttt{CreateDID}~\land \\
          \pcind \pcind \text{no \texttt{CreateDID} appeared before}: \\
          \pcind \pcind (\cdot,\cdot,\lbrace (\lbl_i,\typ_i,\pk_i)
          \rbrace_{i\in[n]}) \gets \tx \\
          \pcind \pcind \st \gets \lbrace (\lbl_i,\typ_i,\pk_i) \rbrace_{i \in [n]} \\
          \pcind \pcif \texttt{op} = \texttt{UpdateDID}: \\
          \pcind \pcind (\cdot,\cdot,(\st_d,\st_a)) \gets \tx \\
          \pcind \pcind \lbrace \lbl^d_i \rbrace_{i\in[n^d]} \gets \st_d \\ 
          \pcind \pcind \lbrace (\lbl^a_i,\typ^a_i,\pk^a_i)
          \rbrace_{i\in[n^a]} \gets \st_a \\
          \pcind \pcind \st \gets (\st \setminus \st_d) \cup \st_a \\
          \pcind \lbl \gets \LMKL(\st) \\
          \pcind \pcif \lbl = \emptyset \lor \P_{Atala}(\st) = 0: \text{skip} \\
          \pcind \pcelse \sid' \gets ((P,\lbl),\sid) \\
          \pcind \ucsend~(\uccmd{Verif},\sid',\tx,\sig,\pk)~\text{to}~\IdealFSig \\
          \pcind \ucrecv~(\uccmd{Verifd},\sid',\tx,b)~\text{from}~\IdealFSig \\
          \pcind \pcif b = 0: \text{skip} \\
          \ucio{Output}~(\uccmd{Read},\sid,(\did,\st))
        }    
      \end{minipage}
    }
  \end{framed}
  \caption{\texttt{Create} and \texttt{Read} operations in Atala
    PRISM DID-based PKI protocol, \RealPKIDIDAtala, realising \IdealGPKIDID
    in the $(\IdealGRO,\IdealFSig,\IdealGdledger)$-hybrid model.}
  \label{fig:atalapkidid1}
\end{figure}

\begin{figure}[ht!]
  \begin{framed}
    \scalebox{0.9}{
      \begin{minipage}[t]{\textwidth}
        \textrm{Party $P$ running operation \cmd with arguments \arg, in
          session \sid.} \\
      \end{minipage}
    }
    \vspace*{0.5em}
    
    \scalebox{0.90}{          
      \begin{minipage}[t]{0.6\textwidth}
        \procedure[linenumbering=on]{$\pcif \cmd = \uccmd{Update} \land
          \arg = (did, \sval_d, \sval_a)$}{
          % \pcif \exists i~\suchthat~\typ_i \notin [\MasterKey,\AuthKey,
          % \CommKey,\IssueKey]~\lor \\
          % \pcind \nexists i~\suchthat~\typ_i = \MasterKey
          % \lor \exists i,j~\suchthat~\lbl_i = \lbl_j: abort \\
          % \pcif \exists i~\suchthat~\lbl_i \notin [\MasterKey,\AuthKey,
          % \CommKey,\IssueKey] \lor
          % \nexists i~\suchthat~\lbl_i = \MasterKey: \\
          % \pcind abort \\
          % \text{Run $(\uccmd{Read},\sid,\did)$ to fetch $(\did,\sval)$} \\
          % \pcif \sval = \bot: abort \\
          \text{Fetch}~
          (\did, \sval=\lbrace (\lbl_i,\typ_i,\pk_i) \rbrace_{i\in[n]})
          ~\text{from storage} \\
          \st_a \gets \emptyset;~
          \lbrace (\lbl_i,\typ_i,\pk_i) \rbrace_{i\in[n]} \gets \sval \\
          \lbrace \lbl^d_i \rbrace_{i\in[n_d]} \gets \sval_d;~
          \lbrace (\lbl^a_i,\typ^a_i) \rbrace_{i\in[n_a]} \gets \sval_a \\
          \sval' \gets (\sval \setminus \sval_d) \cup \sval_a
          \pccomment{\todo{Syntax refinement needed here.}} \\
          \pcif \P_{Atala}(\sval') = 0: abort \\          
          \pcfor i \in [n_a] \\
          \pcind \ucsend~(\uccmd{KeyGen},((P,\lbl_i),\sid))~\text{to}~\IdealFSig \\
          \pcind \ucrecv~(\uccmd{VerKey},((P,\lbl_i),\sid),\pk^a_i)~\text{from}~
          \IdealFSig \\
          \pcind \st_a \gets \st_a \cup \lbrace (\lbl^a_i,\typ^a_i,\pk^a_i) \rbrace \\
          % \text{Locally store}~(\did,\lbrace (i,\lbl_i,\pk_i) \rbrace_{i\in[\tilde{n}]} \\
          \tx \gets (\texttt{UpdateDID}, \did, \sval_d, \st_a) \\
          \lbl \gets \LMKL(\sval) \\
          \ucsend~(\uccmd{Sign},((P,\lbl),\sid),\tx)~\text{to}~\IdealFSig \\
          \ucrecv~(\uccmd{Signed},((P,\lbl),\sid),\tx,\sig)~\text{from}~\IdealFSig \\
          \ucsend~(\uccmd{Append},(\tx,\sig))~\text{to}~\IdealGdledger \\
          \text{Locally store}~(\did,\sval) \\
          \ucio{Output}~(\uccmd{Updated},\sid,(\did,\lbrace \pk^a_i \rbrace_{i\in[n^a]}))
        }
      \end{minipage}
    }
    \scalebox{0.9}{
      \begin{minipage}[t]{0.45\textwidth}
        \procedure[linenumbering=on]{$\pcif \cmd = \uccmd{Deact} \land \arg =
          (did, \sval)$}{       
          \text{Run as a \texttt{Update} operation} \\
          \text{with $\sval_d \gets \sval$ and $\sval_a \gets \emptyset$}
        }
      \end{minipage}
    }
  \end{framed}
  \caption{\texttt{Update} and \texttt{Deactivate} operations in Atala
    PRISM DID-based PKI protocol, \RealPKIDIDAtala, realising \IdealGPKIDID
    in the $(\IdealGRO,\IdealFSig,\IdealGdledger)$-hybrid model.}
  \label{fig:atalapkidid2}
\end{figure}

\subsubsection{Notes for Implementations.} %

We acknowledge that, in the pseudocode given in \figref{fig:atalapkidid1} and
\figref{fig:atalapkidid2}, we have favoured notational simplicity over
implementation efficiency or
convenience. For instance, it is not really needed that the master key signing a
transaction is the one with the lowest index -- any such key can actually do it.
However, it makes notation much easier. Similarly, in practice it may be easier
to use different ``function arguments'' to denote the set of keys to remove and
a set of keys to add from/to an existing DID. However, notationally it is
simpler to follow the chosen (equivalent) approach. Finally, key indexes don't
need to follow a sequential order.

Somehow related, it is of course not needed to fetch the contents of the whole
blockchain whenever we want to get an up to date DID Document. Instead, the
party running the operation can keep a local copy, and just fetch the new
operations since it last synced. However, again the given simplified notion of
a blockchain (via \IdealGdledger) eases notation -- and seems good enough for a
first modelling effort.

\subsection{Security of \RealPKIDIDAtala}
\label{ssec:sec-didatala}

Before going into the proofs, observe that the \uccmd{Create} (in lines 1-2) and
\uccmd{Update} (in lines 6-7) operations in \RealPKIDIDAtala include a policy on
the received key labels and types. Concretely, this policy requires that all
labels are unique, all types are of a known key type, and that at least one of
them is of type \MasterKey. In the sequel, we use $\P_{Atala}$ to refer to this
policy.

\todo{Do we mention somewhere that {\sid}s have structure $(P,\sid')$? This is
  used for \Sim to know which party is generating which keys.}

\begin{theorem}[UC-security of \RealPKIDIDAtala]
  \label{thm:sec-didatala}
  \RealPKIDIDAtala UC-realizes $\IdealGPKIDID$ in the $(\IdealGRO,
  \IdealFSig, \IdealGdledger)$-hybrid model.
\end{theorem}

\begin{proof}
  We show that \RealPKIDIDAtala UC-realizes \IdealGPKIDID, when configured
  with policy $\P_{Atala}$. For this, we give a simulator \Sim such that, for
  any environment \Env, $\Exec_{\RealPKIDIDAtala\adv,\Env} \approx
  \Exec_{\phiDID,\Sim,\Env}$,
  where \phiDID is a protocol composed with the same parties as
  \RealPKIDIDAtala but where each party is simply a dummy party that forwards
  every message from \Env to the ideal functionality \IdealGPKIDID, and every
  response from \IdealGPKIDID back to \Env. The simulator \Sim runs internally a
  copy of the real-world adversary \adv, as well as the ideal functionalities
  \IdealGRO, \IdealFSig and \IdealGdledger, and any involved party.

  \begin{description}
  \item[Simulating DID Create.] %
    % 
    % Keys generation
    If \Sim receives a message $(\uccmd{Create},\sid,\lbrace (\lbl_i,\typ_i)
    \rbrace_{i\in[n]})$ from \IdealGPKIDID, with $\sid=(P,\sid')$, it simulates
    the key generation process of $n$ signing key pairs run in the real
    protocol, by sending $n$ queries $(\uccmd{KeyGen},((P,\lbl_i),\sid))$ to
    \adv~on behalf of \IdealFSig, and receiving the $n$ corresponding
    $(\uccmd{VerKey},((P,\lbl_i),\sid),\pk_i)$ responses.
    %
    % Hash computation
    \Sim then concatenates all $(\lbl_i,\typ_i,\pk_i)$ tuples into \st, and
    simulates a query to \IdealGRO, to set \did as ``$did:prism: h$'' as DID
    value.
    %
    % Tx sim
    With the DID value, \Sim simulates for \adv~the preparation of transaction
    \tx. That is, \Sim uses the master key with lowest label, $\lbl \gets
    \LMKL(\st)$, and simulates a query $(\uccmd{Sign},((P,\lbl),\sid),\tx)$ to
    \IdealFSig, obtaining from \adv~$(\uccmd{Signed},((P,\lbl),\sid),\tx,\sig)$.
    %
    % BB append
    \Sim also simulates appending the $(\tx,\sig)$ tuple to the bulletin board,
    by sending $(\uccmd{Append},(\tx,\sig))$ to \adv~on behalf of \IdealGdledger.
    %
    % Output
    Finally, \Sim sends the $(\uccmd{CreateOk},\sid,\did,\lbrace \pk_i
    \rbrace_{i\in[n]})$ to \IdealGPKIDID, and stores the tuple $(P,\did,\lbrace
    (\lbl_i,\typ_i,\pk_i)\rbrace_{i\in[n]})$.
    %
    % Additional considerations
    \todo{The following should probably go to the argumentation part.}
    Note that \did is generated by an ideal random oracle (thus, uniformly at
    random), just as in the real protocol \RealPKIDIDAtala. Hence, the
    probability that \IdealGPKIDID halts because $\did \in \DID$ is negligible,
    and \IdealGPKIDID outputs $(\uccmd{Created},\sid,\did,\lbrace \val_i=\pk_i
    \rbrace_{i\in[n]})$ to \IdealGPKIDID.
  \item[Simulating DID Read.]
    To simulate read operations, when \Sim receives $(\uccmd{Read},\sid,\did)$
    from \IdealGPKIDID, it first simulates for \adv~the call to \IdealGdledger, to
    get the current status $L$ of the bulletin board. Then, \Sim reproduces the
    processing a normal party would do: from older to newer entry in $L$ related
    to \did, applies the updates in a sequential order -- ignoring any
    \uccmd{UpdateDID} operation before a \uccmd{CreateDID}, and only applying
    the first \uccmd{CreateDID}. If, after applying an update, $\P_{Atala}$ is
    not met by the resulting set of keys and labels, \Sim skips the update.
    Otherwise, \Sim simulates for \adv~a verification query to \IdealFSig for
    the transaction corresponding to the update, and skips the update if the
    signature fails.
    %
    Finally, if after processing all transactions related to \did, the resulting
    contents differ from those stored by \Sim in its record $(P,\did,\sval)$,
    \Sim aborts \todo{Do we need this?}. Otherwise, \Sim returns the processed
    contents.    
  \item[Simulating DID Update.]
    % Load DID
    Upon receiving a request from dummy party $P$, \IdealGPKIDID fetches
    $(P,\did,\sval)$ from its records, and checks if $(\sval \setminus \sval_d)
    \cup \sval_a$ meets $\P_{Atala}$. If either step fails, \IdealGPKIDID
    aborts. Else, it sends $(\uccmd{Update},\sid,\did,\sval_d,\sval_a)$ to \Sim.
    \Sim then retrieves $(P,\did,\sval)$ from its records \todo{this is kept at
      some point, no?)} and simulates for \adv~the calls to \IdealFSig for the
    keys in the received $\sval_a$. \Sim gets the key with the lowest label from
    \sval, with label $\lbl$, bundles the transaction \tx, and simulates a call
    to \IdealFSig to get a signature over \tx. Finally, \Sim simulates the call
    to \IdealGdledger, and returns $(\uccmd{Updated},\sid,(\did,\lbrace \val_i
    \rbrace_{i\in[n_a]}))$ to \IdealGPKIDID, where the $\val_i$'s are the values
    output by \adv~when creating the new keys via simulated calls to \IdealFSig.
  \item[Simulating DID Deactivate.] Directly follows from DID Update.      
  \end{description}
  \todo{This to take into account when arguing indistinguishability}:
  \begin{itemize}
  \item The records of \Sim are always consistent with \IdealGdledger.
  \end{itemize}

  % General comments
  Now, observe that, in all ideal operations, the policy $\P_{Atala}$ is checked
  on the resulting data structure. If the policy is not satisfied, the
  functionality aborts. This is also the case for the real world.

  % Create comments
  Regarding create operations, note that, both in the simulation and in the real
  world, the resulting \did value is produced in a consistent way -- via a call
  to \IdealGRO, using the contents of the DID document to create.
  % Key generation and signature comments
  Similarly, the (verification) keys used and signatures are defined by \adv~
  both the real and ideal world -- in the latter, as part of \Sim's simulation
  of \IdealFSig.
  
\end{proof}


%%% Local Variables:
%%% mode: latex
%%% TeX-master: "prism-protocol"
%%% End:
