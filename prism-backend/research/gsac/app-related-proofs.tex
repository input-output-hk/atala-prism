\section{Proofs for Relationships with Other Schemes}
\label{app:related-proofs}

\subsection{Security of \CUASGS}

We prove that our \CUASGS construction is a secure group signature scheme,
according to \cite{bsz05}, if the underlying \CUASGen construction is secure.

\begin{theorem}[Anonymity of \CUASGS]
  If the base \CUASGen construction is anonymous according to
  \defref{def:anonymity-uas}, then \CUASRing is an anonymous group signature
  scheme.
\end{theorem}

\begin{proof}
  Assume an adversary \adv~ against anonymity in \CUASGS. This directly leads
  to an adverary \advB against anonymity of the $(\fissue^1,\feval^0,
  \finsp^{\upk})-\CUASGen$ instance.

  Concretely, observe that calls to \adv's $Ch,Open,SndToU,WReg,USK,CrptU$
  oracles can be directly simulated by calls to \advB's oracles $\CHALb,
  \OPEN,\HUGEN,\WREG,\CUGEN$ oracles. Thus, if \adv~ guesses its bit $b$
  with non-negligible probability, this directly leads to a non-negligible
  probability of guessing \advB's challenge bit.
  \qed
\end{proof}

\begin{theorem}[Traceability of \CUASGS]
  If the base \CUASGen construction has unforgeable signing according to
  \defref{def:sign-forge-uas}, then \CUASGS is a traceable group signature
  scheme.
\end{theorem}

\begin{proof}
  Assume an adversary \adv~ against traceability in \CUASGS. We build an
  adversary \advB against sign unforgeability of the $(\fissue^1,\feval^0,
  \finsp^{\upk})-\CUASGen$ instance.

  \advB's simulation of \adv~ environment is as in the proof for anonymity
  of \CUASGS. \advB simply outputs whatever \adv~ outputs in line 2 of its
  traceability game. Note that, as \CUASGS is a specific instantiation of
  \CUASGen, whenever \CUASGS accepts a signature, so does \CUASGen. Concretely,
  assume that \adv~ wins the traceability game in \figref{fig:model-gs} with
  $\upk \notin \HU \cup \CU$ at line 5. This implies that the check at line 11
  of the sign unforgeability game (\figref{fig:exp-uas-unfor-sign}) for \CUASGen
  outputs $0$ for all known {\uid}s. Similarly, if \adv~ wins the traceability
  game with condition $\Judge(\dots)=0$ at line 5, then \advB wins its sign
  unforgeability game with the corresponding check at line 6 in
  \figref{fig:exp-uas-unfor-sign}.
  %
  \qed
\end{proof}

\begin{theorem}[Non-frameability of \CUASGS]
  If the base \CUASGen construction is non-frameable according to
  \defref{def:frame-uas}, then \CUASGS is a non-frameable group signature
  scheme.
\end{theorem}

\begin{proof}
  Assume an adversary \adv~ against non-frameability in \CUASGS. We build an
  adversary \advB against non-frameability of the $(\fissue^1,\feval^0,
  \finsp^{\upk})-\CUASGen$ instance.

  \advB's simulation of \adv~ environment is as in the proof for anonymity
  of \CUASGS. \advB simply outputs whatever \adv~ outputs at line 2 of its
  game. Observe that a winning condition for \adv~ directly implies a winning
  condition for \advB. Concretely, if $\upk \in \HU \land \PRVUK[\upk] \neq
  \bot$ in \adv's game, then $\exists \uid \in \HU~\suchthat~\UK[\uid]=\usk$
  for \advB (line 6 of its game). Similarly, if \Judge returns $1$ in \adv's
  game, \advB's \Judge also returns $1$ at line 4. Finally, if \adv~ does not
  corrupt the owner of \upk, then $\uid \in \HU$ in \advB's game, and if \adv
  does not query its $SIGN$ oracle, then there does not exist a matching entry
  in $\SIG[\uid]$ for \advB either. Consequently, as stated, \advB wins its
  non-frameability every time that \adv does.
  \qed
\end{proof}

\subsection{Security of \CUASRing}

We prove that our \CUASRing construction is anonymous and unforgeable, if
the underlying \CUASGen construction is secure.

\begin{theorem}[Anonymity of \CUASRing]
  If the base \CUASGen construction is anonymous according to
  \defref{def:anonymity-uas}, then \CUASRing is an anonymous ring signature
  scheme.
\end{theorem}

\begin{proof}
  Assume $\adv$ is an adversary against ring signature anonymity for \CUASRing.
  Then, we build \advB against anonymity of \CUASGen. Concretely, \advB first
  generates $n$ keypairs by running its \IGEN (with $\fissue^{\sring}$) and
  \HUGEN oracles, and runs \OGEN at least once (with $\finsp^0$). Then, invokes
  $\adv$ passing as public keys the public keys of the $n$ generated issuers
  (users in \CUASRing). $\adv$'s $SIGN(i,R,\msg)$ queries are processed as
  follows. If user $i$ has not yet produced a credential for ring $R$, \advB
  makes a call to its \OBTAIN oracle, using $R$ as attribute set. If such a
  credential was already produced, then reuse it. Then, \advB uses the
  corresponding user and credential identifiers as parameters to a query to its
  own \SIGN oracle (specifying $\feval^{\attrs}$ as signature evaluation
  function). Eventually, $\adv$ outputs two user indexes, a ring $R$, and
  message \msg. If the checks at line 4 of \figref{fig:model-rs} pass, \advB
  generates the challenge signature by calling its own \CHALb oracle, and passes
  the result to $\adv$. Finally, \advB outputs whatever $\adv$ outputs.
  %
  \qed
\end{proof}

\begin{theorem}[Unforgeability of \CUASRing]
  If the base \CUASGen construction is non-frameable according to
  \defref{def:frame-uas}, then \CUASRing is an unforgeable ring signature
  scheme.
\end{theorem}

\begin{proof}
  Assume $\adv$ against unforgeability of \CUASRing wins the game. We build
  \advB winning the non-frameability game of \CUASGen with the same probability.
  %
  \advB prepares the environment as the adversary \advB against anonymity,
  and answers \adv's queries to the $SIGN$ oracle also as in the anonymity
  proof. To answer \adv's queries to its $CORR$ oracle, \advB leverages the
  \ICORR oracle in the non-frameability game for \UAS. Finally, \advB outputs
  whatever $\adv$ outputs.

  Note that, if $\adv$ wins its game, it means that the public keys of the
  issuers in $R$ all belong to uncorrupted users. Since the signature is
  accepted by $Verify$ at line 3 of the ring signature unforgeability game,
  there exists some \uid with an \usk matching the check at line 6 in the
  non-frameability game. Moreover, since $\adv$ never queried its $SIGN$
  oracle, then by construction there cannot exist a matching entry in the
  $\SIG[\uid]$ table. Thus, \advB wins the non-frameability game whenever
  $\adv$ wins its unforgeability game. Since \CUASGen is non-frameable, then
  this probability must be negligible.
  %
  \qed
\end{proof}

\subsection{Security of \CUASAC}

We prove that \CUASAC is an anonymous and unforgeable anonymous credential
scheme, according to \cite{fhs19}, if the underlying \CUASGen construction
is anonymous and has unforgeable signing. Note that issuance unforgeability
of \CUASGen is not needed, as the model in \cite{fhs19} does not allow
leveraging other credentials not any fixed custom issuance policy. Thus,
signature unforgeability from \UAS is enough. Note that AC schemes modelled
as in \cite{fhs19} do not have issuance anonymity, as the user public key
is passed to the issuer.

\begin{theorem}[Anonymity of \CUASAC]
  \label{thm:anon-cuasac}
  If the base \CUASGen construction has signature anonymity according to
  \defref{def:sign-anonymity-uas}, then \CUASAC is an anonymous AC scheme
  according to \cite{fhs19}.
\end{theorem}

\begin{proof}
  Given \adv~ against anonymity of \CUASAC as defined in \cite{fhs19}, we build
  an adversary \advB against anonymity of \CUASGen as defined in
  \defref{def:sign-anonymity-uas}.

  To simulate oracle calls by \adv, \advB simply redirects to its corresponding
  oracle, i.e.: for calls to $HU,CU,Obtain,Show,LoR$, \advB redirects to
  \HUGEN,\CUGEN,\OBTAIN,\SIGN,\CHALb, respectively. Note that ``translating''
  the inputs from the former into inputs for the latter is trivial.
  Obviously, if \adv~ wins in its anonymity game, then so does \advB with the
  same probability.
  % 
  \qed
\end{proof}

\begin{theorem}[Unforgeability of \CUASAC]
  \label{thm:forge-cuasac}
  If the base \CUASGen construction has sign unforgeability according to
  \defref{def:sign-forge-uas}, then \CUASAC is an unforgeable AC scheme.
\end{theorem}

\begin{proof}
  Given \adv~ against unforgeability of \CUASAC as defined in \cite{fhs19}, we
  build an adversary \advB against sign unforgeability of \CUASGen as defined in
  \defref{def:sign-forge-uas}.

  To simulate oracle calls by \adv, \advB acts as in the anonymity proof. Assume
  that \adv~ wins its game. Note that, since $b=1$ in line 5 of the
  unforgeability game of \figref{fig:model-ac}, then by construction, \Verify
  also returns $1$ in the sign unforgeability game of \UAS. In addition, if
  there is no corrupt user with a credential containing $D$, then this leads to
  \advB winning at line 9 in \figref{fig:exp-uas-unfor-sign}, as no credential
  owned by a corrupt user that can be extracted can make the $\feval^{\dattrs}$
  function output $\yeval^0=D$. Thus, \advB wins with the same probability as
  \adv~ does.  
  %
  \qed
\end{proof}

\subsection{Security of \CUASDAC} 

We prove that \CUASDAC is an anonymous and unforgeable DAC scheme, according to
the model in \figref{fig:model-dac}, if the underlying \CUASGen construction has
anonymous issuance and signing, and has unforgeable issuance and signing. 

\begin{theorem}[Issuance anonymity of \CUASDAC]
  If the base \CUASGen construction has issuance anonymity according to
  \defref{def:issue-anonymity-uas}, then \CUASDAC also has issuance anonymity.
\end{theorem}

\begin{proof}
  Given \adv~ against issuance anonymity of \CUASDAC as defined in
  $\Exp^{iss-anon-b}_{\adv,dac}(1^\secpar)$, we build an adversary \advB against
  issuance anonymity of \CUASGen as defined in \defref{def:issue-anonymity-uas}.

  To simulate oracle calls by \adv, \advB simply redirects to its corresponding
  oracle, i.e.: for calls to $HU,CU,HI,CI,Obtain,Show,IssLoR$, \advB redirects
  to \HUGEN,\CUGEN,\IGEN,\ICORR,\OBTAIN,\SIGN,\OBTCHALb, respectively. Note that
  ``translating'' the inputs from the former into inputs for the latter is
  trivial. Obviously, if \adv~ wins in its issuance anonymity game, then so does
  \advB with the  same probability.
  % 
  \qed
\end{proof}

\begin{theorem}[Signature anonymity of \CUASDAC]
  If the base \CUASGen construction has signature anonymity according to
  \defref{def:sign-anonymity-uas}, then \CUASDAC also has signature anonymity.
\end{theorem}

\begin{proof}
  This is the same as in \thmref{thm:anon-cuasac}.
  % 
  \qed
\end{proof}

\begin{theorem}[Unforgeability of \CUASAC]
  If the base \CUASGen construction has issuance and signature unforgeability
  according to \defref{def:issue-forge-uas} and \defref{def:sign-forge-uas}, and
  non-frameability according to \defref{def:frame-uas}, then \CUASAC is an
  unforgeable AC scheme.
\end{theorem}

\begin{proof}
  First, we define the $ExtractChain(\utrans)$ function for our construction.
  Note that \utrans is a tuple $(r,\Sig)$\footnote{This follows for the approach
    to transform non-interactive signature and verification in \UAS to interactive
    presentations, as described in \secref{ssec:variants-gsac}.}, where $r$ is a
  fresh random number produced by the environment (acting as honest verifier),
  and \Sig is a \CUASGen signature over $r$. Thus, $ExtractChain$ leverages
  \ExtractSign from \CUASGen (see \secref{ssec:security-uas}) to get the
  credential and associated \ipk used to produce \Sig. Note that, for this
  \ipk, $\ipk = \ipk_{n-1}$ in $\Exp^{forge}_{\adv,dac}$. In \CUASDAC, we can
  leverage $\ipk_{n-1}$ to fetch from the public list a signature $\Sig_{n-1}$
  generated by the issuer owner of $\ipk_{n-1}$. Again, we apply \ExtractSign
  from \CUASGen, to get the credential one layer up in the chain.
  $ExtractChain$ repeats this step, until reaching the top level credential,
  issued by $\ipk_0$.

  For the proof, we first replace within $Setup$ the $\NIZKSetup$ algorithms for
  issuance and signing with their corresponding $\NIZKSetup$. Due to their
  zero-knowledge property, the resulting game is indistinguishable. Also, note
  that $ExtractChain$ succeeds to extract the corresponding credential chain,
  due to simulation extractability of the NIZKs. 

  From the previous, the same reasoning as in \thmref{thm:forge-cuasac} applies
  to an adversary \adv~winning at line 5 of $\Exp^{forge}_{\adv,dac}$. Next,
  suppose that, from \utrans obtained at line 6, $ExtractChain$ outputs a
  delegation chain that makes \adv~win at line 10. This means that an
  uncorrupted issuer at position $i-1$ of the chain did not own a credential
  containing $D'$, but the issuer at position $i$ does. Then, either of the
  following must happen:

  \begin{itemize}
  \item The credential of the issuer at position $i$ is a forgery, meaning that
    it is a valid signature by the issuer at position $i-1$, who did not produce
    it. This can be used to break signature non-frameability of \CUASGen.
  \item The credential of the issuer at position $i$ is not a forgery, meaning
    that it is a valid signature by the issuer at position $i-1$, who produced
    it. But this contradicts the conditions required by $\fissue^{n,\dattrs}$,
    and therefore breaks issuance unforgeability of \CUASGen.
  \end{itemize}
  %
  \qed
\end{proof}

\subsection{Security of \CUASMPS}

As stated, we restrict to privacy-1 of MPS. Unforgeability-1 in MPS is
equivalent to sign unforgeability in \UAS. Unforgeability-2 in MPS is
equivalent to non-frameability in \UAS. Finally note that, in UAS, we do not
explicitly model extractability, but rather make it an inherent requirement in
the unforgeability and non-frameability definitions. Thus, our \CUASMPS
construction also ensures MPS-extractability.

\begin{theorem}[Privacy-1 of \CUASMPS]
  If the base \CUASGen construction has signature anonymity according to
  \defref{def:sign-anonymity-uas}, then \CUASMPS satisfies privacy-1
  according to \needcite.
\end{theorem}

\begin{proof}
  The proof is direct, by observing that the models are equivalent.
  
  First, note that, in MPS the adversary is required to output a function $F$,
  challenge message, and a pair of user secret keys and witnesses (in \UAS, the
  witnesses are the credentials). This is precisely what \CHALb requires as
  input in our \UAS modelling.

  Next, in MPS, validity of the output of $F$ is checked. In MPS, this
  translates to checking that $F$ does not ouptut $0$. In UAS, this is checked
  in the \SIGN and \CHALb oracles, which abort if the output produced by \feval
  does not belong in \rngfeval -- a well defined set of acceptable outputs by
  the chosen \feval.

  The MPS adversary is then allowed to make more oracle queries, and outputs a
  response bit. Just as in \UAS. Thus, if \CUASMPS satisfies signature
  anonymity, it is also an MPS scheme satisfying privacy-1.  
  % 
  \qed
\end{proof}

\begin{theorem}[Extractability of \CUASMPS]
  If the base \CUASGen construction has signature unforgeability according to
  \defref{def:sign-forge-uas}, then it is an extractable MPS scheme.
\end{theorem}

\begin{proof}
  The proof is direct, as the conditions checked in the extractability
  experiment in MPS are also checked in the signature unforgeability experiment
  for \UAS.

  Concretely, \Verify must accept the signature, and the ouptut of \feval and
  \finsp, computed with the values extracted by \ExtractSign, must respectively
  match those output by the adversary (for \feval), and computed honestly by the
  signature unforgeability (for \finsp).  
  \qed
\end{proof}

\begin{theorem}[Unforgeability-1 of \CUASMPS]
  If the base \CUASGen construction has signature unforgeability according to
  \defref{def:sign-forge-uas}, then \CUASMPS satisfies unforgeability-1 as
  defined in \needcite.
\end{theorem}

\begin{proof}
  For an MPS scheme to satisfy type-1 unforgeability, it has to be extractable.
  This is directly satisfied by signature unforgeability in \UAS.
  %
  Next, the adversary wins in the unforgeability-1 MPS experiment if it manages
  to produce a signature that has not been produced via a call to their \SIGN
  oracle, nor it can be associated to an corrupt user. Indeed, if that happens,
  then such adversary can be used to build an adversary winning the signature
  unforgeability game against \CUASMPS, through winning condition at line 12 --
  as \IdentifySig does not return 1 for any known \uid. Thus, if \CUASMPS has
  signature unforgeability, it also satisfies unforgeability-1 of MPS schemes.
  %
  \qed
\end{proof}

\begin{theorem}[Unforgeability-2 of \CUASMPS]
  If the base \CUASGen construction has non-frameability according to
  \defref{def:frame-uas}, then \CUASMPS satisfies unforgeability-2 as
  defined in \needcite.
\end{theorem}

\begin{proof}
  For an MPS scheme to have type-2 unforgeability, it has to be extractable.
  This is directly satisfied by non-frameability in UAS (a similar reasoning
  as for signature unforgeability and extractability applies here).
  %
  Next, in MPS, the adversary wins if the produced signature has not been
  output by the \SIGN oracle, but Extract outputs the identifier of an honest
  user. This is the same check done in lines 9 and 10 of the non-frameability
  game in \UAS. Therefore, if \CUASMPS satisfies non-frameability as defined in
  \defref{def:frame-uas}, it also satisfies type-2 unforgeability in MPS.
  \qed
\end{proof}

%%% Local Variables:
%%% mode: latex
%%% TeX-master: "uas-paper"
%%% End:
