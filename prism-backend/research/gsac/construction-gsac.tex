\subsection{Generic Construction of \GSAC}
\label{ssec:generic-gsac}

\subsubsection{Building Blocks}
\label{sssec:generic-gsac-bblocks}

\todo{This needs refinement, but ok for now.}

\paragraph{Commitment schemes.} %
Defined as a tuple $(\CSetup,\CCommit)$. Algorithm $\Cparm \gets
\CSetup(\Csecpar)$ produces the parameters for committing to values. $\Ccom
\gets \CCommit(\Cparm, \msg; r)$ produces a commitment \Ccom over message \msg,
from which we may omit randomness $r$. \todo{Informally define hiding and binding,
  leaving formal definitions to the appendix. Mention that we may only need binding
  on occasions.}

\paragraph{Encryption.} %
Defined as a tuple $(\ESetup,\EKeyGen,\EEnc,\EDec)$. Algorithm $\Eparm \gets
\ESetup(\Esecpar)$ produces public parameters for the other algorithms.
$(\Eek,\Edk) \gets \EKeyGen(\Eparm)$ generates the encryption-decryption key
pair, algorithm $\Ec \gets \EEnc(\Eek,\msg)$ encrypts message \msg with
encryption key \Eek, producing ciphertext \Ec, and $\msg/\bot \gets \EDec(\Edk,
\Ec)$ decrypts ciphertexts using decryption key \Edk. \todo{Informally define
  security properties we'll need.}

\paragraph{Non-Interactive Zero-Knowledge.} %
We use non-interactive zero-knowledge proofs of knowledge (NIZK), in the CRS
model \needcite. Informally, a NIZK scheme over an NP relation \NIZKRel is
defined as a tuple $(\NIZKSetup^\NIZKRel,\NIZKProve^\NIZKRel,
\NIZKVerify^\NIZKRel)$. Algorithm $\NIZKcrs \gets \NIZKSetup^\NIZKRel
(\NIZKsecpar)$ produces the common reference string \NIZKcrs. $\NIZKproof/\bot
\gets \NIZKProve^\NIZKRel(\NIZKcrs,\NIZKw,\NIZKx)$ creates a NIZK proof of
knowledge of witness \NIZKw for \NIZKx such that $(\NIZKw,\NIZKx) \in \NIZKRel$.
$1/0 \gets \NIZKVerify^\NIZKRel(\NIZKcrs,\NIZKx,\NIZKproof)$ verifies the proof.
\todo{Informally define security properties we'll need.}

\paragraph{Signatures over Blocks of Committed Messages, with proofs.} %
We use signature schemes that allow signing messages, or commitment to messages,
in blocks, and are compatible with (efficient) proof systems over the produced
signature and signed (commitments to) messages. For this purpose, we define such
schemes as a tuple $(\SBCMSetup,\SBCMKeyGen,\SBCMSign,\SBCMVerify)$. Algorithm
$\SBCMparm \gets \SBCMSetup(\SBCMsecpar)$ produces the public parameters for the
scheme. $(\SBCMvk,\SBCMsk) \gets \SBCMKeyGen(\SBCMparm)$ produces a
verification-signing key
pair. Algorithm $\SBCMsig \gets \SBCMSign(\SBCMsk,\Ccomset,\msgset)$ produces a
signature over a set of committed messages \Ccomset and a set of messages
\msgset, where either \Ccomset or \msgset may be empty. $1/0 \gets
\SBCMVerify(\SBCMvk,\SBCMsig,\Ccomset,\msgset)$ verifies a signature \SBCMsig
over commitment set \Ccomset and message set \msgset. In addition, the produced
signatures must be compatible with (efficient) NIZK proofs of knowledge of a
signature, and of (arbitrary) claims over the signed (committed) messages.
\todo{Informally define security properties we'll need.}

\subsubsection{Definition}
\label{sssec:generic-gsac-definition}

In the following algorithms, we make use of three different NP relations for
\NIZK proof systems:

\begin{description}
\item[$\NIZKRel_{\Issue}$:] Produced by users requesting a credential. It is
  defined as $\NIZKRel_{\Issue} = \lbrace (\usk,\credid), (\Cusk,\Ccredid) :
  \Cusk = \CCommit(\usk) \land \Ccredid = \CCommit(\credid) \rbrace$.
\item[$\NIZKRel_{\Sign}$:] Produced by users when signing a message. It is
  defined as $\NIZKRel_{\Sign} = \lbrace (\usk,\credid,\Attrs,\msg,\SBCMsig),
  (\Ec,\Ec',\DAttrs,M) : M = \CCommit(\msg) \land \Ec = \EEnc(\opk,\usk) \land
  \Ec' = \EEnc(\opk,\credid) \land \SBCMVerify(\ipk,\SBCMsig,\Attrs) = 1 \land
  \usk \in \Attrs \land \credid \in \Attrs \land \DAttrs \subseteq \Attrs \rbrace$.
\item[$\NIZKRel_{\Open}$:] Produced by the opener when opening a signature. It
  is defined as $\NIZKRel_{\Open} = \lbrace (\usk,\credid), (\Ec,\Ec') :
  \usk = \EDec(\osk,\Ec) \land \credid = \EDec(\osk,\Ec') \rbrace$.
\end{description}

In a nutshell \todo{describe high-level approach}.

\paragraph{$\Setup(\secpar,\nattrs) \rightarrow \parm$.} %
Sets up the public parameters \todo{can be distributed across responsible
  parties?}. Namely: $\Cparm \gets \CSetup(\secpar)$, $\Eparm \gets
\ESetup(\secpar)$, $\SBCMparm \gets \SBCMSetup(\secpar)$, $\NIZKcrs \gets
\NIZKSetup(\secpar)$. Outputs $\parm \gets (\Cparm,\SBCMparm,\Eparm,\NIZKcrs)$.

\paragraph{$\IKeyGen(\parm) \rightarrow (\ipk,\isk)$.} %
Generates the signing key pair for the issuer, by parsing \parm as
$(\cdot,\SBCMparm,\cdot,\cdot)$ and running $(\ipk,\isk) \gets
\SBCMKeyGen(\SBCMparm)$.

\paragraph{$\OKeyGen(\parm) \rightarrow (\opk,\osk)$.} %
Generates the encryption key pair for the opener, by parsing \parm as
$(\cdot,\SBCMparm,\Eparm,\cdot)$ and running $(\opk,\osk) \gets
\EKeyGen(\Eparm)$.

\paragraph{$\UKeyGen(\parm) \rightarrow (\upk,\usk)$.} %
Generates users' key pairs by choosing a random value within the attribute space
\AttrSpace, and committing to it. Concretely: $\usk \getr \AttrSpace$, $\upk
\gets \CCommit(\usk; r)$.

\paragraph{$\langle \Obtain(\gpk,\usk,\Attrs),\Issue(\gpk,\isk,\Attrs) \rangle
  \rightarrow \langle \cred/\bot,\utrans/\bot \rangle$.} %

\paragraph{$\Sign(\gpk,\usk,\cred,\DAttrs,\msg) \rightarrow \sig$.} %

\paragraph{$\Verify(\gpk,\sig,\DAttrs,\msg) \rightarrow 1/0$.} %

\paragraph{$\Open(\gpk,\osk,\trans,\sig,\DAttrs,\msg)
  \rightarrow (\upk,\Attrs,\oproof)/\bot$.} %

\paragraph{$\Judge(\gpk,\upk,\Attrs,\oproof,\sig,\DAttrs,\msg)
  \rightarrow 1/0$.} %

\subsection{Security of the generic \GSAC construction}
\label{ssec:gsac-const-sec}

\todo{XX}

\subsection{A Concrete Instantiation for \GSAC}
\label{ssec:instantiation-gsac}

Intuitively, the credentials that we generate for our \GSAC construction are
Pedersen commitments to $(\usk,\credid,\Attrs)$ tuples; i.e., they have
the following structure: $\gen{h}_0^r\gen{h}_1^{\usk}\gen{h}_2^{\credid}
\prod_{i \in [3,\nattrs+2]}\gen{h}_i^{\Attrs_i}$, where the exponent in
$\gen{h_0}$ is a fresh random value that ensures hiding, the exponents in
$\gen{h_1}$ and $\gen{h_2}$ encode the user private key and (unique) credential
identifier, and the remaining attributes are encoded in different
exponentitaions. This will actually be part of a BBS+ signature, which makes it
easy to add subsequent and efficient zero-knowledge proofs. Concretely, we will
be using several \NIZK proof systems, one for the $\langle \Obtain,\Issue
\rangle$ interactive protocol, another for signing, and a third one for opening.
Each will have its own relation, that we define in the corresponding algorithm.
The concrete algorithms are as follows.

\paragraph{$\Setup(\secpar,\nattrs) \rightarrow \parm$.} %
Generates a bilinear group $\BB = (p,\GG_1,\GG_2,\GG_T,\gen{g}_1,\gen{g}_2,e) \gets
\PGen(\secpar)$, $\nattrs+5$ additional generators $\gen{g},\gen{h},\gen{h}_0,...,
\gen{h}_{\nattrs+2}$ of $\GG_1$. Returns $\parm \gets
(\secpar,\nattrs,\BB,\gen{g},\gen{h}_0,...,\gen{h}_{\nattrs+2})$.

\paragraph{$\IKeyGen(\parm) \rightarrow (\ipk,\isk)$.} %
Parses \parm as $(\secpar,\dots)$ and runs $\NIZKcrs \gets \NIZKSetup(\secpar)$.
Outputs $\isk \gets \ZZ^*_p$, and $\ipk \gets (\NIZKcrs,\gen{g}_2^{\isk})$.

\paragraph{$\OKeyGen(\parm) \rightarrow (\opk,\osk)$.} %
Outputs $\osk \gets \ZZ^*_p$, and $\opk \gets \gen{g}^{\osk}$. This is the
opener's ElGamal encryption key pair \needcite.

\paragraph{$\UKeyGen(\parm) \rightarrow (\upk,\usk)$.} %
Outputs $\usk \gets \ZZ^*_p$, and $\upk \gets \gen{h}_1^{\usk}$. \upk will
simply be used to compute Pedersen commitments \needcite to \usk. However,
it is useful to precompute it, and we treat it as a sort of ``public key''.

\paragraph{$\langle \Obtain(\gpk,\usk,\Attrs),\Issue(\gpk,\isk,\Attrs) \rangle
  \rightarrow \langle \cred/\bot,\utrans/\bot \rangle$.} %
This interactive protocol is essentially a BBS+ signing  process. The user
commits to its private key and to a fresh random credential identifier, proves
knowledge of the corresponding private key and credential identifier, and
asks the issuer to sign these commitments, along with any other arbitraty set
of revealed attributes \Attrs. Note that the commitment to the private key is
hiding, while the commitment to the credential identifier is not. This is
because the private key can be shared across credentials, and we need to hide
it to prevent linking at issuance time. On the other hand, the credential
identifier is freshly chosen at random for each credential. Of course, although
both will be formally included as attributes in the credential, neither will be
revealed during signing. But still, including them will be crucial for opening.
For proving knowledge of the private key and credential identifier, we define
relation $\NIZKRel_{\Issue} = \lbrace (r, \usk, \credid),
(\Cusk,\Ccredid): \Cusk = \gen{h}_0^r\gen{h}_1^{\usk} \land \Ccredid =
\gen{h}_2^{\credid}\rbrace$.

\begin{itemize}
\item \underline{User}: Compute fress randomness $r,\credid \getr \ZZ^*_p,
  \Cusk \gets \gen{h}_0^r\gen{h}_1^{\usk}$, $\Ccredid \gets \gen{h}_2^{\credid}$
  and $\NIZKproof \gets \NIZKProve^{\NIZKRel_{\Issue}}(\NIZKcrs,(r,\usk,\credid),
  (\Cusk,\Ccredid))$. Send $(\Cusk,\Ccredid,\NIZKproof)$ to Issuer.
\item \underline{Issuer}: Run $\NIZKVerify^{\NIZKRel_{\Issue}}(\NIZKcrs,(\Cusk,
  \Ccredid),\NIZKproof)$ and return $0$ if it fails. Else, compute
  $x,\tilde{s} \getr \ZZ^*_p, A \gets
  (\gen{g}_1\Cusk h_0^{\tilde{s}} \Ccredid \prod_{i \in \Attrs}
  \gen{h}_i^{\Attrs_i})^{1/(\isk+x)}$.
  Send $(A,x,\tilde{s})$ to User, and output $\utrans \gets
  ((\Cusk,\Ccredid),(A,x,\tilde{s}),\Attrs,\NIZKproof)$.
\item \underline{User}: If $A = 1_{\GG_1}$: return $0$. Else, compute
  $s \gets r + \tilde{s}$. If $e(A,\gen{g}_2)^xe(A,\ipk) \neq
  e(\gen{g}_1\Cusk\gen{h}_0^{\tilde{s}}\Ccredid\prod_{i \in \Attrs}\gen{h}_i^{\Attrs_i},
  \gen{g}_2)$: return $0$. Else, return
  $(\credid,A,x,s)$.
\end{itemize}

\paragraph{$\Sign(\gpk,\usk,\cred,\DAttrs,\msg) \rightarrow \sig$.} %
To create a signature, the user first randomizes its BBS+ credential \cred and
encrypts its public key \upk, and the (commitment to the) credential identifier
\credid, with the opener's encryption key. Then, we extend
the usual NIZK protocol for proving knowledge of BBS+ signatures \cite{cdl16b}
to, in addition to proving knowledge of a signature (which, for us is the user
credential), also prove that the attributes encoding \credid and \usk within
the credential match the encrypted values. Finally, the user commits to \msg,
and includes the commitment within the proof. For this, we define relation
$\NIZKRel_{\Sign}= \lbrace (r,r',\usk,\credid,\Attrs,r_2,r_3,s',\msg),(\DAttrs,
c_1,c_2,c'_1,c'_2,M):
M = \gen{h}^\msg \land c_1 = g^r \land c_2 = \opk^r\gen{h}_1^{\usk} \land
c'_1 = g^{r'} \land c'_2 = \opk^{r'}\gen{h_2}^{\credid} \land
\hat{A}/d = (A')^{-x}\gen{h}_0^{r_2} \land
\gen{g}_1 \prod_{i \in \DAttrs} \gen{h}_i^{\Attrs_i} =
d^{r_3}\gen{h}_0^{-s'}\gen{h}_1^{-\usk}\gen{h}_2^{-\credid}
\prod_{i \notin \DAttrs} \gen{h}_i^{-\Attrs_i} \rbrace$.

\begin{itemize}
\item Parse \gpk as $(\ipk,\opk)$, and \cred as $(A,x,s)$.
\item Parse \cred as $(\credid,A,x,s)$ and re-randomize it as $r_1,r_2 \getr
  \ZZ^*_p, r_3 \gets r_1^{-1}, s' \gets s - r_2r_3$, $A' \gets A^{r_1},
  \hat{A} \gets (A')^{-x}(\gen{g}_1\gen{h}_0^s\gen{h}_1^{\usk}\gen{h}_2^{\credid}
  \prod_{i \in \Attrs}\gen{h}_i^{\Attrs_i})^{r_1}$,
  $d \gets (\gen{g}_1\gen{h}_0^s\gen{h}_1^{\usk}\gen{h}_2^{\credid}
  \prod_{i \in \Attrs}\gen{h}_i^{\Attrs_i})^{r_1}\gen{h}_0^{-r_2}$.
\item Encrypt $\upk=\gen{h}_1^{\usk}$ and $\Ccredid=\gen{h}_2^{\credid}$
  with ElGamal as $r,r' \getr \ZZ^*_Paleolithic's,
  c \gets (c_1 = \gen{g}^r,c_2 = \opk^r\gen{h}_1^{\usk})$,
  $c' \gets (c'_1 = \gen{g}^{r'},c'_2 = \opk^{r'}\gen{h}_2^{\credid})$
\item Compute $M \gets \gen{h}^\msg$ and
  $\NIZKproof \gets \NIZKProve^{\NIZKRel_{\Sign}}(\NIZKcrs,
  (r,r',\usk,\credid,\Attrs,r_2,r_3,s',\msg), (\DAttrs,c_1,c_2,c'_1,c'_2,M))$.
\item Return $\sig \gets (c=(c_1,c_2),c'=(c'_1,c'_2),(A',\hat{A},d),\NIZKproof)$.
\end{itemize}

\paragraph{$\Verify(\gpk,\sig,\DAttrs,\msg) \rightarrow 1/0$.} %
Parse \gpk as $(\ipk,\opk)$ and $\sig$ as $(c=(c_1,c_2),c'=(c'_1,c'_2),
(A',\hat{A},d), \NIZKproof)$. Check that $A' \neq 1_{\GG_1}$ and $e(A',\ipk) =
e(\hat{A},\gen{g}_2)$. Compute $M \gets \gen{h}^\msg,$ and return
$\NIZKVerify^{\NIZKRel_{\Sign}}(\NIZKcrs,(\DAttrs,c_1,c_2,c'_1,c'_2,M),
\NIZKproof)$.

\paragraph{$\Open(\gpk,\osk,\trans,\sig,\DAttrs,\msg)
  \rightarrow (\upk,\Attrs,\oproof)/\bot$.} %
For open algorithms, we define the following relation for correct decryption of
ElGamal ciphertexts: $\NIZKRel_{\Open} = \lbrace (\osk),(\msg,c_1,c_2):
c_2/c_1^{\osk} = \msg \rbrace$. Given $\NIZKRel_{\Open}$, the opener proceeds as
follows.

First, check that $\Verify$ accepts the signature, and abort otherwise. If \sig
is accepted, then parse \sig as $(c=(c_1,c_2),c'=(c'_1,c'_2),\cdot,\NIZKproof)$.
Set $\upk \gets c_2/c_1^{\osk}$, and $\Ccredid \gets c'_2/{c'_1}^{\osk}$. Find
entry $((\Cusk,\Ccredid),(A,x,\tilde{s}),\Attrs,\NIZKproof_{\Issue})$ in \trans,
and return $\bot$ if it does not exist. If it exists, compute proofs of correct
decryption
by running $\NIZKproof_1 \gets \NIZKProve^{\NIZKRel_{\Open}}(\NIZKcrs,\osk,
(\upk,c_1,c_2))$ and $\NIZKproof_2 \gets \NIZKProve^{\NIZKRel_{\Open}}(\NIZKcrs,
\osk,(\Ccredid,c'_1,c'_2))$, and return $(\upk,\Attrs,\oproof = (\Cusk,
\Ccredid,(A,x,\tilde{s}),\NIZKproof_1,\NIZKproof_2,\NIZKproof_{\Issue}))$.

\paragraph{$\Judge(\gpk,\upk,\Attrs,\oproof,\sig,\DAttrs,\msg)
  \rightarrow 1/0$.} %
First, check that $\Verify$ accepts \sig, and abort otherwise. Then, parse
\sig as $(c=(c_1,c_2),c'=(c'_1,c'_2),\cdot,\NIZKproof_\sig)$, and parse \oproof
as $(\Cusk,\Ccredid,(A,x,\tilde{s}),\NIZKproof_1,\NIZKproof_2,
\NIZKproof_{\Issue})$. To verify
the opening proof, we have to: (1) run $\NIZKVerify^{\NIZKRel_{\Open}}(\NIZKcrs,
(\upk,c_1,c_2),\NIZKproof_1)$ and $\NIZKVerify^{\NIZKRel_{\Open}}(\NIZKcrs,
(\Ccredid,c'_1,c'_2),\NIZKproof_2)$, proving correctness of the respective
decryptions; (2) verify the BBS+ signature by checking $e(A,\gen{g}_2)^x
e(A,\ipk) \stackrel{?}{=} e(\gen{g}_1\Cusk\gen{h}_0^{\tilde{s}}\Ccredid
\prod_{i \in \Attrs}\gen{h}_i^{\Attrs_i},g_2)$; and (3) run
$\NIZKVerify^{\NIZKRel_{\Issue}}(\NIZKcrs,(\Cusk,\Ccredid),\NIZKproof_{\Issue})$
proving that only the credential request was made by the owner of \usk. If
either check fails, the proof is rejected.

%%% Local Variables:
%%% mode: latex
%%% TeX-master: "uas"
%%% End:
