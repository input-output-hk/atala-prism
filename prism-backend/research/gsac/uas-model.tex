\section{Model for \UAS Schemes}
\label{sec:model-uas}

Universal Anonymous Signatures.

\subsection{Syntax}
\label{ssec:syntax}

In the following, we assume a setting with multiple groups. For simplicity,
we assume that each has its own issuer and inspector.

\todo{Are generalizations where different groups can share issuer or inspector
  straight forward?}

\todo{The notation for multisets is not strictly correct. Still, leaving it as
  is for now.} 

\commentwho{Jesus}{I'm trying to define \Sign, \Verify, \Inspect and \Judge such
  that multiple credentials can be used to produce a single signature, proving
  claims over arbitrary subsets of all the credentials owned by a user. \Verify
  and, consequently, \Inspect and \Judge need to receive all the \gpk's.
  \Inspect only requires one inspector's secret key, though, and will thus only
  produce that inspector's
  function of the signature and message (this is to avoid complex interactive
  protocols between many inspectors). This, however, creates a quite annoying
  syntax. It may just be easier to restrict signatures to claims over the
  attributes contained in \emph{one} credential.}

\commentwho{Jesus}{Maybe we could remove the attribute sets \uattrs, \dattrs,
  \tuattrs, \tdattrs, from the syntax, and assume that they are part of the
  signatures and/or credentials. This would certainly lead to a cleaner syntax,
  but, what impact would that have in other definitions, etc? Specifically, can
  we support selective disclosure credentials without them? (if so, \fissue,
  \feval and \finsp would be essential in doing so.); Right now, I am more in
  favour of tryint to remove the attribute sets from the definitions, as much
  as possible (and see if we can still support plain selective disclosure).}


\begin{description}
\item[$\parm \gets \Setup(\secpar)$.] Given a security parameter \secpar,
  returns a global system parameter variable \parm.
\item[$(\ipk,\isk) \gets \IKeyGen(\parm,\fissue)$.] Given global system
  parameters \parm, and the function \fissue to be used to check that credential
  requestors meet the conditions to be issued a credential, an issuer runs
  \IKeyGen to generate its issuing key pair. Hereafter, we assume that the
  public part \ipk is added to the group public key \gpk, as well as \fissue.
\item[$(\opk,\osk) \gets \OKeyGen(\parm,\feval,\finsp)$.] Given global system
  parameters \parm, and two functions \feval and \finsp, an inspector runs
  \OKeyGen to generate its inspecting key pair. The function \feval defines
  requirements that signed message-signature pairs must meet. The function
  \finsp defines the type of utility that will be extractable from those pairs
  that do not meet \feval. Hereafter, we assume that the public part \opk is
  added to the group public key \gpk, along with \feval and \finsp.
\item[$(\upk,\usk) \gets \UKeyGen(\parm)$.] Given global system parameters
  \parm, returns a user's key pair.
\item[$\langle \cred/\bot,\utrans/\bot \rangle \gets
  \langle
  \Obtain(\usk,\uattrs,\dattrs,
  \ldblbrace (gpk_i,\cred_i,\tuattrs_i,\tdattrs_i)\rdblbrace_{i \in \Issuers}),
  \Issue(\isk,\upk,\dattrs,
  \ldblbrace (\gpk_i,\tdattrs_i)\rdblbrace_{i \in \Issuers})
  \rangle$.] %
  This interactive protocol lets a user with key pair (\upk,\usk) running the
  \Obtain process, obtain a credential \cred from an issuer in the system, on
  attribute set $\uattrs \cup \dattrs$, where the attributes in $\uattrs$ (and
  the ones in the $\tuattrs_i$'s) are kept hidden from the issuer, while those
  in \dattrs (as well as those in the $\tdattrs_i$'s) are revealed. Moreover,
  $\uattrs \cup \dattrs$ must be a subset of $\bigcup_i (\tuattrs_i \cup
  \tdattrs_i)$. In order to get the credential, the user may leverage previously
  obtained credentials $\cred_i$ from a multiset of other issuers (including the
  one to whom a new credential is being requested) with public keys $\gpk_i$.
  Moreover, each credential $\cred_i$ can be used to prove claims over
  undisclosed attributes $\tuattrs_i$. The user outputs the produced credential
  \cred, while the issuer outputs the protocol transcript \utrans for the
  produced credential.
\item[$\sig \gets \Sign(\usk,\msg,\uattrs,\dattrs,
  \ldblbrace (\gpk_i,\cred_i,\tuattrs_i,\tdattrs_i)\rdblbrace_{i \in \Issuers})$.]
  The user with with secret key \usk, who obtained credentials $\cred_i$ from a
  multiset of issuers in the system, produces a signature \sig over message
  \msg. The produced signature proves arbitrary claims over the set of
  disclosed attributes $\bigcup_i \dattrs_i$ and undisclosed attributes
  $\bigcup_i \uattrs_i$, which must be a subset of the corresponding disclosed
  (resp. undisclosed) attributes \tdattrs (resp. \tuattrs) in the credentials
  used to generate the signature.
\item[$1/0 \gets
  \Verify(\sig,\msg,\dattrs, \ldblbrace \gpk_i \rdblbrace_{i \in \Issuers})$.]
  Checks whether \sig is a valid signature, over message \msg, satisfying
  arbitrary claims over disclosed attributes $\dattrs$, as well as
  potentially other undisclosed attributes endorsed by issuers in the systsem
  with public keys $\bigcup_i \gpk_i$.
\item[$(\y,\iproof)/\bot \gets
  \Inspect(\osk,\trans,\sig,\msg,\dattrs,
  \ldblbrace \gpk_i \rdblbrace_{i \in \Issuers})$.] %
  \commentwho{Jesus}{I do not like this description. It's complicated. This needs
    to be rephrased from scratch.}
  Executed by the inspector with opening key \osk. Receives a signature \sig
  over message \msg, produced by a set of credentials satisfying claims on
  disclosed attributes $\dattrs$, as well as potentially undisclosed
  attributes, all endorsed by a multiset of issuers with public keys $\gpk_i$. 
  If \trans is a valid set of transcripts of the execution of a
  $\langle\Obtain,\Issue\rangle$ interactive protocol where the signer obtained
  the credential associated to the inspector's group%
  \footnote{Note: we are not constraining that the group to which the inspector
    with key \osk belongs to, issued only one of the credentials used to
    generate the signature being inspected.}, then the function outputs
  a value $\y$ derived from the signer's credential and public key, and the
  signed message, as well as a proof of correct inspection.
\item[$1/0 \gets \Judge(\y,\iproof,\sig,\msg,\dattrs,
  \ldblbrace \gpk_i\rdblbrace_{i \in \Issuers})$.] %
  Checks if \iproof is a valid inspection correctness proof for the value \y,
  obtained by applying \Inspect to the the pair signature \sig over \msg. 
\end{description}

The correctness and security properties are defined with the help of the
following sets of oracles, and global variables that help oracles and games
keep consistent state.

\paragraph{Issuance, evaluation, and inspection private functions.} %
We emphasize that, both in our syntax definition, as well as on the following
modelling, we make use of three different and abstract functions: \fissue,
\feval and \finsp. The three functions are run by users (maybe, on user-private
data) as part of different algorithms in an \UAS scheme. In all cases, the
user has to prove correctness of their computation. We introduce them next,
\todo{and will give concrete examples in \secref{sec:uac-instantiation}.}

\todo{Domains (attributes, messages, etc.) are not correctly defined here. But
  for now, just leave it as is to at least convey the idea. Also, this needs
  refinement in general: what properties (if any) do these functions need to
  meet in order to be meaningful?}
\begin{description}
\item[$\fissue: (\uattrs,\dattrs) \rightarrow 0/1$.] Governs what customized
  conditions an issuer requires in order to issue credentials, when receiving
  a request from a user with a set of disclosed attributes \dattrs, and
  undisclosed attributes \uattrs (which need to be kept private from the
  issuer). \fissue returns 1 to accept a request, 0 to reject it.  
\item[$\feval: (\msg,\scred) \rightarrow 0/1$.] Governs customized conditions
  to define whether the owner of a set of credentials -- or, more specifically,
  credentials encoding certain attributes -- should be able to sign certain
  messages.
\item[$\finsp: (\upk,\scred,\sig) \rightarrow \bin^*$.] Defines what utility
  value, derived from the user's public key, credentials, and produced signature
  (or a subset thereof), should be extractable by the group inspector.
\end{description}

\paragraph{Global Variables.} %
All the honestly generated users (i.e., all honestly generated user key pairs,
since we assume a one-to-one relationship between user and user key pair), as
well as all honestly generated credentials, are assigned identifiers. Week's
typically write as \uid for users' identifiers, and \cid for credentials'. The
adversary can
refer to any individual user or credential using the corresponding identifier --
even though he may not know the actual contents of the key pairs or credentials.
In games that involve challenge user or credential identifiers, we use \cuid and
\ccid to refer to these challenge user/credential.
%
The games, usually through the oracles, keep track of the users (through
set \UK) and credentials (through table \CRED) that are created as a result of
oracle calls by the adversary, as well as the attributes that each credential
is associated (through table \ATTR), and the user to which each credential has
been issued to (through table \OWNR). We use the user and credential identifiers
to reference specific user key pairs, credentials, and credential attributes.
For instance, $\UK[\uid]$ refers to the user key pair corresponding to user
with identifier \uid, $\PUBUK[\uid]$ refers to the public key of that key pair,
and $\PRVUK[\uid]$ to the private key; credential $\CRED[\cid]$ refers to the
credential data associated to the credential with identifier \cid; $\ATTR[\cid]$
is the set of attributes that was assigned to the credential with identifier
\cid; and $\OWNR[\cid]$ is set to the \uid corresponding to the user that
credential \cid was issued to.
%
Additionally, the games keep track of all honest and corrupt users that have
been generated, through sets \HU and \CU, respectively. They also keep track of
the signatures that have been honestly produced, through the \SIG table. Since
signatures are produced through credentials, the \SIG table is indexed with
{\cid}s. For the anonymity game, we also need to keep track of the challenge
signatures that the adversary has obtained, in order to prevent trivial wins
by allowing any of them to be opened. 
%
All global variables are initially set by the games to empty values (denoted
with $\bot$), and all tables/sets are initialized as empty tables/sets, denoted
with $\emptyset$. Also, for readability, we abuse the syntax as follows: we
write $\CRED[\uid]$ to mean $\CRED[\cid]$ for all $\cid$ such that
$\OWNR[\cid] = \uid$; \todo{Something else?}

\paragraph{Oracles.} %
Oracles are the interface of the adversary with the corresponding games. In
other words: through these oracles, the game environment exposes to the adversary
functionality that could otherwise be executed only by honest parties with
private knowledge -- knowledge that would make the adversary capable of
trivially breaking the security properties formalized in the experiments.
In the game-based definitions of our \GSAC model, we leverage the following
oracles, which are formally defined in \figref{fig:oracles1} and
\figref{fig:oracles2}. Note that we identify issuers and inspectors by the group
identifier.

\begin{description}
\item[\HIGEN.] Adds a new honest issuer to the game, by honestly generating
  its key pair.
\item[\CIGEN.] Adds a new corrupt issuer to the game or, if the specified issuer
  already exists and is honest, corrupts it.
\item[\HOGEN.] Adds a new honest inspector to the game, by honestly generating
  its key pair.
\item[\COGEN.] Adds a new corrupt inspector to the game or, if the specified
  inspector already exists and is honest, corrupts it. 
\item[\HUGEN.] Adds a new honest user to the game, by honestly generating
  the user's key pair.
\item[\CUGEN.] Adds a new corrupt user to the game or, if the specified
  user already exists and is honest, corrupts it.
\item[\OBTISS.] Lets the adversary add a new honestly generated credential to
  the game, on behalf of an honest user.
\item[\OBTAIN.] Enables the adversary to play the role of a dishonest issuer
  in games that support it, by letting it interact with honest users who want to
  receive credentials.
\item[\ISSUE.] Allows the adversary to play the role of dishonest users,
  requesting an honest issuer to produce credentials for them.
\item[\SIGN.] Lets the adversary get signatures from credentials belonging
  to honest users.
\item[\OPEN.] Given an honestly produced signature, lets the adversary learn
  which credential was used to produce it.
\item[\CHALb.] Upon receiving two challenge credentials, a common intersecting
  set of attributes, and a message, returns a signature produced by one of these
  two credentials, defined by the bit $b$, which is established in the anonymity
  game.
\end{description}

{%\setlength\intextsep{\sep}
  \begin{figure*}[htp!]
    \centering
    \scalebox{0.9}{

      \begin{minipage}[t]{0.55\textwidth}

        \procedure{$\HIGEN(\gid,\fissue)$}{%
          \pcif \gid \in \HI \lor \gid \in \CI: \pcreturn \bot \\
          (\ipk,\isk) \gets \IKeyGen(\parm) \\
          \IK[\gid] \gets (\ipk,\isk,\fissue) \\
          \HI \gets \HI \cup \lbrace \gid \rbrace \\
          \GK[\gid] \gets (\ipk,\fissue,\cdot,\cdot,\cdot) \\
          \pcreturn \ipk \\
        }        
        
        \procedure{$\CIGEN(\gid,\ipk,\fissue)$}{%
          \pcif \gid \in \CI: \pcreturn \bot \\
          \CI \gets \CI \cup \lbrace \gid \rbrace \\          
          \pcif \gid \in \HI: \\
          \pcind \HI \gets \HI \setminus \lbrace \gid \rbrace; \\
          \pcind \pcreturn \IK[\gid] \\
          \pcelse: \IK[\gid] = (\ipk,\bot,\fissue) \\ 
          \GK[\gid] \gets (\ipk,\fissue,\cdot,\cdot,\cdot) \\         
          \pcreturn \top \\
        }

        \procedure{$\HUGEN(\uid)$}{%
          \pcif \uid \in \HU \lor \uid \in \CU: \pcreturn \bot \\
          (\upk,\usk) \gets \UKeyGen(\parm) \\
          \UK[\uid] \gets (\upk,\usk);
          \HU \gets \HU \cup \lbrace  \uid \rbrace \\
          \pcreturn \top
        }        
        
      \end{minipage}
    }
    \scalebox{0.9}{
      
      \begin{minipage}[t]{.5\textwidth}

        \procedure{$\HOGEN(\gid,\feval,\finsp)$}{%
          \pcif \gid \in \HO \lor \gid \in \CO: \pcreturn \bot \\
          (\opk,\osk) \gets \OKeyGen(\parm) \\
          \OK[\gid] \gets (\opk,\osk,\feval,\finsp) \\
          \HO \gets \HO \cup \lbrace  \gid \rbrace \\
          \GK[\gid] \gets (\cdot,\cdot,\opk,\feval,\finsp) \\          
          \pcreturn \opk \\
        }        
        
        \procedure{$\COGEN(\gid,\opk,\feval,\finsp)$}{%
          \pcif \gid \in \CO: \pcreturn \bot \\
          \CO \gets \CO \cup \lbrace \gid \rbrace \\          
          \pcif \gid \in \HO: \\
          \pcind \HO \gets \HO \setminus \lbrace \gid \rbrace; \\
          \pcind \pcreturn \OK[\gid]) \\
          \pcelse: \OK[\gid] = (\opk,\bot,\feval,\finsp) \\
          \GK[\gid] \gets (\cdot,\cdot,\opk,\feval,\finsp) \\
          \pcreturn \top \\
        }

        \procedure{$\CUGEN(\uid,\upk)$}{%
          \pcif \uid \in \CU: \pcreturn \bot \\
          \CU \gets \CU \cup \lbrace \uid \rbrace \\          
          \pcif \uid \in \HU: \\
          \pcind \HU \gets \HU \setminus \lbrace \uid \rbrace; \\
          \pcind \pcreturn (\UK[\uid],\CRED[\uid]) \\
          \pcelse: \UK[\uid] = (\upk,\bot) \\          
          \pcreturn \top
        }
        
      \end{minipage}
      
    }

    \caption{Detailed oracles available in our model (1/2). Oracles for
      generating key material for users, issuers, and inspectors.}
    \label{fig:oracles1}
  \end{figure*}
}

{%\setlength\intextsep{\sep}
  \begin{figure*}[htp!]
    \centering
    \scalebox{0.9}{

      \begin{minipage}[t]{0.55\textwidth}

        \procedure{$\ISSUE(\cid,\uid,\gid,\dattrs,
          \ldblbrace (\sgid,\scid,\tdattrs) \rdblbrace)$}{%
          \pcif \uid \notin \CU: \pcreturn \bot \\
          \pcif \gid \notin \HI: \pcreturn \bot \\
%          \pcif \exists \cid
          \pcif \exists \gid' \in \sgid~\st~\gid' \notin \HI: \pcreturn \bot \\
          \pcif \exists \cid' \in \scid~\st~\CRED[\cid'] \neq \bot: \pcreturn \bot \\
          \langle \cdot, \utrans \rangle \gets
          \langle \adv, \\
          \pcind \pcind
          \Issue(\PRVIK[\gid],\PUBUK[\uid],
          \ldblbrace (\GK[\sgid],\tdattrs)\rdblbrace) \rangle \\
          \trans[\cid] \gets \utrans \\
          \OWNR[\cid] \gets \uid;~\ATTR[\cid] \gets (\scid, \bot, \dattrs) \\
          \pcreturn \top \\          
        }                

        \procedure{$\OBTAIN(\cid,\uid,\gid,\uattrs,\dattrs,
          \ldblbrace (\sgid,\scid,\tuattrs,\tdattrs) \rdblbrace)$}{%
          \pcif \uid \notin \HU: \pcreturn \bot \\
          \pcif \exists \cid' \in \scid~\st~\CRED[\cid'] \neq \bot: \pcreturn \bot \\
          \langle \cred, \cdot \rangle \gets
          \langle \Obtain(\PRVUK[\uid],\uattrs,\dattrs, \\
          \hspace*{65pt}
          \ldblbrace (\GK[\sgid],\CRED[\scid],\tuattrs,\tdattrs) \rdblbrace), \\
          \hspace*{50pt} \adv \rangle \\
          \CRED[\cid] \gets \cred \\
          \OWNR[\cid] \gets \uid;~\ATTR[\cid] \gets (\scid, \uattrs, \dattrs) \\
          \pcreturn \top \\
        }

        \procedure{$\INSPECT(\gid,\sig,\msg)$}{%
          \textrm{Let}~\uid~\textrm{be s.t.}~(\sig,\msg,\scid,\uattrs,\dattrs,\sgid)
          \in \SIG[\uid] \\
          \pcif \sig \in \CSIG: \pcreturn \bot \\
          (\y,\oproof) \gets
          \Inspect(\PRVOK[\gid],\trans,\sig,\msg,\dattrs,\GK[\sgid]) \\
          \pcreturn (\y,\oproof)
        }
        
      \end{minipage}
    }
    \scalebox{0.9}{
      
      \begin{minipage}[t]{.5\textwidth}

        \procedure{$\OBTISS(\cid,\uid,\gid,\uattrs,\dattrs,
          \ldblbrace (\sgid,\scid,\tuattrs,\tdattrs) \rdblbrace)$}{%
          \pcif \uid \notin \HU: \pcreturn \bot \\
          \pcif \gid \notin \HI: \pcreturn \bot \\          
          \pcif \exists \gid' \in \sgid~\st~\gid' \notin \HI: \pcreturn \bot \\
          \pcif \exists \cid' \in \scid~\st~\CRED[\cid'] \neq \bot:
          \pcreturn \bot \\
          \langle \cred, \utrans \rangle \gets
          \langle \Obtain(\PRVUK[\uid],\uattrs,\dattrs,\\
          \hspace*{75pt} \ldblbrace (\GK[\sgid],
          \CRED[\scid],\tuattrs,\tdattrs) \rdblbrace), \\
          \hspace*{60pt} \Issue(\PRVIK[\gid],\PUBUK[\uid],
          \ldblbrace (\GK[\sgid],\tdattrs)
          \rdblbrace ) \rangle \\
          \trans[\cid] \gets \utrans;~\CRED[\cid] \gets \cred \\
          \OWNR[\cid] \gets \uid;~\ATTR[\cid] \gets (\scid,\uattrs,\dattrs) \\
          \pcreturn \top \\
        }


        \procedure{$\SIGN(\msg,\uid,\uattrs,\dattrs,
          \ldblbrace (\sgid,\scid,\tuattrs,\tdattrs) \rdblbrace )$}{%
          \pcif \uid \notin \HU: \pcreturn \bot \\
          \cred \gets \CRED[\cid] \\
          \sig \gets \Sign(\PRVUK[\uid],\msg,\uattrs,\dattrs, \\
          \hspace*{42pt}
          \ldblbrace (\GK[\sgid],\CRED[\scid],\tuattrs,\tdattrs) \rdblbrace) \\
          \SIG[\uid] \gets \SIG[\uid] \cup
          \lbrace (\sig,\msg,\scid,\uattrs,\dattrs,\sgid) \rbrace \\
          \pcreturn \sig \\
        }                

        \procedure{$\CHALb(\msg,
          (\uid_{0,1},\uattrs_{0,1},\dattrs_{0,1},\ldblbrace (\sgid,\cscid_{0,1},\tuattrs_{0,1},\tdattrs_{0,1}) \rdblbrace))$}{%
          \pcif \cuid_0 \notin \HU \lor \cuid_1 \notin \HU: \pcreturn \bot \\
          \pcif \feval(\msg,\CRED[\scid_0]) \neq \feval(\msg,\CRED[\scid_1]):
          \pcreturn \bot \\
          \csig \gets \Sign(\PRVUK[\uid_b],\msg,\uattrs_b,\dattrs_b, \\
          \hspace*{42pt}
          \ldblbrace (\GK[\sgid],\CRED[\scid_b],\tuattrs_b,\tdattrs_b) \rdblbrace) \\
          \CSIG \gets \CSIG \cup
          \lbrace (\csig,\msg,\cscid_b,\uattrs_b,\dattrs_b,\sgid) \rbrace \\
          \pcreturn \csig
        }
        
      \end{minipage}
      
    }

    \caption{Detailed oracles available in our model (2/2). Oracles for
      obtaining credentials, signatures, and processing them.}
    \label{fig:oracles}
  \end{figure*}
}

\paragraph{Correctness.} %
Correctness of \UAS schemes is formalized through the experiment in
\figref{fig:exp-uas-corr}. It states that a signature for some message \msg,
revealing attributes \dattrs, which was honestly produced though a credential
that was obtained by an honest user interacting with an honest issuer, with a
set of attributes \attrs such that $\dattrs \subseteq \attrs$, must be accepted
by \Verify. Moreover, an honestly produced correctness proof of opening for such
signature, revealing the public key pair of the user, must also be accepted by
\Judge.

\begin{figure}[htp!]
  \procedure{$\ExpCorrect(1^\secpar)$}{%
    \parm \gets \Setup(1^\secpar) \\
    (\uid,\msg,\uattrs,\dattrs,
    \ldblbrace (\sgid,\scid,\tuattrs,\tdattrs) \rdblbrace)
    \gets \adv^{\HIGEN,\HOGEN,\HUGEN,\OBTISS}(\parm) \\
    \pcif \dattrs \nsubseteq \bigcup_{\cid \in \scid} \DATTR[\cid] \lor
    \bigcup \tdattrs \nsubseteq \dattrs: \pcreturn \bot \\
    \sig \gets \Sign(\PRVUK[\uid],\msg,\uattrs,\dattrs,
    \ldblbrace (\GK[\sgid],\CRED[\scid],\tuattrs,\tdattrs) \rdblbrace ) \\
    \pccomment{Signature fails to verify} \\
    \pcif \Verify(\sig,\msg,\dattrs,\GK[\sgid]) = 0: \pcreturn 0 \\
    \pccomment{An inspector among those involved in credentials used to produce
      \sig cannot inspect it} \\
    \pcif \exists \gid \in \sgid~\st~
    \Inspect(\PRVOK[\gid],\trans,\sig,\msg,\dattrs,
    \ldblbrace \GK[\sgid] \rdblbrace) = \bot: \pcreturn 0 \\
    \pccomment{Some of the produced $(\y,\iproof)$ pairs is rejected by
      \Judge \todo{This must be conditioned on \feval = 0!}} \\
    \pcif \exists \gid \in \sgid~\st~
    (\y,\iproof) \gets \Inspect(\PRVOK[\gid],\trans,\sig,\msg,\dattrs,
    \ldblbrace \GK[\sgid] \rdblbrace) \land \\
    \pcind
    \Judge(\y,\iproof,\sig,\msg,\dattrs,\ldblbrace \GK[\sgid] \rdblbrace) = 0):
    \pcreturn 0 \\
    \textrm{Let}~(\fissue^\gid,\feval^\gid,\finsp^\gid)~\textrm{denote the
      issue, evaluation and inspection functions defined} \\
    \textrm{by each $\gid \in \sgid$} \pccomment{\todo{Come up with some
        notation for this}} \\
    \pccomment{\fissue rejects some issued credential} \\
    \pcif \exists \gid \in \sgid~\st~\fissue^\gid(\tuattrs,\tdattrs) = 0:
    \pcreturn 0 \\
    \pccomment{\feval rejects the signature-message pair} \\
    \pcif \exists \gid \in \sgid~\st~\feval^\gid(\msg,\sig) = 0: \pcreturn 0 \\
    \pccomment{\y is the wrong value according to \finsp} \\
    \pcif \exists \gid \in \sgid~\st~
    \y \neq \finsp^\gid(\PUBUK[\uid],\CRED[\scid])): \pcreturn 0 \\
    \pcreturn 1
  }
  \caption{Correctness experiment for \UAS schemes. \todo{Reorganize this.}}
  \label{fig:exp-uas-corr}
\end{figure}

\subsection{Security Properties}
\label{ssec:security}

\paragraph{Anonymity.} %
In group signatures, anonymity captures that no adversary must be able to learn,
from any group signature, the identity (e.g., member index) of its signer. In 
anonymous credentials, it requires that no adversary should learn anything about
the holder of a successfully shown credential, beyond that he owns a credential
attesting for the claimed function of the attributes it contains. In both GS and
AC, it is also typically required that
multiple signatures/presentations by the users are unlinkable. The approach to
formally state this property is in both cases frequently the same: the adversary
picks two (honest) users (or credentials in the AC case), the game randomly
chooses one of them, and lets the adversary request challenge
signatures/presentations from it. The adversary wins if it succeeds in guessing
which was the chosen user/credential better than guessing at random. In group
signatures, the game must also restrict the adversary from opening challenge
signatures. In anonymous credentials, the game must further constraint the
adversary to output credentials that have some common function of (a subset of)
its attributes.

In our notion of anonymity for \UAS, we need to merge the previous constraints.
Furthermore, a key difference with group signatures is that the game requires
the adversary to output credential identifiers, rather than user identifiers.
Specifically, this means that the adversary may actually output two credentials
that belong to the same user. Therefore, in some sense, the anonymity we get is
more general than that of group signatures. Moreover, in order to prevent
trivial wins by the adversary, we have to restrict that the challenge signatures
belong to a group that has been ``programmed'' with the same evaluation and
inspection functions, which is again a generalization over group signatures,
as discussed in the sequel. The formal specification of the anonymity game is
given in \figref{fig:exp-uas-anonb}. 

\begin{figure}[htp!]
  \procedure{$\ExpAnonb(1^\secpar)$}{%
     \parm \gets \Setup(1^\secpar) \\
     (\cscid_0,\cscid_1,\status) \gets
     \adv^{\lbrace\HI,\CI,\HO,\CO,\HU,\CU\rbrace\GEN,\OBTAIN,\SIGN,\INSPECT}
     (\choose,\parm) \\
     b^* \gets
     \adv^{\lbrace\HI,\CI,\HO,\CO,\HU,\CU\rbrace,\GEN,\OBTAIN,\SIGN,\INSPECT,\CHALb}
     (\guess,\status) \\
     \pcreturn b^*
  }
  \caption{Anonymity experiment for \UAS schemes.}
  \label{fig:exp-uas-anonb}
\end{figure}

\begin{definition}{(Anonymity of \UAS)}
  We define the advantage \AdvAnon of $\adv$ against \ExpAnonb as
  $\AdvAnon=|\Pr\lbrack\ExpAnono(1^\secpar)=1\rbrack-
  \Pr\lbrack\ExpAnonz(1^\secpar)=1\rbrack|$.
  %
  An \UAS scheme satisfies anonymity if, for any p.p.t. adversary $\adv$,
  \AdvAnon is a negligible function of $1^\secpar$.
\end{definition}

\paragraph{Discussion on the generality of anonymity in \UAS schemes.} %
This notion of anonymity is more general than that of group signatures in the
sense that calls to the \INSPECT oracle reveal an arbitrary function of the
identity of the user -- which can certainly be the member index itself, as it
is frequent in group signatures, ore any other function computable from the
user public key, its credentials, and produced signature. \todo{Delve more in
  this.}

\paragraph{Traceability.} %
Traceability is one of the unforgeability-related properties in group
signatures. It captures that any signature accepted by \Verify needs to open
to one of the users that joined the group. While there is no traceability notion
in anonymous credentials, it is natural to map it to their unforgeability
property; if only because both require the issuer to be honest. Unforgeability
in anonymous credentials typically ensures that no adversary can get a verifier
to accept a credential presentation requiring a set of attributes that is not
contained in one of the credentials controlled by the adversary.

Our notion of traceability for \UAS is inspired by both. First, we restrict to
signatures that are accepted by \Verify. For any such signature, \Inspect has to
create a valid output (i.e., must not abort). The output of \Inspect has to be
accepted by \Judge too. Moreover, there must exist one user (either honest or
corrupt) who used his public key to obtain one or more credentials that allowed
him to produce a signature which, once inspected, returns \y.

\begin{figure}[htp!]
  \procedure{$\ExpTrace(1^\secpar)$}{%
    \parm \gets \Setup(1^\secpar) \\
    \pccomment{\todo{Right now, in \COGEN we may not know \osk! Then, the checks
        in the last part of the game may not be executable!}} \\
     (\sig,\dattrs,\msg) \gets
     \adv^{\lbrace \HI,\CO,\HU,\CU \rbrace \GEN,\OBTISS,\ISSUE,\SIGN,\INSPECT}
     (\parm) \\
     \pcreturn 1~\pcif \Verify(\sig,\msg,\dattrs,\GK[\sgid]) = 1~\land~( \\
     \pcind \pccomment{All \Obtain transcripts fail to open \sig} \\
     \pcind \nexists \gid \in \sgid:
     \Inspect(\PRVOK[\gid],\trans,\sig,\msg,\dattrs,\sgid) = \bot~\lor \\
     \pcind \pccomment{Such transcript exists, but ...} \\
     \pcind \exists\gid \in \sgid~\st~
     \Inspect(\PRVOK[\gid],\trans,\sig,\msg,\dattrs) =
     (\y,\iproof)~\land (\\
     \pcind \pcind \pccomment{... the proof is rejected by \Judge ...} \\
     \pcind \pcind \Judge(\y,\iproof,\sig,\msg,\dattrs,\GK[\sgid]) = 0~\lor \\
     \pcind \pcind \pccomment{... or \Judge accepts the proof, but inspect
       returns an invalid value} \\
     \pcind \pcind \nexists \uid \in \CU \cup \HU~\st~
     \y = \finsp(\PUBUK[\uid],\CRED[\uid],\sig) \\
     \pcind ) \\
     ) \\
     \pcreturn 0
  }
  \caption{Traceability experiment for \UAS schemes.}
  \label{fig:exp-uas-trace}
\end{figure}

\begin{definition}{(Traceability of \UAS)}
  We define the advantage \AdvTrace of $\adv$ against \ExpTrace as
  $\AdvTrace=\Pr\lbrack\ExpTrace(1^\secpar)=1\rbrack$.
  %
  A \UAS scheme satisfies traceability if, for any p.p.t. adversary $\adv$,
  \AdvTrace is a negligible function of $1^\secpar$.
\end{definition}

\paragraph{Discussion on the generality of traceability in \UAS schemes.} %
Note that this notion of traceability is, again, strictly more general than
both group signatures and anonymous credentials. On the group signatures side,
it all depends on the \Inspect function. If we set it to the traditional
inspection function, where it returns the identity (index) of the signer, then
our notion of traceability becomes exactly that of conventional group
signatures. On the other hand, \Inspect may be set to a non-injective function,
meaning that it can be possible that different user public keys, credentials and
signature tuples, may produce the same output. In that case, though, our notion
of traceability ensures that there must exist \emph{at least one} user capable
of issuing that signature.
%
On the anonymous credentials side, note that our definition allows that users
combine multiple credentials in a ``presentation''. However, in anonymous
credentials, it is frequent to require that all the attributes proven in a
presentation must be contained in a single credential. Note though that our
definition requires all credentials to be tied to the same user identity.
Therefore, it is a natural extension as (in the real life), it is very frequent
to combine multiple identifying documents to prove that you are entitled to
something. Moreover, any specific instantiation that restricts signatures
to be associated to just one credential directly leads to the previously
mentioned ``one credential per presentation'' rule.


\paragraph{Non-frameability.} %
Non-frameability variants are a core unforgeability-type property in group
signatures. However, no
similar property is modeled for anonymous credentials (\todo{see the discussion
  in \secref{sec:introduction} for further detail}). It is a quite strong
property, as it must be ensured even in the presence of dishonest issuer and
opener. Intuitively, it prevents the adversary from creating a signature that
frames an honest user. Depending on the inspection capabilities of the scheme,
this framing could be done in different ways; i.e., by convincing third parties
that signatures by different (possibly corrupt) users are linked, or directly
by having open proofs output the identity of a user who did not create the
signature being opened.
%
In \UAS schemes, in order for a user to be framed...

\begin{figure}[htp!]
  \procedure{$\ExpNonframe(1^\secpar)$}{%
    \parm \gets \Setup(1^\secpar) \\
     (\sig,\dattrs,\msg,\upk,\cred,\oproof) \gets
     \adv^{\lbrace\CI,\CO,\HU,\CU\rbrace\GEN,\OBTAIN,\SIGN}(\parm) \\
     \pcreturn 1~\pcif \Verify(\gpk,\sig,\dattrs,\msg) = 1 \land
     \OWNR[\cred] \in \HU \\%\land \cred \notin \CCRED~\land\\
     \pcind \Judge(\gpk,\upk,\cred,\oproof,\sig,\dattrs,\msg) = 1~\land \\
     \pcind \pccomment{\sig was not produced by \SIGN, or \upk and \cred do not match} \\
     \pcind (\sig \notin \SIG[\OWNR[\cred]] \lor \PUBUK[\OWNR[\cred]] \neq \upk) \\
     \pcreturn 0
  }
  \caption{Non-frameability experiment for \GSAC schemes.}
  \label{fig:exp-gsac-frame}
\end{figure}

\begin{definition}{(Non-frameability of \GSAC)}
  We define the advantage \AdvNonframe of $\adv$ against \ExpNonframe as
  $\AdvNonframe=\Pr\lbrack\ExpNonframe(1^\secpar)=1\rbrack$.
  %
  A \GSAC scheme satisfies non-frameability if, for any p.p.t. adversary $\adv$,
  \AdvNonframe is a negligible function of $1^\secpar$.
\end{definition}

\paragraph{Discussion.} %
\todo{Argue that our properties cover (and augment) a direct combination
  (even if possible) of their GS and AC counterparts. Specifically, anonymity
  of \GSAC is stronger than anonymity in GS and AC, and the same for trace and
  non-frame.}

\todo{It would be nice to prove that a \GSAC scheme to which we remove the
  \Open/\Judge functions becomes an AC scheme. And conversely, a \GSAC scheme
  where all credentials have no attributes, and where we restrict to only
  one credential per user, becomes a conventional GS scheme.}

\subsection{Transformation to Interactive Presentations}
\label{ssec:interactivetransform}

\todo{Is it possible to give it in some generic sense?}

%%% Local Variables:
%%% mode: latex
%%% TeX-master: "gsac"
%%% End:
