\documentclass{llncs}%[10pt]{llncs}
\pagestyle{plain}

\usepackage[n,advantage,operators,sets,adversary,landau,probability,notions,
logic,ff,mm,primitives,events,complexity,asymptotics,keys]{cryptocode}
\usepackage{amssymb}
\usepackage{xspace}
\usepackage[normalem]{ulem}
\usepackage{hyperref}
\usepackage{mathtools}

\title{Universal Anonymous Signatures}
\author{}

% Custom shorthands for this paper
\definecolor{darkgreen}{rgb}{0.1,0.7,0.1}
\newcommand{\todo}[1]{{\colorbox{red}{\bf TODO:}\textcolor{red}{#1}}}
\newcommand{\doubt}[1]{{\colorbox{blue}
    {\textcolor{white}{\bf DOUBT:}}\textcolor{blue}{#1}}}
\newcommand{\response}[1]{{\colorbox{orange}{\bf RESPONSE:}
                                             \textcolor{orange}{#1}}}
\newcommand{\modified}[1]{{\textcolor{orange}{#1}}}
\newcommand{\commentwho}[2]{{\colorbox{darkgreen}{{\bf #1:}}
    \textcolor{darkgreen}{#2}}}
\newcommand{\comment}[1]{{\textcolor{orange}{\# #1}}}
%\newcommand{\note}[1]{{\colorbox{gray}{\bf NOTE:}\textcolor{gray} {#1}}}
\newcommand{\needcite}{[\colorbox{red}{?}]}
\newcommand{\figref}[1]{Fig. \ref{#1}}
\newcommand{\tabref}[1]{Table \ref{#1}}
\newcommand{\lstref}[1]{Listing \ref{#1}}
\newcommand{\secref}[1]{Section \ref{#1}}
\newcommand{\appref}[1]{Appendix \ref{#1}}
\newcommand{\defref}[1]{Definition \ref{#1}}

%\renewcommand{\qedsymbol}{$\blacksquare$}
\newcommand{\br}[1]{\ensuremath{\lbrack #1 \rbrack}}
\newcommand{\gen}[1]{\ensuremath{#1}}

\mathchardef\mhyphen="2D

\def\parm{\ensuremath{param}\xspace}
\def\Exp{\ensuremath{\mathsf{Exp}}\xspace}
\def\Adv{\ensuremath{\mathsf{Adv}}\xspace}
\def\status{\ensuremath{\mathsf{st}}\xspace}

% Symbols for GSAC
\def\GSAC{\ensuremath{\mathsf{GSAC}}\xspace}
\def\HUGEN{\ensuremath{\mathsf{HUGEN}}\xspace}
\def\CUGEN{\ensuremath{\mathsf{CUGEN}}\xspace}
\def\OBTAIN{\ensuremath{\mathsf{OBTAIN}}\xspace}
\def\OBTISS{\ensuremath{\mathsf{OBTISS}}\xspace}
\def\ISSUE{\ensuremath{\mathsf{ISSUE}}\xspace}
\def\SIGN{\ensuremath{\mathsf{SIGN}}\xspace}
\def\OPEN{\ensuremath{\mathsf{OPEN}}\xspace}
\def\CHALb{\ensuremath{\mathsf{CHAL}_b}\xspace}
\def\ATTR{\ensuremath{\mathsf{ATTR}}\xspace}
\def\OWNR{\ensuremath{\mathsf{OWNR}}\xspace}
\def\CRED{\ensuremath{\mathsf{CRED}}\xspace}
\def\USK{\ensuremath{\mathsf{\MakeUppercase{\usk}}}\xspace}
\def\PUBUSK{\ensuremath{\mathsf{P}\USK}\xspace}
\def\PRVUSK{\ensuremath{\mathsf{S}\USK}\xspace}
\def\HU{\ensuremath{\mathsf{HU}}\xspace}
\def\CU{\ensuremath{\mathsf{CU}}\xspace}
\def\SIG{\ensuremath{\mathsf{SIG}}\xspace}
\def\CSIG{\ensuremath{\mathsf{CSIG}}\xspace}
\def\CCRED{\ensuremath{\mathsf{CC}}\xspace}
\def\secpar{\ensuremath{\kappa}\xspace}
\def\param{\ensuremath{\param}\xspace}
\def\isk{\ensuremath{isk}\xspace}
\def\ipk{\ensuremath{ipk}\xspace}
\def\osk{\ensuremath{osk}\xspace}
\def\opk{\ensuremath{opk}\xspace}
\def\gpk{\ensuremath{gpk}\xspace}
\def\usk{\ensuremath{usk}\xspace}
\def\upk{\ensuremath{upk}\xspace}
\def\Attrs{\ensuremath{A}\xspace}
\def\DAttrs{\ensuremath{D}\xspace}
\def\cred{\ensuremath{cred}\xspace}
\def\utrans{\ensuremath{reg}\xspace}
\def\trans{\ensuremath{\MakeUppercase{reg}}\xspace}  
\def\msg{\ensuremath{m}\xspace}
\def\sig{\ensuremath{\sigma}\xspace}
\def\oproof{\ensuremath{\pi}\xspace}
\def\Setup{\ensuremath{Setup}\xspace}
\def\IKeyGen{\ensuremath{IKG}\xspace}
\def\OKeyGen{\ensuremath{OKG}\xspace}
\def\UKeyGen{\ensuremath{UKG}\xspace}
\def\Obtain{\ensuremath{Obtain}\xspace}
\def\Issue{\ensuremath{Issue}\xspace}
\def\Sign{\ensuremath{Sign}\xspace}
\def\Verify{\ensuremath{Verify}\xspace}
\def\Open{\ensuremath{Open}\xspace}
\def\Judge{\ensuremath{Judge}\xspace}
\def\ExpCorrect{\ensuremath{\Exp_{\GSAC,\adv}^{corr}}\xspace}
\def\ExpAnonb{\ensuremath{\Exp_{\GSAC,\adv}^{anon-b}}\xspace}
\def\ExpAnonz{\ensuremath{\Exp_{\GSAC,\adv}^{anon-0}}\xspace}
\def\ExpAnono{\ensuremath{\Exp_{\GSAC,\adv}^{anon-1}}\xspace}
\def\AdvAnon{\ensuremath{\Adv_{\GSAC,\adv}^{anon}}\xspace}
\def\ExpTrace{\ensuremath{\Exp_{\GSAC,\adv}^{trace}}\xspace}
\def\AdvTrace{\ensuremath{\Adv_{\GSAC,\adv}^{trace}}\xspace}
\def\ExpNonframe{\ensuremath{\Exp_{\GSAC,\adv}^{frame}}\xspace}
\def\AdvNonframe{\ensuremath{\Adv_{\GSAC,\adv}^{frame}}\xspace}
\def\choose{\ensuremath{\mathsf{choose}}\xspace}
\def\guess{\ensuremath{\mathsf{guess}}\xspace}
\def\uid{\ensuremath{\mathsf{uid}}\xspace}
\def\cuid{\ensuremath{\mathsf{uid}^*}\xspace}
\def\cid{\ensuremath{\mathsf{cid}}\xspace}
\def\ccid{\ensuremath{\mathsf{cid}^*}\xspace}
\def\csig{\ensuremath{\mathsf{\sigma}^*}\xspace}


%%% Local Variables: 
%%% mode: pdflatex
%%% TeX-master: "gsac.tex"
%%% End:

% Symbols for GSAC
\def\GSAC{\ensuremath{\mathsf{GSAC}}\xspace}
\def\WREG{\ensuremath{\mathsf{WREG}}\xspace}
\def\RREG{\ensuremath{\mathsf{RREG}}\xspace}
\def\HUGEN{\ensuremath{\mathsf{HUGEN}}\xspace}
\def\CUGEN{\ensuremath{\mathsf{CUGEN}}\xspace}
\def\OBTAIN{\ensuremath{\mathsf{OBTAIN}}\xspace}
\def\OBTISS{\ensuremath{\mathsf{OBTISS}}\xspace}
\def\ISSUE{\ensuremath{\mathsf{ISSUE}}\xspace}
\def\SIGN{\ensuremath{\mathsf{SIGN}}\xspace}
\def\OPEN{\ensuremath{\mathsf{OPEN}}\xspace}
\def\CHALb{\ensuremath{\mathsf{CHAL}_b}\xspace}
\def\ATTR{\ensuremath{\mathsf{ATTR}}\xspace}
\def\OWNR{\ensuremath{\mathsf{OWNR}}\xspace}
\def\CRED{\ensuremath{\mathsf{CRED}}\xspace}
\def\USK{\ensuremath{\mathsf{\MakeUppercase{\usk}}}\xspace}
\def\PUBUSK{\ensuremath{\mathsf{P}\USK}\xspace}
\def\PRVUSK{\ensuremath{\mathsf{S}\USK}\xspace}
\def\HU{\ensuremath{\mathsf{HU}}\xspace}
\def\CU{\ensuremath{\mathsf{CU}}\xspace}
\def\SIG{\ensuremath{\mathsf{SIG}}\xspace}
\def\CSIG{\ensuremath{\mathsf{CSIG}}\xspace}
\def\CCRED{\ensuremath{\mathsf{CC}}\xspace}
\def\secpar{\ensuremath{\kappa}\xspace}
\def\nattrs{\ensuremath{n}\xspace}
\def\param{\ensuremath{\param}\xspace}
\def\isk{\ensuremath{isk}\xspace}
\def\ipk{\ensuremath{ipk}\xspace}
\def\osk{\ensuremath{osk}\xspace}
\def\opk{\ensuremath{opk}\xspace}
\def\gpk{\ensuremath{gpk}\xspace}
\def\usk{\ensuremath{usk}\xspace}
\def\upk{\ensuremath{upk}\xspace}
\def\Attrs{\ensuremath{\mathsf{A}}\xspace}
\def\DAttrs{\ensuremath{\mathsf{D}}\xspace}
\def\cred{\ensuremath{cred}\xspace}
\def\utrans{\ensuremath{reg}\xspace}
\def\trans{\ensuremath{\MakeUppercase{reg}}\xspace}  
\def\msg{\ensuremath{m}\xspace}
\def\sig{\ensuremath{\sigma}\xspace}
\def\oproof{\ensuremath{\pi}\xspace}
\def\Setup{\ensuremath{Setup}\xspace}
\def\IKeyGen{\ensuremath{IKG}\xspace}
\def\OKeyGen{\ensuremath{OKG}\xspace}
\def\UKeyGen{\ensuremath{UKG}\xspace}
\def\Obtain{\ensuremath{Obtain}\xspace}
\def\Issue{\ensuremath{Issue}\xspace}
\def\Sign{\ensuremath{Sign}\xspace}
\def\Verify{\ensuremath{Verify}\xspace}
\def\Open{\ensuremath{Open}\xspace}
\def\Judge{\ensuremath{Judge}\xspace}
\def\ExpGSACCorrect{\ensuremath{\Exp_{\GSAC,\adv}^{corr}}\xspace}
\def\ExpGSACAnonb{\ensuremath{\Exp_{\GSAC,\adv}^{anon-b}}\xspace}
\def\ExpGSACAnonz{\ensuremath{\Exp_{\GSAC,\adv}^{anon-0}}\xspace}
\def\ExpGSACAnono{\ensuremath{\Exp_{\GSAC,\adv}^{anon-1}}\xspace}
\def\AdvGSACAnon{\ensuremath{\Adv_{\GSAC,\adv}^{anon}}\xspace}
\def\ExpGSACTrace{\ensuremath{\Exp_{\GSAC,\adv}^{trace}}\xspace}
\def\AdvGSACTrace{\ensuremath{\Adv_{\GSAC,\adv}^{trace}}\xspace}
\def\ExpGSACNonframe{\ensuremath{\Exp_{\GSAC,\adv}^{frame}}\xspace}
\def\AdvGSACNonframe{\ensuremath{\Adv_{\GSAC,\adv}^{frame}}\xspace}
\def\choose{\ensuremath{\mathsf{choose}}\xspace}
\def\guess{\ensuremath{\mathsf{guess}}\xspace}
\def\uid{\ensuremath{\mathsf{uid}}\xspace}
\def\cuid{\ensuremath{\mathsf{uid}^*}\xspace}
\def\cid{\ensuremath{\mathsf{cid}}\xspace}
\def\ccid{\ensuremath{\mathsf{cid}^*}\xspace}
\def\csig{\ensuremath{\mathsf{\sigma}^*}\xspace}
\def\credid{\ensuremath{cid}\xspace}
\def\Ccredid{\ensuremath{\Ccom_{\credid}}\xspace}
\def\Cusk{\ensuremath{\Ccom_{\usk}}\xspace}
\def\attr{\ensuremath{attr}\xspace}

%%% Local Variables: 
%%% mode: pdflatex
%%% TeX-master: "gsac.tex"
%%% End:

% Symbols for UAS
\def\UAS{\ensuremath{\mathsf{UAS}}\xspace}
\def\IGEN{\ensuremath{\mathsf{IGEN}}\xspace}
\def\OGEN{\ensuremath{\mathsf{OGEN}}\xspace}
\def\UGEN{\ensuremath{\mathsf{UGEN}}\xspace}
\def\ICORR{\ensuremath{\mathsf{ICORR}}\xspace}
\def\OCORR{\ensuremath{\mathsf{OCORR}}\xspace}
\def\UCORR{\ensuremath{\mathsf{UCORR}}\xspace}
\def\CCORR{\ensuremath{\mathsf{CCORR}}\xspace}
\def\HUGEN{\ensuremath{\mathsf{HUGEN}}\xspace}
\def\CUGEN{\ensuremath{\mathsf{CUGEN}}\xspace}
\def\ISET{\ensuremath{\mathsf{ISET}}\xspace}
\def\OBTAIN{\ensuremath{\mathsf{OBTAIN}}\xspace}
\def\OBTISS{\ensuremath{\mathsf{OBTISS}}\xspace}
\def\ISSUE{\ensuremath{\mathsf{ISSUE}}\xspace}
\def\SIGN{\ensuremath{\mathsf{SIGN}}\xspace}
\def\OPEN{\ensuremath{\mathsf{OPEN}}\xspace}
\def\INSPECT{\ensuremath{\mathsf{INSPECT}}\xspace}
\def\CHALb{\ensuremath{\mathsf{SIGCHAL}_b}\xspace}
\def\OBTCHALb{\ensuremath{\mathsf{OBTCHAL}_b}\xspace}
\def\SIMSETUP{\ensuremath{\mathsf{SimSetup}}\xspace}
\def\SIMOBTAIN{\ensuremath{\mathsf{SIMOBTAIN}}\xspace}
\def\SIMSIGN{\ensuremath{\mathsf{SIMSIGN}}\xspace}
\def\SIMOPEN{\ensuremath{\mathsf{SIMOPEN}}\xspace}
\def\ATTR{\ensuremath{\mathsf{ATT}}\xspace}
\def\UATTR{\ensuremath{\mathsf{U}\ATTR}\xspace}
\def\DATTR{\ensuremath{\mathsf{D}\ATTR}\xspace}
\def\OWNR{\ensuremath{\mathsf{OWN}}\xspace}
\def\GRP{\ensuremath{\mathsf{GRP}}\xspace}
\def\ISR{\ensuremath{\mathsf{ISR}}\xspace}
\def\CCRED{\ensuremath{\mathsf{CCRD}}\xspace}
\def\IK{\ensuremath{\mathsf{IK}}\xspace}
\def\OK{\ensuremath{\mathsf{OK}}\xspace}
\def\UK{\ensuremath{\mathsf{UK}}\xspace}
\def\GK{\ensuremath{\mathsf{GK}}\xspace}
\def\PUBIK{\ensuremath{\mathsf{P}\IK}\xspace}
\def\PRVIK{\ensuremath{\mathsf{S}\IK}\xspace}
\def\PUBOK{\ensuremath{\mathsf{P}\OK}\xspace}
\def\PRVOK{\ensuremath{\mathsf{S}\OK}\xspace}
\def\PUBUK{\ensuremath{\mathsf{P}\UK}\xspace}
\def\PRVUK{\ensuremath{\mathsf{S}\UK}\xspace}
\def\II{\ensuremath{\mathsf{I}}\xspace}
\def\OO{\ensuremath{\mathsf{O}}\xspace}
\def\UU{\ensuremath{\mathsf{U}}\xspace}
\def\HI{\ensuremath{\mathsf{HI}}\xspace}
\def\CI{\ensuremath{\mathsf{CI}}\xspace}
\def\HO{\ensuremath{\mathsf{HO}}\xspace}
\def\CO{\ensuremath{\mathsf{CO}}\xspace}
\def\HU{\ensuremath{\mathsf{HU}}\xspace}
\def\CU{\ensuremath{\mathsf{CU}}\xspace}
\def\CC{\ensuremath{\mathsf{CC}}\xspace}
\def\GEN{\ensuremath{\mathsf{GEN}}\xspace}
\def\CORR{\ensuremath{\mathsf{CORR}}\xspace}
\def\SIG{\ensuremath{\mathsf{SIG}}\xspace}
\def\CSIG{\ensuremath{\mathsf{CSIG}}\xspace}
%\def\CCRED{\ensuremath{\mathsf{CC}}\xspace}
\def\secpar{\ensuremath{\kappa}\xspace}
%\def\param{\ensuremath{\param}\xspace}
\def\isk{\ensuremath{isk}\xspace}
\def\ipk{\ensuremath{ipk}\xspace}
\def\sipk{\ensuremath{\mathbf{\ipk}}\xspace}
\def\osk{\ensuremath{osk}\xspace}
\def\opk{\ensuremath{opk}\xspace}
\def\gpk{\ensuremath{gpk}\xspace}
\def\sgpk{\ensuremath{\mathbf{\gpk}}\xspace}
\def\usk{\ensuremath{usk}\xspace}
\def\upk{\ensuremath{upk}\xspace}
\def\uk{\ensuremath{uk}\xspace}
\def\Cred{\ensuremath{C}\xspace}
\def\sCred{\ensuremath{\mathbf{C}}\xspace}
\def\credid{\ensuremath{cid}\xspace}
\def\scredid{\ensuremath{\boldsymbol{cid}}\xspace}
\def\cred{\ensuremath{crd}\xspace}
\def\scred{\ensuremath{\mathbf{\cred}}\xspace}
\def\utrans{\ensuremath{reg}\xspace}
\def\trans{\ensuremath{\mathbf{reg}}\xspace}
\def\IdentifyCred{\ensuremath{\mathsf{IdentifyCred}}\xspace}
\def\IdentifySig{\ensuremath{\mathsf{IdentifySig}}\xspace}
\def\ExtractSetup{\ensuremath{\mathsf{ExtSetup}}\xspace}
\def\ExtractSign{\ensuremath{\mathsf{ExtSign}}\xspace}
\def\ExtractIssue{\ensuremath{\mathsf{ExtIss}}\xspace}
\def\c{\ensuremath{c}\xspace}
\def\y{\ensuremath{y}\xspace}
\def\yissue{\ensuremath{y_{is}}\xspace}
\def\yeval{\ensuremath{y_{ev}}\xspace}
\def\tyeval{\ensuremath{\tilde{y}_{ev}}\xspace}
\def\ceval{\ensuremath{\c_{ev}}\xspace}
\def\Yeval{\ensuremath{Y_{ev}}\xspace}
\def\TYeval{\ensuremath{\tilde{Y}_{ev}}\xspace}
\def\yinsp{\ensuremath{y_{op}}\xspace}
\def\tyinsp{\ensuremath{\tilde{y}_{op}}\xspace}
\def\cinsp{\ensuremath{\c_{op}}\xspace}
\def\sig{\ensuremath{\sigma}\xspace}
\def\Sig{\ensuremath{\Sigma}\xspace}
\def\iproof{\ensuremath{\pi}\xspace}
\def\tiproof{\ensuremath{\tilde{\pi}}\xspace}
\def\Setup{\ensuremath{Setup}\xspace}
\def\KeyGen{\ensuremath{KG}\xspace}
\def\IKeyGen{\ensuremath{IKG}\xspace}
\def\OKeyGen{\ensuremath{OKG}\xspace}
\def\UKeyGen{\ensuremath{UKG}\xspace}
\def\ISet{\ensuremath{ISet}\xspace}
\def\Obtain{\ensuremath{Obt}\xspace}
\def\Issue{\ensuremath{Iss}\xspace}
\def\Sign{\ensuremath{Sign}\xspace}
\def\Verify{\ensuremath{Verify}\xspace}
\def\Inspect{\ensuremath{Inspect}\xspace}
\def\Judge{\ensuremath{Judge}\xspace}
\def\ExpCorrect{\ensuremath{\Exp_{\UAS,\adv}^{corr}}\xspace}
\def\CorrectIssue{\ensuremath{CorrectIssue}\xspace}
\def\CorrectEval{\ensuremath{CorrectEval}\xspace}
\def\CorrectEvalInspect{\ensuremath{CorrectEvalInspect}\xspace}
\def\CorrectInspect{\ensuremath{CorrectInspect}\xspace}
\def\ExpExtractIssue{\ensuremath{\Exp_{\UAS,\adv}^{ext-iss}}\xspace}
\def\ExpExtractSign{\ensuremath{\Exp_{\UAS,\adv}^{ext-sig}}\xspace}
\def\ExpIdentifyCred{\ensuremath{\Exp_{\UAS,\adv}^{id-cred}}\xspace}
\def\ExpIdentifySign{\ensuremath{\Exp_{\UAS,\adv}^{id-sig}}\xspace}
\def\ExpSimAnonb{\ensuremath{\Exp_{\UAS,\adv}^{sim-anon-b}}\xspace}
\def\ExpSigAnonb{\ensuremath{\Exp_{\UAS,\adv}^{sig-anon-b}}\xspace}
\def\ExpIssAnonb{\ensuremath{\Exp_{\UAS,\adv}^{iss-anon-b}}\xspace}
\def\ExpSigAnonz{\ensuremath{\Exp_{\UAS,\adv}^{sig-anon-0}}\xspace}
\def\ExpIssAnonz{\ensuremath{\Exp_{\UAS,\adv}^{iss-anon-0}}\xspace}
\def\ExpSigAnonzo{\ensuremath{\Exp_{\UAS,\adv}^{sig-anon-01}}\xspace}
\def\ExpSigAnono{\ensuremath{\Exp_{\UAS,\adv}^{sig-anon-1}}\xspace}
\def\ExpIssAnono{\ensuremath{\Exp_{\UAS,\adv}^{iss-anon-1}}\xspace}
\def\ExpSigAnonoz{\ensuremath{\Exp_{\UAS,\adv}^{sig-anon-10}}\xspace}
\def\ExpSimAnono{\ensuremath{\Exp_{\UAS,\adv}^{sim-anon-1}}\xspace}
\def\ExpSimAnonz{\ensuremath{\Exp_{\UAS,\adv}^{sim-anon-0}}\xspace}
\def\AdvSigAnon{\ensuremath{\Adv_{\UAS,\adv}^{sig-anon}}\xspace}
\def\AdvIssAnon{\ensuremath{\Adv_{\UAS,\adv}^{iss-anon}}\xspace}
\def\AdvSimAnon{\ensuremath{\Adv_{\UAS,\adv}^{sim-anon}}\xspace}
\def\AdvIdCred{\ensuremath{\Adv_{\UAS,\adv}^{id-cred}}\xspace}
\def\AdvIdSign{\ensuremath{\Adv_{\UAS,\adv}^{id-sig}}\xspace}
\def\ExpTrace{\ensuremath{\Exp_{\UAS,\adv}^{trace}}\xspace}
\def\AdvTrace{\ensuremath{\Adv_{\UAS,\adv}^{trace}}\xspace}
\def\IssueTrace{\ensuremath{IssTrace}\xspace}
\def\IssueForge{\ensuremath{IssForge}\xspace}
\def\ExpForge{\ensuremath{\Exp_{\UAS,\adv}^{forge}}\xspace}
\def\ExpForgeIssue{\ensuremath{\Exp_{\UAS,\adv}^{iss-forge}}\xspace}
\def\AdvForge{\ensuremath{\Adv_{\UAS,\adv}^{forge}}\xspace}
\def\AdvForgeIssue{\ensuremath{\Adv_{\UAS,\adv}^{iss-forge}}\xspace}
\def\ExpForgeSign{\ensuremath{\Exp_{\UAS,\adv}^{sig-forge}}\xspace}
\def\AdvForgeSign{\ensuremath{\Adv_{\UAS,\adv}^{sig-forge}}\xspace}
\def\EvalForge{\ensuremath{EvForge}\xspace}
\def\EvalInspectForge{\ensuremath{EvInForge}\xspace}
\def\InspectForge{\ensuremath{InForge}\xspace}
\def\ExpNonframe{\ensuremath{\Exp_{\UAS,\adv}^{frame}}\xspace}
\def\AdvNonframe{\ensuremath{\Adv_{\UAS,\adv}^{frame}}\xspace}
\def\ExpNonframeSign{\ensuremath{\Exp_{\UAS,\adv}^{frame-sign}}\xspace}
\def\AdvNonframeSign{\ensuremath{\Adv_{\UAS,\adv}^{frame-insp}}\xspace}
\def\ExpNonframeInsp{\ensuremath{\Exp_{\UAS,\adv}^{frame-insp}}\xspace}
\def\AdvNonframeInsp{\ensuremath{\Adv_{\UAS,\adv}^{frame-insp}}\xspace}
\def\ExpExtractIssue{\ensuremath{\Exp_{\UAS,\adv}^{ext-issue}}\xspace}
\def\AdvExtractIssue{\ensuremath{\Adv_{\UAS,\adv}^{ext-issue}}\xspace}
\def\ExpExtractSign{\ensuremath{\Exp_{\UAS,\adv}^{ext-sign}}\xspace}
\def\AdvExtractSign{\ensuremath{\Adv_{\UAS,\adv}^{ext-sign}}\xspace}
\def\choose{\ensuremath{\mathsf{choose}}\xspace}
\def\guess{\ensuremath{\mathsf{guess}}\xspace}
\def\Oanonc{\ensuremath{\mathcal{O}_{anon-b}_{\choose}}\xspace}
\def\Oanong{\ensuremath{\mathcal{O}_{anon-b}_{\guess}}\xspace}
\def\OExt{\ensuremath{\mathcal{O}_{ext}}\xspace}
\def\OId{\ensuremath{\mathcal{O}_{id}}\xspace}
\def\OAnon{\ensuremath{\mathcal{O}_{anon-b}}\xspace}
\def\OIssAnon{\ensuremath{\mathcal{O}_{iss-anon-b}}\xspace}
\def\OSigAnon{\ensuremath{\mathcal{O}_{sig-anon-b}}\xspace}
\def\Osimanon{\ensuremath{\mathcal{O}_{sim-anon-b}}\xspace}
\def\Otrace{\ensuremath{\mathcal{O}_{trace}}\xspace}
\def\Oforge{\ensuremath{\mathcal{O}_{forge}}\xspace}
\def\Oforgeissue{\ensuremath{\mathcal{O}_{iss-forge}}\xspace}
\def\Oforgesign{\ensuremath{\mathcal{O}_{sig-forge}}\xspace}
\def\Oframe{\ensuremath{\mathcal{O}_{frame}}\xspace}
\def\gid{\ensuremath{\mathsf{gid}}\xspace}
\def\iid{\ensuremath{\mathsf{iid}}\xspace}
\def\oid{\ensuremath{\mathsf{oid}}\xspace}
\def\sgid{\ensuremath{\boldsymbol{\mathsf{\gid}}}\xspace}
\def\siid{\ensuremath{\boldsymbol{\mathsf{\iid}}}\xspace}
\def\uid{\ensuremath{\mathsf{uid}}\xspace}
\def\suid{\ensuremath{\boldsymbol{\mathsf{uid}}}\xspace}
\def\cuid{\ensuremath{\mathsf{uid}^*}\xspace}
\def\cid{\ensuremath{cid}\xspace}
\def\cidi{\ensuremath{cid^I}\xspace}
\def\cidu{\ensuremath{cid^U}\xspace}
\def\scid{\ensuremath{\boldsymbol{cid}}\xspace}
\def\ccid{\ensuremath{cid^*}\xspace}
\def\ccidi{\ensuremath{cid^{I*}}\xspace}
\def\ccidu{\ensuremath{cid^{U*}}\xspace}
\def\cscid{\ensuremath{\scid^*}\xspace}
\def\csig{\ensuremath{\mathsf{\sigma}^*}\xspace}
\def\cSig{\ensuremath{\mathsf{\Sigma}^*}\xspace}
\def\LangIss{\ensuremath{\Lang_{iss}}\xspace}
\def\RelIss{\ensuremath{\NIZKRel_{iss}}\xspace}
\def\LangIns{\ensuremath{\Lang_{op}}\xspace}
\def\RelIns{\ensuremath{\NIZKRel_{op}}\xspace}
\def\LangVerIns{\ensuremath{\Lang_{verop}}\xspace}
\def\LangSig{\ensuremath{\Lang_{sig}}\xspace}
\def\RelSig{\ensuremath{\NIZKRel_{sig}}\xspace}
\def\fissue{\ensuremath{f_{is}}\xspace}
\def\rngfissue{\ensuremath{R_{is}}\xspace}
\def\feval{\ensuremath{f_{ev}}\xspace}
\def\rngfeval{\ensuremath{R_{ev}}\xspace}
\def\varrngfeval{\ensuremath{r_{ev}}\xspace}
\def\finsp{\ensuremath{f_{op}}\xspace}
\def\rngfinsp{\ensuremath{R_{op}}\xspace}
\def\varrngfinsp{\ensuremath{r_{op}}\xspace}
\def\famfissue{\ensuremath{\mathcal{F}_{is}}\xspace}
\def\famfeval{\ensuremath{\mathcal{F}_{ev}}\xspace}
\def\famfinsp{\ensuremath{\mathcal{F}_{op}}\xspace}
\def\Issuers{\ensuremath{\mathbb{I}}\xspace}
\def\CUASGen{\ensuremath{\Pi_{\UAS}}\xspace}
\def\CUASGenHideIss{\ensuremath{\CUASGen^{hide-iss}}\xspace}
\def\CUASGenInt{\ensuremath{\CUASGen^{int}}\xspace}
\def\trap{\ensuremath{\tau}\xspace}
\def\extract{\ensuremath{\mathsf{ex}}\xspace}
\def\extracttrap{\ensuremath{\tau_{\extract}}\xspace}
\def\sring{\ensuremath{\mathbf{ring}}\xspace}
\def\CUASRing{\ensuremath{\Pi_{\UAS}^{ring}}\xspace}
\def\CUASGS{\ensuremath{\Pi_{\UAS}^{gs}}\xspace}
\def\CUASAC{\ensuremath{\Pi_{\UAS}^{ac}}\xspace}
\def\CUASGSMDO{\ensuremath{\Pi_{\UAS}^{gsmdo}}\xspace}
\def\CUASDAC{\ensuremath{\Pi_{\UAS}^{dac}}\xspace}
\def\CUASRAC{\ensuremath{\Pi_{\UAS}^{rac}}\xspace}
\def\CUASMPS{\ensuremath{\Pi_{\UAS}^{mps}}\xspace}
\def\proofins{\ensuremath{\NIZKproof_{op}}\xspace}
\def\proofsig{\ensuremath{\NIZKproof_{sig}}\xspace}
\def\crsiss{\ensuremath{\NIZKcrs_{iss}}\xspace}
\def\crssig{\ensuremath{\NIZKcrs_{sig}}\xspace}
\def\crsins{\ensuremath{\NIZKcrs_{op}}\xspace}

%%% Local Variables: 
%%% mode: pdflatex
%%% TeX-master: "gsac.tex"
%%% End:


\begin{document}
\maketitle


\begin{abstract}
  Much work has been done in the last 4 decades on privacy-preserving
  identities and authentication that tries to incorporate some sort of utility
  and accountability without losing (too much) privacy. Among the most relevant
  efforts we can find group signatures and anonymous credentials.
  %
  Anonymous credentials (AC) let users obtain attestations by trusted
  issuers on certain attributes -- thus implicitly defining the concept of
  identity as ``a set of attributes'' -- and allow credential owners to prove
  claims on these attributes. Indeed, this has proven to be very
  flexible, to the point that it has served as inspiration to design W3C's
  Verifiable Credentials, a relevant industry-led effort. However, the
  accountability offered by anonymous credentials is quite limited; frequently
  restricted to blacklisting, if present at all.
  %
  On the other hand, group signatures (GS) have traditionally focused on plain
  authentication without attributes (except very few exceptions), making instead
  more emphasis on the accountability side. That is, not only allowing full
  deanonymization, but also variants such as sequential linking, user-controlled
  anonymity, or even full unlinkability.
  %
  Not surprisingly, GS and AC share very common syntax, models and frequently
  also constructions.

  We argue that, for the real world, both functionalities are essential. Taking
  advantage of their similarities, we first propose a trivial combination, that
  we refer to with the original \GSAC acronym. Through \GSAC, we
  showcase the benefits of even a simple combination. We give a common syntax,
  security model, and generic construction for \GSAC.
  %
  Next, we acknowledge that even \GSAC has strong limitations. Namely, it is too
  strict in the rules one can apply during issuance of credentials, and signing
  and opening of the produced signatures. While generalizations along these
  lines have been explored independently in the GS and AC domains, to the best
  of our knowledge, no single scheme offers full flexibility.
  We define a syntax, security model, and give a generic
  construction for a scheme meeting these requirements. Given its high
  flexibility to cater for different needs, we name it Universal Anonymous
  Signatures (\UAS). Indeed, an \UAS scheme lets issuers decide on issuance
  policies, which easily enable advanced use cases such as delegation, or
  issuance conditioned identity-related claims. It can also be used to build
  schemes that allow signers prove arbitrary claims on their identities, as well
  as deciding on which should be the opening authority that can apply some
  commonly agreed deanonymization function over the signer's identity (which is
  now well-defined, given the incorporation of attributes from the AC domain).
  %
  The techniques and building blocks we leverage are well known. The challenge
  is, rather, on the definitional side.
\end{abstract}

\section{Introduction}
\label{sec:introduction}

Privacy is a human right \needcite. It is also one of the most threatened ones
as a consequence of digitalization and hyperconnectivity \needcite. Maybe one
of the reasons behind this is that, on many occasions, it conflicts with another
human right: security \needcite. The most representative occasion in which this
happens is when we, as individuals%
\footnote{This, of course, also happens when authenticating machines -- as they
  carry to some extent information of some individual(s).}%
, want to gain access to some (possibly
digital) good or service. In that context, one has to prove having the legitimate
rights to access that good or service. However, in the process of doing so, the
entity verifying that we have the rights, typically learns something about us,
like our real name. This has been traditionally fine in physical systems, as the
impact was mostly local. However, with the advent of digitalization and
hyperconnectivity, this usually means that something related to our identity is
being logged in some server, who knows where and with what security controls,
and probably vulnerable to a number of criminal activities \needcite.

Of course, the cryptographic community has not been stranger to the need of
providing satisfactory solutions that enable both privacy and security. We
refer to all works trying to address to this need as \emph{privacy-preserving
  authentication} mechanisms. For decades, a huge amount of research, with many
variations, has been produced in the field. Sadly, still with limited impact in
the real world. Within this context, the goal of this review is manyfold: from a
practical perspective, serve
as a hub to security and privacy professionals that may find themselves in need
to balance privacy and security in some real world application; from a more
academic one, provide a (somehow detailed) overview of the vast amount of
literature in the field, which is already overwhelming enough to make it hard
to find the most appropriate variant of some property we may find ourselves
in need to provide; and finally, attempt to throw some insight into why such a
vast academic work has so far failed to gain more real world adoption.

In order to achieve this, %
we go over the main cryptographic primitives used to build privacy-preserving
authentication mechanisms. For all of them, we first provide a high-level
intuition of the approach to protect users' privacy. Then, we overview the
most prominent security models since their inception, and complement this
with some relevant variations concerning the provided functionality. We also
summarise the main results from an efficiency standpoint, and the main
challenges found so far both in theory and practice. To gain some initial
insight into the academic impact of each primitive, we provide (approximate)
numbers on publications in the main cryptography, security and privacy venues.
Finally, with a more practical mindset, we overview the applications that each
primitive has ``in theory'', and the extent to which they have been deployed
in the real world.

\note{Argue/mention that focus is on primitives that can be used for signing.
  This therefore excludes protocols that can only be used for anonymous
  authentication, but not anonymous singing per se (e.g., PAKE, PASTA, PESTO...
  and probably also Nymble, PEREA, BLAC...)}

\note{Create a graph of relationships between basic primitives and their
  applications. E.g., much of
  the work in SRP group signatures and DAA seems to be derived from randomizable
  signatures (which, afaik, are also a foundation for anonymous credentials).}

%%% Local Variables:
%%% mode: latex
%%% TeX-master: "sok-privsig"
%%% End:

\section{Preliminaries}
\label{sec:preliminaries}

\subsection{Cryptographic Building Blocks}
\label{ssec:bbs}

\todo{Briefly describe algorithms they use.}

\paragraph{Append-Only Bulletin Boards.} %
%\todo{For now, simplified notion as in \cite{acc+20}.}

\paragraph{Digital Signatures.} %
%Maybe build on \IdealFSig from \cite{canetti03}.

\paragraph{Merkle Trees.}

\paragraph{Public Key Infrastructures.} %
% Maybe build on \IdealFCA from \cite{canetti03}.

\paragraph{Ideal functionality specification.} %

\iffalse
We take the basic approach to see DID documents as simple dictionaries
(i.e., lists of labels with corresponding types and values), associated to a
common label, the DID. Based on this, we define the operations that compose an
ideal basic functionality for DIDs, as follows:

\begin{description}
\item[Create interface.] To create a new DID, the functionality exposes a
  \uccmd{Create} interface that receives a tuple of label and key type values.
  Then, it asks the adversary to provide a value to each of the specified
  labels, and the DID value. If the DID value does not exist in the records of
  the functionality, a new entry (including the dictionary) is created. In order
  to meet the authentication requirement, the ideal functionality takes note of
  the party creating the DID, so that subsequent changes in the DID can only be
  done by this party. We note that more advanced predicates could be defined, by
  this bare minimum seems to be enough for most (current) DID methods.
  \todo{Give evidence of the latter?}
\item[Read interface.] The contents of existing DIDs can be fetched via the
  \uccmd{Read} interface. It receives a DID value and, if it exists in the
  functionality's records, returns the associated contents. Note that, if such
  entry exists, then it is authentic by definition, and thus trivially meets
  the authentication requirements.
\item[Update interface.] The contents of existing DIDs can be modified via the
  \uccmd{Update} interface. It requires to specify the actual DID and two lists:
  $\sval_d$ with the labels to remove, and $\sval_a$ with the lables of new keys
  to add (along with their types). The DID must exist in the records of the
  functionality, and the call must originate from the same party that created
  it. Note that this covers variants such as using update only to add new keys
  (e.g. by specifying an empty $\sval_d$), or modifications of existing labels
  (e.g., by adding an existing label in $\sval_d$, and also in $\sval_a$).
\item[Deactivate interface.] Deactivating a DID is actually a simplification of
  the update operation, without specifying new keys to add. This is possible
  via the \uccmd{Deact} interface. Note that it is not strictly necessary to
  specify this interface, as it is a special case of \uccmd{Update}. However, we
  explicitly add it for proximity to the actual DID specification. \todo{Just
    remove it?}
\end{description}

Our expectation is that this simple approach allows to capture a large
subset of all possible DID method implementations. Note that the functional
restrictions we impose seem to be unavoidable. Namely, we do not specify how
the actual identifiers are computed; instead, we allow the adversary to pick
them. To update a DID, we only require such DID to exist and that the caller
is the party that created it, but otherwise the DID's contents can be set to
something completely new. Finally, deactivating labels within an existing DID
requires that these labels exist -- and, again, that the caller is the same
party who created the DID. Concretely, note that any further structure beyond
the label-type-value dictionary approach, is left as an implementation detail.
However, enforcing additional structure seems unavoidable in practice (e.g.,
requiring that all DIDs include keys of certain type). Since this definitely
has an impact on the behaviour of the ideal functionality, we parameterize it
with a policy \P, which can be used as a black box: the functionality invokes
\P with a set of labels and key types, and receives either $1$ to denote
acceptance, or $0$ to denote rejection. However, it is not realistic to assume
that, even corrupted parties will enforce such policies. Thus, our approach is
to defer policy checking in the ideal functionality until after having
interacted with the simulator \Sim. Intuitively then, if the output of the
ideal functionality along with \Sim is indistinguishable to that of the real
protocol with any real world adversary, then the policy is correctly enforced
by the real protocol. Looking ahead, the real protocol in turn ``delegates''
this policy-checking to an ideal append-only bulletin board. \todo{This captures
  the idea, but refine and re-structure.}

Note that we also allow the adversary to pick the values associated to the
labels in new and updated DIDs. This is a strong requirement that follows the
argumentation in \cite{canetti03} for ideal digital signatures: namely, this
means that there is absolutely any requirement on the way in which this values
are computed (they might not even be conventional public keys). Indeed, this
seems to match the permissive philosophy of defining DIDs in the W3C DID
specification.

We capture the resulting notion in \figref{fig:fpkidid}.

\begin{figure}
  \begin{framed}
    \scalebox{0.9}{
      \begin{minipage}[t]{\textwidth}
        \textrm{Upon receiving a call $\IdealFPKIDID^{\P}(\cmd,\sid,\arg)$
          from some party $P$ to run command \cmd, in session \sid, with arguments
          \arg, $\IdealFPKIDID^{\P}$ operates as follows.}
        \textrm{If the received \cmd and \arg pair does not meet any of
          the following options, the request is ignored.}
        \textrm{\P is a boolean predicate defining concrete conditions for
          acceptable DID creations, updates and deactivations.
        }
        \textrm{\todo{We assume that the functionality keeps track of the
            received requests.}}
      \end{minipage}
    }
    \vspace*{0.5em}

    \scalebox{0.9}{
      \begin{minipage}[t]{0.55\textwidth}
        \procedure[linenumbering]{$\pcif \cmd = \uccmd{Create} \land \arg =
          \lbrace (\lbl_i,\typ_i) \rbrace_{i\in[n]}$}{
          \pcif \P(\lbrace (\lbl_i, \typ_i) \rbrace_{i\in[n]}) = 0:
          \text{abort} \\                    
          \ucsend~(\uccmd{Create}, \sid, \arg)~\text{to}~\Sim \\
          \ucrecv~(\uccmd{CreateOk}, \sid, \did,
          \sval=\lbrace \val_i \rbrace_{i\in[n]}) \\
          \hspace*{2.8em}\text{from}~\Sim \\
          \pcif (\cdot,\did,\cdot) \notin \DID: \\
          \pcind \DID \gets \DID \cup \lbrace (P,\did,\lbrace
          (\lbl_i,\typ_i,\val_i) \rbrace_{i\in[n]}) \rbrace \\
          \ucio{Output}~(\uccmd{Created},\sid,\did,\sval)
        }
      \end{minipage}
    }
    \scalebox{0.9}{
      \begin{minipage}[t]{0.5\textwidth}
        \procedure[linenumbering]{$\pcif \cmd = \uccmd{Read} \land
          \arg = did$}{
          \ucsend~(\uccmd{Read},\sid,\arg)~\text{to}~\Sim \\
          \ucrecv~(\uccmd{ReadOk},\sid)~\text{from}~\Sim \\          
          \pcif (\cdot,did,\sval) \notin \DID: abort \\
          \pcif \P(\sval) = 0: abort \\
          \ucio{Output}~(\uccmd{Read},\sid,(did,\sval))~\text{to}~P
        }    
      \end{minipage}
    }
    \vspace*{0.25em} \\
    \scalebox{0.9}{
      \begin{minipage}[t]{0.55\textwidth}
        % \arg = (did, \sval= \lbrace (\lbl_i,\typ_i) \rbrace_{i\in[n]})\$}{
        \procedure[linenumbering]{$\pcif \cmd = \uccmd{Update} \land
          \arg = (did, \sval_d,\sval_a)$}{          
          % \sval_{del} \gets \sval_0 \setminus \sval;
          % \sval_{add} \gets \sval \setminus \sval_{del} \\          
          % n'' \gets |\sval_{add}| \\
          \pcif (P,\did,\sval_0) \notin \DID: \text{abort} \\
          \pcif \P((\sval \setminus \sval_d) \cup \sval_a) = 0: \text{abort} \\
          \ucsend~(\uccmd{Update}, \sid,\did,\sval_d,\sval_a)~\text{to}~\Sim \\
          \ucrecv~(\uccmd{UpdtOk},\sid,\did,\lbrace \val^a_i \rbrace_{i\in[n^a]})~
          \text{from}~\Sim \\
          \lbrace (\lbl_i^a,\typ_i^a) \rbrace_{i\in[n^a]} \gets \sval_a;~
          \lbrace \lbl_i^d \rbrace_{i\in[n^d]} \gets \sval_d \\          
          \sval_1 \gets (\sval_0 \setminus \sval_d) \cup
          \lbrace (\lbl^a_i,\typ^a_i,\val^a_i) \rbrace_{i\in[n^a]} \\          
          \DID \gets \DID \setminus \lbrace (P,\did,\sval_0) \rbrace \cup
          \lbrace (P, \did, \sval_1) \rbrace \\
          \ucio{Output}~(\uccmd{Updated},\sid, (\did,\lbrace \val_a \rbrace_{i\in[n^a]}))
        }
      \end{minipage}
    }
    \scalebox{0.9}{
      \begin{minipage}[t]{0.5\textwidth}
        \procedure[linenumbering]{$\pcif \cmd = \uccmd{Deact} \land \arg =
          (did, \lbrace \lbl_i \rbrace_{i\in[n]})$}{          
          \pcif (P,did,\sval_0) \notin \DID: abort \\
          \sval_1 \gets \sval_0 \setminus \lbrace \lbl_i \rbrace_{i\in[n]} \\
          \pcif \P(\sval_1) = 0: abort \\
          \ucsend~(\uccmd{Deact}, \sid, \arg)~\text{to}~\Sim \\
          \ucrecv~(\uccmd{DeactOk},\sid)~\text{from}~\Sim \\          
          \pcind \DID \gets \DID \setminus \lbrace(P,did,\sval_0)\rbrace~\cup \\
          \hspace*{7em}\lbrace (P,did,\sval_1) \rbrace \\
          \ucio{Output}~(\uccmd{Deacted},\sid,\did)
        }
      \end{minipage}
    }
  \end{framed}
  \caption{Ideal PKI functionality, \IdealFPKI. We abuse set
    notation, and write $\lbrace (a_i,\dots) \rbrace_{i\in[n]} \setminus
    \lbrace (b_j,\dots) \rbrace_{j\in[m]}$ to mean subtracting from the first
    set the elements present in the second set, whenever $a_i = b_j$ for some
    $i,j$, even if the size of the $(a_i,\dots)$ and $(b_j,\dots)$ tuples do not
    match.}
  \label{fig:fpki}
\end{figure}
\fi

\subsection{Self-Sovereign Identities}
\label{ssec:ssi}

\paragraph{Decentralised Identifiers (DIDs).} %
Much of the SSI community is built on top of DIDs, which is a W3C Recommendation
since 2022\footnote{\url{https://www.w3.org/TR/did-core}. Last access, 3rd
  November, 2022.}. Essentially, the DID specification defines how to associate
public keys and additional metadata to unique identifiers, leveraging verifiable
data registries (VDR). Whereas there is no concrete requirement on these VDRs,
the typical choice are blockchains, which leads to the ``Decentralised''
adjective for DIDs. The DID specification describes the syntax that identifiers
must follow, as well as the operations (CRUD: Create, Read, Update, Deactivate)
that any DID-compatible mechanism must implement.

That is, a mechanism following the DID specification must enable any user to be
able to register a unique identifier, associate public key material to it, and
manange it. Thus, it is natural to see such a mechamism as an equivalent to
traditional Public Key Infrastructures (PKIs), where different instantiations of
such a mechanism -- e.g., with different approaches to implement the CRUD
methods required by the DID specification, or based on different VDRs -- can be
seen as a different type of DID-based PKI.

\todo{Maybe expand. Also, describe the difference between DID Document, and DID.
  Also describe the concept of DID controller, essential to prevent manipulation
  of DIDs and their documents.}

\paragraph{Verifiable Credentials (VCs).} %
\todo{Describe: Introduction to VCs. Relating to DIDs.}

\subsection{Overview of Atala PRISM}
\label{ssec:overview-prism}

In a nutshell, Atala PRISM builds VCs and DIDs on top of the Cardano blockchain.
More concretely, any entity can create (and manage) a DID, and issuers can issue
VCs associated to a previously existing DID, where Cardano is the chosen VDR.
Users can later leverage VCs to authenticate against any potential verifier, in
a manner that all the information that the verifier needs to (cryptographically)
authenticate the user is contained in the Cardano blockchain.

\todo{Ellaborate more.}


%%% Local Variables:
%%% mode: latex
%%% TeX-master: "prism-protocol"
%%% End:

\section{\GSAC: Merging Group Signatures and Anonymous Credentials}
\label{sec:gsac}

\todo{Some intro here}

\subsection{Model for \GSAC}
\label{ssec:model-gsac}

While group signatures and anonymous credentials share quite common syntax and
security properties, which are even frequently formalised in a similar way,
there are aspects that need to be considered with care.

The first choice that has to be made is whether to make sign/show and
verification a non-interactive or an interactive process -- in group signatures,
we are in the former case, whereas in anonymous credentials, we are in the
latter. While both options are of course valid, this has direct impact in the
modelling. We choose the non-interactive approach, and then sketch how to
generically (and trivially) translate it into an interactive process. The reason
is that this results in a more flexible building block, suitable for a wider set
of scenarios. And, syntactically, it seems more natural to combine two
non-interactive processes into an interactive one, than the other way around.

We also need to take into account that, in group signature schemes, users only
get one membership credential -- typically bound to their personal secret
key --, which they then use to create as many group signatures as they want. On
the other hand, in anonymous credentials, users can get as many credentials as
they desire, which can (typically) then be showed also arbitrarily many times.
In this regard, we follow a somehow intermediate approach: users own a single
personal secret key, which they can use to get as many credentials as they wish.
Subsequently, they can use any of those credentials to create \GSAC signatures.
This has, as expected, some nuanced impact in the modelling. For instance, in
group signatures, tracing and non-frameability are frequently dependent on the
open function, which (among other things) returns an index pointing to the join
transcript associated with the user who produced the signature given by the
adversary as a response to the game's challenge. In our approach, though, each
user may have many such transcripts -- one per credential that the user
obtains. However, note that what actually provides meaning to these notions
is not the index; but the associated user key pair, which is moreover necessary
to prove that the opening has been performed honestly. Thus, with some
restructuring of the corresponding games, we can still capture the main
meaning of these traceability and non-frameability notions from group
signatures. This also impacts anonymity, since in group signatures, this is
modelled by challenging the adversary to guess which user (which implicitly
translates to ``which user credential'') out of two challenge users was
leveraged to produce the challenge signature. But again, in our case, users may
have multiple credentials. However, it seems sufficient to adopt a similar
anonymity flavour to that of anonymous credentials, where the adversary has to
distinguish between presentations (signatures) produced from two arbitrary
credentials, independently on whether these credentials were obtained by the
same user, or not.
%
Getting closer to the anonymity definition of anonymous credentials also has the
side effect to introduce the notions of attributes for free into the domain of
group signatures -- which, up to now, had considered membership credentials
\emph{without} attributes. Yet another side effect of introducing multiple
attributes (if we think from a group signature perspective) is that, in order
to avoid trivial wins, the anonymity game has to enforce that both challenge
credentials reveal the same predicate on their attributes.

In the model described below, we follow closely the definitions from
\cite{bsz05} for group signatures, and \cite{fhs19} for anonymous credentials
(which are in turn influenced by \cite{bsz05}).

\subsubsection{Syntax for \GSAC schemes}
\label{sssec:syntax-gsac}

\begin{description}
\item[$\parm \gets \Setup(\secpar)$.] Given a security parameter \secpar,
  returns a global system parameter variable \parm.
\item[$(\ipk,\isk) \gets \IKeyGen(\parm)$.] Given global system parameters
  \parm, returns the issuer's key pair. Hereafter, we assume that the public
  part \ipk is added to the group public key \gpk.
\item[$(\opk,\osk) \gets \OKeyGen(\parm)$.] Given global system parameters
  \parm, returns an opener's key pair. Hereafter, we assume that the public part
  \opk is added to the group public key \gpk.
\item[$(\upk,\usk) \gets \UKeyGen(\parm)$.] Given global system parameters
  \parm, returns a user's key pair.
\item[$\langle \cred/\bot,\utrans/\bot \rangle \gets
  \langle \Obtain(\gpk,\usk,\attrs),\Issue(\gpk,\isk,\attrs) \rangle$.]
  This interactive protocol lets a user with key pair (\upk,\usk) running the
  \Obtain process, obtain a credential \cred with attributes \attrs, by
  communicating with an issuer, with private key \isk, running the \Issue
  counterpart. The user outputs the produced credential \cred, while the issuer
  outputs the protocol transcript \utrans for the produced credential.
\item[$\sig \gets \Sign(\gpk,\usk,\cred,\dattrs,\msg)$.] The user with
  with secret key \usk, who obtained a credential \cred, produces a signature
  \sig over message \msg, revealing subset of attributes \dattrs in \cred.
\item[$1/0 \gets \Verify(\gpk,\sig,\dattrs,\msg)$.] Checks whether \sig is a
  valid signature revealing the attribute set \dattrs of its signing credential,
  over message \msg.
\item[$(\upk,\attrs,\oproof)/\bot \gets
  \Open(\gpk,\osk,\sig,\dattrs,\msg)$.]
  Given a signature \sig over message \msg, produced by a credential with
  attributes containing the set \dattrs, returns the public key \upk of the
  signer, and the attribute set \attr contained in the credential used to
  generate the signature, along with a proof of opening correctness \oproof; or
  $\bot$ if the opening process fails.
\item[$1/0 \gets \Judge(\gpk,\upk,\attrs,\oproof,\sig,\dattrs,
  \msg)$.] Checks if \oproof is a valid opening proof for the
  statement: ``The owner of a credential over attribute set \attrs, and owner
  of public key \upk, created signature \sig, which reveals attribute set
  $\dattrs \subseteq \attrs$''.
\end{description}

The correctness and security properties are defined with the help of the
following sets of oracles, and global variables that help oracles and games
keep consistent state.

\paragraph{Global variables.} %
All the honestly generated users (i.e., all honestly generated user key pairs,
since we assume a one-to-one relationship between user and user key pair), as
well as all honestly generated credentials, are assigned identifiers. We
typically write \uid for users' identifiers, and \cid for credentials'. The
adversary can
refer to any individual user or credential using the corresponding identifier --
even though he may not know the actual contents of the key pairs or credentials.
In games that involve challenge user or credential identifiers, we use \cuid and
\ccid to refer to these challenge user/credential.
%
The games keep track of the user keys (through set \UK) and credentials (through
table \CRED) that are created as a result of oracle calls by the adversary. We
use the user and credential identifiers to reference specific user key pairs,
credentials, and credential attributes.
For instance, $\UK[\uid]$ refers to the user key pair corresponding to user
with identifier \uid, $\PUBUK[\uid]$ refers to the public key of that key pair,
and $\PRVUK[\uid]$ to the private key; credential $\CRED[\cid]$ refers to the
credential data associated to the credential with identifier \cid; $\ATTR[\cid]$
is the set of attributes that was assigned to the credential with identifier
\cid; and $\OWNR[\cid]$ is set to the \uid corresponding to the user that
credential \cid was issued to.
%
Additionally, the games keep track of all honest and corrupt users that have
been generated, through sets \HU and \CU, respectively. They also keep track of
the signatures that have been honestly produced, through the \SIG table. Since
signatures are produced through credentials, the \SIG table is indexed with
{\cid}s. For the anonymity game, we also need to keep track of the challenge
signatures that the adversary has obtained, in order to prevent trivial wins
by allowing any of them to be opened. 
%
All global variables are assumed to be initially set by the games to empty
values, and all tables/sets are initialized as empty tables/sets. Also, for
readability, we abuse the syntax as follows: we
write $\CRED[\uid]$ to mean $\CRED[\cid]$ for all $\cid$ such that
$\OWNR[\cid] = \uid$.% ; we also sometimes use \upk as a ``synonym'' of \uid, as it
% is possible to do a reverse lookup from the \UK table (i.e., $\CRED[\upk]$ can
% be replaced by $\CRED[\uid]~\st~\UK[\uid]=(\upk,\cdot)$); \todo{Something else?}

\paragraph{Oracles.} %
Oracles are the interface of the adversary with the corresponding games. In
other words: through these oracles, the game enviroment exposes to the adversary
functionality that could otherwise be executed only by honest parties with
private knowledge -- knowledge that would make the adversary capable of
trivially breaking the security properties formalized in the experiments.
In the game-based definitions of our \GSAC model, we leverage the following
oracles, which are formally defined in \appref{app:gsac-formal}.

\begin{description}
\item[\RREG.] Reads the given transcript table entry.
\item[\WREG.] Sets a transcript table entry to the given value.
\item[\HUGEN.] Adds a new honest user to the game, by honestly generating
  the user's key pair.
\item[\CUGEN.] If the specified user identifier is not already in the game,
  sets the user public key to a value given by the adversary. If it already
  exists, and is associated to an honest user, then it reveals to the adversary
  the user's secret key.
\item[\OBTISS.] Lets the adversary add a new honestly generated credential to
  the game, on behalf of an honest user.
\item[\OBTAIN.] Enables the adversary to play the role of a dishonest issuer
  in games that support it, by letting it interact with honest users who want to
  receive credentials.
\item[\ISSUE.] Allows the adversary to play the role of dishonest users,
  requesting an honest issuer to produce credentials for them.
\item[\SIGN.] Lets the adversary get signatures from credentials belonging
  to honest users.
\item[\OPEN.] Given an honestly produced signature, lets the adversary learn
  the public key of the user who produced it.
\item[\CHALb.] Upon receiving two challenge credentials, a common intersecting
  set of attributes, and a message, returns a signature produced by one of these
  two credentials, defined by the bit $b$, which is established in the anonymity
  game.
\end{description}

\paragraph{Correctness.} %
Correctness of \GSAC schemes is formalized through the experiment in
\figref{fig:exp-gsac-corr}. It states that a signature for some message \msg,
revealing attributes \dattrs, which was honestly produced through a credential
that was obtained by an honest user interacting with an honest issuer, with a
set of attributes \attrs such that $\dattrs \subseteq \attrs$, must be accepted
by \Verify. Moreover, an honestly produced correctness proof of opening for such
signature, revealing the public key pair of the user, must also be accepted by
\Judge.

\begin{definition}{(Correctness of \GSAC)}
  \label{def:correctness-gsac}
  A \GSAC scheme is correct if, for any p.p.t. adversary $\adv$,
  $\Pr[\ExpGSACCorrect(1^\secpar) = 1]$ is a negligible function of \secpar.
\end{definition}

\begin{figure}[htp!]
  \procedure{$\ExpGSACCorrect(1^\secpar)$}{%
     \parm \gets \Setup(1^\secpar) \\
     (\ipk,\isk) \gets \IKeyGen(\parm);~(\opk,\osk) \gets \OKeyGen(\parm);~ 
     \gpk \gets (\ipk,\opk) \\
    (\cid,\dattrs,\msg) \gets \adv^{\HUGEN,\OBTISS,\RREG}(\gpk) \\
    \pcif \dattrs \nsubseteq \ATTR[\cid] :
    \pcreturn \bot \\
    \sig \gets \Sign(\gpk,\PRVUK[\OWNR[\cid]],\CRED[\cid],\dattrs,\msg) \\
    (\upk,\attrs,\oproof) \gets \Open(\gpk,\osk,\sig,\dattrs,\msg) \\
    \pcif \Verify(\gpk, \sig, \dattrs,\msg) = 0 \lor
    \Judge(\gpk,\upk,\attrs,\oproof,\sig,\dattrs,\msg) = 0: \\
     \pcind \pcreturn 1 \\
    \pcreturn 0
  }
  \caption{Correctness experiment for \GSAC schemes.}
  \label{fig:exp-gsac-corr}
\end{figure}

\subsubsection{Security Properties of \GSAC Schemes}
\label{sssec:security-gsac}

\paragraph{Anonymity.} %
In group signatures, anonymity captures that no adversary must be able to learn,
from any group signature, anything about its signer. In anonymous credentials
(with selective disclosure), it requires that no adversary should learn anything
about the holder of a credential that has been successfully shown, beyond that
he owns a credential containing the revealed attributes. In both GS and AC, it
is also typically required that
multiple signatures/presentations by the users are unlinkable. The approach to
formally state this property is in both cases frequently the same: the adversary
picks two (honest) users (or credentials in the AC case), the game randomly
chooses one of them, and lets the adversary request challenge
signatures/presentations from it. The adversary wins if it succeeds in guessing
which was the chosen user/credential better than guessing at random. In group
signatures, the game must also restrict the adversary from opening challenge
signatures. In anonymous credentials, the game must further constraint the
adversary to pick credentials that have some common subset of attributes, and
to use that common subset to request the challenge presentations. Here, we need
to take into account both. Furthermore, a key difference with group signatures
is that the game requires the adversary to output credential identifiers, rather
than user identifiers. Specifically, this means that the adversary may actually
output two credentials that belong to the same user. Therefore, in some sense,
the anonymity we get is more general than that of group signatures, as in \GSAC,
like in anonymous credentials, the adversary may request signatures from
different credentials, but belonging to the same user. The formal
specification of the anonymity game is given in \figref{fig:exp-gsac-anonb}.

\begin{figure}[htp!]
  \procedure[linenumbering]{$\ExpGSACAnonb(1^\secpar)$}{%
     \parm \gets \Setup(1^\secpar) \\
     (\ipk,\isk) \gets \IKeyGen(\parm);~(\opk,\osk) \gets \OKeyGen(\parm);~ 
     \gpk \gets (\ipk,\opk) \\
     (\cid_0,\cid_1,\status) \gets
     \adv^{\HUGEN,\CUGEN,\WREG,\OBTAIN,\SIGN,\OPEN}(\choose,\gpk,\isk) \\
     b^* \gets
     \adv^{\HUGEN,\CUGEN,\WREG,\OBTAIN,\SIGN,\OPEN,\CHALb}(\guess,\gpk,\isk,\status) \\
     \pcreturn b^*
  }
  \caption{Anonymity experiment for \GSAC schemes.}
  \label{fig:exp-gsac-anonb}
\end{figure}

\begin{definition}{(Anonymity of \GSAC)}
  \label{def:anonymity-gsac}
  We define the advantage \AdvGSACAnon of $\adv$ against \ExpGSACAnonb as
  $\AdvGSACAnon=|\Pr\lbrack\ExpGSACAnono(1^\secpar)=1\rbrack-
  \Pr\lbrack\ExpGSACAnonz(1^\secpar)=1\rbrack|$.
  %
  A \GSAC scheme satisfies anonymity if, for any p.p.t. adversary $\adv$,
  \AdvAnon is a negligible function of $1^\secpar$.
\end{definition}

\paragraph{Traceability.} %
Traceability is one of the unforgeability-related properties in group
signatures. It captures that any signature accepted by \Verify needs to open
to one of the users that joined the group. While there is no traceability notion
in anonymous credentials, it is natural to map it to their unforgeability
property; if only because both require the issuer to be honest. Unforgeability
in anonymous credentials typically ensures that no adversary can get a verifier
to accept a credential presentation requiring a set of attributes that is not
contained in one of the credentials controlled by the adversary.
%
Our notion of traceability for \GSAC combines both requirements. It assumes an
honest issuer, as otherwise the adversary can create untraceable credentials at
will. The game then lets the adversary add honest and corrupt users, create
honest signatures, and open them. The adversary wins if, after this interaction,
is able to produce a $(\sig,\dattrs,\msg)$ tuple that is accepted by \Verify,
but either cannot be opened, it can be opened but the proof is rejected by
\Judge, or even though it is accepted by \Judge, there is no credential
that contains the set of attributes \dattrs. We formally define traceability in
the \ExpTrace experiment in \figref{fig:exp-gsac-trace}.

\begin{figure}[htp!]
  \procedure[linenumbering]{$\ExpGSACTrace(1^\secpar)$}{%
    \parm \gets \Setup(1^\secpar) \\
    (\ipk,\isk) \gets \IKeyGen(\parm);~(\opk,\osk) \gets \OKeyGen(\parm);~ 
    \gpk \gets (\ipk,\opk) \\
    (\sig,\dattrs,\msg) \gets
    \adv^{\HUGEN,\CUGEN,\RREG,\OBTISS,\ISSUE,\SIGN,\OPEN}(\gpk,\osk) \\
    \pcif \Verify(\gpk,\sig,\dattrs,\msg) = 0: \pcreturn 0 \\
    \pcif \Open(\gpk,\osk,\trans,\sig,\dattrs,\msg) = \bot: \pcreturn 1 \\
    \textrm{Let}~(\upk,\attrs,\oproof) \gets \Open(\gpk,\osk,\trans,\sig,
    \dattrs,\msg) \\
    \pcif \Judge(\gpk,\upk,\attrs,\oproof,\sig,\dattrs,\msg) = 0: \pcreturn 1 \\
    \pcelse \pcif \forall \uid \in \CU \cup \HU, \nexists (\uid,\cdot,\attrs)
    \in \CRED \lor \dattrs \not\subseteq \attrs: \pcreturn 1 \\
    \pcreturn 0
  }
  \caption{Traceability experiment for \GSAC schemes.}
  \label{fig:exp-gsac-trace}
\end{figure}

\begin{definition}{(Traceability of \GSAC)}
  \label{def:trace-gsac}
  We define the advantage \AdvGSACTrace of $\adv$ against \ExpGSACTrace as
  $\AdvGSACTrace=\Pr\lbrack\ExpGSACTrace(1^\secpar)=1\rbrack$.
  %
  A \GSAC scheme satisfies traceability if, for any p.p.t. adversary $\adv$,
  \AdvGSACTrace is a negligible function of $1^\secpar$.
\end{definition}

\paragraph{Non-frameability.} %
Non-frameability variants are a core unforgeability-type property in group
signatures. However, no similar property is modelled for anonymous credentials
(see the discussion in \todo{\secref{sec:introduction}} for further detail). It
is a quite strong
property, as it must be ensured even in the presence of dishonest issuer and
opener. Intuitively, it prevents the adversary from creating a signature that
frames an honest user. Depending on the inspection capabilities of the scheme,
this framing could be done in different ways; i.e., by convincing third parties
that signatures by different (possibly corrupt) users are linked, or directly
by having open proofs output the identity of a user who did not create the
signature being opened.
%
In \GSAC schemes, in order for a user to be framed, the adversary first needs to
create a $(\sig,\dattrs,\msg,\upk,\oproof)$ tuple such that the signature,
attributes, and message are accepted by \Verify, and which, along with the user
public key, and open correctness proof, are accepted by \Judge.
Then, the adversary wins the game if the owner of the credential is honest, but
the signature was not produced via the \SIGN oracle.

%, or if the set of attribute disclosed in the signature are not contained in the
% set of attributes revealed during \Open.

\begin{figure}[htp!]
  \procedure[linenumbering]{$\ExpGSACNonframe(1^\secpar)$}{%
    \parm \gets \Setup(1^\secpar) \\
     (\ipk,\isk) \gets \IKeyGen(\parm);~(\opk,\osk) \gets \OKeyGen(\parm);~ 
     \gpk \gets (\ipk,\opk) \\
     (\sig,\dattrs,\msg,\upk,\attrs,\oproof) \gets
     \adv^{\HUGEN,\CUGEN,\WREG,\OBTAIN,\SIGN}(\gpk,\isk,\osk) \\
     \pcif \Verify(\gpk,\sig,\dattrs,\msg) = 0: \pcreturn 0 \\
     \pcif \Judge(\gpk,\upk,\attrs,\oproof,\sig,\dattrs,\msg) = 0: \pcreturn 0 \\
     \textrm{Let}~\uid~\textrm{be st}~\PUBUK[\uid] = (\upk,\cdot) \\
     \pcif \textrm{no such \uid exists}: \pcreturn 0 \\
     \pcif \uid \in \HU \land (\sig \notin \SIG[\uid]
     \lor \dattrs \not\subseteq \attrs): \pcreturn 1 \\
     \pcreturn 0
  }
  \caption{Non-frameability experiment for \GSAC schemes.}
  \label{fig:exp-gsac-frame}
\end{figure}

\begin{definition}{(Non-frameability of \GSAC)}
  \label{def:frame-gsac}
  We define the advantage \AdvGSACNonframe of $\adv$ against \ExpGSACNonframe as
  $\AdvGSACNonframe=\Pr\lbrack\ExpGSACNonframe(1^\secpar)=1\rbrack$.
  %
  A \GSAC scheme satisfies non-frameability if, for any p.p.t. adversary $\adv$,
  \AdvNonframe is a negligible function of $1^\secpar$.
\end{definition}

%%% Local Variables:
%%% mode: latex
%%% TeX-master: "uas"
%%% End:

\subsection{\GSACGen: A Generic Construction for \GSAC}
\label{ssec:generic-gsac}

\todo{Do we mention earlier that GSAC only supports selective disclosure?}

\subsubsection{Definition}
\label{sssec:generic-gsac-definition}

In the following algorithms, we make use of three different NP relations for
\NIZK proof systems:

\begin{description}
\item[$\NIZKRel_{\Issue}$:] Produced by users requesting a credential. It is
  defined as $\NIZKRel_{\Issue} = \lbrace \usk, \Ccom :
  \Ccom = \CCommit(\usk; r) \rbrace$.
\item[$\NIZKRel_{\Sign}$:] Produced by users when signing a message. It is
  defined as $\NIZKRel_{\Sign} = \lbrace (\usk,\Ccom,\Attrs,\msg,\SBCMsig),
  (\Cmsg,\Ec,\DAttrs) : \Cmsg = \CCommit(\msg) \land \Ccom =
  \CCommit(\usk; 0) \land \Ec = \EEnc(\opk,\Ccom)
  \land \SBCMVerify(\ipk,\SBCMsig,\Attrs \cup \lbrace \usk \rbrace) = 1
  \land \DAttrs \subseteq \Attrs \rbrace$.
\item[$\NIZKRel_{\Open}$:] Used by the opener when opening a signature. It
  is defined as $\NIZKRel_{\Open} = \lbrace (\osk), (\Ec,\msg) :
  \msg = \EDec(\osk,\Ec) \rbrace$.
\end{description}

In a nutshell \todo{describe high-level approach}.

\paragraph{$\Setup(\secpar,\nattrs) \rightarrow \parm$.} %
Sets up the public parameters \todo{can be distributed across responsible
  parties?}. Namely: $\Cparm \gets \CSetup(\secpar)$, $\Eparm \gets
\ESetup(\secpar)$, $\SBCMparm \gets \SBCMSetup(\secpar)$, $\NIZKcrs \gets
\NIZKSetup(\secpar)$. Outputs $\parm \gets (\Cparm,\SBCMparm,\Eparm,\NIZKcrs)$.

\paragraph{$\IKeyGen(\parm) \rightarrow (\ipk,\isk)$.} %
Generates the signing key pair for the issuer, by parsing \parm as
$(\cdot,\SBCMparm,\cdot,\cdot)$ and running $(\ipk,\isk) \gets
\SBCMKeyGen(\SBCMparm)$.

\paragraph{$\OKeyGen(\parm) \rightarrow (\opk,\osk)$.} %
Generates the encryption key pair for the opener, by parsing \parm as
$(\cdot,\SBCMparm,\Eparm,\cdot)$ and running $(\opk,\osk) \gets
\EKeyGen(\Eparm)$.

\paragraph{$\UKeyGen(\parm) \rightarrow (\upk,\usk)$.} %
Generates users' key pairs by choosing a random value within the attribute space
\AttrSpace, and committing to it. Concretely: $\usk \getr \AttrSpace$, $\upk
\gets \CCommit(\usk; 0)$.

\paragraph{$\langle \Obtain(\gpk,\usk,\Attrs),\Issue(\gpk,\isk,\Attrs) \rangle
  \rightarrow \langle \cred/\bot,\utrans/\bot \rangle$.} %
In a nutshell, the user commits to a random credential identifier, and requests
a signature over commitments of its user secret key and credential identifier,
as well as the attributes in \Attrs. The user proves knowledge of the committed
values, and the issuer sends the signature (credential) in return. More
concretely:

\begin{itemize}
\item \underline{User}: Commit to the user secret key by running $\Ccom \gets
  \CCommit(\usk)$. Generate proof $\NIZKproof \gets
  \NIZKProve^{\NIZKRel_{\Issue}}(\NIZKcrs,\usk,\Ccom)$. Send $(\Ccom,
  \NIZKproof)$ to the issuer.
\item \underline{Issuer}: Run $\NIZKVerify^{\NIZKRel_{\Issue}}(\NIZKcrs,
  \C,\NIZKproof)$, and return $\bot$ if it fails. Else, create the credential
  by computing $\SBCMsig \gets \SBCMSign(\isk,\C,\Attrs)$. Send \SBCMsig to the
  user, and output $\utrans \gets (\C,\SBCMsig,\Attrs,\NIZKproof)$.
\item \underline{User}: Check the signature by running $\SBCMVerify(\ipk,
  \SBCMsig,\Attrs \cup \lbrace \usk \rbrace)$, and return $\bot$ if
  verification fails. Otherwise, return $\cred \gets \SBCMsig$.
\end{itemize}

\paragraph{$\Sign(\gpk,\usk,\cred,\DAttrs,\msg) \rightarrow \sig$.} %
Commit to the message and recompute the user public key by running
$\Cmsg \gets \CCommit(\msg), \Ccom \gets \CCommit(\usk; 0)$.
Encrypt \Ccom as $\Ec \gets \EEnc(\opk,\Ccom)$ and create a NIZK proof using
$\NIZKRel_{\Sign}$ as $\NIZKproof \gets \NIZKProve^{\NIZKRel_{\Sign}}
(\NIZKcrs,(\usk,\Ccom,\Attrs,\msg,\cred),
(\Cmsg,\Ec,\DAttrs))$. Output $\sig \gets (\Ec,\NIZKproof)$.

\paragraph{$\Verify(\gpk,\sig,\DAttrs,\msg) \rightarrow 1/0$.} %
Parse \sig as $(\Ec,\NIZKproof)$, compute $\Cmsg \gets \CCommit(\msg)$ and
return $\NIZKVerify^{\NIZKRel_{\Sign}}(\NIZKcrs,(\Cmsg,\Ec,\DAttrs),
\NIZKproof)$.

\paragraph{$\Open(\gpk,\osk,\trans,\sig,\DAttrs,\msg)
  \rightarrow (\upk,\oproof)/\bot$.} %
First, verify the signature with $\Verify(\gpk,\sig,\DAttrs,\msg)$ and
return $\bot$ if verification fails. Else, parse \sig as $(\Ec,\NIZKproof)$
and run $\upk \gets \EDec(\osk,\Ec)$. Compute proof of correct decryption
as $\oproof_{\Open} \gets \NIZKProve^{\NIZKRel_{\Open}}(\NIZKcrs,\osk,(\Ec,
\upk))$. Return $(\upk,\oproof_{\Open})$.

\paragraph{$\Judge(\gpk,\upk,\Attrs,\oproof,\sig,\DAttrs,\msg)
  \rightarrow 1/0$.} %
First, verify the signature with $\Verify(\gpk,\sig,\DAttrs,\msg)$ and
return $\bot$ if verification fails. If verification suceeds, parse
\sig as $(\Ec,\cdot)$ and return $\NIZKVerify^{\NIZKRel_{\Open}}(\NIZKcrs,(\Ec,
\upk))$.


%%% Local Variables:
%%% mode: latex
%%% TeX-master: "uas"
%%% End:

\subsection{Correctness and Security of \GSACGen}
\label{ssec:security-gsac}

\begin{theorem}[Correctness of \GSACGen]
  \label{thm:correctness-gsac}
  If the underlying schemes for \todo{xxx}, our generic construction \GSACGen
  satisfies correctness as defined in \defref{def:correctness-gsac}.
\end{theorem}

\begin{proof}[\thmref{thm:correctness-gsac}]
  \todo{XXX}
\end{proof}

\begin{theorem}[Anonymity of \GSACGen]
  \label{thm:anonymity-gsac}
  If the NIZK system used for $\NIZKRel_{\Sign}$ is zero-knowledge and
  simulation-extractable, our \GSACGen construction satisfies anonymity as
  defined in \defref{def:anonymity-gsac}.
\end{theorem}

\begin{proof}[\thmref{thm:anonymity-gsac}]
  %
  \qed
\end{proof}

\begin{theorem}[Traceability of \GSACGen]
  \label{thm:trace-gsac}
  If the underlying scheme for \todo{xxx}, then our \GSACGen construction
  satisfies traceability as defined in \defref{def:trace-gsac}, except
  with negligible probability.
\end{theorem}

\todo{\usk belongs to \AttrSpace! I think this can lead to malleability attacks.
  Make them disjoint?}

\begin{proof}[\thmref{thm:trace-gsac}]
  \qed
\end{proof}

\begin{theorem}[Non-frameability of \GSACGen]
  \label{thm:frame-gsac}
  If the underlying schemes for \todo{xxx}, then our \GSACGen construction
  satisfies non-frameability as defined in \defref{def:frame-gsac}, except with
  negligible probability.
\end{theorem}

\begin{proof}[\thmref{thm:frame-gsac}]
  \qed
\end{proof}

%%% Local Variables:
%%% mode: latex
%%% TeX-master: "gsac"
%%% End:


%%% Local Variables:
%%% mode: latex
%%% TeX-master: "uas"
%%% End:

\section{From \GSAC to \UAS}
\label{sec:uas}

\todo{Some intro here}

\section{Model for \UAS Schemes}
\label{sec:model-uas}

We now define our model for Universal Anonymous Signatures. At a very high
level, it is a combination of the main models for group signatures and
anonymous credentials -- we try to extract the most useful features from both.
Instead of considering just one group, as in group signatures, we directly and
explicitly include support for multiple issuers as in anonymous credentials.
Moreover, instead of letting each user own a single credential, we allow
multiple credentials per user, even within the same group. This of course
makes us put forward refined notions of anonymity, traceability and
non-frameability. Note that, property-wise, we keep the three main properties
from group signatures, rather than sticking to the two usual properties of
anonymous credentials (anonymity and unforgeability). This is a consequence
of keeping the inspection capability from group signatures, which we consider
crucial in order to ensure some sort of accountability even in very adversarial
contexts where the issuer may be corrupt.
%
But, as stated in the introduction, we go beyond just combining anonymous
credentials and group signatures, and include support for arbitrary policies
during issuance, signing, and inspection. Consequently, our model needs to
capture the flexibility that these policies provide. In short, manipulations of
the issuance policy are only considered forgeries if the adversary manages to
create a valid signature using a credential that was obtained by dishonestly
manipulating the issuance policy (we discuss this further in the traceability
definition). Signing policies are taken into account both in the anonymity
property, where we need to restrict that both credentials used to produce the
challenge signatures always satisfy the policy, and in traceability.
\commentwho{Jesus}{Think a bit more about the following claim, and update if
  needed.}
We cannot give any assurance on the signing policies when the issuers are
corrupt, and thus they are not considered in the non-frameability property.
Inspection policies are considered both in traceability and non-frameability,
where the adversary wins if it manages to somehow alter the correct output of
\Inspect. What is even suprising, is that given the possible non-injectivity of
the inspection function lets us reach an even stronger notion of anonymity.
Namely, \uline{we can allow the adversary to open challenge signatures, as
  long as the inspection function outputs the same value for both}. We give more
details on these aspects in the actual definitions of the security properties.
%
Finally, from a functional perspective, we require that both issuer and
inspector fix the issuance predicate \fissue and inspection function \finsp when
generating their public keys. Indeed, both will be part of the common group key.
This also makes our definitions easier, as supporting multiple issuance
predicate and inspection functions per group would require to consider many
special cases -- and, anyway, that would probably end up being equivalent to
just requiring that a new group needs to be created per each $(\fissue,\finsp)$
pair. On the other hand, signing policies can be defined at signing time (in
practice, most probably by the verifiers). This is in line with the usual
practice in anonymous credentials, that let users prove arbitrary claims on
their credentials, as long as they are met by the contained attributes.

\subsection{Syntax}
\label{ssec:syntax}

In the following, we assume a setting with multiple groups. For simplicity,
we assume that each has its own issuer and inspector.

\todo{Are generalizations where different groups can share issuer or inspector
  straight forward?}

\todo{The notation for multisets is not strictly correct. Still, leaving it as
  is for now.} 

\begin{description}
\item[$\parm \gets \Setup(\secpar)$.] Given a security parameter \secpar,
  returns a global system parameter variable \parm.
\item[$(\ipk,\isk) \gets \IKeyGen(\parm,\fissue)$.] Given global system
  parameters \parm, and the function \fissue to be used to check that credential
  requestors meet the conditions to be issued a credential, an issuer runs
  \IKeyGen to generate its issuing key pair. Hereafter, we assume that the
  public part \ipk is added to the group public key \gpk, as well as \fissue.
\item[$(\opk,\osk) \gets \OKeyGen(\parm,\finsp)$.] Given global system
  parameters \parm, and function \finsp, an inspector runs \OKeyGen to generate
  its inspecting key pair. The function \finsp defines the type of utility that
  will be extractable from those pairs that do not meet conditions defined
  at signing time (typically, by verifiers). Hereafter, we assume that the
  public part \opk is added to the group public key \gpk, along with \finsp.
\item[$(\upk,\usk) \gets \UKeyGen(\parm)$.] Given global system parameters
  \parm, returns a user's key pair.
\item[$\langle \cred/\bot,\utrans/\bot \rangle \gets
  \langle
  \Obtain(\usk,\attrs,\ldblbrace (gpk_i,\cred_i)\rdblbrace_{i \in \Issuers}),
  \Issue(\isk,\upk,\attrs,\ldblbrace \gpk_i) \rdblbrace_{i \in \Issuers})
  \rangle$.] %
  This interactive protocol lets a user with key pair (\upk,\usk) running the
  \Obtain process, obtain a credential \cred from an issuer in the system, on
  attribute set $\attrs$ plus, possibly, blinded attributes attested by
  previously obtained credentials $\cred_i$, from a multiset of other issuers
  (which may include the one to whom a new credential is being requested) with
  public keys $\gpk_i$. The user outputs the produced credential \cred, while
  the issuer outputs the protocol transcript \utrans for the produced
  credential. To ease notation, we assume that the group to which each
  credential belongs is available from the credential itself, and therefore
  omit the $\gpk_i$ values from the syntax, unless necessary to avoid ambiguity.
\item[$\sig \gets \Sign(\gpk,\usk,\cred,\msg,\feval)$.]
  The user with with secret key \usk and credential \cred, produces a signature
  \sig over message \msg, meeting the conditions defined by \feval.
\item[$1/0 \gets \Verify(\gpk,\sig,\msg,\feval)$.] Checks whether \sig is a
  valid signature, over message \msg, from a user in the group with public key
  \gpk, for signing function \feval.
\item[$(\y,\iproof)/\bot \gets \Inspect(\gpk,\osk,\trans,\sig,\msg,\feval)$.] %
  Executed by the inspector with private key \osk. Receives a signature \sig
  over message \msg and evaluation function \feval. If \trans contains a set of
  valid transcripts corresponding
  to the $\langle\Obtain,\Issue\rangle$ interactive protocol execution that
  issued the credential used to produce \sig, the function outputs a value $\y$
  derived from the signer's public key, credential, message, and signature, as
  well as a proof of correct inspection. We sometimes abuse notation, and write
  $\trans[\uid]$ to mean that the \Inspect function is operating only for the
  obtain transcripts related to user \uid.
\item[$1/0 \gets \Judge(\gpk,\y,\iproof,\sig,\msg,\feval)$.] %
  Checks if \iproof is a valid inspection correctness proof for the value \y,
  obtained by applying \Inspect to the the signature \sig over message \msg. 
\end{description}

The correctness and security properties are defined with the help of the
following sets of oracles, and global variables that help oracles and games
keep consistent state.

\paragraph{Issuance, evaluation, and inspection functions.} %
We emphasize that, both in our syntax definition, as well as on the following
modelling, we make use of three different and abstract functions: \fissue,
\feval and \finsp. The three functions are introduced to allow customized
governance of the resulting instantiation of an \UAS scheme. They will be
defined by different parties, but in all cases, they are run by users (maybe,
on user-private data). Also, in all cases, the user has to prove correctness of
their computation. We introduce them next, \todo{and will give concrete examples
  in \secref{sec:uac-instantiation}.}

\begin{description}
\item[$\fissue: (\upk, \attrs,\scred) \rightarrow 0/1$.] Defined by issuers,
  governs what customized conditions an issuer requires in order to issue
  credentials, when receiving a request from user with public key \upk, for
  attributes \attrs. \fissue may run checks on a (possibly empty) set of
  additional credentials \scred, which may further be used for blind issuance
  \commentwho{Jesus}{can it?}. \fissue returns $1$ to accept a request, $0$ to
  reject it.  
\item[$\feval: (\upk,\cred,\msg) \rightarrow \varrngfeval$.] Can be defined by any
  party, although we anticipate that this will typically be done by either
  signers, verifiers, or some governance organization. It defines customized
  conditions to be met by the owner of a user key \upk and credential \cred,
  in order to sign a concrete message \msg. Its outputs  $\varrngfeval$ must
  belong in a well defined set \rngfeval, with the restriction that it must
  return $0$ when the conditions for signing are not met.
\item[$\finsp: (\varrngfeval,\upk,\cred,\msg) \rightarrow \varrngfinsp$.]
  Defines what utility value, derived from the user's public key, credential,
  and message should be extractable by the group inspector. Note that \finsp
  also receives as input a value in the range of the \feval function, \rngfeval.
  This will allow inspection logic to depend on the value produced by the
  evaluation function. The output of \finsp is a value \varrngfinsp, which must
  belong in a well defined set \rngfinsp.
\end{description}

\paragraph{Helper Function \Identify.} In addition, we make use of a helper
function \Identify for some of our definitions. This function receives
a signature and a user secret key, and determines whether the latter was used
to produce the former. This is in line with previous works on DAA
\cite{bfg+11,cdl16} and group signatures without traditional open functions
\cite{dl21,fgl21,gl19}. More concretely, the \Identify function, defined as $1/0
\gets \Identify(\usk,\sig)$, conveys meaning to our definitions conditioned on
the fact that, given a signature \sig accepted by \Verify, \Identify returns $1$
for exactly one \usk.

\paragraph{Global Variables.} %
The environment manages several global variables in the games posed to the
adversary. Users are referred to with user identifiers, \uid; for credentials,
we use \cid; for groups, \gid. For credentials and groups, we use bold font to
denote sets: i.e., \scid and \sgid denote sets of credential and group
identifiers. All tables/sets are initialized as empty tables/sets, denoted
with $\emptyset$.

\begin{description}
\item[Tables for parties]:
  \begin{description}
  \item[\HU and \CU.] Keep track of honest (\HU) and corrupted (\CU) users;
    i.e., they are sets of {\uid}s.
  \item[\HI and \CI.] Keep track of honest (\HI) and corrupted (\CI) issuers.
    Since we assume only one issuer per group, they are sets of {\gid}s.
  \item[\HO and \CO.] Keep track of honest (\HO) and corrupted (\CO) inspectors.
    Since we assume only one inspector per group, they are sets of {\gid}s.
  \end{description}
\item[Tables for keys]:
  \begin{description}
  \item[\UK, \PUBUK and \PRVUK.] \UK maintains user key pairs $(\upk,\usk)$.
    To refer to the key pair of a specific user, we use $\UK[\uid]$. \PUBUK
    is a shorthand to refer to the public part, and \PRVUK refers to the
    private part -- we may also index both by \uid.
  \item[\IK, \PUBIK and \PRVIK.] Same as \UK, but for issuer key pairs. In
    addition, \IK also includes the \fissue function, which is part of the
    public key.
  \item[\OK, \PUBOK, \PRVOK.] Same as \UK, but for inspector key pairs. In
    addition, \OK also includes the \finsp function, which is part of the
    public key.
  \item[\GK.] We bundle public keys for issuers and inspectors into a common
    public key for each group. We use the table \GK for that purpose which,
    consequently, can also be indexed by \gid.
  \end{description}
\item[Tables for credentials-related data]:
  \begin{description}
  \item[\CRED.] Stores information related to credentials obtained by honest
    users. Thus, it is indexable by \cid. More specifically, it stores tuples of
    the form $(\uid,\cred,\gid,\attrs,\scid)$, where \uid is the identity of the
    owner of the credential, \cred (when available) is the credential itself,
    \gid is the identifier of the group to which the credential belongs, \attrs
    are the attributes included in \cred, and \scid are the identifiers of any
    additional credential that \uid used to request \cred.
  \item[\OWNR.] For notational convenience, when we write $\OWNR[\cid]$ we mean
    ``\uid such that $\CRED[\cid] = (\uid, \cdot, \cdot, \cdot, \cdot)$''.
  \item[\ATTR.] For notational convenience, when we write $\ATTR[\cid]$ we mean
    ``\attrs such that $\CRED[\cid] = (\cdot, \cdot, \cdot, \attrs, \cdot)$''.
  \item[\GRP.] For notational convenience, when we write $\GRP[\cid]$ we mean
    ``\gid such that $\CRED[\cid] = (\cdot, \gid, \cdot, \cdot, \cdot)$''.
  \end{description}
\item[Tables for signatures]:
  \begin{description}
  \item[\SIG.] Maintains signatures generated via the \SIGN oracle, on behalf
    of honest users.
  \item[\CSIG.] Maintains challenge signatures, generated via the \CHALb oracle,
    by one of the challenge users in the anonymity game.
  \end{description}
\end{description}

\paragraph{Oracles.} %
Oracles are the interface of the adversary with the corresponding games. In
other words: through these oracles, the game environment exposes to the adversary
functionality that could otherwise be executed only by honest parties with
private knowledge -- knowledge that would make the adversary capable of
trivially breaking the security properties formalized in the experiments.
In the game-based definitions of our \UAS model, we leverage the following
oracles, which are formally defined in \figref{fig:oracles1} and
\figref{fig:oracles2}. 

\begin{description}
\item[\IGEN.] Adds a new issuer to the game, generating its keypair and setting
  the associated issuance function.
\item[\OGEN.] Adds a new inspector to the game, generating its key pair and
  setting the associated evaluation and inspection functions.
\item[\ICORR.] Corrupts an existing (and honest) issuer, by giving its secret
  key to the adversary.
\item[\OCORR.] Corrupts an existing (and honest) inspector, by giving its secret
  key to the adversary.  
\item[\HUGEN.] Adds a new honest user to the game, by honestly generating
  the user's key pair.
\item[\CUGEN.] Adds a new corrupt user to the game or, if the specified
  user already exists and is honest, corrupts it.
\item[\OBTISS.] Lets the adversary add a new honestly generated credential to
  the game, on behalf of an honest user.
\item[\OBTAIN.] Enables the adversary to play the role of a dishonest issuer
  in games that support it, by letting it interact with honest users who want to
  receive credentials.
\item[\ISSUE.] Allows the adversary to play the role of dishonest users,
  requesting an honest issuer to produce credentials for them.
\item[\SIGN.] Lets the adversary get signatures from credentials belonging
  to honest users.
\item[\OPEN.] Given an honestly produced signature, lets the adversary learn
  which credential was used to produce it.
\item[\CHALb.] Upon receiving two challenge credentials, a common intersecting
  set of attributes, and a message, returns a signature produced by one of these
  two credentials, defined by the bit $b$, which is established in the anonymity
  game.
\end{description}

{%\setlength\intextsep{\sep}
  \begin{figure*}[htp!]
    \centering
    \scalebox{0.9}{

      \begin{minipage}[t]{0.55\textwidth}

        \procedure{$\IGEN(\gid,\fissue)$}{%
          \pcif \gid \in \HI \lor \gid \in \CI: \pcreturn \bot \\
          (\ipk,\isk) \gets \IKeyGen(\parm) \\
          \IK[\gid] \gets ((\ipk,\fissue),\isk) \\
          \HI \gets \HI \cup \lbrace \gid \rbrace \\
          \GK[\gid] \gets ((\ipk,\fissue),\cdot) \\
          \pcreturn \ipk \\
        }

        \procedure{$\ICORR(\gid)$}{%
          \pcif \gid \in \CI \lor \gid \notin \HI: \pcreturn \bot \\
          \HI \gets \HI \setminus \lbrace \gid \rbrace \\
          \CI \gets \CI \cup \lbrace \gid \rbrace \\
          \pcreturn \isk \\
        }        

        \procedure{$\HUGEN(\uid)$}{%
          \pcif \uid \in \HU \lor \uid \in \CU: \pcreturn \bot \\
          (\upk,\usk) \gets \UKeyGen(\parm) \\
          \UK[\uid] \gets (\upk,\usk);
          \HU \gets \HU \cup \lbrace  \uid \rbrace \\
          \pcreturn \top
        }        
        
      \end{minipage}
    }
    \scalebox{0.9}{
      
      \begin{minipage}[t]{.5\textwidth}

        \procedure{$\OGEN(\gid,\feval,\finsp)$}{%
          \pcif \gid \in \HO \lor \gid \in \CO: \pcreturn \bot \\
          (\opk,\osk) \gets \OKeyGen(\parm) \\
          \OK[\gid] \gets ((\opk,\finsp),\osk) \\
          \HO \gets \HO \cup \lbrace \gid \rbrace \\
          \GK[\gid] \gets (\cdot,(\opk,\finsp)) \\
          \pcreturn \opk \\
        }

        \procedure{$\OCORR(\gid)$}{%
          \pcif \gid \in \CO \lor \gid \notin \HO: \pcreturn \bot \\
          \HO \gets \HO \setminus \lbrace \gid \rbrace \\
          \CO \gets \CO \cup \lbrace \gid \rbrace \\
          \pcreturn \osk \\
        }        
        
        \procedure{$\CUGEN(\uid,\upk)$}{%          
          \pcif \uid \in \CU: \pcreturn \bot \\
          \CU \gets \CU \cup \lbrace \uid \rbrace \\          
          \pcif \uid \in \HU: \\
          \pcind \HU \gets \HU \setminus \lbrace \uid \rbrace; \\
          \pcind \pcreturn (\UK[\uid],\CRED[\uid]) \\
          \pcelse: \UK[\uid] = (\upk,\bot) \\          
          \pcreturn \top
        }
        
      \end{minipage}
      
    }

    \caption{Detailed oracles available in our model (1/2). Oracles for
      generating key material for users, issuers, and inspectors.}
    \label{fig:oracles1}
  \end{figure*}
}

{%\setlength\intextsep{\sep}
  \begin{figure*}[htp!]
    \centering
    \scalebox{0.9}{

      \begin{minipage}[t]{0.55\textwidth}

        \procedure{$\ISSUE(\cid,\uid,\gid,\attrs,\ldblbrace\scid\rdblbrace)$}{%
          \pcif \uid \notin \CU: \pcreturn \bot \\          
          \pcif \gid \notin \HI: \pcreturn \bot \\
          \pcif \exists \gid' \in \sgid~\st~\gid' \notin \HI: \pcreturn \bot \\
          \pcif \CRED[\cid] \neq \bot: \pcreturn \bot \\
          \langle \cdot, \utrans \rangle \gets
          \langle \adv, \\
          \hspace*{45pt}
          \Issue(\PRVIK[\gid],\PUBUK[\uid], \attrs,
          \ldblbrace \CRED[\scid] \rdblbrace) \rangle \\
          \trans[\cid] \gets \utrans \\
          \CRED[\cid] \gets (\uid, \cdot, \gid, \attrs, \scid) \\
          \pcreturn \top \\          
        }                

        \procedure{$\OBTAIN(\cid,\uid,\gid,\attrs,\ldblbrace\scid\rdblbrace)$}{%
          \pcif \uid \notin \HU: \pcreturn \bot \\
          \pcif \exists \cid' \in \scid~\st~\CRED[\cid'] \neq \bot: \pcreturn \bot \\
          \langle \cred, \cdot \rangle \gets
          \langle \Obtain(\PRVUK[\uid],\attrs, 
          \ldblbrace \CRED[\scid] \rdblbrace), \\
          \hspace*{50pt} \adv \rangle \\
          \CRED[\cid] \gets (\uid, \cdot, \gid, \attrs, \scid) \\          
          \pcreturn \top \\
        }

        \procedure{$\INSPECT(\gid,\sig,\msg)$}{%
          \textrm{Let}~\uid~\textrm{be s.t.}~(\cid,\sig,\msg,\feval)
          \in \SIG[\uid] \\
          (\y,\iproof) \gets
          \Inspect(\GK[\gid],\PRVOK[\gid],\trans,\sig,\msg,\feval) \\
          \pcif \CSIG[\sig] \neq \bot: \\
          \pcind \textrm{Parse $\CSIG[\sig]$ as $(\sig_{1-b},\msg,\feval)$} \\
          \pcind (\y_{1-b},\iproof_{1-b}) \gets
          \Inspect(\GK[\gid],\PRVOK[\gid],\\
          \hspace*{107pt} \trans,\sig_{1-b},\msg,\feval) \\
          \pcind \pcif \y_{1-b} \neq \y: \pcreturn \bot \\
          \pcreturn (\y,\iproof)
        }
        
      \end{minipage}
    }
    \scalebox{0.9}{
      
      \begin{minipage}[t]{.5\textwidth}

        \procedure{$\OBTISS(\cid,\uid,\gid,\attrs,\ldblbrace\scid\rdblbrace)$}{%
          \pcif \uid \notin \HU: \pcreturn \bot \\
          \pcif \gid \notin \HI: \pcreturn \bot \\
          \pcif \exists \gid' \in \sgid~\st~\gid' \notin \HI: \pcreturn \bot \\
          \pcif \exists \cid' \in \scid~\st~\CRED[\cid'] \neq \bot:
          \pcreturn \bot \\
          \langle \cred, \utrans \rangle \gets
          \langle \Obtain(\PRVUK[\uid],\attrs,
          \ldblbrace \CRED[\scid] \rdblbrace), \\
          \hspace*{60pt} \Issue(\PRVIK[\gid],\PUBUK[\uid],\attrs,
          \ldblbrace \CRED[\scid] \rdblbrace ) \rangle \\
          \trans[\cid] \gets \utrans \\
          \CRED[\cid] \gets (\uid, \cred, \gid, \attrs, \scid) \\
          \pcreturn \top \\
        }

        \procedure{$\SIGN(\uid,\cid,\msg,\feval)$}{%
          \pcif \uid \notin \HU \lor \CRED[\cid] = \bot: \pcreturn \bot \\
          \sig \gets \Sign(\GK[\GRP[\cid]],\PRVUK[\uid],\msg,\CRED[\cid],\feval) \\
          \SIG[\uid] \gets \SIG[\uid] \cup
          \lbrace (\cid,\sig,\msg,\feval) \rbrace \\
          \pcreturn \sig \\
        }                

        \procedure{$\CHALb(\cuid_{0,1},\ccid_{0,1},\msg,\feval)$}{%
          \pcif \cuid_0 \notin \HU \lor \cuid_1 \notin \HU: \pcreturn \bot \\
          \pcif \gid = \GRP[\ccid_0] \neq \GRP[\ccid_1]: \pcreturn \bot \\
          \pcif \gid \in \CO: \pcreturn \bot \\
          \pcif \feval(\PUBUK[\uid_0],\CRED[\scid_0],\msg) \neq \\
          \pcind \feval(\PUBUK[\uid_1],\CRED[\scid_1],\msg):
          \pcreturn \bot \\
          \csig_b \gets \Sign(\GK[\gid],\PRVUK[\cuid_b],\CRED[\ccid_b],
          \msg,\feval) \\
          \csig_{1-b} \gets \Sign(\GK[\gid],\PRVUK[\cuid_{1-b}],\CRED[\ccid_{1-b}],
          \msg,\feval) \\          
          \CSIG[\csig_b] \gets 
          \lbrace (\csig_{1-b},\msg,\feval) \rbrace \\
          \pcreturn \csig_b
        }
        
      \end{minipage}
      
    }

    \caption{Detailed oracles available in our model (2/2). Oracles for
      obtaining credentials, signatures, and processing them.}
    \label{fig:oracles2}
  \end{figure*}
}

\paragraph{Correctness.} %
Correctness of \UAS schemes is formalized through the experiment in
\figref{fig:exp-uas-corr}. It states that a signature over any arbitrary message
and valid function \feval \todo{Should we explicitly check \feval in the game?
  If so, how?},
produced honestly by an honest user \uid with credential \cid, is accepted by
\Verify. Moreover, upon issuance of the credential \cid, it must have met the
conditions set by \fissue. Similarly, the value returnd by \Inspect must be
accepted by \Judge, and must match the output of applying \finsp on \feval, the
credential, user key, and message.

\begin{figure}[htp!]
  \procedure{$\ExpCorrect(1^\secpar)$}{%
    \parm \gets \Setup(1^\secpar) \\
    (\uid,\cid,\msg,\feval)
    \gets \adv^{\IGEN,\OGEN,\HUGEN,\OBTISS}(\parm) \\
    \sig \gets \Sign(\GK[\cid],\PRVUK[\uid],\cid,\msg,\feval) \\
    \pcif \Verify(\GK[\cid],\sig,\msg,\feval) = 0: \pcreturn 1~
    \pccomment{\sig fails to verify} \\
    \pcif \CorrectIssue(\uid,\cid) = 0: \pcreturn 1~
    \pccomment{\cid should not have been issued} \\
    % \pcif \CorrectEval(\uid,\cid,\msg,\feval)
    % = 0: \pcreturn 1~
    % \pccomment{\msg should not have been signed} \\
    \pcif \CorrectEvalInspect(\uid,\cid,\msg,\sig,\feval) = 0: \pcreturn 1~
    \pccomment{The combination of evaluation and inspection is wrong} \\
    \pcreturn 0 \\
  }

  \procedure{$\CorrectIssue(\uid,\cid)$}{
    \gid \gets \GRP[\cid] \\
    \textrm{Parse}~\ATTR[\cid]~\textrm{as}~(\attrs^\cid,\scid^\cid);~
    \textrm{Parse}~\GK[\gid]~\textrm{as}~
    ((\cdot,\fissue^\gid),\cdot) \\        
    \pcif \fissue^\gid(\PUBUK[\uid],\attrs^\cid,\CRED[\scid]^\cid) = 0: 
    \pcreturn 0 \\
    \pcreturn 1 \\   
  }

  % \procedure{$\CorrectEval(\uid,\cid,\msg,\feval)$}{
  %   \pcif \feval(\PUBUK[\uid],\CRED[\cid],\msg) = 0: \pcreturn 0 \\
  %   \pcreturn 1 \\
  % }

  \procedure{$\CorrectEvalInspect(\uid,\cid,\msg,\sig,\feval)$}{
    \textrm{Let}~\gid \gets \GRP[\cid];~
    \textrm{Parse}~\GK[\gid]~\textrm{as}~(\cdot,(\cdot,\finsp^\gid)) \\
    \pccomment{\sig cannot be inspected} \\
    \pcif \Inspect(\GK[\gid],\PRVOK[\gid],\trans,\sig,\msg,\feval) = \bot: \pcreturn 0 \\
    \pccomment{The produced $(\y,\iproof)$ pair is rejected by \Judge
      % \todo{This must be conditioned on \feval = 0!}
    } \\
    \pcif (\y,\iproof) \gets \Inspect(\GK[\gid],\PRVOK[\gid],\trans,\sig,\msg,\feval)
    \land
    \Judge(\GK[\gid],\y,\iproof,\sig,\msg) = 0: \\
    \pcind \pcreturn 0 \\
    \pccomment{\y is the wrong value} \\
    \pcif \y \neq \finsp^\gid(\feval(\PUBUK[\uid],\CRED[\cid],\msg),
    \PUBUK[\uid],\CRED[\cid],\msg)):
    \pcreturn 0 \\
    \pcreturn 1
  }
  
  \caption{Correctness experiment for \UAS schemes.}
  \label{fig:exp-uas-corr}
\end{figure}

\subsection{Security Properties}
\label{ssec:security}

\paragraph{Anonymity.} %
In group signatures, anonymity captures that no adversary must be able to learn,
from any group signature, the identity (e.g., member index) of its signer. In 
anonymous credentials, it requires that no adversary should learn anything about
the holder of a successfully shown credential, beyond that he owns a credential
attesting for the claimed predicate of the attributes it contains. In both GS
and AC, it is also typically required that
multiple signatures/presentations by the users are unlinkable. The approach to
formally state this property is in both cases frequently the same: the adversary
picks two (honest) users (or credentials in the AC case), the game randomly
chooses one of them, and lets the adversary request challenge
signatures/presentations from it. The adversary wins if it succeeds in guessing
which was the chosen user/credential better than guessing at random. In group
signatures, the game must also restrict the adversary from opening challenge
signatures. In anonymous credentials, the game must further constraint the
adversary to output credentials that both meet the same (challenge) predicate
of their attributes.

In our notion of anonymity for \UAS, we need to merge the previous constraints.
Furthermore, a key difference with group signatures is that the game requires
the adversary to output credential identifiers, rather than user identifiers.
Specifically, this means that the adversary may actually output two credentials
that belong to the same user. Therefore, in some sense, the anonymity we get is
more general than that of group signatures. Moreover, in order to prevent
trivial wins by the adversary, we have to restrict that the challenge signatures
belong to a group that has been ``programmed'' with the same evaluation and
inspection functions, which is again a generalization over group signatures,
as discussed in the sequel. Other than this, the overall approach is the same:
the adversary picks two honestly obtained credentials, and one of them (set
by the $b$ value defining the $anon-b$ game) is used to program the \CHALb
oracle. The formal specification of the anonymity game is
given in \figref{fig:exp-uas-anonb}, where $\Oanonc \gets (\lbrace\HU,\CU\rbrace
\GEN,\lbrace\II,\OO\rbrace\GEN,\lbrace\II,\OO\rbrace\CORR,\OBTAIN,\SIGN,
\INSPECT)$ and $\Oanong \gets (\lbrace\HU,\CU\rbrace\GEN,\lbrace\II,\OO\rbrace
\GEN,\lbrace\II,\OO\rbrace\CORR,\OBTAIN,\SIGN,\INSPECT,\CHALb)$

\begin{figure}[htp!]
  \procedure{$\ExpAnonb(1^\secpar)$}{%
     \parm \gets \Setup(1^\secpar) \\
     (\ccid_0,\ccid_1,\status) \gets \adv^{\Oanonc}(\choose,\parm) \\
     b^* \gets \adv^{\Oanong} (\guess,\status) \\
     \pcreturn b^*
  }
  \caption{Anonymity experiment for \UAS schemes.}
  \label{fig:exp-uas-anonb}
\end{figure}

\begin{definition}{(Anonymity of \UAS)}
  We define the advantage \AdvAnon of $\adv$ against \ExpAnonb as
  $\AdvAnon=|\Pr\lbrack\ExpAnono(1^\secpar)=1\rbrack-
  \Pr\lbrack\ExpAnonz(1^\secpar)=1\rbrack|$.
  %
  An \UAS scheme satisfies anonymity if, for any p.p.t. adversary $\adv$,
  \AdvAnon is a negligible function of $1^\secpar$.
\end{definition}

\paragraph{Discussion on the generality of anonymity in \UAS schemes.} %
This notion of anonymity is more general than that of group signatures in the
sense that calls to the \INSPECT oracle reveal an arbitrary function of the
identity of the user -- which can certainly be the member index itself, as it
is frequent in group signatures, or any other function computable from the
user public key, its credentials, and produced signature. Moreover, note that
our notion is even stronger than the conventional ``CCA-like'' anonymity notion
that gives the adversary access to the open oracle, only restricting it when
trying to open challenge signature. Precisely the introduction of generic
inspection functions allows us to \uline{\emph{let the adversary open
    challenge signatures $\csig_b$}} as long as the counterpart $\csig_{1-b}$
makes \Inspect produce the same \y value as output. This is certainly not
possible when using conventional opening, since it directly outputs the identity
of the signer -- which cannot be the same for different challenge users.

\paragraph{Traceability.} %
% Traceability is one of the unforgeability-related properties in group
% signatures. It captures that any signature accepted by \Verify needs to open
% to one of the users that joined the group. While there is no traceability notion
% in anonymous credentials, it is natural to map it to their unforgeability
% property; if only because both require the issuer to be honest. Unforgeability
% in anonymous credentials typically ensures that no adversary can get a verifier
% to accept a credential presentation requiring a set of attributes that is not
% contained in one of the credentials controlled by the adversary.

% Our notion of traceability for \UAS is inspired by both. First, we restrict to
% signatures that are accepted by \Verify. For any such signature, \Inspect has to
% create a valid output (i.e., must not abort). The output of \Inspect has to be
% accepted by \Judge too. Moreover, there must exist one user (either honest or
% corrupt) who used his public key to obtain one or more credentials that allowed
% him to produce a signature which, once inspected, returns \y.
In group signatures, traceability ensures that, in the presence of an honest
issuer, every signature accepted by the verify algorithm must have been created
by a user who joined the group.  With conventional opening, this is essentially
checked by requesting the adversary to produce a signature, opening it, and
checking whether or not it comes from a group member; i.e., this ensures that,
as long as there are verifiable inspections, inspection cannot be forged in the
presence of an honest issuer.
%
The somehow equivalent property in anonymous credentials is unforgeability. It
requires that, again in the presence of an honest issuer, no adversary can
succeed in a credential presentation for attributes (or predicates of them) that
are not contained in a legitimately issued credential it controls. In other
words, that the evaluation of the credential attributes cannot be forged.
%
Both notions have an equivalent in \UAS. That is, we need to ensure that, in the
presence of an honest issuer, no signature accepted by \Verify can result in a
wrong inspection, nor it can have been obtained by a wrong evaluation of some
credential. Note, however, that \UAS offer the adversary yet another way to
break traceability/unforgeability. Namely, an adversary can attempt to
illegitimately obtain a credential that later enables it to produce untraceable
signatures; i.e., signatures that evaluate correctly, or for which \Inspect
returns the appropiate value, yet the adversary should not have been able to
obtain a credential that allowed it to produce such signatures. Our notion of
traceability/unforgeability thus challenges the adversary to produce a signature
that breaks any of those conditions. The formal
definition of the traceability experiment is given in \figref{fig:exp-uas-trace},
where $\Otrace \gets \lbrace\HU,\CU\rbrace\GEN,\IGEN,\OGEN,\OCORR,\OBTISS,
\ISSUE,\SIGN,\INSPECT$.

\begin{figure}[htp!]
    \procedure{$\ExpTrace(1^\secpar)$}{%
      \parm \gets \Setup(1^\secpar) \\
      (\gid,\sig,\msg,\feval) \gets \adv^{\Otrace}(\parm) \\
      \pcif \Verify(\GK[\gid],\sig,\msg,\feval) = 0: \pcreturn 0 \\
      \pcif \EvalInspectForge(\gid,\sig,\msg,\feval) = 1 \\%\land
%      \InspectForge(\gid,\sig,\msg,\feval) = 1: \\
      \pcind \pcreturn 1 \\
      \pcreturn 0 \\
    }
        
    % \procedure{$\EvalForge(\gid,\sig,\msg,\feval)$}{%
    %   \pcif \exists \uid~\st~\Identify(\uid,\sig) = 1~\land \\
    %   \pcind \pccomment{\uid does not own a \cred that satisfies \feval...} \\
    %   \pcind (\nexists \cred \in \CRED[\uid]~\st~
    %   \feval(\PUBUK[\uid],\cred,\msg) \neq 0 ~\lor \\
    %   \hspace*{13pt} \pccomment{... or \uid does, but \cred was itself obtained
    %     fraudulently} \\
    %   \hspace*{13pt} (\exists \cred \in \CRED[\uid]~\st~
    %   \feval(\PUBUK[\uid],\cred,\msg) \neq 0~\land \\
    %   \pcind \pcind \fissue(\PUBUK[\uid],\attrs^\cid,\scred^\cid) = 0, \\
    %   \pcind \pcind \textrm{where $\CRED[\cid]=\cred$ and
    %     $(\attrs^\cid,\scred^\cid) \gets \ATTR[\cid]$})): \\
    %   \pcind \pcreturn 1 \\
    %   \pcelse \pcreturn 0 \\
    % }
    
    \procedure{$\EvalInspectForge(\gid,\sig,\msg,\feval)$}{%
      \pccomment{All \Obtain transcripts fail to inspect \sig} \\
      \pcif \Inspect(\PRVOK[\gid],\trans,\sig,\msg) \neq \bot~\lor \\
      \pcind \pccomment{A valid transcript exists, but ...} \\
      \pcind \exists \uid~\st~\Inspect(\PRVOK[\gid],\trans[\uid],\sig,\msg) =
      (\y,\iproof)~\land (\\
      \pcind \pcind \pccomment{... the proof is rejected by \Judge ...} \\
      \pcind \pcind \Judge(\y,\iproof,\sig,\msg,\GK[\sgid]) = 0~\lor \\
      \pcind \pcind \pccomment{... or \uid does not own a \cred for which
        \finsp outputs \y} \\
      \pcind \pcind \nexists \cred \in \CRED[\uid]~\st~
      \y = \finsp(\feval(\PUBUK[\uid],\cred,\msg),
      \PUBUK[\uid],\cred,\msg)~\lor \\
      \pcind \pcind \pccomment{... or \uid does, but \cred was itself obtained
        fraudulently} \\
      \pcind \pcind (\exists \cred \in \CRED[\uid]~\st~
      y = \finsp(\feval(\PUBUK[\uid],\cred,\msg),
      \PUBUK[\uid],\cred,\msg)~\land \\
      \pcind \pcind \pcind \fissue(\PUBUK[\uid],\attrs^\cid,\scred^\cid) = 0, \\
      \pcind \pcind \pcind \textrm{where $\CRED[\cid]=\cred$ and
        $(\attrs^\cid,\scred^\cid) \gets \ATTR[\cid]$})): \\      
      \pcind \pcreturn 1 \\      
      \pcreturn 0
    }
  \caption{Traceability experiment for \UAS schemes.}
  \label{fig:exp-uas-trace}
\end{figure}

\begin{definition}{(Traceability of \UAS)}
  We define the advantage \AdvTrace of $\adv$ against \ExpTrace as
  $\AdvTrace=\Pr\lbrack\ExpTrace(1^\secpar)=1\rbrack$.
  %
  A \UAS scheme satisfies traceability if, for any p.p.t. adversary $\adv$,
  \AdvTrace is a negligible function of $1^\secpar$.
\end{definition}

\paragraph{Discussion on the generality of traceability in \UAS schemes.} %
The notion of traceability we present for \UAS is strictly more general than
the corresponding one for group signatures. This is again a direct
consequence of the fact that \Inspect, which can return an arbitrary function
of the signer's credential (and signed message), is a generalization of the
conventional \Open -- if we make \Inspect return just the identity of the
signer, then we get something similar to the usual notion of traceability.
Although, even in that case, we need to take into account attributes, and the
fact that the same user may obtain multiple credentials (that is why, even when
having \Inspect return the identity of the signer, our notion is not exactly
the same). And, in this sense, the unforgeability notion that \UAS have is
equivalent to that of anonymous credentials. It would seem, though, that we do
not need the traceability part of group signatures; after all, it is the
protection against wrong claims on attributes what enables meaningful and
flexible authentication. However, the type of protection against misuses of
the inspection functionality that we can get with an honest issuer (as in
traceability) is much higher than without an honest user (as in
non-frameability). Specifically, with an honest issuer we can ensure that
the adversary cannot even alter the value returned by \Inspect on signatures
by corrupt users, nor the output of the signing predicate \feval. Whereas, with
a corrupt issuer, all we can ensure is that
the adversary cannot forge a signature from an honest user for which \Inspect
returns the same value as a signature by that honest user would produce; and,
certainly, a corrupt issuer can arbitrarily issue credentials meeting any
desired predicate \feval .Traceability is, therefore, a core property to ensure
accountability.

Note also that we have implicitly assumed an unforgeability relaxation in the
sense that we do not consider a forgery the fact that an adversary can obtain
credentials that do not meet the \fissue predicate \emph{as long as they are
  not used to produced untraceable signatures}. Indeed, this is in line with
previous work \cite[Section 3.3.3]{ckl+15}, and it does make sense as, in the
real world, being able to fraudulently obtain credentials that cannot be later
used to authenticate, does not pose a risk. However, an adversary breaking
the issue policy defined by \fissue to get a credential and later using it to
create a valid signature (i.e., meeting \feval or \finsp) is considered a
forgery by our definition.

\paragraph{Non-frameability.} %
Non-frameability variants are a core unforgeability-type property in group
signatures. However, no
similar property is modeled for anonymous credentials (\todo{see the discussion
  in \secref{sec:introduction} for further detail}). It is a quite strong
property, as it must be ensured even in the presence of dishonest issuer and
opener. Intuitively, it prevents the adversary from creating a signature that
frames an honest user. Depending on the inspection capabilities of the scheme,
this framing could be done in different ways; i.e., by convincing third parties
that signatures by different (possibly corrupt) users are linked, or directly
by having inspection proofs output the identity of a user who did not create the
signature being inspected.

The notion of non-frameability in \UAS schemes is unavoidably more subtle than
in group signatures, though. To see this, we note that, by allowing any
arbitrary inspection function \finsp to be used, it can be perfectly valid to
have a signature produced by a corrupted user output the same \y value upon
inspection than the one that \finsp outputs when inspecting a signature by an
honest user. As a concrete example, imagine an inspection function that returns
the banking account number (or cryptocurrency address) to be used to request
payment of a fine. In this example, a corrupted user and an honest user could
both have delegated this to some agency\footnote{\todo{Maybe, come up with a
    better example. How about spam filters that apply some sort of classification
    engine? \finsp could be this engine, and emails by many different users could
    be classified just as ``Spam'' or ``Not spam''}}, so it makes perfect sense
that the inspection function returns the same value for both users.
%
Thus, in \UAS schemes, we must be more subtle, by necessity. Note that we cannot
just require that no adversary can produce a signature for which inspection
returns the same value as it would for a signature honestly generated by an
uncorrupted user. This is again the case since inspection may be non-injective,
which means that inspection of a signature by some corrupted user may
legitimately produce the same value as that of an honest user. Thus, we need
to further refine the definition by requiring that (i) no adversary can produce
a signature that is recognized (via \Identify) as originating from an honest
user \uid, (ii) who owns a credential that makes \finsp return a valid \y value
(i.e., generated by \Inspect along with a proof, and accepted by \Judge), (iii)
without having queried the \SIGN oracle before. This non-frameability property
for \UAS schemes is formally defined in \figref{fig:exp-uas-frame}, where
$\Oframe \gets \lbrace\HU,\CU\rbrace\GEN,
\lbrace\II,\OO\rbrace\GEN,\lbrace\II,\OO\rbrace\CORR,\OBTAIN,\SIGN$

\begin{figure}[htp!]
  \procedure{$\ExpNonframe(1^\secpar)$}{%
    \parm \gets \Setup(1^\secpar) \\
     (\gid,\sig,\msg,\feval,\y,\iproof) \gets \adv^{\Oframe}(\parm) \\
     \pcreturn 1~\pcif \Verify(\GK[\gid],\sig,\msg,\feval) = 1 \land
     \Judge(\GK[\gid],\y,\iproof,\sig,\msg) = 1~\land \\
     \pcind \pccomment{\adv~did not query \SIGN to get \sig, yet \sig originates
       from an honest \uid, owning a credential that makes \finsp
       evaluate to \y} \\
     \pcind \exists \uid \in \HU~\st~\Identify(\PRVUK[\uid],\sig) = 1 \land
     (\cdot,\sig,\msg,\cdot) \notin \SIG[\uid]~\land\\
     \pcind \exists \cred \in \CRED[\uid]~\st~
     \finsp(\feval(\PUBUK[\uid],\cred,\msg),\PUBUK[\uid],\cred,\msg) = \y \\
%     \pccomment{\todo{I don't think we can give any assurance about \feval here,
%         given that issuers are corrupt. But still, think about it.}} \\
     \pcreturn 0
  }
  \caption{Non-frameability experiment for \UAS schemes.}
  \label{fig:exp-uas-frame}
\end{figure}

\begin{definition}{(Non-frameability of \GSAC)}
  We define the advantage \AdvNonframe of $\adv$ against \ExpNonframe as
  $\AdvNonframe=\Pr\lbrack\ExpNonframe(1^\secpar)=1\rbrack$.
  %
  A \GSAC scheme satisfies non-frameability if, for any p.p.t. adversary $\adv$,
  \AdvNonframe is a negligible function of $1^\secpar$.
\end{definition}

\paragraph{Discussion on the generality of non-frameability in \UAS schemes.} %
Anonymous credentials do not have non-frameability property and, thus, it is
hard to make a comparison. However, we can draw some connections with AC schemes
that support revocation, as revocation is somehow equivalent to linking, which
is a type of inspection available in group signatures. In this sense, note that
basic revocation (without straight deanonymization) can be trivially achieved
through our generic \Inspect function. For instance, one could set \finsp to
be a pseudorandom number seeded with the user's public key (or credential). In
this sense, \Inspect could be essentially seen as a Verifiable Random Function.
If we compare with group signatures, our notion is again more general than the
conventional one. Indeed, in typical group signatures, \Open returns the
identity of the signer, and thus non-frameability does not allow the same value
to be returned when opening signatures by different users. Otherwise, a
corrupted user controlled by the adversary could be able to create a signature
opening to some honest user (hence, framing him). But this is again possible
with our definition which, in case of making \Inspect equal \Open, becomes
equivalent to conventional non-frameability. However, it also allows more
generic situations in which signatures by different users may return the same
value. 

\subsection{Functionality ``Transformations'' and Special Cases}
\label{ssec:transformations}

\commentwho{Jesus}{This section is aimed at showing the ``universality'' of
  our model. If I'm not mistaken, ideally, it should support the following
  variations (except the transformation to interactive presentations,
  probably).}

\subsubsection{Transformation to Interactive Presentations}
\label{sssec:interactivetransform}

\todo{Is it possible to give it in some generic sense?}

\subsubsection{Combining Multiple Credentials in One Presentation}

\todo{I think this might be possible if we can use \fissue so that any user can
  create arbitrary groups, where he is the issuer, an can thus run
  \Obtain,\Issue protocols to combine credentials from other groups. This is
  probably inefficient (or, at least, more than doing so wihtout having to
  create ad hoc groups), but at least allows the enhanced functionality...}

\subsubsection{Restricting to Conventional GS and AC Schemes}

\todo{It would be nice to prove that a \GSAC scheme to which we remove the
  \Open/\Judge functions becomes an AC scheme. And conversely, a \GSAC scheme
  where all credentials have no attributes, and where we restrict to only
  one credential per user, becomes a conventional GS scheme.}


\subsubsection{Group Signatures with Message Dependent Opening}

\todo{Describe what type of \finsp would be needed to achieve a functionality
  similar to that of GS-MDO. Does this fit our model straight away? Maybe also
  show how other types of GS could be ``mimicked'' (e.g., GS only with linking).
  This would also be good to test if the model is as generic as I think it is}

\subsubsection{Ring Signatures}

\todo{Can we come up with a (\fissue,\feval,\finsp) tuple that allows us to
  somehow mimic ring signatures (i.e., ``a group signature without issuer nor
  opening''? Does the model support this?}
  
%%% Local Variables:
%%% mode: latex
%%% TeX-master: "uas"
%%% End:

\section{A Generic \UAS Construction}
\label{sec:gen-construction}

In this section, we give a generic construction of an \UAS scheme, based on
generic building blocks. In \secref{sec:instantiation}, we give a concrete
instantiation.

\subsection{Building Blocks}
\label{ssec:bblocks}

\subsubsection{Commitments}

\todo{Introduce, and summarise correctness and security.}

\begin{description}
  \item[$\Cparm \gets \CSetup(\Csecpar).$]
  \item[$(\Ccom,\Copn) \gets \CCommit(\Cparm,\msg).$] 
  \item[$\Cproof \gets \CProve(\Ccom,\Copn,\msg).$]    
  \item[$1/0 \gets \CVerify(\Cproof,\Ccom).$] 
\end{description}

\subsubsection{Signatures with Proofs of Knowledge}

As put forward in \cite{cl02}, a generic way to build anonymous credentials
requires a digital signature scheme with efficient protocols for proving
knowledge of a digital signature, and signing committed values. This approach
also has much in common with the Sign-Randomize-Prove (SRP) approach to build
group signatures \needcite. Thus, even a priori, it seems a good core component
to build something that is actually in betwween GS and AC. In the sequel, we use
\RS to refer to schemes of this type, which must provide the functionality
specified by the following syntax:

\begin{description}
\item[$(\RSvk,\RSsk) \gets \RSKeyGen(\RSsecpar,\RSlen)$.] Generates a key pair,
  consisting on the signig key \RSsk and the verification key \RSvk, which
  support sigining blocks of up to \RSlen messages.
\item[$\RSsig \gets \RSSign(\RSsk,\msgset,\cmsgset)$.] Given signing key \RSsk,
  a set of messages \msgset, and a set of commitments to messages, where
  $|\msgset + \cmsgset| \le \RSlen$, returns a signature \RSsig.
\item[$1/0 \gets \RSVerify(\RSvk,\RSsig,\amsgset)$.] Given a verification key
  \RSvk, a signature \RSsig, supposedly over set of messages \amsgset which may
  contain commitments to messages, returns $1$ or $0$. \todo{Assume that, within
    \amsgset, it is possible to know which are commitments and wich plain
    messages?}
\item[$\RSproof/\bot \gets \RSProve(\RSsig,\amsgset,\D)$.] Given signature
  \RSsig, over set of messages (and commitments to messages) \amsgset, out of
  which those indexed by set \D are to be revealed, creates a proof of knowledge
  \RSproof of \RSsig.
\item[$1/0 \gets \RSProveVer(\RSproof,\msgset)$.] Given a proof of knowledge of
  a signature over a set of messages, which includes \msgset, returns $1$ or
  $0$.
\end{description}

\todo{Summarise correctness and security.}

\subsubsection{Digital Signatures}

\todo{Introduce, and summarise correctness and security.}

\begin{description}
\item[$(\Svk,\Ssk) \gets \SKeyGen(\Ssecpar)$.]
\item[$\Ssig \gets \SSign(\Ssk,\msg)$.]
\item[$1/0 \gets \SVerify(\Svk,\Ssig,\msg)$.]  
\end{description}

\subsubsection{Encryption}

\todo{Introduce, and summarise correctness and security.}

\begin{description}
\item[$(\Eek,\Edk) \gets \EKeyGen(\Esecpar)$.]
\item[$\Ec \gets \EEnc(\Eek,\msg)$.]
\item[$\msg \gets \EDec(\Edk,\Ec)$.]
\end{description}

\subsubsection{Non-Interactive ZK Proofs of Knowledge}

\todo{Introduce, and summarise correctness and security.}

\begin{description}
\item[$\NIZKcrs\gets \NIZKSetup(\NIZKsecpar)$.]
\item[$\NIZKproof \gets \NIZKProve(\NIZKcrs,\NIZKx,\NIZKw)$.]
\item[$1/0 \gets \NIZKVerify(\NIZKcrs,\NIZKproof,\NIZKx)$.]
\end{description}

We use $\NIZK^\Lang$ to denote a \NIZK which is used to prove statements on a
given NP language \Lang.

\subsubsection{Signature Proofs of Knowledge}

\todo{Introduce, and summarise correctness and security.}

\begin{description}
\item[$\SPKproof \gets \SPKProve(\SPKmsg,\SPKx,\SPKw)$.]
\item[$1/0 \gets \SPKVerify(\SPKproof,\SPKmsg,\NIZKx)$.]
\end{description}

We use $\SPK^\Lang$ to denote an \SPK which is used to prove statements on a
given NP language \Lang.

\subsection{Generic Construction \CUASGen}

\todo{Many variables need renaming here.}

\paragraph{$\parm \gets \Setup(\secpar)$.} %
\begin{itemize}
  \item Parse \secpar as $(\Csecpar,\NIZKsecpar,\RSsecpar,\Ssecpar,\Esecpar)$
  \item $\Cparm \gets \CSetup(\Csecpar)$
  \item $\NIZKparm \gets \NIZKSetup(\NIZKsecpar)$
  \item Return $(\Cparm,\NIZKsecpar,\RSsecpar,\Ssecpar,\Esecpar)$
\end{itemize}

\paragraph{$(\ipk,\isk) \gets \IKeyGen(\parm,\fissue)$.} %
\begin{itemize}
\item Parse \parm as $(\Cparm,\NIZKsecpar,\RSsecpar,\Ssecpar,\Esecpar)$
\item $(\RSvk,\RSsk) \gets \RSKeyGen(\RSsecpar)$  
\item $\sig_{\fissue} \gets \RSSign(\RSsk,\fissue,\bot)$
\item $\ipk \gets (\RSvk,\fissue,\sig_{\fissue})$
\item $\isk \gets \RSsk$
\item Return $(\ipk,\isk)$
\end{itemize}

\paragraph{$(\opk,\osk) \gets \OKeyGen(\parm,\finsp)$.} %
\begin{itemize}
\item Parse \parm as $(\Cparm,\NIZKsecpar,\RSsecpar,\Ssecpar,\Esecpar)$
\item $(\Svk,\Ssk) \gets \SKeyGen(\Ssecpar)$
\item $(\Eek,\Edk) \gets \EKeyGen(\Esecpar)$
\item $\sig_{\finsp} \gets \SSign(\Ssk,\finsp)$
\item $\opk \gets (\Svk,\Eek,\finsp,\sig_{\finsp})$
\item $\osk \gets (\Ssk,\Edk)$
\item Return $(\opk,\osk)$
\end{itemize}

\paragraph{$(\upk,\usk) \gets \UKeyGen(\parm)$.} %
\begin{itemize}
\item Parse \parm as $(\Cparm,\NIZKsecpar,\RSsecpar,\Ssecpar,\Esecpar)$
\item $(\Svk,\Ssk) \gets \SKeyGen(\Ssecpar)$
\item Return $(\Svk,\Ssk)$
\end{itemize}

\paragraph{$\langle \cred/\bot,\utrans/\bot \rangle \gets
  \langle
  \Obtain(\usk,\attrs,\ldblbrace (gpk_i,\cred_i)\rdblbrace_{i \in \Issuers}),
  \Issue(\isk,\upk,\attrs,\ldblbrace \gpk_i \rdblbrace_{i \in \Issuers})
  \rangle$.} %
We define the NP language $\LangIss = \lbrace (\usk,
\ldblbrace \cred_i \rdblbrace_{i \in \Issuers}): \fissue(\usk,\attrs,\ldblbrace
(\gpk_i,\cred_i) \rdblbrace_{i \in \Issuers}) = 1 \land \Ccom = \CCommit(\usk)
\rbrace$ \todo{\fissue needs to ensure that all {\cred}s are linked to \usk. How
  to capture this?} The interactive protocol from a user to obtain a credential
from an issuer of the system is as follows:

\begin{itemize}
\item Issuer: $\ch \gets \bin^*$; Send \ch to User.
\item User:
  \begin{itemize}    
  \item $\Ccom_{\usk} \gets \CCommit(\usk)$
  % \item $\SPKproof^{\Lang} \gets
  %   \SPK \lbrace (\usk): \Ccom_{\usk} = \CCommit(\Cparm,\usk) \rbrace (\ch)$
  \item $\SPKproof^{\LangIss} \gets \SPKProve^{\LangIss}
    (\ch,(\attrs,\Ccom_{\usk}, \ldblbrace \gpk_i \rdblbrace_{i \in \Issuers}),
    (\usk,\ldblbrace \cred_i \rdblbrace_{i \in \Issuers}))$
  \item Send $(\attrs,\Ccom_{\usk},\SPKproof^{\LangIss})$ to Issuer
  \end{itemize}
\item Issuer:
  \begin{itemize}
%  \item $1 \stackrel{?}{=} \SPKVerify(\SPKproof^{\Lang},\Ccom_{\upk})$
  \item $1 \stackrel{?}{=} \SPKVerify(\SPKproof^{\LangIss},\ch,
    (\attrs,\Ccom_{\usk}, \ldblbrace \gpk_i \rdblbrace_{i \in \Issuers}))$
  \item $\cred \gets \RSSign(\isk,\attrs,\Ccom_{\usk})$
  \item $\utrans \gets (\Ccom_{\usk},\cred,\attrs)$
  \item Send \cred to User and output \utrans.
  \end{itemize}
\item User:
  \begin{itemize}
  \item $1 \stackrel{?}{=} \RSVerify(\RSvk,\cred,(\attrs,\Ccom_{\usk}))$
  \item Output \cred
  \end{itemize}   
\end{itemize}

\paragraph{$\sig \gets \Sign(\gpk,\usk,\cred,\msg,\feval)$.} %
We define the NP language $\LangSig = \lbrace (\upk,\cred,\y):
\SVerify(\sig,\upk) = 1~\land~\Ec = \EEnc(\Eek,\y)~\land~
\y = \finsp(\feval(\upk,\cred,\msg),\upk,\cred,\msg)\rbrace$. The
signing algorithm is defined as follows:
\todo{Argue that the composition of \finsp and \feval is not redundant even
  though both depend on the same arguments. Namely, because \feval is defined
  on a per-signature basis (akin to the policies of ACs), and \finsp is defined
  once per group (akin to the open function in GSs). Hence, \finsp cannot just
  be defined to ``contain'' \feval at setup time.}

\begin{itemize}
\item Parse \gpk as $(\ipk,\opk)$
\item Parse \ipk as $(\RSvk,\fissue,\sig_{\fissue})$; Verify $\sig_{\fissue}$
\item Parse \opk as $(\Svk,\Eek,\finsp,\sig_{\finsp})$; Verify $\sig_{\finsp}$
\item $\sig' \gets \SSign(\usk,\msg)$
%\item $b \gets \feval(\upk,\cred,\msg)$
\item % If $b=0$: 
  $\y \gets \finsp(\feval(\upk,\cred,\msg),
  \upk,\cred,\msg)$%; else $\y = \bot$.
\item $\Ec \gets \EEnc(\Eek,\y)$
\item $\SPKproof^{\LangSig} \gets \SPKProve^{\LangSig}(\msg,(\Ec,\sig'),
  (\upk,\cred,\y))$
\item Return $(\Ec,\sig',\SPKproof^{\LangSig})$
\end{itemize}

\paragraph{$1/0 \gets \Verify(\gpk,\sig,\msg,\feval)$.} %
\begin{itemize}
\item Parse \sig as $(\Ec,\sig',\SPKproof^{\LangSig})$
\item Return $\SPKVerify^{\LangSig}(\SPKproof^{\LangSig},\msg,(\Ec,\sig'))$ 
\end{itemize}

\paragraph{$(\y,\iproof)/\bot \gets \Inspect(\gpk,\osk,\trans,\sig,\msg,\feval)$.} %

We follow the approach of \cite{bsz05} for verifiable openings. Namely, we
define an NP language $\LangVerIns = \lbrace (\Eek,\Edk,r):
(\Eek,\Edk) = \EKeyGen(\Esecpar;r) \land \y = \EDec(\Edk,\c) \rbrace$.

\begin{itemize}
\item Parse \osk as $(\Ssk,\Edk)$; \gpk as
  $(\cdot,(\Svk,\Eek,\finsp,\sig_{\finsp}))$
\item  $1 \stackrel{?}{=} \Verify(\gpk,\sig,\msg,\feval)$  
\item $\y \gets \EDec(\Edk,\c)$
\item $\NIZKproof^{\LangVerIns} \gets \NIZKProve(\NIZKcrs,(\c,\y),(\Eek,\Edk,r))$
\item Return $\NIZKproof^{\LangVerIns}$
\end{itemize}

\paragraph{$1/0 \gets \Judge(\gpk,\y,\iproof,\sig,\msg)$.} %

\begin{itemize}
\item  $1 \stackrel{?}{=} \Verify(\gpk,\sig,\msg,\feval)$
\item Parse \sig as $(\c,\cdot,\cdot)$  
\item Return $\NIZKVerify(\NIZKcrs,\iproof,(\c,\y))$
\end{itemize}

%%% Local Variables:
%%% mode: latex
%%% TeX-master: "uas"
%%% End:


%%% Local Variables:
%%% mode: latex
%%% TeX-master: "uas"
%%% End:

\section{Efficiency and Experimental Analysis}
\label{sec:analysis}

\subsection{Computational and Communication Costs}
\label{ssec:asymptotic-analysis}

\comment{This section is just a placeholder for now. We should have some
  asymptotic analysis or similar.}

\subsection{Benchmarks}
\label{ssec:benchmarks}

\comment{This section is just a placeholder for now. Eventually, we'd like
  to have some PoC to report.}

%%% Local Variables:
%%% mode: latex
%%% TeX-master: "uas"
%%% End:

\section{Conclusion}
\label{sec:conclusion}

In this work, we have described how a direct combination of the group signatures
(GS) and anonymous credentials (AC) research lines results in a scheme that can
bring benefit to real world use cases. Going further, we generalized such simple
combination into Universal Anonymous Signatures, or \UAS. Our model for \UAS
covers straight away a large subset of GS and AC variants, while still giving
system designers the flexibility to decide on the specific tradeoff between
privacy, utility, accountability and computational overhead that fits their
setting best. More concretely, our model for \UAS supports arbitrary issuance
policies, signature evaluation functions, and opening functions, without losing
definitional meaning. We also gave a generic construction from well known
building blocks, and prove that it meets our security model.

Even though the generalization effort we do is significant, there are still
ways to go further.
%
For instance, we do not hide the issuer(s) of the
credential(s) employed to produce a signature (resp., request a new credential).
Thus, we can only ensure privacy among signatures (resp., credential requests)
that involve credentials issued by the same issuer set. While hiding the issuers
of the involved credentials is certainly possible (we show how to do it at
signing time, in order to build ring signatures), we argue that it may be
desirable to leave it as is, at least, initially. The reason being that issuers
are still what conveys trust to a system where credentials contain attributes
attested by these issuers. Indeed, if it is already hard for verifiers to decide
whether or not to trust a non-anonymous issuer, one can imagine that it would be
even harder to decide whether or not to trust an anonymous one. This question
is core to the domain of trust registries, for instance.
%
A second desirable improvement, perhaps clearer from a technical point of view,
is to extend our model to support non-trivial utility and accountability across
multiple signatures from the same signer. More concretely, let our \feval and
\finsp functions (and perhaps also \fissue?) operate on multiple signatures by
the same \usk. That can lead to an extended \UAS scheme which would support
advanced use cases such as rate limiting (or $k$-TAA \cite{asm06}); or might
be even privately training an AI model based on one's authenticated
past activity, and prove that the model has been trained honestly.

%%% Local Variables:
%%% mode: latex
%%% TeX-master: "uas"
%%% End:


\bibliographystyle{splncs04}
\bibliography{uas}

\appendix

\section{Cryptographic Building Blocks}
\label{app:crypto-building-blocks}

\iffalse
\subsection{Digital Signatures}
\label{sapp:digital-signatures}

We rely on digital signatures as a core building block. A digital signature
provides the functionality defined by the following syntax:

\begin{description}
\item[$\Sparm \gets \SSetup(\Ssecpar)$.] It produces public parameters for the
  other algorithms, given an input security parameter \Ssecpar.
\item[$(\Svk,\Ssk) \gets \SKeyGen(\Sparm)$.] Generates a verification-signing
  key pair.
\item[$\Ssig \gets \SSign(\Ssk,\msg)$.] Signs message \msg with signing key
  \Ssk, producing signature \Ssig,  
\item[$1/0 \gets \SVerify(\Svk,\Ssig,\msg)$.] Checks whether \Ssig is a valid
  signature over \msg, under verification key \Svk.
\end{description}

A digital signature scheme is correct if honestly generated signatures, using
honestly generated key pairs, are always accepted by \SVerify. 
%
A digital signature scheme \S has existential unforgeability if, for all p.p.t.
adversaries $\adv$, $\Pr[\ExpEUF(\Ssecpar) = 1]$ is a negligible function of the
security parameter.

\begin{figure}[ht!]
  \scalebox{0.9}{  
    \begin{minipage}[t]{\textwidth}
      \centering
      \procedure{$\ExpEUF(\Ssecpar)$}{%
        \Sparm \gets \SSetup(\Ssecpar) \\
        (\Svk,\Ssk) \gets \SKeyGen(\Sparm) \\
        (\Ssig,\msg) \gets \adv^{\SSign(\Ssk,\cdot)}(\Svk) \\
        \pcif \SVerify(\Svk,\Ssig,\msg) = 0: \pcreturn 0 \\
        \pcif \msg~\textrm{was not queried to \SSign}: \pcreturn 1 \\
        \pcreturn 0
      }
    \end{minipage}
  }
  \label{fig:euf-game}
  \caption{Existential Unforgeability Game.}
\end{figure}
\fi

\subsection{Public-Key Encryption}
\label{sapp:pk-encryption}

A public-key encryption scheme is defined by the following algorithms:

\begin{description}
\item[$\Eparm \gets \ESetup(\Esecpar)$.] Produces public parameters \Eparm given
  a security parameter \Esecpar.
\item[$(\Eek,\Edk) \gets \EKeyGen(\Eparm)$.] Given public parameters \Eparm,
  produces an encryption-decryption key pair $(\Eek,\Edk)$.
\item[$\Ec \gets \EEnc(\Eek,\msg)$.] Encrypts message \msg with encryption key
  \Eek, producing ciphertext \Ec.
\item[$\msg \gets \EDec(\Edk,\Ec)$.] Decrypts ciphertext \Ec with decryption key
  \Edk.
\end{description}

A public-key encryption scheme is correct if, given a honestly generated key
pair $(\Eek,\Edk)$, produced with honestly generated parameters \Eparm,
$\Pr[\EDec(\Edk,\EEnc(\Eek,\msg))=\msg] = 1$ with overwhelming probability.

A public-key encryption scheme has IND-CCA2 security if
$\Pr[\ExpINDCCAiio(\Esecpar) = 1] - \Pr[\ExpINDCCAiiz(\Esecpar) = 1]|$ is
a negligible function of \Esecpar, for any p.p.t. adversary \adv, where
\ExpINDCCAiib is as defined in \figref{fig:indcca2-game}.

\begin{figure}[ht!]
  \scalebox{0.9}{  
    \begin{minipage}[t]{\textwidth}
      \centering      
      \procedure{$\ExpINDCCAiib(\Esecpar)$}{%
        \Eparm \gets \ESetup(\Esecpar) \\
        (\Eek,\Edk) \gets \EKeyGen(\Eparm) \\
        b^* \gets \adv^{\ELR(b,\cdot,\cdot),\EDEC(\Edk,\cdot)}(\Eek),
        ~\textrm{where:} \\
        \pcind \ELR(b,\msg_0,\msg_1)~\textrm{returns}~\EEnc(\Eek,\msg_b),
        ~\textrm{and} \\
        \pcind \EDEC(\Edk,\Ec)~\textrm{returns}~\EDec(\Edk,\Ec) \\
        \pcif \EDEC~\textrm{has been called with an output of \ELR, abort} \\
        \pcreturn b^*
      }
    \end{minipage}
  }
  \label{fig:indcca2-game}
  \caption{IND-CCA2 Game.}
\end{figure}

\subsection{Commitments}
\label{sapp:commitments}

Although we don't expose commitments directly in our \UAS scheme or
constructions, they are an essential part of \SBCM schemes. Thus, we overview
them briefly next. In a nutshell, a commitment scheme is defined by the
following algorithms:

\begin{description}
\item[$\Cparm \gets \CSetup(\Csecpar)$.] Given a security parameter \Csecpar,
  returns the public parameters \Cparm to commit messages.
\item[$\Ccom \gets \CCommit(\Cparm,\msg;r)$.] Given the public parameters and
  a message \msg, outputs a commitment \Ccom to \msg, for which randomness $r$
  from some predefined randomness space $\mathcal{R}$ is used.
\end{description}

Opening a commitment \Ccom means revealing the message \msg and randomness $r$
that were used to produce \Ccom. Commitment schemes are required to be binding
and (usually) hiding:

\begin{description}
\item[Binding.] Intuitively, the binding property of commitment schemes means
  that no adversary can change the message that has been committed to. More
  formally, $\Pr[\ExpComBind(\Csecpar) = 1]$ must be a negligible function of
  the security parameter.
\item[Hiding.] The hiding property captures that no adversary should be able to
  learn the message that was committed, when given only the commitment. This is
  formally defined through \ExpComHideb, where $|\Pr[\ExpComHideb(\Csecpar)=1|
  b=1] - \Pr[\ExpComHideb(\Csecpar)=1|b=0]|$ must be a negligible function of
  the security parameter.
\end{description}

\begin{figure}[ht!]
  \scalebox{0.9}{
    \begin{minipage}[t]{0.5\textwidth}
      \procedure{$\ExpComBind(\Csecpar)$}{%
        \Cparm \gets \CSetup(\Csecpar) \\
        (\msg_0,r_0,\msg_1,r_1) \gets \adv(\Cparm) \\
        \Ccom_0 \gets \CCommit(\Cparm,\msg_0,r_0) \\
        \Ccom_1 \gets \CCommit(\Cparm,\msg_1,r_1) \\
        \pcif \msg_0 \neq \msg_1 \land \Ccom_0 = \Ccom_1: \pcreturn 1 \\
        \pcreturn 0
      }
    \end{minipage}
    \begin{minipage}[t]{0.5\textwidth}
      \procedure{$\ExpComHideb(\secpar)$}{%
        \Cparm \gets \CSetup(\secpar) \\
        b' \gets \adv^{COM(\Cparm,\cdot,\cdot)},~\textrm{where}: \\
        \pcind COM(\Cparm,\msg_0,\msg_1): \\
        \pcind \pcind r \getr \mathcal{R};~\pcreturn \CCommit(\Cparm,\msg_b,r) \\
        % (\msg_0,\msg_1,st) \gets \adv(\Cparm) \\
        % r \getr \mathcal{R} \\
        % \Ccom \gets \CCommit(\Cparm,\msg_b,r) \\
        % b' \gets \adv(st,\Ccom) \\
        \pcreturn b'
      }
    \end{minipage}
  }
  \label{fig:com-games}
  \caption{Games for commitment schemes.}
\end{figure}

\paragraph{Commitments on Blocks of Messages.} We also use an extension
of commitment schemes that allows committing to multiple messages at once. The
properties we need are the same, and their definitions are extended in the
natural way. Namely, $\CCommit$ receives a vector/block of messages, \msgset
instead of a single message. In the games, the adversary returns sets of
messages and, in the binding game, the comparison $\msg_0 \neq \msg_1$ now
compares sets $\msgset_0$ and $\msgset_1$, which must differ in at least one
element. This extension is straight-forward, for instance, from Pedersen
commitments \cite{bcc+15}.

\subsection{Simulation-Extractable Non-Interactive Zero-Knowledge
  Proofs of Knowledge}
\label{sapp:nizk}

Let \NIZKRel be an NP relation defined by pairs of elements $(\NIZKx,\NIZKw)$,
where \NIZKx is a statement and \NIZKw a witness proving that $(\NIZKx,\NIZKw)
\in \NIZKRel$. For concrete relations, we write $\NIZKRel = \lbrace (\NIZKx),
(\NIZKw): f(x,w) \rbrace$, where $f(x,w)$ is a Boolean predicate denoting the
concrete conditions that \NIZKx and \NIZKw need to meet. The set of all \NIZKx
such that there exists a \NIZKw for which $(\NIZKx,\NIZKw) \in \NIZKRel$ is the
language, or \NIZKLang, for \NIZKRel. $\NIZKx \notin \NIZKLang$ means that
there is no $\NIZKw$ such that $(\NIZKx,\NIZKw) \in \NIZKRel$.

We use non-interactive zero-knowledge proofs of knowledge (NIZKPoK, or, for
short, NIZK) over NP relations, in the Common Reference String (CRS) model. A
NIZK system is a tuple $(\NIZKSetup,\NIZKProve,\NIZKVerify)$, defined as follows
\cite{gos06}:

\begin{description}
\item[$\NIZKcrs \gets \NIZKSetup(\NIZKsecpar)$.] Generates a CRS \NIZKcrs from
  security parameters \NIZKsecpar.
\item[$\NIZKproof \gets \NIZKProve(\NIZKcrs,\NIZKx,\NIZKw)$.] Given \NIZKcrs,
  statement \NIZKx, and witness \NIZKw, creates a proof \NIZKproof.
\item[$1/0 \gets \NIZKVerify(\NIZKcrs,\NIZKproof,\NIZKx)$.] Checks whether
  \NIZKproof is a valid proof for \NIZKx.
\end{description}

Any zero-knowledge proof of knowledge must meet completeness, soundness and
zero-knowledge,properties. We further need an extra property, called
\emph{simulation-extractability} \cite{cl06}, which amplifies the security
requirements of (simulation-) soundness.
%
To define more formally the properties we need, we have to define three extra
algorithms:

\begin{description}
\item[$(\NIZKcrs,\NIZKtrap) \gets \NIZKSimSetup(\NIZKsecpar)$.] Produces a
  \NIZKcrs as the \NIZKSetup algorithm, along with a trapdoor \NIZKtrap.
\item[$\NIZKproof \gets \NIZKSim(\NIZKcrs,\NIZKtrap,\NIZKx)$.] Given a trapdoor
  \NIZKtrap produced by \NIZKSimSetup, and a statement $\NIZKx \in \NIZKLang$,
  produces a simulated proof \NIZKproof of $\NIZKx \in \NIZKLang$.
\item[$\NIZKw \gets \NIZKExtract(\NIZKcrs,\NIZKtrap,\NIZKx,\NIZKproof)$.] Given
  a trapdoor \NIZKtrap produced by \NIZKSimSetup, and a proof \NIZKproof for
  $\NIZKx \in \NIZKLang$, returns a valid \NIZKw for \NIZKx.
\end{description}

When we want to make explicit the NP relation \NIZKRel to which the previous
algorithms refer to, we use $\NIZKSetup^\NIZKRel,\NIZKProve^\NIZKRel$, 
$\NIZKVerify^\NIZKRel$, etc., and omit the \NIZK prefix and super-index when
clear from context. Altogether, the tuple $(\Setup,\Prove,\Verify,\SimSetup,
\Sim,\Extract)$ needs to meet the following properties:

\paragraph{Completeness.} %
Ensures that, for any $(\NIZKx,\NIZKw) \in \NIZKRel$, any honest prover will be
able to create a proof \NIZKproof that is accepted by any honest verifier, with
overwhelming probability. More precisely, $\Pr\lbrack\ExpNIZKComp(\NIZKsecpar)
\rbrack = 1$ with overwhelming probability, for any p.p.t. \adv, for
\ExpNIZKComp in \figref{fig:nizk-games}.

\paragraph{Soundness.} %
Ensures that no adversary can create proofs accepted by \Verify, for
statements $\NIZKx \notin \NIZKLang$, except with negligible probability. That
is, for \ExpNIZKSound as in \figref{fig:nizk-games}, $\Pr\lbrack\ExpNIZKSound
(\NIZKsecpar)\rbrack=1$ with overwhelming probability. If this holds only
against p.p.t. adversaries, we talk of Non-Interactive \emph{arguments}, while
if soundness holds even against unbounded adversaries, we talks about
Non-Interactive proofs.

\paragraph{Zero-knowledge.} %
Intuitively, captures that no information can be learned from a statement and
proof pair, beyond their validity (or not). This is captured by requiring the
adversary to distinguish between a run in the real world ($b=0$), where the
setup is done with \Setup, and $\adv$ has access to an honest prover
\Prove; and a run in an ideal world ($b=1$), where the setup is replaced by
\SimSetup, and proofs are simulated with the help of the trapdoor produced by
\SimSetup, except when $\adv$ specifies $(\NIZKx,\NIZKw) \notin \NIZKRel$. Note
that, in the context of simulation-extractable NIZK, this property not only
requires that the simulated proofs are indistinguishable to the real ones; it
also requires that \SimSetup is indistinguishable from \Setup. All this is
formalised by requiring that $|\Pr[\ExpNIZKZKb(\NIZKsecpar) = 1 | b = 1] -
\Pr[\ExpNIZKZKb(\NIZKsecpar) = 1 | b = 0]$ be a negligible function of \secpar,
where \ExpNIZKZKb is as defined in \figref{fig:nizk-games}.

\paragraph{Simulation-Extractability.} As stated, simulation-extractability
is an extension to simulation soundness. In a nutshell,
simulation-extractability requires that, even after having received a polynomial
number of simulated proofs of knowledge, no adversary can output a valid proof
of knowledge from which no witness can be extracted. It implies simulation
soundness, which ``just'' requires that no adversary can produce a valid proof
after having seen polynomially many simulated proofs (but does not guarantee
extraction). Formally, for simulation-extractability we require that
$\Pr[\ExpNIZKSimExt(\NIZKsecpar)] = 1$ is a negligible function of \secpar,
where \ExpNIZKSimExt is defined in \figref{fig:nizk-games}.

\begin{figure}[ht!]
  \scalebox{0.9}{
    \begin{minipage}[t]{0.5\textwidth}
      \procedure{$\ExpNIZKComp(\NIZKsecpar)$}{%
        \NIZKcrs \gets \Setup(\NIZKsecpar) \\
        (\NIZKx,\NIZKw) \gets \adv(\NIZKcrs) \\
        \pcif (\NIZKx,\NIZKw) \notin \NIZKRel: \pcreturn 0 \\
        \NIZKproof \gets \Prove(\NIZKcrs,\NIZKx,\NIZKw) \\
        b \gets \Verify(\NIZKcrs,\NIZKproof,\NIZKx) \\
        \pcreturn b \\
      }
      \procedure{$\ExpNIZKSimExt(\NIZKsecpar)$}{%
        (\NIZKcrs,\NIZKtrap) \gets \SimSetup(\NIZKsecpar) \\
        (\NIZKx,\NIZKproof) \gets \adv^{\Sim'(\NIZKcrs,\NIZKtrap,\cdot,\cdot)}
        (\NIZKcrs) \\
        \pcind \textrm{Where}~\Sim'(\NIZKcrs,\NIZKtrap,\NIZKx,\NIZKw)~\textrm{returns} \\
        \pcind \pcind
        \Sim(\NIZKcrs,\NIZKtrap,\NIZKx)~\pcif (\NIZKx,\NIZKw) \in \NIZKRel \\
        \pcind \pcind \bot~\pcif (\NIZKx,\NIZKw) \notin \NIZKRel \\
        \NIZKw' \gets \Extract(\NIZKcrs,\NIZKtrap,\NIZKx,\NIZKproof) \\
        \pcif \Verify(\NIZKcrs,\NIZKproof,\NIZKx) = 1 \land
        (\NIZKx,\NIZKw') \notin \NIZKRel~\land \\
        \pcind \NIZKx~\textrm{was not queried to $\Sim$ via $\Sim'$}: \\
        \pcind \pcreturn 1 \\
        \pcreturn 0
      }    
    \end{minipage}
    \begin{minipage}[t]{0.5\textwidth}
      \procedure{$\ExpNIZKSound(\NIZKsecpar)$}{%
        \NIZKcrs \gets \Setup(\NIZKsecpar) \\
        (\NIZKproof,\NIZKx) \gets \adv(\NIZKcrs) \\
        \pcif \NIZKx \notin \NIZKLang \land
        \Verify(\NIZKcrs,\NIZKproof,\NIZKx): \\
        \pcind \pcreturn 0 \\
        \pcreturn 1 \\
      }
      
      \procedure{$\ExpNIZKZKb(\NIZKsecpar)$}{%
        \pcif b = 0: \\
        \pcind \NIZKcrs \gets \Setup(\NIZKsecpar) \\
        \pcind \pcreturn \adv^{\Prove(\NIZKcrs,\cdot,\cdot)}(\NIZKcrs) \\
        \pcif b = 1: \\
        \pcind (\NIZKcrs,\NIZKtrap) \gets \SimSetup(\secpar) \\
        \pcind \pcreturn \adv^{\Sim'(\NIZKcrs,\NIZKtrap,\cdot,\cdot)}(\NIZKcrs),~
        \textrm{where} \\
        \pcind \Sim'(\NIZKcrs,\NIZKtrap,\NIZKx,\NIZKw)~\textrm{returns} \\
        \pcind \pcind \Sim(\NIZKcrs,\NIZKtrap,\NIZKx)~\pcif (\NIZKx,\NIZKw)
        \in \NIZKRel \\
        \pcind \pcind \bot~\pcif (\NIZKx,\NIZKw) \notin \NIZKRel
      }    
    \end{minipage}
  }
  \label{fig:nizk-games}
  \caption{Games for Simulation-Extractable NIZK schemes.}
\end{figure}

As studied in \cite{cl06}, simulation-extractable NIZKPoKs formalise the concept
of ``signatures of knowledge'' (see, e.g., \cite{cs97}). Which basically means
that, given an $(\NIZKx,\NIZKw)$ pair from an NP relation, we can treat \NIZKx
as a public key, and \NIZKw as its corresponding private key, and leverage them
to build digital signature schemes -- with the advantage of being able to do so
while proving arbitrary claims, as long as they can be represented as an NP
relation. We note that, given a simulation-extractable NIZK system, it is
straightforward to build a signature of knowledge by adding the message to be
signed in the statement of the NIZK.

\iffalse
\subsection{Signatures over Blocks of Messages}
\label{sapp:sbm}

A signature scheme on blocks of messages (\SBM) allows a signer to create a
single signature over a set of messages. The resulting signature is typically
more concise than just creating multiple signatures, and typical schemes
\cite{cl02,asm06,ps16} are additionally compatible with efficient proof
protocols over the signed messages. The functionality offered by an \SBM scheme
is as follow:

\begin{description}
\item[$\SBMparm \gets \SBMSetup(\SBMsecpar)$.] It produces public parameters
  for the other algorithms, given an input security parameter \SBMsecpar.
\item[$(\SBMvk,\SBMsk) \gets \SBMKeyGen(\SBMparm)$.] Generates a
  verification-signing key pair.
\item[$\SBMsig \gets \SBMSign(\SBMsk,\smsg)$.] Produces a signature \sig, over
  a block of messages \smsg, using signing key \SBMsk.
\item[$1/0 \gets \SBMVerify(\SBMvk,\SBMsig,\widetilde{\smsg})$.] Checks
  whether \SBMsig is a valid signature over the set of messages \smsg, under
  verification key \SBMvk.
\end{description}

An \SBM scheme must satisfy correctness and unforgeability properties.

\paragraph{Correctness.} %
Informally, an \SBM scheme is correct if signatures generated between an honest
party running running \SBMSign over \smsg, for honestly generated parameters
and key pairs, results in a signature over $\smsg \cup \overline{\smsg}$ that is
accepted by \SBMVerify.

\paragraph{Unforgeability.} %
It must be unfeasible for an adversary to produce signatures over blocks of
messages that have not been signed by the signer. More formally, an \SBM scheme
is unforgeable if, for all p.p.t. adversaries $\adv$, $\Pr[\ExpSBMEUF
(\SBMsecpar) = 1]$, as defined in \figref{fig:sbm-games}, is a negligible
function of the security parameter. 

\begin{figure}[ht!]
  \scalebox{0.9}{
    \begin{minipage}[t]{\textwidth}
      \centering    
      \procedure{$\ExpSBMEUF(\SBMsecpar)$}{%
        \SBMparm \gets \SBMSetup(\SBMsecpar) \\
        (\SBMvk,\SBMsk) \gets \SBMKeyGen(\SBMparm) \\
        (\SBMsig,\smsg) \gets \adv^{\SSign(\SBMsk,\cdot)}(\SBMvk) \\
        \pcif \SBMVerify(\SBMvk,\SBMsig,\smsg) = 0: \pcreturn 0 \\
        \pcif \smsg~\textrm{was not queried to \SBMSign}: \pcreturn 1 \\
        \pcreturn 0
      }
    \end{minipage}
  }
  \label{fig:sbm-games}
  \caption{Unforgeability game for \SBM schemes.}
\end{figure}
\fi

\subsection{Signatures over Blocks of Committed Messages}
\label{sapp:sbcm}

For our generic constructions, we use interactive signing protocols between a
user and a signer, where the user has a block of messages to sign blindly, and
both receive a common block of messages to be also included in the resulting
signature. This is precisely the case of partially blind signatures, that
collapse to blind signatures when there is no common message between user
and signer; and to conventional signatures when the user does not input a
message to be blindly signed \cite{ao00}. Partially blind signatures, as
blind signatures \cite{ps96}, cannot be modelled with the conventional security
against existential forgeries. Simply because the user is actually expected to
be able to create signatures on messages unknown to the signer, which formally
translates into the impossibility to check whether a signature output by the
adversary is over a message that has been queried to the signing oracle or not.
Thus, instead of using the conventional existential unforgeability property,
(partially) blind signature schemes move to the ``one-more'' paradigm, which
demands that no adversary can produce $n+1$ distinct signatures after having
interacted with the signing oracle at most $n$ times.

To the best of our knowledge, models of existing schemes for signing blocks of
messages like \cite{cl02,asm06,ps16,cdl16b} target the case of signing blocks
of \emph{plain} messages, and are subsequently informally extended to support
signing commitments to blocks of messages via interactive protocols. However,
they do not support signing both committed and plain messages (although the
extension is trivial); and, more importantly, do not give security models of
the resulting construction, nor of course prove its security. While extending
the constructions seems straightforward, the modelling needs to be changed due
to the mentioned nuance of the conventional EUF notion not being compatible with
interactive signing protocols where the signer does not learn (some of) the
signed message(s). As we use this variant as a generic building block,
we briefly model such a scheme for Signatures over Blocks of Committed Messages
(\SBCM).

The syntax for an \SBCM scheme is as follows:

\begin{description}
\item[$\SBCMparm \gets \SBCMSetup(\SBCMsecpar)$.] It produces public parameters
  for the other algorithms, given an input security parameter \SBCMsecpar.
\item[$(\SBCMvk,\SBCMsk) \gets \SBCMKeyGen(\SBCMparm)$.] Generates a
  verification-signing key pair.
\item[$(\SBCMcom,\pi,r) \gets \SBCMBlind(\SBCMvk,\osmsg,\smsg)$.] A user
  computes commitment \SBCMcom to request a signature over messages \osmsg (in
  committed form) and \smsg (in plain form), to signer with verification key
  \SBCMvk. The output is the commitment \SBCMcom, the randomness $r$ used to
  compute it, and a proof $\pi$ proving knowledge of \osmsg and \smsg in
  \SBCMcom.
\item[$\SBCMbsig \gets \SBCMSign(\SBCMsk,\SBCMcom,\pi,\smsg)$.] The
  signer, with signing key \SBCMsk, produces a blind signature \SBCMbsig over
  the messages committed to in commitment \SBCMcom, as well as the messages in
  \smsg, with associated proof $\pi$.
\item[$\SBCMsig \gets \SBCMUnblind(\SBCMvk,\SBCMbsig,\SBCMcom,r,\osmsg,\smsg)$.]
  A user, who requested a signature over \osmsg and \smsg, where \SBCMcom is a
  commitment over \osmsg using randomness $r$, finalises the signature,
  computing \SBCMsig from the signer's partial signature \SBCMbsig.
\item[$1/0 \gets \SBCMVerify(\SBCMvk,\SBCMsig,\osmsg,\smsg)$.] Checks
  whether \SBCMsig is a valid signature over the set of messages \osmsg and
  \smsg, under verification key \SBCMvk.
\end{description}

The correctness and security properties are then defined as follows.

\paragraph{Correctness.} %
Informally, an \SBCM scheme is correct if signatures generated between an honest
party running \SBCMBlind, an honest signer running \SBCMSign fed with the output
of \SBCMBlind and matching \smsg and signing key pair, and the user finally
running \SBCMUnblind over the partial signature by the signer and leveraging
the same randomness as in \SBCMBlind, produces a signature over \osmsg and \smsg
that is accepted by \SBCMVerify. 

\paragraph{Unforgeability.} %
It must be unfeasible for an adversary to produce signatures over blocks of
messages that have not been signed (in plain, or committed shape) by the
signer. More formally, an \SBCM scheme is unforgeable if, for all p.p.t.
adversaries $\adv$, $\Pr[\ExpSBCMOMF(\SBCMsecpar) = 1]$, as defined in
\figref{fig:sbcm-games}, is a negligible function of the security parameter.
Note that this follows the ``one-more-forgery'' type of definition of blind
signatures \cite{bold02}.

\paragraph{Blindness.} %
Finally, the signer must not learn the plaintext values of the messages that are
signed in committed form. Note that this is a weaker notion than the usual
blindness property of (partially) blind signature schemes, where it is
additionally required that the adversary cannot link a signature to the signing
process that produced it. Informally, we capture this basically as the blinding
notion of a commitment scheme -- and formally define it in \ExpSBCMBlindb in
\figref{fig:sbcm-games}.
%
Note that, in the definition, we explicitly do not give back to the adversary
\adv~any full signature (i.e., after running \SBCMUnblind) obtained from values
returned by \adv, as this will allow the adversary to trivially check what
messages (among the ones he chose) was signed. While this may seem a too weak
notion, it is good enough for our needs, as in our \UAS construction we never
share actual signatures, but zero-knowledge proofs of knowledge of such
signatures.

An \SBCM scheme is blind if, for all p.p.t.
adversaries $\adv$, $|\Pr[\ExpSBCMBlindo(\SBCMsecpar) = 1] -
\Pr[\ExpSBCMBlindz(\SBCMsecpar) = 1]|$ is a negligible function of the security
parameter.

\begin{figure}[ht!]
  \scalebox{0.85}{
    \begin{minipage}[t]{0.62\textwidth}
      \centering      
      \procedure[linenumbering]{$\ExpSBCMOMF(\secpar)$}{%
        \parm \gets \Setup(\secpar) \\
        (\vk,\sk) \gets \KeyGen(\parm) \\
        \lbrace(\sig_i,\overline{\smsg}_i, \smsg_i)\rbrace_{i\in[n]} \gets
        \adv^{\Sign(\sk,\cdot,\cdot,\cdot)}(\vk) \\
        \pcif \exists i \in [n]~\st~\Verify(\vk,\sig_i,\osmsg_i,\smsg_i) = 0: \\
        \pcind \pcreturn 0 \\
        \pcif \exists i \neq j \in [n]~\st \\
        \pcind \smsg_i = \smsg_j \land \osmsg_i = \osmsg_j: \pcreturn 0 \\
        \pcif \adv~\textrm{called}~\Sign(\sk,\cdot,\cdot,\cdot)~
        \textrm{more than $n$ times}: \\
        \pcind \pcreturn 0 \\
        \pcreturn 1
      }
      % \procedure[linenumbering]{$\ExpSBCMEUF(\secpar)$}{%
      %   \parm \gets \SBCMSetup(\secpar) \\
      %   (\vk,\sk) \gets \SBCMKeyGen(\parm) \\
      %   (\sig,\overline{\smsg}, \smsg) \gets
      %   \adv^{\langle \cdot, \SBCMSign(\sk,\cdot) \rangle}(\vk) \\
      %   \pcif \SBCMVerify(\vk,\sig,\overline{\smsg},\smsg) = 0: \\
      %   \pcind \pcreturn 0 \\
      %   \overline{\smsg'} \gets \Extract(\sig,\utrans) \\
      %   \pcind \textrm{where \utrans is the signing transcript for \sig} \\
      %   \pcreturn \smsg' \neq \smsg
      % }
    \end{minipage}
      % \vspace*{0.5em}
    \begin{minipage}[t]{0.43\textwidth}
      \procedure[linenumbering]{$\ExpSBCMBlindb(\secpar)$}{%
        \parm \gets \Setup(\secpar) \\
        (\vk,\sk) \gets \KeyGen(\parm) \\
        % (\osmsg_0,\osmsg_1) \getr M \pccomment{$M \coloneqq$ message space} \\
        % (\st,\smsg) \gets \adv(\vk,\sk) \\
        % \pcfor d \in \bin: \\
        % \pcind (\com_d,\pi_d,r_d) \gets \SBCMCom(\vk,\osmsg_d,\smsg) \\
        % \pcind (\st,\bsig_d) \gets \adv(\st,\sk,\com_d,\pi_d,\smsg) \\
        % \pcind \sig_d \gets \SBCMUnblind(\vk,\bsig_d,r_d,\osmsg_d,\smsg) \\
        % b^* \gets \adv(\st,\sig_b,\sig_{1-b}) \\       
        b^* \gets \adv^{BLIND(\cdot,\cdot,\cdot)}(\vk),~\textrm{where:} \\
        \pcind BLIND(\osmsg_0,\osmsg_1,\smsg): \\
        \pcind \pcind (\com,\pi,r) \gets \Blind(\vk,\osmsg_b,\smsg); \\
        \pcind \pcind \pcreturn (\com,\pi) \\
        \pcreturn b = b^*
      }
    \end{minipage}
  }
  \label{fig:sbcm-games}
  \caption{Games for \SBCM schemes.
  }
\end{figure}

\paragraph{Augmented NIZKs.} %
Constructions of \SBCM (e.g. \cite{asm06}), as well as the formalisation we just
described, require that the user proves, in zero-knowledge (via a NIZK),
knowledge of the messages to be signed in committed
form -- and the blinding factor used to hide them. Note that, in such
constructions, it is easy to extend their NIZK so that, instead of simply
proving knowledge of the messages signed in committed form, the party running
\SBCMBlind proves, in zero knowledge, some arbitrary claim over the messages to
be signed (both those in committed or plaintext form). This can be done by
extending the initial NIZK into a more general one. 
%
It is direct that, if the base \SBCM scheme satisfies blindness, then extending
it with an arbitrary zero-knowledge NIZK proof over the signed messages, still
maintains blindness (the adversary cannot distinguish which committed message
set is signed, in the blindness game).
%
We use this extension in our \CUASGen construction. Note that, under this
extension, we may also need to provide extra information to the \SBCMBlind,
\SBCMSign and \SBCMUnblind algorithms, as part of the extended statement being
proven, and new witnesses (even though some may not be part of the messages to
be signed). To make this explicit, we use the following notation to denote
that we use a NIZK for relation $\NIZKRel$, with values $x'$ and $w'$ that are
part of the statement being proven, but are not contained within the actual
\smsg or \osmsg:

\begin{itemize}
\item $(\Ccom,\pi,r) \gets \SBCMBlind^\NIZKRel(\vk,\osmsg,\smsg;x',w')$. Like
  \SBCMBlind, but for relation \NIZKRel, which requires extra values $x'$ and
  $w'$ for its statement and witnesses, beyond what may be included in \smsg and
  \osmsg, respectively.
\item $\SBCMbsig \gets \SBCMSign^\NIZKRel(\sk,\Ccom,\pi,\smsg;x')$. Like
  \SBCMSign, but for relation \NIZKRel, which requires extra values $x'$
  for its statement, beyond what may be included in \smsg.
\item $\SBCMsig \gets \SBCMUnblind^\NIZKRel(\vk,\SBCMbsig,\Ccom,r,\osmsg,\smsg;
  x',w')$. Like \SBCMUnblind, but for relation \NIZKRel, which requires extra
  values $x'$ and $w'$ for its statement and witnesses, beyond what may be
  included in \smsg and \osmsg, respectively.
\end{itemize}

\paragraph{Proofs of knowledge of a \SBCM signature.} %
Finally, in addition, we require that the produced signatures must be compatible
with (efficient) NIZK proofs of knowledge of a signature.

\subsubsection{An Instantiation of \SBCM with BBS+}

Next, we give an instantiation of an \SBCM scheme, based on BBS+ signatures.
We emphasise again that this is essentially equivalent to the protocol for
signing committed block of messages in \cite{asm06} and, also, to the equivalent
ones in \cite{cl02,ps16} (although not for BBS+ signatures). The main difference
being that we allow merging committed blocks of messages and blocks of
(plaintext) messages into the same signature.
%
Note also that, in our instantiation, we just include a generic \NIZK, for some
relation over witness $\overline{\smsg}$ (i.e., the messages to be blindly
signed), and statement $(\Ccom,\smsg)$ (i.e., their block commitment, and the
plainly signed messages). This is intentional, as we want to support cases where
proving arbitrary predicates is possible (as opposed to ``just'' proving that
the commitment is over the messages in $\overline{\smsg}$).
%
When we want to make explicit the relation over which the employed \NIZK is
defined, we add a $\NIZKRel$ superscript to the algorithms.

\paragraph{$\SBCMparm \gets \SBCMSetup(\SBCMsecpar,\nattrs,\tnattrs)$.} %
Generates a bilinear group $\BB = (p,\GG_1,\GG_2,\GG_T,\gen{g}_1,\gen{g}_2,e)
\gets \PGen(\SBCMsecpar)$, and $\nattrs+\tnattrs+1$ additional generators
$\gen{g}$, $\gen{h}_1,...,\gen{h}_{\nattrs}$,$\gen{\th}_1,...,\gen{\th}_{\tnattrs}$
of $\GG_1$. Returns $\SBCMparm \gets (\SBCMsecpar,\nattrs,\tnattrs,\BB,
\gen{g},\gen{h}_1,...,\gen{h}_{\nattrs},\gen{\th}_1,...,\gen{\th}_{\tnattrs})$.
We assume that \SBCMparm is available to all other algorithms, even when not
explicitly passed as an argument.

\paragraph{$(\SBCMvk,\SBCMsk) \gets \SBCMKeyGen(\SBCMparm)$.} %
Parses \SBCMparm as $(\cdot,\cdot,(p,\GG_1,\GG_2,\GG_T,\gen{g}_1,\gen{g}_2,e),
\dots$ $\NIZKcrs \gets \NIZKSetup(\secpar)$. Outputs $\SBCMsk \gets \ZZ^*_p$,
and $\SBCMvk \gets (\NIZKcrs,\gen{g}_2^{\SBCMsk})$.

\paragraph{$(\Ccom,\pi,r) \gets \SBCMBlind(\SBCMvk,\osmsg,\smsg)$.} %
If $|\smsg |>\nattrs$ or $|\osmsg|>\tnattrs$,
abort. Else, fetch fresh randomness $r \getr \ZZ^*_p$, compute $\Ccom \gets
\gen{g}^r\prod_{i \in [|\overline{\smsg}|]}\gen{\th}_i^{\overline{\smsg}_i}$,
and $\NIZKproof \gets \NIZKProve(\NIZKcrs,(r,\overline{\smsg}),\Ccom)$.
Output $(\Ccom,\NIZKproof,r)$.

\paragraph{$\SBCMbsig \gets \SBCMSign(\SBCMsk,\Ccom,\NIZKproof,\smsg)$.} %
If $|\smsg|>\nattrs$, abort. Else, run $\NIZKVerify(\NIZKcrs,\Ccom,\NIZKproof)$
and return $0$ if it fails. Else, compute $x,\tilde{s} \getr \ZZ^*_p, A \gets
(\gen{g}_1\Ccom \gen{g}^{\tilde{s}} \prod_{i \in |\smsg|}
\gen{h}_i^{\smsg_i})^{1/(\SBCMsk+x)}$. Return $\SBCMbsig \gets (A,x,\tilde{s})$.

\paragraph{$\SBCMsig \gets \SBCMUnblind(\SBCMvk,\SBCMbsig,\Ccom,r,
  \osmsg,\smsg)$.} %
Parse \SBCMbsig as $(A,x,\tilde{s})$
If $A = 1_{\GG_1}$: return $0$. Else, compute $s \gets r + \tilde{s}$. If
$e(A,\gen{g}_2)^xe(A,\SBCMvk) \neq e(\gen{g}_1\Ccom\gen{g}^{\tilde{s}}
\prod_{i \in |\smsg|}\gen{h}_i^{\smsg_i},\gen{g}_2)$: return $0$. Else, return
$(A,x,s)$.

% \paragraph{$\SBCMsig/\bot \gets \langle \SBCMCom(\SBCMvk,\overline{\smsg},
%   \smsg), \SBCMSign(\SBCMsk,\smsg) \rangle$.} %

% \begin{itemize}
% \item \underline{User}: If $|\smsg|>\nattrs$ or $|\overline{\smsg}|>\tnattrs$,
%   abort. Else, fetch fresh randomness $r \getr \ZZ^*_p$, compute $\Ccom \gets
%   \gen{g}^r\prod_{i \in [|\overline{\smsg}|]}\gen{\th}_i^{\overline{\smsg}_i}$,
%   and $\NIZKproof \gets \NIZKProve(\NIZKcrs,(r,\overline{\smsg}),\Ccom)$.
%   Send $(\Ccom,\NIZKproof)$ to Issuer.
% \item \underline{Signer}: If $|\smsg|>\nattrs$, abort. Else, run $\NIZKVerify
%   (\NIZKcrs,\Ccom,\NIZKproof)$ and return $0$ if it fails. Else, compute
%   $x,\tilde{s} \getr \ZZ^*_p, A \gets (\gen{g}_1\Ccom \gen{g}^{\tilde{s}}
%   \prod_{i \in |\smsg|}\gen{h}_i^{\smsg_i})^{1/(\SBCMsk+x)}$. Send
%   $(A,x,\tilde{s})$ to User.
% \item \underline{User}: If $A = 1_{\GG_1}$: return $0$. Else, compute
%   $s \gets r + \tilde{s}$. If $e(A,\gen{g}_2)^xe(A,\SBCMvk) \neq
%   e(\gen{g}_1\Ccom\gen{g}^{\tilde{s}}\prod_{i \in |\smsg|}\gen{h}_i^{\smsg_i},
%   \gen{g}_2)$: return $0$. Else, return $(A,x,s)$.
% \end{itemize}

\paragraph{$1/0 \gets \SBCMVerify(\SBCMvk,\SBCMsig,\overline{\smsg},\smsg)$.} %
To verify a signature \SBCMsig, for message set $\overline{\smsg}$ that was
signed as a block commitment, and message set \smsg, signed as plaintext, parse
\SBCMsig as $(A,x,s)$ and check if $e(A,\gen{g}_2^x\SBCMvk) =
e(\gen{g}_1\gen{g}^s\prod_{i \in |\overline{\smsg}|}\gen{h}^{\overline{\smsg}_i}
\prod_{i \in |\smsg|}\gen{h}^{\smsg_i},\gen{g}_2)$

\paragraph{Augmented NIZKs.} %
If needed, it is straightforward to augment the proof \NIZKproof used
in \SBCMBlind, \SBCMSign and \SBCMUnblind  with an extended proof that also
proves that $(x,w) \in \NIZKRel$, for some other relation \NIZKRel (where
$\osmsg$ is part of $w$), thus implementing the augmented interface described
earlier for \SBCM schemes.

\paragraph{Proving Knowledge of Signature.} %
Proving knowledge of a BBS+ signature as produced in our \SBCM variant
is essentially the same as in \cite{asm06,cdl16b}, only needing to account for
the different basis for messages signed in committed and plain form.

\paragraph{Correctness.} Correctness is direct from correctness of the
commitment scheme, the NIZK system, and BBS+.

\paragraph{OMF security.} The proof for OMF security is essentially the
same as that of EUF security for BBS+ signatures \cite{cdl16b}, which gives a
reduction against the $q$-SDH problem. There is only one exception: in the proof
for BBS+ EUF-security, when the adversary against EUF-security queries a
signature, the simulator (the adversary against $q$-SDH) has to simulate the
signature. For this, it requires the $l$ messages in the message set to be
signed, which in plain BBS+ it is not a problem, as they are received in the
clear. In the case of \SBCM, however, some of the messages to be signed are
received as part of a commitment. Note however that, in \SBCM, besides receiving
the commitment, we also receive a NIZK proof of knowledge of the corresponding
messages. Thus, the adversary against $q$-SDH in the OMF case only needs to
extract the witnesses from the proof. This is doable after soundness of the NIZK
and binding of the commitment scheme. %
If the NIZK also has online-extractability, then we do not need to impose
further requirements. If it is not, then the \SBCM scheme should not allow
parallel signing requests, or limit the total number of signatures to be
created to a logarithmic function of the security parameter (which is probably
not reasonable in most cases; although it may be for group signature-like
settings). We refer to \cite[Lemma 1]{cdl16b} for the full details of the proof
of EUF security in BBS+.
%
% \jesus{If needed, do the actual proof. Should essentially be a copy-paste of
%   that in \cite{cdl16b}...}

\paragraph{Blindness.}
Blindness of \SBCM is a direct consequence of the hiding property of the
underlying block commitment scheme. Indeed, assume an adversary $\adv$ against
blindness of \SBCM. Then, $\adv$ can be used by \advB against the hiding
property of the commitment scheme in a straightforward manner. Namely, \advB
sets up the environment for \adv by running the \KeyGen algorithm using the
\vk it receives. Then, for every query that \adv makes to its $BLIND$ oracle,
\advB checks that the length of \smsg is acceptable (i.e., less than the
maximum length allowed by \SBCM), and forwards the query to its own $COM$
oracle using $\osmsg_0$ and $\osmsg_1$. To the output \Ccom of $COM$, \advB
appends a simulated proof $\pi$, and forwards $(\Ccom,\pi)$ to \adv. Finally,
\advB outputs whatever \adv~outputs. The simulation is correct due to the
zero-knowledge property of the underlying NIZK. Thus, \advB wins whenever
\adv~does.

%%% Local Variables:
%%% mode: latex
%%% TeX-master: "uas-paper"
%%% End:

\section{GSAC Detailed Formalisation}
\label{app:gsac-formal}

\subsection{Detailed Oracles}
\label{sapp:gsac-oracles}

{%\setlength\intextsep{\sep}
  \begin{figure*}[ht!]
    \centering
    \scalebox{0.9}{

      \begin{minipage}[t]{0.55\textwidth}

        \procedure{$\RREG(i)$}{%
          \pcreturn \trans[i] \\
        }

        \procedure{$\HUGEN(\uid)$}{%
          \pcif \uid \in \HU \lor \uid \in \CU: \pcreturn \bot \\
          (\upk,\usk) \gets \UKeyGen(\parm) \\
          \UK[\uid] \gets (\upk,\usk);
          \HU \gets \HU \cup \lbrace  \uid \rbrace \\
          \pcreturn \top \\
        }        
        
        \procedure{$\CUGEN(\uid,\upk)$}{%
          \pcif \uid \in \CU: \pcreturn \bot \\
          \CU \gets \CU \cup \lbrace \uid \rbrace \\          
          \pcif \uid \in \HU: \\
          \pcind \HU \gets \HU \setminus \lbrace \uid \rbrace; \\
          \pcind \pcreturn (\UK[\uid],\CRED[\uid]) \\
          \pcelse: \UK[\uid] = (\upk,\bot) \\          
          \pcreturn \top \\
        }

        \procedure{$\OBTISS(\uid,\cid,\Attrs)$}{%
          \pcif \uid \in \CU \lor \uid \notin \HU: \pcreturn \bot \\
          \pcif \CRED[\cid] \neq \bot: \pcreturn \bot \\
          \langle \cred, \utrans \rangle \gets
          \langle \Obtain(\gpk,\PRVUK[\uid],\Attrs), \\
          \pcind \pcind \pcind \pcind \pcind \pcind
          \Issue(\gpk,\isk,\Attrs) \rangle \\
          \trans[\cid] \gets \utrans;~\CRED[\cid] \gets \cred \\
          \OWNR[\cid] \gets \uid;~\ATTR[\cid] \gets \Attrs \\
          \pcreturn \top \\
        }        

        \procedure{$\OBTAIN(\uid,\cid,\Attrs)$}{%
          \pcif \uid \in \CU \lor \uid \notin \HU: \pcreturn \bot \\
          \pcif \CRED[\cid] \neq \bot: \pcreturn \bot \\
          \langle \cred, \cdot \rangle \gets
          \langle \Obtain(\gpk,\PRVUK[\uid],\Attrs),\adv \rangle \\
          \CRED[\cid] \gets \cred \\
          \OWNR[\cid] \gets \uid;~\ATTR[\cid] \gets \Attrs \\
          \pcreturn \top \\
        }
        
      \end{minipage}
    }
    \scalebox{0.9}{
      
      \begin{minipage}[t]{.5\textwidth}

        \procedure{$\WREG(i,\rho)$}{%
          \trans[i] \gets \rho \\
        }        

        \procedure{$\ISSUE(\uid,\cid,\Attrs)$}{%
          \pcif \uid \notin \CU: \pcreturn \bot \\
          \pcif \CRED[\cid] \neq \bot: \pcreturn \bot \\
          \langle \cdot, \utrans \rangle \gets
          \langle \adv, \Issue(\gpk,\isk,\Attrs) \rangle \\
          \trans[\cid] \gets \utrans \\
          \OWNR[\cid] \gets \uid;~\ATTR[\cid] \gets \Attrs \\
          \pcreturn \top \\          
        }        

        \procedure{$\SIGN(\cid,\DAttrs,\msg)$}{%
          \uid \gets \OWNR[\cid] \\
          \pcif \uid \notin \HU: \pcreturn \bot \\
          \cred \gets \CRED[\cid] \\
          \sig \gets \Sign(\gpk,\PRVUK[\uid],\cred,\DAttrs,\msg) \\
          \SIG[\cid] \gets \SIG[\cid] \cup \lbrace (\sig,\DAttrs,\msg) \rbrace \\
          \pcreturn \sig \\
        }

        \procedure{$\OPEN(\sig)$}{%
          \textrm{Let}~\cid~\textrm{be s.t.}~(\sig,\DAttrs,\msg) \in \SIG[\cid] \\
          \pcif \textrm{no such \cid exists, or there is more than one}: \\
          \pcind \pcreturn \bot \\
          \pcif \sig \in \CSIG: \pcreturn \bot \\
          (\upk,\oproof) \gets \Open(\gpk,\osk,\sig,\DAttrs,\msg) \\
%          \OSIG \gets \OSIG \cup \lbrace (\sig,\upk,\cred) \rbrace \\
%          \CCRED \gets \CCRED \cup \lbrace \cid \rbrace \\
          \pcreturn (\upk,\oproof) \\
        }

        \procedure{$\CHALb(\ccid_0,\ccid_1,\DAttrs,\msg)$}{%
          \pcif \DAttrs \nsubseteq \ATTR[\ccid_0] \cap \ATTR[\ccid_1]:
          \pcreturn \bot \\
          \pcif \cuid_0 \neq \bot \lor \cuid_1 \neq \bot: \pcreturn \bot \\
          \cuid_0 = \OWNR[\ccid_0];~\cuid_1 = \OWNR[\ccid_1] \\
          \pcif \cuid_0 \notin \HU \lor \cuid_1 \notin \HU: \pcreturn \bot \\
 %         \pcif \ccid_0 \in \CCRED \lor \ccid_1 \in \CCRED: \pcreturn \bot \\
          \csig \gets \Sign(\gpk,\PRVUK[\cuid_b],\CRED[\ccid_b],
          \DAttrs,\msg) \\
          \CSIG \gets \CSIG \cup \lbrace \csig \rbrace \\
          \pcreturn \csig
        }
        
      \end{minipage}
      
    }

    \caption{Detailed oracles available in our model for \GSAC schemes.}
    \label{fig:oracles}
  \end{figure*}
}

\subsection{Detailed Proofs}
\label{sapp:gsac-proofs}

\subsection{Concrete Instantiation with BBS+}
\label{sapp:gsac-instantiation}

Intuitively, the credentials that we generate for our \GSAC construction are
Pedersen commitments to $(\usk,\credid,\Attrs)$ tuples; i.e., they have
the following structure: $\gen{h}_0^r\gen{h}_1^{\usk}\prod_{i \in [2,\nattrs+1]}
\gen{h}_i^{\Attrs_i}$, where the exponent in $\gen{h_0}$ is a fresh random value
that ensures hiding, the exponent in $\gen{h_1}$ encodes the user private
key, and the remaining attributes are encoded in different
exponentiations. This will actually be part of a BBS+ signature, which makes it
easy to add subsequent and efficient zero-knowledge proofs. Concretely, we will
be using several \NIZK proof systems, one for the $\langle \Obtain,\Issue
\rangle$ interactive protocol, another for signing, and a third one for opening.
Each will have its own relation, that we define in the corresponding algorithm.
The concrete algorithms are as follows.

\paragraph{$\Setup(\secpar,\nattrs) \rightarrow \parm$.} %
Generates a bilinear group $\BB = (p,\GG_1,\GG_2,\GG_T,\gen{g}_1,\gen{g}_2,e) \gets
\PGen(\secpar)$, $\nattrs+4$ additional generators $\gen{g},\gen{h},\gen{h}_0,...,
\gen{h}_{\nattrs+1}$ of $\GG_1$. Returns $\parm \gets
(\secpar,\nattrs,\BB,\gen{g},\gen{h},\gen{h}_0,...,\gen{h}_{\nattrs+1})$.

\paragraph{$\IKeyGen(\parm) \rightarrow (\ipk,\isk)$.} %
Parses \parm as $(\secpar,\dots)$ and runs $\NIZKcrs \gets \NIZKSetup(\secpar)$.
Outputs $\isk \gets \ZZ^*_p$, and $\ipk \gets (\NIZKcrs,\gen{g}_2^{\isk})$.

\paragraph{$\OKeyGen(\parm) \rightarrow (\opk,\osk)$.} %
Outputs $\osk \gets \ZZ^*_p$, and $\opk \gets \gen{g}^{\osk}$. This is the
opener's ElGamal encryption key pair \needcite.

\paragraph{$\UKeyGen(\parm) \rightarrow (\upk,\usk)$.} %
Outputs $\usk \gets \ZZ^*_p$, and $\upk \gets \gen{h}_1^{\usk}$. \upk will
simply be used to compute Pedersen commitments \needcite to \usk. However,
it is useful to precompute it, and we treat it as a sort of ``public key''.

\paragraph{$\langle \Obtain(\gpk,\usk,\Attrs),\Issue(\gpk,\isk,\Attrs) \rangle
  \rightarrow \langle \cred/\bot,\utrans/\bot \rangle$.} %
This interactive protocol is essentially a BBS+ signing  process. The user
commits to its private key, proves knowledge of the corresponding private
key, and asks the issuer to sign the commitment, along with any other arbitraty
set of revealed attributes \Attrs. For proving knowledge of the private key, we
define relation $\NIZKRel_{\Issue} = \lbrace (r, \usk), \Ccom: \Ccom =
\gen{h}_0^r\gen{h}_1^{\usk}\rbrace$.

\begin{itemize}
\item \underline{User}: Fetch fresh randomness $r \getr \ZZ^*_p,
  \Ccom \gets \gen{h}_0^r\gen{h}_1^{\usk}$, and compute $\NIZKproof \gets
  \NIZKProve^{\NIZKRel_{\Issue}}(\NIZKcrs,(r,\usk),
  \Ccom)$. Send $(\Ccom,\NIZKproof)$ to Issuer.
\item \underline{Issuer}: Run $\NIZKVerify^{\NIZKRel_{\Issue}}(\NIZKcrs,\Ccom,
  \NIZKproof)$ and return $0$ if it fails. Else, compute
  $x,\tilde{s} \getr \ZZ^*_p, A \gets
  (\gen{g}_1\Ccom h_0^{\tilde{s}} \prod_{i \in \Attrs}
  \gen{h}_i^{\Attrs_i})^{1/(\isk+x)}$.
  Send $(A,x,\tilde{s})$ to User, and output $\utrans \gets
  (\Ccom,(A,x,\tilde{s}),\Attrs,\NIZKproof)$.
\item \underline{User}: If $A = 1_{\GG_1}$: return $0$. Else, compute
  $s \gets r + \tilde{s}$. If $e(A,\gen{g}_2)^xe(A,\ipk) \neq
  e(\gen{g}_1\Ccom\gen{h}_0^{\tilde{s}}\prod_{i \in \Attrs}\gen{h}_i^{\Attrs_i},
  \gen{g}_2)$: return $0$. Else, return
  $(A,x,s)$.
\end{itemize}

\paragraph{$\Sign(\gpk,\usk,\cred,\DAttrs,\msg) \rightarrow \sig$.} %
To create a signature, the user first randomizes its BBS+ credential \cred and
encrypts its public key \upk (note that $\upk=\gen{h}_1^{\usk}$ with the
opener's encryption key. Then, we extend the usual NIZK protocol for proving
knowledge of BBS+ signatures \cite{cdl16b} to, in addition to proving knowledge
of a signature (which, for us is the user credential), also prove that the
attribute encoding \usk within the credential match the encrypted value.
Finally, the user commits to \msg, and includes this commitment within the
proof. For this, we define relation
$\NIZKRel_{\Sign}= \lbrace (r,\usk,\Attrs,r_2,r_3,s',\msg),(\DAttrs,c_1,c_2,
\Cmsg): \Cmsg = \gen{h}^\msg \land c_1 = g^r \land c_2 = \opk^r\gen{h}_1^{\usk}
\land \hat{A}/d = (A')^{-x}\gen{h}_0^{r_2} \land
\gen{g}_1 \prod_{i \in \DAttrs} \gen{h}_i^{\Attrs_i} =
d^{r_3}\gen{h}_0^{-s'}\gen{h}_1^{-\usk}
\prod_{i \notin \DAttrs} \gen{h}_i^{-\Attrs_i} \rbrace$.

\begin{itemize}
\item Parse \gpk as $(\ipk,\opk)$, and \cred as $(A,x,s)$.
\item Re-randomize \cred as $r_1,r_2 \getr
  \ZZ^*_p, r_3 \gets r_1^{-1}, s' \gets s - r_2r_3$, $A' \gets A^{r_1},
  \hat{A} \gets (A')^{-x}(\gen{g}_1\gen{h}_0^s\gen{h}_1^{\usk}
  \prod_{i \in \Attrs}\gen{h}_i^{\Attrs_i})^{r_1}$,
  $d \gets (\gen{g}_1\gen{h}_0^s\gen{h}_1^{\usk}
  \prod_{i \in \Attrs}\gen{h}_i^{\Attrs_i})^{r_1}\gen{h}_0^{-r_2}$.
\item Encrypt $\upk=\gen{h}_1^{\usk}$ with ElGamal as $r \getr \ZZ^*_p,
  c \gets (c_1 = \gen{g}^r,c_2 = \opk^r\gen{h}_1^{\usk})$.
\item Compute $\Cmsg \gets \gen{h}^\msg$ and
  $\NIZKproof \gets \NIZKProve^{\NIZKRel_{\Sign}}(\NIZKcrs,
  (r,\usk,\Attrs,r_2,r_3,s',\msg), (\DAttrs,c_1,c_2,\Cmsg))$.
\item Return $\sig \gets (c=(c_1,c_2),(A',\hat{A},d),\NIZKproof)$.
\end{itemize}

\paragraph{$\Verify(\gpk,\sig,\DAttrs,\msg) \rightarrow 1/0$.} %
Parse \gpk as $(\ipk,\opk)$ and $\sig$ as $(c=(c_1,c_2),
(A',\hat{A},d), \NIZKproof)$. Check that $A' \neq 1_{\GG_1}$ and $e(A',\ipk) =
e(\hat{A},\gen{g}_2)$. Compute $\Cmsg \gets \gen{h}^\msg,$ and return
$\NIZKVerify^{\NIZKRel_{\Sign}}(\NIZKcrs,(\DAttrs,c_1,c_2,\Cmsg),
\NIZKproof)$.

\paragraph{$\Open(\gpk,\osk,\sig,\DAttrs,\msg)
  \rightarrow (\upk,\oproof)/\bot$.} %
For open algorithms, we define the following relation for correct decryption of
ElGamal ciphertexts: $\NIZKRel_{\Open} = \lbrace \osk,(c_1,c_2,\msg):
c_2/c_1^{\osk} = \msg \rbrace$. Given $\NIZKRel_{\Open}$, the opener first,
checks that $\Verify$ accepts the signature, and aborts otherwise. If \sig
is accepted, then parses \sig as $(c=(c_1,c_2),\cdot,\cdot)$.
Sets $\upk \gets c_2/c_1^{\osk}$. Finally, computes the proof of correct
decryption by running $\oproof \gets \NIZKProve^{\NIZKRel_{\Open}}(\NIZKcrs,
\osk,(\upk,c_1,c_2))$, and returns $(\upk,\oproof)$.

\paragraph{$\Judge(\gpk,\upk,\oproof,\sig,\DAttrs,\msg)
  \rightarrow 1/0$.} %
First, check that $\Verify$ accepts \sig, and abort otherwise. If the signature
is accepted, then parse \sig as $(c=(c_1,c_2),\cdot,\cdot)$ and return
$\NIZKVerify^{\NIZKRel_{\Open}}(\NIZKcrs,(c_1,c_2,\upk),\oproof)$.

% \commentwho{Jesus}{For consistency with \UAS, maybe remove the need to return
%   \Attrs in \Open, and return \DAttrs instead. Then, we do not need to pass
%   \trans as a parameter to \Open, as \DAttrs is already attested for in the
%   signature being opened. It also makes sense from the point of view that we
%   are already revealing \upk, which would allow tracing; but otherwise don't
%   reveal anything else about the attributes of \upk -- attributes that whoever
%   requested the signature (the verifier) didn't seem to consider necessary, as
%   s/he only requested \DAttrs. Still, if we make this modification, mention the
%   possibility to return \Attrs, and the option to do it via adding \trans as a
%   parameter to \Open (or including an encryption of all attributes in the
%   signature, which is probably unrealistic.)}

%%% Local Variables:
%%% mode: latex
%%% TeX-master: "uas"
%%% End:


\end{document}

%%% Local Variables:
%%% mode: latex
%%% TeX-master: "uas"
%%% End:
