\documentclass{llncs}%[10pt]{llncs}
\pagestyle{plain}

\usepackage[n,advantage,operators,sets,adversary,landau,probability,notions,
logic,ff,mm,primitives,events,complexity,asymptotics,keys]{cryptocode}
\usepackage{amssymb}
\usepackage{xspace}
\usepackage[normalem]{ulem}
\usepackage{hyperref}
\usepackage{mathtools}
\usepackage{multirow}
\usepackage{footnote}
\makesavenoteenv{tabular}
\makesavenoteenv{table}

%% Set/unset to activate/deactivate author comments
\def\authnotes{1}

%%% Allows line-breaks in math formulae after commas
\AtBeginDocument{%
  \mathchardef\mathcomma\mathcode`\,
  \mathcode`\,="8000 
}
{\catcode`,=\active
  \gdef,{\mathcomma\discretionary{}{}{}}
}

%%%

\setcounter{tocdepth}{2}

\title{Universal Anonymous Signatures}
\author{Jesus Diaz\inst{1} \and Markulf Kohlweiss\inst{2}}

\institute{Input Output Global,\\
\email{jesus.diazvico@iohk.io},\\
\and
Input Output Global and University of Edinburgh,\\
\email{markulf.kohlweiss@ed.ac.uk}}

% Custom shorthands for this paper
\definecolor{darkgreen}{rgb}{0.1,0.7,0.1}
\newcommand{\todo}[1]{{\colorbox{red}{\bf TODO:}\textcolor{red}{#1}}}
\newcommand{\doubt}[1]{{\colorbox{blue}
    {\textcolor{white}{\bf DOUBT:}}\textcolor{blue}{#1}}}
\newcommand{\response}[1]{{\colorbox{orange}{\bf RESPONSE:}
                                             \textcolor{orange}{#1}}}
\newcommand{\modified}[1]{{\textcolor{orange}{#1}}}
\newcommand{\commentwho}[2]{{\colorbox{darkgreen}{{\bf #1:}}
    \textcolor{darkgreen}{#2}}}
\newcommand{\comment}[1]{{\textcolor{orange}{\# #1}}}
%\newcommand{\note}[1]{{\colorbox{gray}{\bf NOTE:}\textcolor{gray} {#1}}}
\newcommand{\needcite}{[\colorbox{red}{?}]}
\newcommand{\figref}[1]{Fig. \ref{#1}}
\newcommand{\tabref}[1]{Table \ref{#1}}
\newcommand{\lstref}[1]{Listing \ref{#1}}
\newcommand{\secref}[1]{Section \ref{#1}}
\newcommand{\appref}[1]{Appendix \ref{#1}}
\newcommand{\defref}[1]{Definition \ref{#1}}

%\renewcommand{\qedsymbol}{$\blacksquare$}
\newcommand{\br}[1]{\ensuremath{\lbrack #1 \rbrack}}
\newcommand{\gen}[1]{\ensuremath{#1}}

\newcommand{\setind}[2]{\ensuremath{\set{#1}_{#2}}}

\mathchardef\mhyphen="2D
\def\suchthat{\ensuremath{s.t.}\xspace}

% General
\def\secpar{\ensuremath{1^\kappa}\xspace}
\def\Gen{\ensuremath{Gen}\xspace}
\def\Hash{\ensuremath{Hash}\xspace}
\def\tx{\ensuremath{tx}\xspace}
\def\getr{\ensuremath{\stackrel{\$}{\gets}}\xspace}
\def\arg{\ensuremath{arg}\xspace}
\def\msg{\ensuremath{m}\xspace}

% UC
\def\IdealF{\ensuremath{\mathcal{F}}\xspace}
\def\IdealG{\ensuremath{\mathcal{G}}\xspace}
\newcommand{\ucio}[1]{\ensuremath{\text{#1}}}
\newcommand{\uccmd}[1]{\ensuremath{\mathtt{#1}}}
\def\cmd{\ensuremath{cmd}\xspace}
\def\sid{\ensuremath{sid}\xspace}
\def\ssid{\ensuremath{ssid}\xspace}
\def\ucsend{\ucio{Send}\xspace}
\def\ucrecv{\ucio{Recv}\xspace}
\def\Sim{\ensuremath{\mathcal{S}}\xspace}
\def\Env{\ensuremath{\mathcal{E}}\xspace}
\def\Exec{\ensuremath{\text{EXEC}}}

% Signatures
\def\sig{\ensuremath{\sigma}\xspace}
\def\Sig{\ensuremath{\mathsf{S}}\xspace}
\def\IdealFSig{\ensuremath{\IdealF_{SIG}}\xspace}
\def\SIG{\ensuremath{\mathsf{SG}}\xspace}

% Hashes
\def\IdealGRO{\ensuremath{\IdealG_{RO}}\xspace}
\def\hashlist{\ensuremath{HL}\xspace}

% Key Types
\def\mpk{\ensuremath{mpk}\xspace}
\def\msk{\ensuremath{msk}\xspace}
\def\ipk{\ensuremath{ipk}\xspace}
\def\isk{\ensuremath{isk}\xspace}
\def\apk{\ensuremath{apk}\xspace}
\def\ask{\ensuremath{ask}\xspace}
\def\VK{\ensuremath{\mathsf{VK}}\xspace}

% Clock
\def\IdealGclock{\ensuremath{\IdealG_{clock}}\xspace}

% BB
\def\ldgLedger{\ensuremath{L}\xspace}
\def\IdealGledger{\ensuremath{\IdealG_{Ledger}}\xspace}
\def\IdealGdledger{\ensuremath{\IdealGledger^\delta}\xspace}
\def\ldgHonest{\ensuremath{H}\xspace}
\def\ldgMap{\ensuremath{M}\xspace}
\def\ldgState{\ensuremath{\Sigma}\xspace}
\def\ldgUtxo{\ensuremath{U}\xspace}
\def\ldgTx{\ensuremath{tx}\xspace}
\def\ldgBlock{\ensuremath{B}\xspace}
\def\ldgHead{\ensuremath{Head}\xspace}

% VDR
\def\VDR{\ensuremath{\mathsf{VDR}}\xspace}

% DIDs
\def\IdealGPKIDID{\ensuremath{\IdealG^{\P,\delta}_{\mathsf{PKIdvu}}}\xspace}
\def\did{\ensuremath{did}\xspace}
\def\DID{\ensuremath{\mathsf{DID}}\xspace}
\def\DIDCreate{\ensuremath{\DID.Create}\xspace}
\def\DIDRead{\ensuremath{\DID.Read}\xspace}
\def\DIDUpdate{\ensuremath{\DID.Update}\xspace}
\def\DIDReadLdg{\ensuremath{\mathsf{ParseLedger}}\xspace}
\def\didOp{\ensuremath{op}\xspace}
\def\didOP{\ensuremath{\mathsf{OP}}\xspace}
\def\didOWN{\ensuremath{\mathsf{OWN}}\xspace}
\def\typ{\ensuremath{t}\xspace}
\def\lbl{\ensuremath{l}\xspace}
\def\LBL{\ensuremath{L}\xspace}
\def\val{\ensuremath{v}\xspace}
\def\sval{\ensuremath{\mathbf{v}}\xspace}
\def\VAL{\ensuremath{V}\xspace}
\def\P{\ensuremath{\mathtt{P}}\xspace}
% \def\Pc{\ensuremath{\mathtt{P}_{\mathtt{C}}}\xspace}
% \def\Pu{\ensuremath{\mathtt{P}_{\mathtt{U}}}\xspace}
% \def\Pd{\ensuremath{\mathtt{P}_{\mathtt{D}}}\xspace}

% VCs
\def\VC{\ensuremath{\mathsf{VC}}\xspace}
\def\VCCreate{\ensuremath{\VC.Create}\xspace}
\def\VCPublish{\ensuremath{\VC.Publish}\xspace}
\def\VCRevoke{\ensuremath{\VC.Revoke}\xspace}
\def\VCShow{\ensuremath{\VC.Show}\xspace}
\def\VCVerify{\ensuremath{\VC.Verify}\xspace}

% Misc
\def\IdealFCA{\ensuremath{\IdealF_{CA}}\xspace}

% Atala
\def\RealPKIDIDAtala{\ensuremath{\Pi_{DID}^{Atala}}\xspace}
\def\phiDID{\ensuremath{\phi_{DID}}\xspace}
\def\MasterKey{\texttt{M}\xspace}
\def\AuthKey{\texttt{A}\xspace}
\def\CommKey{\texttt{C}\xspace}
\def\IssueKey{\texttt{I}\xspace}
\def\LMKL{\texttt{LMKL}\xspace}

%%% Local Variables: 
%%% mode: pdflatex
%%% TeX-master: "prism-protocol"
%%% End:

% Symbols for GSAC
\def\GSAC{\ensuremath{\mathsf{GSAC}}\xspace}
\def\WREG{\ensuremath{\mathsf{WREG}}\xspace}
\def\RREG{\ensuremath{\mathsf{RREG}}\xspace}
\def\HUGEN{\ensuremath{\mathsf{HUGEN}}\xspace}
\def\CUGEN{\ensuremath{\mathsf{CUGEN}}\xspace}
\def\OBTAIN{\ensuremath{\mathsf{OBTAIN}}\xspace}
\def\OBTISS{\ensuremath{\mathsf{OBTISS}}\xspace}
\def\ISSUE{\ensuremath{\mathsf{ISSUE}}\xspace}
\def\SIGN{\ensuremath{\mathsf{SIGN}}\xspace}
\def\OPEN{\ensuremath{\mathsf{OPEN}}\xspace}
\def\CHALb{\ensuremath{\mathsf{CHAL}_b}\xspace}
\def\ATTR{\ensuremath{\mathsf{ATTR}}\xspace}
\def\OWNR{\ensuremath{\mathsf{OWNR}}\xspace}
\def\CRED{\ensuremath{\mathsf{CRED}}\xspace}
\def\USK{\ensuremath{\mathsf{\MakeUppercase{\usk}}}\xspace}
\def\PUBUSK{\ensuremath{\mathsf{P}\USK}\xspace}
\def\PRVUSK{\ensuremath{\mathsf{S}\USK}\xspace}
\def\HU{\ensuremath{\mathsf{HU}}\xspace}
\def\CU{\ensuremath{\mathsf{CU}}\xspace}
\def\SIG{\ensuremath{\mathsf{SIG}}\xspace}
\def\CSIG{\ensuremath{\mathsf{CSIG}}\xspace}
\def\CCRED{\ensuremath{\mathsf{CC}}\xspace}
\def\secpar{\ensuremath{\kappa}\xspace}
\def\nattrs{\ensuremath{n}\xspace}
\def\param{\ensuremath{\param}\xspace}
\def\isk{\ensuremath{isk}\xspace}
\def\ipk{\ensuremath{ipk}\xspace}
\def\osk{\ensuremath{osk}\xspace}
\def\opk{\ensuremath{opk}\xspace}
\def\gpk{\ensuremath{gpk}\xspace}
\def\usk{\ensuremath{usk}\xspace}
\def\upk{\ensuremath{upk}\xspace}
\def\Attrs{\ensuremath{\mathsf{A}}\xspace}
\def\DAttrs{\ensuremath{\mathsf{D}}\xspace}
\def\cred{\ensuremath{cred}\xspace}
\def\utrans{\ensuremath{reg}\xspace}
\def\trans{\ensuremath{\MakeUppercase{reg}}\xspace}  
\def\msg{\ensuremath{m}\xspace}
\def\sig{\ensuremath{\sigma}\xspace}
\def\oproof{\ensuremath{\pi}\xspace}
\def\Setup{\ensuremath{Setup}\xspace}
\def\IKeyGen{\ensuremath{IKG}\xspace}
\def\OKeyGen{\ensuremath{OKG}\xspace}
\def\UKeyGen{\ensuremath{UKG}\xspace}
\def\Obtain{\ensuremath{Obtain}\xspace}
\def\Issue{\ensuremath{Issue}\xspace}
\def\Sign{\ensuremath{Sign}\xspace}
\def\Verify{\ensuremath{Verify}\xspace}
\def\Open{\ensuremath{Open}\xspace}
\def\Judge{\ensuremath{Judge}\xspace}
\def\ExpGSACCorrect{\ensuremath{\Exp_{\GSAC,\adv}^{corr}}\xspace}
\def\ExpGSACAnonb{\ensuremath{\Exp_{\GSAC,\adv}^{anon-b}}\xspace}
\def\ExpGSACAnonz{\ensuremath{\Exp_{\GSAC,\adv}^{anon-0}}\xspace}
\def\ExpGSACAnono{\ensuremath{\Exp_{\GSAC,\adv}^{anon-1}}\xspace}
\def\AdvGSACAnon{\ensuremath{\Adv_{\GSAC,\adv}^{anon}}\xspace}
\def\ExpGSACTrace{\ensuremath{\Exp_{\GSAC,\adv}^{trace}}\xspace}
\def\AdvGSACTrace{\ensuremath{\Adv_{\GSAC,\adv}^{trace}}\xspace}
\def\ExpGSACNonframe{\ensuremath{\Exp_{\GSAC,\adv}^{frame}}\xspace}
\def\AdvGSACNonframe{\ensuremath{\Adv_{\GSAC,\adv}^{frame}}\xspace}
\def\choose{\ensuremath{\mathsf{choose}}\xspace}
\def\guess{\ensuremath{\mathsf{guess}}\xspace}
\def\uid{\ensuremath{\mathsf{uid}}\xspace}
\def\cuid{\ensuremath{\mathsf{uid}^*}\xspace}
\def\cid{\ensuremath{\mathsf{cid}}\xspace}
\def\ccid{\ensuremath{\mathsf{cid}^*}\xspace}
\def\csig{\ensuremath{\mathsf{\sigma}^*}\xspace}
\def\credid{\ensuremath{cid}\xspace}
\def\Ccredid{\ensuremath{\Ccom_{\credid}}\xspace}
\def\Cusk{\ensuremath{\Ccom_{\usk}}\xspace}
\def\Cupk{\ensuremath{\Ccom_{\upk}}\xspace}
\def\attr{\ensuremath{attr}\xspace}

%%% Local Variables: 
%%% mode: pdflatex
%%% TeX-master: "gsac.tex"
%%% End:

% Symbols for UAS
\def\UAS{\ensuremath{\mathsf{UAS}}\xspace}
\def\IGEN{\ensuremath{\mathsf{IGEN}}\xspace}
\def\OGEN{\ensuremath{\mathsf{OGEN}}\xspace}
\def\ICORR{\ensuremath{\mathsf{ICORR}}\xspace}
\def\OCORR{\ensuremath{\mathsf{OCORR}}\xspace}
\def\CCORR{\ensuremath{\mathsf{CCORR}}\xspace}
\def\HUGEN{\ensuremath{\mathsf{HUGEN}}\xspace}
\def\CUGEN{\ensuremath{\mathsf{CUGEN}}\xspace}
\def\OBTAIN{\ensuremath{\mathsf{OBTAIN}}\xspace}
\def\OBTISS{\ensuremath{\mathsf{OBTISS}}\xspace}
\def\ISSUE{\ensuremath{\mathsf{ISSUE}}\xspace}
\def\SIGN{\ensuremath{\mathsf{SIGN}}\xspace}
\def\OPEN{\ensuremath{\mathsf{OPEN}}\xspace}
\def\INSPECT{\ensuremath{\mathsf{INSPECT}}\xspace}
\def\CHALb{\ensuremath{\mathsf{SIGCHAL}_b}\xspace}
\def\OBTCHALb{\ensuremath{\mathsf{OBTCHAL}_b}\xspace}
\def\SIMSETUP{\ensuremath{\mathsf{SimSetup}}\xspace}
\def\SIMOBTAIN{\ensuremath{\mathsf{SIMOBTAIN}}\xspace}
\def\SIMSIGN{\ensuremath{\mathsf{SIMSIGN}}\xspace}
\def\SIMOPEN{\ensuremath{\mathsf{SIMOPEN}}\xspace}
\def\ATTR{\ensuremath{\mathsf{ATT}}\xspace}
\def\UATTR{\ensuremath{\mathsf{U}\ATTR}\xspace}
\def\DATTR{\ensuremath{\mathsf{D}\ATTR}\xspace}
\def\OWNR{\ensuremath{\mathsf{OWN}}\xspace}
\def\GRP{\ensuremath{\mathsf{GRP}}\xspace}
\def\ISR{\ensuremath{\mathsf{ISR}}\xspace}
\def\CCRED{\ensuremath{\mathsf{CCRD}}\xspace}
\def\IK{\ensuremath{\mathsf{IK}}\xspace}
\def\OK{\ensuremath{\mathsf{OK}}\xspace}
\def\UK{\ensuremath{\mathsf{UK}}\xspace}
\def\GK{\ensuremath{\mathsf{GK}}\xspace}
\def\PUBIK{\ensuremath{\mathsf{P}\IK}\xspace}
\def\PRVIK{\ensuremath{\mathsf{S}\IK}\xspace}
\def\PUBOK{\ensuremath{\mathsf{P}\OK}\xspace}
\def\PRVOK{\ensuremath{\mathsf{S}\OK}\xspace}
\def\PUBUK{\ensuremath{\mathsf{P}\UK}\xspace}
\def\PRVUK{\ensuremath{\mathsf{S}\UK}\xspace}
\def\II{\ensuremath{\mathsf{I}}\xspace}
\def\OO{\ensuremath{\mathsf{O}}\xspace}
\def\HI{\ensuremath{\mathsf{HI}}\xspace}
\def\CI{\ensuremath{\mathsf{CI}}\xspace}
\def\HO{\ensuremath{\mathsf{HO}}\xspace}
\def\CO{\ensuremath{\mathsf{CO}}\xspace}
\def\HU{\ensuremath{\mathsf{HU}}\xspace}
\def\CU{\ensuremath{\mathsf{CU}}\xspace}
\def\CC{\ensuremath{\mathsf{CC}}\xspace}
\def\GEN{\ensuremath{\mathsf{GEN}}\xspace}
\def\CORR{\ensuremath{\mathsf{CORR}}\xspace}
\def\SIG{\ensuremath{\mathsf{SIG}}\xspace}
\def\CSIG{\ensuremath{\mathsf{CSIG}}\xspace}
%\def\CCRED{\ensuremath{\mathsf{CC}}\xspace}
\def\secpar{\ensuremath{\kappa}\xspace}
%\def\param{\ensuremath{\param}\xspace}
\def\isk{\ensuremath{isk}\xspace}
\def\ipk{\ensuremath{ipk}\xspace}
\def\sipk{\ensuremath{\mathbf{\ipk}}\xspace}
\def\osk{\ensuremath{osk}\xspace}
\def\opk{\ensuremath{opk}\xspace}
\def\gpk{\ensuremath{gpk}\xspace}
\def\sgpk{\ensuremath{\mathbf{\gpk}}\xspace}
\def\usk{\ensuremath{usk}\xspace}
\def\upk{\ensuremath{upk}\xspace}
\def\cred{\ensuremath{crd}\xspace}
\def\scred{\ensuremath{\mathbf{\cred}}\xspace}
\def\utrans{\ensuremath{reg}\xspace}
\def\trans{\ensuremath{\mathbf{reg}}\xspace}
\def\IdentifyCred{\ensuremath{\mathsf{IdentifyCred}}\xspace}
\def\IdentifySig{\ensuremath{\mathsf{IdentifySig}}\xspace}
\def\ExtractSetup{\ensuremath{\mathsf{ExtractSetup}}\xspace}
\def\ExtractSign{\ensuremath{\mathsf{ExtractSig}}\xspace}
\def\ExtractIssue{\ensuremath{\mathsf{ExtractIssue}}\xspace}
\def\c{\ensuremath{c}\xspace}
\def\y{\ensuremath{y}\xspace}
\def\yeval{\ensuremath{y_{ev}}\xspace}
\def\tyeval{\ensuremath{\tilde{y}_{ev}}\xspace}
\def\ceval{\ensuremath{\c_{ev}}\xspace}
\def\Yeval{\ensuremath{Y_{ev}}\xspace}
\def\TYeval{\ensuremath{\tilde{Y}_{ev}}\xspace}
\def\yinsp{\ensuremath{y_{op}}\xspace}
\def\tyinsp{\ensuremath{\tilde{y}_{op}}\xspace}
\def\cinsp{\ensuremath{\c_{op}}\xspace}
\def\sig{\ensuremath{\sigma}\xspace}
\def\Sig{\ensuremath{\Sigma}\xspace}
\def\iproof{\ensuremath{\pi}\xspace}
\def\tiproof{\ensuremath{\tilde{\pi}}\xspace}
\def\Setup{\ensuremath{Setup}\xspace}
\def\IKeyGen{\ensuremath{IKG}\xspace}
\def\OKeyGen{\ensuremath{OKG}\xspace}
\def\UKeyGen{\ensuremath{UKG}\xspace}
\def\Obtain{\ensuremath{Obt}\xspace}
\def\Issue{\ensuremath{Iss}\xspace}
\def\Sign{\ensuremath{Sign}\xspace}
\def\Verify{\ensuremath{Verify}\xspace}
\def\Inspect{\ensuremath{Inspect}\xspace}
\def\Judge{\ensuremath{Judge}\xspace}
\def\ExpCorrect{\ensuremath{\Exp_{\UAS,\adv}^{corr}}\xspace}
\def\CorrectIssue{\ensuremath{CorrectIssue}\xspace}
\def\CorrectEval{\ensuremath{CorrectEval}\xspace}
\def\CorrectEvalInspect{\ensuremath{CorrectEvalInspect}\xspace}
\def\CorrectInspect{\ensuremath{CorrectInspect}\xspace}
\def\ExpExtractIssue{\ensuremath{\Exp_{\UAS,\adv}^{ext-iss}}\xspace}
\def\ExpExtractSign{\ensuremath{\Exp_{\UAS,\adv}^{ext-sig}}\xspace}
\def\ExpIdentifyCred{\ensuremath{\Exp_{\UAS,\adv}^{id-cred}}\xspace}
\def\ExpIdentifySign{\ensuremath{\Exp_{\UAS,\adv}^{id-sig}}\xspace}
\def\ExpSimAnonb{\ensuremath{\Exp_{\UAS,\adv}^{sim-anon-b}}\xspace}
\def\ExpSigAnonb{\ensuremath{\Exp_{\UAS,\adv}^{sig-anon-b}}\xspace}
\def\ExpIssAnonb{\ensuremath{\Exp_{\UAS,\adv}^{iss-anon-b}}\xspace}
\def\ExpSigAnonz{\ensuremath{\Exp_{\UAS,\adv}^{sig-anon-0}}\xspace}
\def\ExpIssAnonz{\ensuremath{\Exp_{\UAS,\adv}^{iss-anon-0}}\xspace}
\def\ExpSigAnonzo{\ensuremath{\Exp_{\UAS,\adv}^{sig-anon-01}}\xspace}
\def\ExpSigAnono{\ensuremath{\Exp_{\UAS,\adv}^{sig-anon-1}}\xspace}
\def\ExpIssAnono{\ensuremath{\Exp_{\UAS,\adv}^{iss-anon-1}}\xspace}
\def\ExpSigAnonoz{\ensuremath{\Exp_{\UAS,\adv}^{sig-anon-10}}\xspace}
\def\ExpSimAnono{\ensuremath{\Exp_{\UAS,\adv}^{sim-anon-1}}\xspace}
\def\ExpSimAnonz{\ensuremath{\Exp_{\UAS,\adv}^{sim-anon-0}}\xspace}
\def\AdvSigAnon{\ensuremath{\Adv_{\UAS,\adv}^{sig-anon}}\xspace}
\def\AdvIssAnon{\ensuremath{\Adv_{\UAS,\adv}^{iss-anon}}\xspace}
\def\AdvSimAnon{\ensuremath{\Adv_{\UAS,\adv}^{sim-anon}}\xspace}
\def\AdvIdCred{\ensuremath{\Adv_{\UAS,\adv}^{id-cred}}\xspace}
\def\AdvIdSign{\ensuremath{\Adv_{\UAS,\adv}^{id-sig}}\xspace}
\def\ExpTrace{\ensuremath{\Exp_{\UAS,\adv}^{trace}}\xspace}
\def\AdvTrace{\ensuremath{\Adv_{\UAS,\adv}^{trace}}\xspace}
\def\IssueTrace{\ensuremath{IssTrace}\xspace}
\def\IssueForge{\ensuremath{IssForge}\xspace}
\def\ExpForgeIssue{\ensuremath{\Exp_{\UAS,\adv}^{iss-forge}}\xspace}
\def\AdvForgeIssue{\ensuremath{\Adv_{\UAS,\adv}^{iss-forge}}\xspace}
\def\ExpForgeSign{\ensuremath{\Exp_{\UAS,\adv}^{sig-forge}}\xspace}
\def\AdvForgeSign{\ensuremath{\Adv_{\UAS,\adv}^{sig-forge}}\xspace}
\def\EvalForge{\ensuremath{EvForge}\xspace}
\def\EvalInspectForge{\ensuremath{EvInForge}\xspace}
\def\InspectForge{\ensuremath{InForge}\xspace}
\def\ExpNonframe{\ensuremath{\Exp_{\UAS,\adv}^{frame}}\xspace}
\def\AdvNonframe{\ensuremath{\Adv_{\UAS,\adv}^{frame}}\xspace}
\def\ExpNonframeSign{\ensuremath{\Exp_{\UAS,\adv}^{frame-sign}}\xspace}
\def\AdvNonframeSign{\ensuremath{\Adv_{\UAS,\adv}^{frame-insp}}\xspace}
\def\ExpNonframeInsp{\ensuremath{\Exp_{\UAS,\adv}^{frame-insp}}\xspace}
\def\AdvNonframeInsp{\ensuremath{\Adv_{\UAS,\adv}^{frame-insp}}\xspace}
\def\ExpExtractIssue{\ensuremath{\Exp_{\UAS,\adv}^{ext-issue}}\xspace}
\def\AdvExtractIssue{\ensuremath{\Adv_{\UAS,\adv}^{ext-issue}}\xspace}
\def\ExpExtractSign{\ensuremath{\Exp_{\UAS,\adv}^{ext-sign}}\xspace}
\def\AdvExtractSign{\ensuremath{\Adv_{\UAS,\adv}^{ext-sign}}\xspace}
\def\choose{\ensuremath{\mathsf{choose}}\xspace}
\def\guess{\ensuremath{\mathsf{guess}}\xspace}
\def\Oanonc{\ensuremath{\mathcal{O}^{anon-b}_{\choose}}\xspace}
\def\Oanong{\ensuremath{\mathcal{O}^{anon-b}_{\guess}}\xspace}
\def\OExt{\ensuremath{\mathcal{O}^{ext}}\xspace}
\def\OId{\ensuremath{\mathcal{O}^{id}}\xspace}
\def\OIssAnon{\ensuremath{\mathcal{O}^{iss-anon-b}}\xspace}
\def\OSigAnon{\ensuremath{\mathcal{O}^{sig-anon-b}}\xspace}
\def\Osimanon{\ensuremath{\mathcal{O}^{sim-anon-b}}\xspace}
\def\Otrace{\ensuremath{\mathcal{O}^{trace}}\xspace}
\def\Oforgeissue{\ensuremath{\mathcal{O}^{iss-forge}}\xspace}
\def\Oforgesign{\ensuremath{\mathcal{O}^{sig-forge}}\xspace}
\def\Oframe{\ensuremath{\mathcal{O}^{frame}}\xspace}
\def\gid{\ensuremath{\mathsf{gid}}\xspace}
\def\iid{\ensuremath{\mathsf{iid}}\xspace}
\def\oid{\ensuremath{\mathsf{oid}}\xspace}
\def\sgid{\ensuremath{\boldsymbol{\mathsf{\gid}}}\xspace}
\def\siid{\ensuremath{\boldsymbol{\mathsf{\iid}}}\xspace}
\def\uid{\ensuremath{\mathsf{uid}}\xspace}
\def\suid{\ensuremath{\boldsymbol{\mathsf{uid}}}\xspace}
\def\cuid{\ensuremath{\mathsf{uid}^*}\xspace}
\def\cid{\ensuremath{\mathsf{cid}}\xspace}
\def\scid{\ensuremath{\boldsymbol{\mathsf{cid}}}\xspace}
\def\ccid{\ensuremath{\mathsf{cid}^*}\xspace}
\def\cscid{\ensuremath{\scid^*}\xspace}
\def\csig{\ensuremath{\mathsf{\sigma}^*}\xspace}
\def\cSig{\ensuremath{\mathsf{\Sigma}^*}\xspace}
\def\LangIss{\ensuremath{\Lang_{is}}\xspace}
\def\RelIss{\ensuremath{\NIZKRel_{is}}\xspace}
\def\LangIns{\ensuremath{\Lang_{op}}\xspace}
\def\RelIns{\ensuremath{\NIZKRel_{op}}\xspace}
\def\LangVerIns{\ensuremath{\Lang_{verop}}\xspace}
\def\LangSig{\ensuremath{\Lang_{sig}}\xspace}
\def\RelSig{\ensuremath{\NIZKRel_{sig}}\xspace}
\def\fissue{\ensuremath{f_{is}}\xspace}
\def\feval{\ensuremath{f_{ev}}\xspace}
\def\rngfeval{\ensuremath{R_{ev}}\xspace}
\def\varrngfeval{\ensuremath{r_{ev}}\xspace}
\def\finsp{\ensuremath{f_{op}}\xspace}
\def\rngfinsp{\ensuremath{R_{op}}\xspace}
\def\varrngfinsp{\ensuremath{r_{op}}\xspace}
\def\famfissue{\ensuremath{\mathcal{F}_{is}}\xspace}
\def\famfeval{\ensuremath{\mathcal{F}_{ev}}\xspace}
\def\famfinsp{\ensuremath{\mathcal{F}_{op}}\xspace}
\def\Issuers{\ensuremath{\mathbb{I}}\xspace}
\def\CUASGen{\ensuremath{\Pi_{\UAS}}\xspace}
\def\trap{\ensuremath{\tau}\xspace}
\def\extract{\ensuremath{\mathsf{ex}}\xspace}
\def\extracttrap{\ensuremath{\tau_{\extract}}\xspace}
\def\sring{\ensuremath{\mathbf{ring}}\xspace}
\def\CUASRing{\ensuremath{\Pi_{\UAS}^{ring}}\xspace}
\def\CUASGS{\ensuremath{\Pi_{\UAS}^{gs}}\xspace}
\def\CUASAC{\ensuremath{\Pi_{\UAS}^{ac}}\xspace}
\def\CUASGSMDO{\ensuremath{\Pi_{\UAS}^{gsmdo}}\xspace}
\def\CUASDAC{\ensuremath{\Pi_{\UAS}^{dac}}\xspace}
\def\CUASRAC{\ensuremath{\Pi_{\UAS}^{rac}}\xspace}
\def\CUASMPS{\ensuremath{\Pi_{\UAS}^{mps}}\xspace}

%%% Local Variables: 
%%% mode: pdflatex
%%% TeX-master: "gsac.tex"
%%% End:


\begin{document}
{\def\addcontentsline#1#2#3{}\maketitle}%\maketitle


\begin{abstract}
  Much work has been done in the last 4 decades on privacy-preserving
  identities and authentication that tries to incorporate some sort of utility
  and accountability without losing (too much) privacy. Among the most relevant
  efforts we can find group signatures (GS) and anonymous credentials (AC).
  %
  AC let users obtain attestations by trusted
  issuers on certain attributes -- thus implicitly defining the concept of
  identity as ``a set of attributes'' -- and allow credential owners to prove
  claims on these attributes. While this has proven to be very
  flexible, accountability in AC schemes is quite limited; frequently restricted
  to blacklisting, if present at all.
  %
  On the other hand, GS have traditionally focused on plain authentication
  without attributes, making instead more emphasis on the accountability side.
  That is, not only allowing full
  deanonymization, but also variants such as sequential linking, user-controlled
  anonymity, or even full unlinkability.
  %
  Not surprisingly, GS and AC share very common syntax, models and frequently
  also constructions.

  We argue that, in the real world, both functionalities are essential. Taking
  advantage of their similarities, we first propose a trivial combination,
  \GSAC, showcasing the benefits even a simple combination brings. We give a
  common syntax, security model, and generic construction for \GSAC.
  %
  Next, we generalize \GSAC into our Universal Anonymous Signatures (\UAS)
  proposal by allowing arbitrary policies at issuance, signing and opening
  time, among other improvements. While generalizations along these
  lines have been explored independently in the GS and AC domains, to the best
  of our knowledge, no single scheme offers such full flexibility.
  We define a syntax, security model, and give a generic construction for a
  scheme meeting these requirements. Furthermore, we show how our model for
  \UAS is a generalization of previous well known schemes in the GS and AC
  domains.
  %
  The techniques and building blocks we leverage are well known. The challenge
  is, rather, on the definitional side. To further evidence the universality of
  \UAS, we show how more restricted schemes can be built from vanilla \UAS.
\end{abstract}

\tableofcontents

\section{Introduction}
\label{sec:introduction}

%%% Local Variables:
%%% mode: latex
%%% TeX-master: "single-mnemonic"
%%% End:

\section{Preliminaries}
\label{sec:preliminaries}

\subsection{Related Work}
\label{ssec:related}

\paragraph{Group Signatures and Anonymous Credentials.} %
\todo{Brief summary of GS and AC here. Maybe split them in different
  paragraphs.}

\paragraph{Attribute-Based Group Signatures.} %
The closest related works to our \GSAC are the series of Attribute-Based Group
Signature schemes (ABGS) published between 2007 and 2014. The initial proposals
\cite{khad07a,khad07b} defined a variant of group signatures in which membership
credentials contained attributes. Yet, there are some
crucial differences that make ABGS too rigid for practice. Ignoring that the
first works \cite{khad07a,khad07b} are modelled as static group signatures
(i.e., they do not allow dynamic joins of new members), each group has a static
``attribute access tree'' associated to it. This tree defines what combinations
of attributes make a signature valid. For instance, an example given in
\cite{khad07b} requires group signatures to originate from an employee who is
``a member of the IT department, and either at least a junior manager from the
cryptography group, or a senior manager from the biometrics team''. If,
eventually, a different attribute access tree is needed, a new group needs to
be created. Additionally, only revocation is considered in \cite{khad07b},
which is a subset of the accountability flavours that can be reached through
the open function of conventional GS schemes. These schemes also lack formalism
in their definitions. For instance, correctness does not cover correctness of
revocation, and the anonymity definition does not seem to consider trivial wins
by the adversary in which the adversary can revoke one of the two challenge
users.
%
\cite{emo09} follows the attribute access tree aproach, but generalizes it so
that attributes can be removed from the access tree, or the required conditions
can be modified -- e.g., in \cite{emo09}, we could remove the ``or a senior
manager from the biometrics team'' condition in the previous example. It also
allows dynamic joins, and models open rather than just revocation. However,
it does not
allow adding new attributes; in that case, a new group has to be created. It
also deviates from the conventional group signatures model, by introducing a
\emph{collision-resistance} property that prevents different users from
combining their membership credentials. While this is just a modelling choice
with no impact in the actual functionality, it seems that this could be
integrated within the existing traceability property, leading to a cleaner
model. \todo{How does removing attributes or changing the access rules affect
  anonymity? Is this considered?}
%
Finally, and as far as we know, \cite{aa14} is the last work in this ABGS line.
They support dynamic joins, open and verifier-local revocation (hence, their
variant is named VLR-ABGS). Furthermore, they include backwards unlinkability,
meaning that a compromise of a user key at epoch $t$ does not affect anonymity
of signatures on previous epochs by that same user. This scheme is still based
on attribute access trees -- concretely, those in \cite{emo09}, and thus
inherits its constraints. It also resorts to many properties, deviating from the
cleaner approaches of GS schemes. Specifically, it defines security of VLR-ABGS
by means of attribute anonymity (which is not formally defined),
backwards-unlinkability user anonymity, traceability, non-frameability,
attribute unforgeability, and collision resistance of attributes. Again, this
are just modelling choices, but it would seem that several of these properties
could be collapsed, hence leading to a cleaner model, much easier to evaluate
(e.g., traceability, attribute unforgeability and collision resistance of
attributes all seem similar).

In summary, the previous line of ABGS seems to lack the flexibility of AC
schemes in regards to what can be proven with attribute-based credentials.
Restricting to AC with selective disclosure, one can get essentially the same
functionality as with ABGS, but with flexible policies. That is, just let
the verifier specify the policy that users need to meet depending on the
situation, and only users revealing a matching subset of attributes will be
accepted. In this respect, anonymity and unforgeability of ACs are equivalent to
anonymity and traceability/unforgeability properties of ABGS. Still, an ABGS
group is tied up to one concrete policy, while a single AC scheme spans
arbitrarily many.
%
On the other hand, what ABGS (or, rather, some of its variants) adds with
respect to AC schemes is the non-frameability property of group signatures,
which opens the possibility to easily incorporate accountability notions.
%
In addition to what we have already mentioned, none of the ABGS works allows
a group member to obtain multiple credentials -- not even from the same issuer.
While obtaining multiple credentials in the setting of GS is not useful, it is
if we include attributes and realize that the same user may have multiple
attributes. This is certainly the case in multiple real life scenarios. For
instance, the same person may obtain two different degrees within the same
university; or an employee of a bank can also be a client. Both situations
require attribute credentials issued by the same entity (the university, or
the bank), none of them being a proper subset of the other, but still being
associated to the same person.

% One difference between our current approach to GSAC, and the existing ABGS
% works, is that in ABGS each group member can only get one credential. On the
% other hand, we are trying to let any group member obtain as many credentials
% as he wishes, possibly with different attributes, albeit all of them related to
% the same user key pair (if a different user key pair is employed, then this is
% considered as a different user).

\subsection{Cryptographic Building Blocks}
\label{ssec:cryptobblocks}

\todo{NIZKs.}

\subsubsection{Notation}
\label{ssec:notation}

\todo{\Attrs and \DAttrs (in \secref{ssec:construction-gsac}, they are treated
  as indices, but I think that's not the case elsewhere).}

\todo{In \secref{ssec:construction-gsac}, we assume that all messages and
  attributes can be encoded as elements in $\ZZ^*_p$.}
  

%%% Local Variables:
%%% mode: latex
%%% TeX-master: "uas"
%%% End:

\section{\GSAC: Merging Group Signatures and Anonymous Credentials}
\label{sec:gsac}

\todo{Some intro here}

\subsection{Model for \GSAC}
\label{ssec:model-gsac}

While group signatures and anonymous credentials share quite common syntax and
security properties, which are even frequently formalised in a similar way,
there are aspects that need to be considered with care.

The first choice that has to be made is whether to make sign/show and
verification a non-interactive or an interactive process -- in group signatures,
we are in the former case, whereas in anonymous credentials, we are in the
latter. While both options are of course valid, this has direct impact in the
modelling. We choose the non-interactive approach, and then sketch how to
generically (and trivially) translate it into an interactive process. The reason
is that this results in a more flexible building block, suitable for a wider set
of scenarios. And, syntactically, it seems more natural to combine two
non-interactive processes into an interactive one, than the other way around.

We also need to take into account that, in group signature schemes, users only
get one membership credential -- typically bound to their personal secret
key --, which they then use to create as many group signatures as they want. On
the other hand, in anonymous credentials, users can get as many credentials as
they desire, which can (typically) then be showed also arbitrarily many times.
In this regard, we follow a somehow intermediate approach: users own a single
personal secret key, which they can use to get as many credentials as they wish.
Subsequently, they can use any of those credentials to create \GSAC signatures.
This has, as expected, some nuanced impact in the modelling. For instance, in
group signatures, tracing and non-frameability are frequently dependent on the
open function, which (among other things) returns an member index. In our
approach, though, each user may have many such transcripts -- one per credential
that the user obtains. However, note that what actually provides meaning to
these notions is not the index; but the attributes in the credential, which is
moreover necessary
to prove that the opening has been performed honestly. Thus, with some
restructuring of the corresponding games, we can still capture the main
meaning of these traceability and non-frameability notions from group
signatures. This also impacts anonymity, since in group signatures, this is
modelled by challenging the adversary to guess which user (which implicitly
translates to ``which user credential'') out of two challenge users was
leveraged to produce the challenge signature. But again, in our case, users may
have multiple credentials. However, it seems sufficient to adopt a similar
anonymity flavour to that of anonymous credentials, where the adversary has to
distinguish between presentations (signatures) produced from two arbitrary
credentials, independently on whether these credentials were obtained by the
same user, or not.
%
Getting closer to the anonymity definition of anonymous credentials also has the
side effect to introduce the notions of attributes for free into the domain of
group signatures -- which, up to now, had considered membership credentials
\emph{without} attributes. Yet another side effect of introducing multiple
attributes (if we think from a group signature perspective) is that, in order
to avoid trivial wins, the anonymity game has to enforce that both challenge
credentials reveal the same predicate on their attributes.

\subsubsection{Syntax for \GSAC schemes}
\label{sssec:syntax-gsac}

\begin{description}
\item[$\parm \gets \Setup(\secpar)$.] Given a security parameter \secpar,
  returns a global system parameter variable \parm.
\item[$(\ipk,\isk) \gets \IKeyGen(\parm)$.] Given global system parameters
  \parm, returns the issuer's key pair. Hereafter, we assume that the public
  part \ipk is added to the group public key \gpk.
\item[$(\opk,\osk) \gets \OKeyGen(\parm)$.] Given global system parameters
  \parm, returns an opener's key pair. Hereafter, we assume that the public part
  \opk is added to the group public key \gpk.
\item[$(\upk,\usk) \gets \UKeyGen(\parm)$.] Given global system parameters
  \parm, returns a user's key pair.
\item[$\langle \cred/\bot,\utrans/\bot \rangle \gets
  \langle \Obtain(\gpk,\usk,\attrs),\Issue(\gpk,\isk,\attrs) \rangle$.]
  This interactive protocol lets a user with key pair (\upk,\usk) running the
  \Obtain process, obtain a credential \cred with attributes \attrs, by
  communicating with an issuer, with private key \isk, running the \Issue
  counterpart. The user outputs the produced credential \cred, while the issuer
  outputs the protocol transcript \utrans for the produced credential.
\item[$\sig \gets \Sign(\gpk,\usk,\cred,\dattrs,\msg)$.] The user with
  with secret key \usk, who obtained a credential \cred, produces a signature
  \sig over message \msg, revealing subset of attributes \dattrs in \cred.
\item[$1/0 \gets \Verify(\gpk,\sig,\dattrs,\msg)$.] Checks whether \sig is a
  valid signature revealing the attribute set \dattrs of its signing credential,
  over message \msg.
\item[$(\upk,\attrs,\oproof)/\bot \gets
  \Open(\gpk,\osk,\sig,\dattrs,\msg)$.]
  Given a signature \sig over message \msg, produced by a credential with
  attributes containing the set \dattrs, returns the public key \upk of the
  signer, and the attribute set \attrs contained in the credential used to
  generate the signature, along with a proof of opening correctness \oproof; or
  $\bot$ if the opening process fails.
\item[$1/0 \gets \Judge(\gpk,\upk,\attrs,\oproof,\sig,\dattrs,
  \msg)$.] Checks if \oproof is a valid opening proof for the
  statement: ``The owner of a credential over attribute set \attrs, and owner
  of public key \upk, created signature \sig, which reveals attribute set
  $\dattrs \subseteq \attrs$''.
\end{description}

The correctness and security properties are defined with the help of the
following sets of oracles and global variables that help oracles and games
keep consistent state.

\paragraph{Global variables.} %
All the honestly generated users (i.e., all honestly generated user key pairs,
since we assume a one-to-one relationship between user and user key pair), as
well as all honestly generated credentials, are assigned identifiers. We
typically write \uid for users' identifiers, and \cid for credentials'. The
adversary can
refer to any individual user or credential using the corresponding identifier --
even though he may not know the actual contents of the key pairs or credentials.
In games that involve challenge user or credential identifiers, we use \cuid and
\ccid to refer to these challenge user/credential.
%
The games keep track of the user keys (through table \UK) and credentials (through
table \CRED) that are created as a result of oracle calls by the adversary. We
use the user and credential identifiers to reference specific user key pairs,
credentials, and credential attributes.
For instance, $\UK[\uid]$ refers to the user key pair corresponding to user
with identifier \uid, $\PUBUK[\uid]$ refers to the public key of that key pair,
and $\PRVUK[\uid]$ to the private key; credential $\CRED[\cid]$ refers to the
credential data associated to the credential with identifier \cid; $\ATTR[\cid]$
is the set of attributes that was assigned to the credential with identifier
\cid; and $\OWNR[\cid]$ is set to the \uid corresponding to the user that
credential \cid was issued to.
%
Additionally, the games keep track of all honest and corrupt users that have
been generated through sets \HU and \CU, respectively. They also keep track of
the signatures that have been honestly produced, through the \SIG table. Since
signatures are produced through credentials, the \SIG table is indexed with
{\cid}s. For the anonymity game, we also need to keep track of the challenge
signatures that the adversary has obtained, in order to prevent trivial wins
by allowing any of them to be opened. 
%
All global variables are assumed to be initially set by the games to empty
values, and all tables/sets are initialized as empty tables/sets. Also, for
readability, we abuse the syntax and
write $\CRED[\uid]$ to mean $\CRED[\cid]$ for all $\cid$ such that
$\OWNR[\cid] = \uid$.% ; we also sometimes use \upk as a ``synonym'' of \uid, as it
% is possible to do a reverse lookup from the \UK table (i.e., $\CRED[\upk]$ can
% be replaced by $\CRED[\uid]~\st~\UK[\uid]=(\upk,\cdot)$); \todo{Something else?}

\paragraph{Oracles.} %
Oracles are the interface of the adversary with the corresponding games. In
other words: through these oracles, the game enviroment exposes to the adversary
functionality that could otherwise be executed only by honest parties with
private knowledge -- knowledge that would make the adversary capable of
trivially breaking the security properties formalized in the experiments.
In the game-based definitions of our \GSAC model, we leverage the following
oracles, which are formally defined in \appref{app:gsac-formal}.

\begin{description}
\item[\HUGEN.] Adds a new honest user to the game, by honestly generating
  the user's key pair.
\item[\CUGEN.] If the specified user identifier is not already in the game,
  sets the user public key to a value given by the adversary. If it already
  exists, and is associated to an honest user, then it reveals to the adversary
  the user's secret key.
\item[\OBTISS.] Lets the adversary add a new honestly generated credential to
  the game, on behalf of an honest user.
\item[\OBTAIN.] Enables the adversary to play the role of a dishonest issuer
  in games that support it, by letting it interact with honest users who want to
  receive credentials.
\item[\ISSUE.] Allows the adversary to play the role of dishonest users,
  requesting an honest issuer to produce credentials for them.
\item[\RREG.] Reads the given transcript table entry.
\item[\WREG.] Sets a transcript table entry to the given value.  
\item[\SIGN.] Lets the adversary get signatures from credentials belonging
  to honest users.
\item[\OPEN.] Given an honestly produced signature, lets the adversary learn
  the public key of the user who produced it.
\item[\CHALb.] Upon receiving two challenge credentials, a common intersecting
  set of attributes, and a message, returns a signature produced by one of these
  two credentials, defined by the bit $b$, which is established in the anonymity
  game.
\end{description}

\paragraph{Correctness.} %
Correctness of \GSAC schemes is formalized through the experiment in
\figref{fig:exp-gsac-corr}. It states that a signature for some message \msg,
revealing attributes \dattrs, which was honestly produced through a credential
that was obtained by an honest user interacting with an honest issuer, with a
set of attributes \attrs such that $\dattrs \subseteq \attrs$, must be accepted
by \Verify. Moreover, an honestly produced correctness proof of opening for such
signature, revealing the public key pair of the user and credential attributes,
must also be accepted by \Judge.

\begin{definition}{(Correctness of \GSAC)}
  \label{def:correctness-gsac}
  A \GSAC scheme is correct if, for any p.p.t. adversary $\adv$,
  $\Pr[\ExpGSACCorrect(1^\secpar) = 1]$ is a negligible function of \secpar.
\end{definition}

\begin{figure}[htp!]
  \procedure[linenumbering]{$\ExpGSACCorrect(1^\secpar)$}{%
     \parm \gets \Setup(1^\secpar) \\
     (\ipk,\isk) \gets \IKeyGen(\parm);~(\opk,\osk) \gets \OKeyGen(\parm);~ 
     \gpk \gets (\ipk,\opk) \\
    (\cid,\dattrs,\msg) \gets \adv^{\HUGEN,\OBTISS,\RREG}(\gpk) \\
    \pcif \dattrs \nsubseteq \ATTR[\cid] :
    \pcreturn \bot \\
    \sig \gets \Sign(\gpk,\PRVUK[\OWNR[\cid]],\CRED[\cid],\dattrs,\msg) \\
    (\upk,\attrs,\oproof) \gets \Open(\gpk,\osk,\sig,\dattrs,\msg) \\
    \pcif \Verify(\gpk, \sig, \dattrs,\msg) = 0 \lor
    \Judge(\gpk,\upk,\attrs,\oproof,\sig,\dattrs,\msg) = 0: \\
     \pcind \pcreturn 1 \\
    \pcreturn 0
  }
  \caption{Correctness experiment for \GSAC schemes.}
  \label{fig:exp-gsac-corr}
\end{figure}

\subsubsection{Security Properties of \GSAC Schemes}
\label{sssec:security-gsac}

\paragraph{Anonymity.} %
In group signatures, anonymity captures that no adversary must be able to learn,
from any group signature, anything about its signer. In anonymous credentials
(with selective disclosure), it requires that no adversary should learn anything
about the holder of a credential that has been successfully shown, beyond that
he owns a credential containing the revealed attributes. In both GS and AC, it
is also typically required that
multiple signatures/presentations by the users are unlinkable. The approach to
formally state this property is in both cases frequently the same: the adversary
requests signatures from one out of two challenge users/credentials, where the
one actually producing the signatures is defined by a bit $b$. The adversary
wins if it succeeds in guessing which was the chosen user/credential better than
guessing at random. In group signatures, the game must also restrict the
adversary from opening challenge
signatures. In anonymous credentials, the game must constraint the
adversary to pick credentials that have some common subset of attributes, and
to use that common subset to request the challenge presentations. Here, we need
to take into account both. Furthermore, a key difference with group signatures
is that the game requires the adversary to output credential identifiers, rather
than user identifiers. Specifically, this means that the adversary may actually
output two credentials that belong to the same user. Therefore, in some sense,
the anonymity we get is more general than that of group signatures, as in \GSAC,
like in anonymous credentials, the adversary may request signatures from
different credentials, but belonging to the same user. The formal
specification of the anonymity game is given in \figref{fig:exp-gsac-anonb}.

\begin{figure}[htp!]
  \procedure[linenumbering]{$\ExpGSACAnonb(1^\secpar)$}{%
     \parm \gets \Setup(1^\secpar) \\
     (\ipk,\isk) \gets \IKeyGen(\parm);~(\opk,\osk) \gets \OKeyGen(\parm);~ 
     \gpk \gets (\ipk,\opk) \\
     b^* \gets
     \adv^{\HUGEN,\CUGEN,\WREG,\OBTAIN,\SIGN,\OPEN,\CHALb}(\gpk,\isk,\status) \\
     \pcreturn b^*
  }
  \caption{Anonymity experiment for \GSAC schemes.}
  \label{fig:exp-gsac-anonb}
\end{figure}

\begin{definition}{(Anonymity of \GSAC)}
  \label{def:anonymity-gsac}
  We define the advantage \AdvGSACAnon of $\adv$ against \ExpGSACAnonb as
  $\AdvGSACAnon=|\Pr\lbrack\ExpGSACAnono(1^\secpar)=1\rbrack-
  \Pr\lbrack\ExpGSACAnonz(1^\secpar)=1\rbrack|$.
  %
  A \GSAC scheme satisfies anonymity if, for any p.p.t. adversary $\adv$,
  \AdvAnon is a negligible function of $1^\secpar$.
\end{definition}

\paragraph{Traceability.} %
Traceability is one of the unforgeability-related properties in group
signatures. It captures that any signature accepted by \Verify needs to open
to one of the users that joined the group. While there is no traceability notion
in anonymous credentials, it is natural to map it to their unforgeability
property; if only because both require the issuer to be honest. Unforgeability
in anonymous credentials typically ensures that no adversary can get a verifier
to accept a credential presentation requiring a set of attributes that is not
contained in one of the credentials controlled by the adversary.
%
Our notion of traceability for \GSAC combines both requirements. It assumes an
honest issuer, as otherwise the adversary can create untraceable credentials at
will. The game then lets the adversary add honest and corrupt users, create
honest signatures, and open them. The adversary wins if, after this interaction,
is able to produce a $(\sig,\dattrs,\msg)$ tuple that is accepted by \Verify,
but either cannot be opened, it can be opened but the proof is rejected by
\Judge, or even though it is accepted by \Judge, there is no credential
that contains the set of attributes \dattrs. We formally define traceability in
the \ExpTrace experiment in \figref{fig:exp-gsac-trace}.

\begin{figure}[htp!]
  \procedure[linenumbering]{$\ExpGSACTrace(1^\secpar)$}{%
    \parm \gets \Setup(1^\secpar) \\
    (\ipk,\isk) \gets \IKeyGen(\parm);~(\opk,\osk) \gets \OKeyGen(\parm);~ 
    \gpk \gets (\ipk,\opk) \\
    (\sig,\dattrs,\msg) \gets
    \adv^{\HUGEN,\CUGEN,\RREG,\OBTISS,\ISSUE,\SIGN}(\gpk,\osk) \\
    \pcif \Verify(\gpk,\sig,\dattrs,\msg) = 0: \pcreturn 0 \\
    \pcif \Open(\gpk,\osk,\trans,\sig,\dattrs,\msg) = \bot: \pcreturn 1 \\
    \textrm{Let}~(\upk,\attrs,\oproof) \gets \Open(\gpk,\osk,\trans,\sig,
    \dattrs,\msg) \\
    \pcif \Judge(\gpk,\upk,\attrs,\oproof,\sig,\dattrs,\msg) = 0: \pcreturn 1 \\
    \pcelse \pcif \forall \uid \in \CU \cup \HU, \nexists (\uid,\cdot,\attrs)
    \in \CRED \lor \dattrs \not\subseteq \attrs: \pcreturn 1 \\
    \pcreturn 0
  }
  \caption{Traceability experiment for \GSAC schemes.}
  \label{fig:exp-gsac-trace}
\end{figure}

\begin{definition}{(Traceability of \GSAC)}
  \label{def:trace-gsac}
  We define the advantage \AdvGSACTrace of $\adv$ against \ExpGSACTrace as
  $\AdvGSACTrace=\Pr\lbrack\ExpGSACTrace(1^\secpar)=1\rbrack$.
  %
  A \GSAC scheme satisfies traceability if, for any p.p.t. adversary $\adv$,
  \AdvGSACTrace is a negligible function of $1^\secpar$.
\end{definition}

\paragraph{Non-frameability.} %
Non-frameability variants are a core unforgeability-type property in group
signatures. However, no similar property is modelled for anonymous credentials.
It is a quite strong
property, as it must be ensured even in the presence of dishonest issuer and
opener. Intuitively, it prevents the adversary from creating a signature that
frames an honest user. Depending on the inspection capabilities of the scheme,
this framing could be done in different ways; i.e., by convincing third parties
that signatures by different (possibly corrupt) users are linked, or directly
by having open proofs output the identity of a user who did not create the
signature being opened.
%
In \GSAC schemes, in order for a user to be framed, the adversary first needs to
create a $(\sig,\dattrs,\msg,\upk,\attrs,\oproof)$ tuple such that the signature,
attributes, and message are accepted by \Verify, and which, along with the user
public key, and open correctness proof, are accepted by \Judge; or, alternatively,
produce a signature revealing \dattrs, which is accepted by \Verify, and \Judge
accepts \attrs as the attribute set in the used credential owned by \upk, but
$\dattrs \not\subseteq \attrs$.
Then, the adversary wins the game if the owner of the credential is honest, but
the signature was not produced via the \SIGN oracle.

%, or if the set of attribute disclosed in the signature are not contained in the
% set of attributes revealed during \Open.

\begin{figure}[htp!]
  \procedure[linenumbering]{$\ExpGSACNonframe(1^\secpar)$}{%
    \parm \gets \Setup(1^\secpar) \\
     (\ipk,\isk) \gets \IKeyGen(\parm);~(\opk,\osk) \gets \OKeyGen(\parm);~ 
     \gpk \gets (\ipk,\opk) \\
     (\sig,\dattrs,\msg,\upk,\attrs,\oproof) \gets
     \adv^{\HUGEN,\CUGEN,\WREG,\OBTAIN,\SIGN}(\gpk,\isk,\osk) \\
     \pcif \Verify(\gpk,\sig,\dattrs,\msg) = 0: \pcreturn 0 \\
     \pcif \Judge(\gpk,\upk,\attrs,\oproof,\sig,\dattrs,\msg) = 0: \pcreturn 0 \\
     \textrm{Let}~\uid~\textrm{be st}~\PUBUK[\uid] = (\upk,\cdot) \\
     \pcif \textrm{no such \uid exists}: \pcreturn 0 \\
     \pcif \uid \in \HU \land (\sig \notin \SIG[\uid]
     \lor \dattrs \not\subseteq \attrs): \pcreturn 1 \\
     \pcreturn 0
  }
  \caption{Non-frameability experiment for \GSAC schemes.}
  \label{fig:exp-gsac-frame}
\end{figure}

\begin{definition}{(Non-frameability of \GSAC)}
  \label{def:frame-gsac}
  We define the advantage \AdvGSACNonframe of $\adv$ against \ExpGSACNonframe as
  $\AdvGSACNonframe=\Pr\lbrack\ExpGSACNonframe(1^\secpar)=1\rbrack$.
  %
  A \GSAC scheme satisfies non-frameability if, for any p.p.t. adversary $\adv$,
  \AdvNonframe is a negligible function of $1^\secpar$.
\end{definition}

%%% Local Variables:
%%% mode: latex
%%% TeX-master: "uas"
%%% End:

\subsection{\GSACGen: A Generic Construction for \GSAC}
\label{ssec:generic-gsac}

In a nutshell, our generic \GSAC construction leverages a signature scheme
on blocks of (committed) messages, where each message to be signed is an
attribute in the produced credential; with the exception of the user secret key,
which is treated as a special attribute. Moreover, the user secret key is also
the only ``attribute'' signed in committed form, to prevent the issuer from
learning it. On top of that, we make use of NIZKs to: (1) during credential
issuance, have users prove knowledge of the secret key; (2) during signing,
have users prove knowledge of the user secret key, unrevealed attributes, and
a credential including all (revealed and unrevealed) attributes; and (3), during
open, to have the opener prove correctness of the decryption of the user
``public'' key. Concerning the user public keys, they are mere commitments to
the user secret key (with $0$ as randomness -- although could be any fixed
value), that serve to univocally identify a user, without revealing its secret
key.

In more detail, in the following algorithms, we make use of three different NP
relations for \NIZK proof systems:

\begin{description}
\item[$\NIZKRel_{\Issue}$:] Produced by users requesting a credential. It is
  defined as $\NIZKRel_{\Issue} = \lbrace \usk, \Ccom :
  \Ccom = \CCommit(\usk; r) \rbrace$.
\item[$\NIZKRel_{\Sign}$:] Produced by users when signing a message. It is
  defined as $\NIZKRel_{\Sign} = \lbrace (\usk,\Ccom,\attrs,\msg,\SBCMsig),
  (\Cmsg,\Ec,\dattrs) : \Cmsg = \CCommit(\msg) \land \Ccom =
  \CCommit(\usk; 0) \land \Ec = \EEnc(\opk,\Ccom)
  \land \SBCMVerify(\ipk,\SBCMsig,\usk,\attrs) = 1
  \land \dattrs \subseteq \attrs \rbrace$.
\item[$\NIZKRel_{\Open}$:] Used by the opener when opening a signature. It
  is defined as $\NIZKRel_{\Open} = \lbrace (\osk), (\Ec,\msg) :
  \msg = \EDec(\osk,\Ec) \rbrace$.
\end{description}

With the help of those \NIZK proof systems, we build the algorithms for
\GSACGen as follows:

\paragraph{$\Setup(\secpar,\nattrs) \rightarrow \parm$.} %
Sets up the public parameters. Namely: $\Cparm \gets \CSetup(\secpar)$, $\Eparm
\gets \ESetup(\secpar)$, $\SBCMparm \gets \SBCMSetup(\secpar)$,
$\NIZKcrs_{\Issue} \gets \NIZKSetup^{\Issue}(\secpar)$, $\NIZKcrs_{\Sign} \gets
\NIZKSetup^{\Sign} (\secpar)$, $\NIZKcrs_{\Open}\gets \NIZKSetup^{\Open}(\secpar)$.
Outputs $\parm \gets (\Cparm,\SBCMparm,\Eparm,\NIZKcrs_{\Issue},\NIZKcrs_{\Sign},
\NIZKcrs_{\Open})$. Note that parts of this process can be left to individual
parties (e.g., \SBCMSetup, $\NIZKSetup^{\Issue}$ to the issuer, or \ESetup and
$\NIZKSetup^{\Open}$ to the opener) who later publish the output, but we
concentrate them here in \Setup for readability.

\paragraph{$\IKeyGen(\parm) \rightarrow (\ipk,\isk)$.} %
Generates the signing key pair for the issuer, by parsing \parm as
$(\cdot,\SBCMparm,\cdot,\cdot)$ and running $(\ipk,\isk) \gets
\SBCMKeyGen(\SBCMparm)$.

\paragraph{$\OKeyGen(\parm) \rightarrow (\opk,\osk)$.} %
Generates the encryption key pair for the opener, by parsing \parm as
$(\cdot,\SBCMparm,\Eparm,\cdot)$ and running $(\opk,\osk) \gets
\EKeyGen(\Eparm)$.

\paragraph{$\UKeyGen(\parm) \rightarrow (\upk,\usk)$.} %
Generates users' key pairs by choosing a random value within the attribute space
\AttrSpace, and committing to it. Concretely: $\usk \getr \AttrSpace$, $\upk
\gets \CCommit(\usk; 0)$.

\paragraph{$\langle \Obtain(\gpk,\usk,\attrs),\Issue(\gpk,\isk,\attrs) \rangle
  \rightarrow \langle \cred/\bot,\utrans/\bot \rangle$.} %
In a nutshell, the user requests a signature over a commitment of its user
secret key, as well as the attributes in \attrs. The user proves knowledge of
the committed value, and the issuer sends the signature (credential) in return.
This is directly an execution of the interactive signing protocol of an \SBCM
scheme, in which the user parses \gpk as $(\ipk,\opk)$ and runs $\SBCMCom(\ipk,
\usk,\attrs)$, and the issuer runs $\SBCMSign(\isk,\attrs)$; in both cases,
using $\NIZKRel_{\Issue}$ as \NIZK relation. The credential \cred output to the
user is the signature produced by the interactive signing protocol, and the
\utrans entry obtained by the issuer is the transcript of the protocol, namely
a $(\Ccom,\attrs,\cred,\pi)$ tuple.

\iffalse

\todo{This is not consistent with the $\langle \SBCMCom,\SBCMSign \rangle$
  definition...}
  
\begin{itemize}
\item \underline{User}: Commit to the user secret key by running $\Ccom \gets
  \CCommit(\usk)$. Generate proof $\NIZKproof \gets
  \NIZKProve^{\NIZKRel_{\Issue}}(\NIZKcrs,\usk,\Ccom)$. Send $(\Ccom,
  \NIZKproof)$ to the issuer.
\item \underline{Issuer}: Run $\NIZKVerify^{\NIZKRel_{\Issue}}(\NIZKcrs,
  \C,\NIZKproof)$, and return $\bot$ if it fails. Else, create the credential
  by computing $\SBCMsig \gets \SBCMSign(\isk,\C,\attrs)$. Send \SBCMsig to the
  user, and output $\utrans \gets (\C,\SBCMsig,\attrs,\NIZKproof)$.
\item \underline{User}: Check the signature by running $\SBCMVerify(\ipk,
  \SBCMsig,\attrs \cup \lbrace \usk \rbrace)$, and return $\bot$ if
  verification fails. Otherwise, return $\cred \gets \SBCMsig$.
\end{itemize}
\fi

\paragraph{$\Sign(\gpk,\usk,\cred,\dattrs,\msg) \rightarrow \sig$.} %
Commit to the message and recompute the user public key by running
$\Cmsg \gets \CCommit(\msg), \Ccom \gets \CCommit(\usk; 0)$.
Encrypt \Ccom as $\Ec \gets \EEnc(\opk,\Ccom)$ and create a NIZK proof using
$\NIZKRel_{\Sign}$ as $\NIZKproof \gets \NIZKProve^{\NIZKRel_{\Sign}}
(\NIZKcrs,(\usk,\Ccom,\attrs,\msg,\cred),
(\Cmsg,\Ec,\dattrs))$. Output $\sig \gets (\Ec,\NIZKproof)$.

\paragraph{$\Verify(\gpk,\sig,\dattrs,\msg) \rightarrow 1/0$.} %
Parse \sig as $(\Ec,\NIZKproof)$, compute $\Cmsg \gets \CCommit(\msg)$ and
return $\NIZKVerify^{\NIZKRel_{\Sign}}(\NIZKcrs,(\Cmsg,\Ec,\dattrs),
\NIZKproof)$.

\paragraph{$\Open(\gpk,\osk,\trans,\sig,\dattrs,\msg)
  \rightarrow (\upk,\oproof)/\bot$.} %
First, verify the signature with $\Verify(\gpk,\sig,\dattrs,\msg)$ and
return $\bot$ if verification fails. Else, parse \sig as $(\Ec,\NIZKproof)$
and run $\upk \gets \EDec(\osk,\Ec)$. Compute proof of correct decryption
as $\oproof_{\Open} \gets \NIZKProve^{\NIZKRel_{\Open}}(\NIZKcrs,\osk,(\Ec,
\upk))$. Return $(\upk,\oproof_{\Open})$.

\paragraph{$\Judge(\gpk,\upk,\attrs,\oproof,\sig,\dattrs,\msg)
  \rightarrow 1/0$.} %
First, verify the signature with $\Verify(\gpk,\sig,\dattrs,\msg)$ and
return $\bot$ if verification fails. If verification suceeds, parse
\sig as $(\Ec,\cdot)$ and return $\NIZKVerify^{\NIZKRel_{\Open}}(\NIZKcrs,(\Ec,
\upk))$.


%%% Local Variables:
%%% mode: latex
%%% TeX-master: "uas"
%%% End:


%%% Local Variables:
%%% mode: latex
%%% TeX-master: "uas"
%%% End:

\section{From \GSAC to \UAS}
\label{sec:uas}

\GSAC is a very interesting exercise that showcases the advantages of direct
combination of group signatures and anonymous credentials. However, it also
inherits a main limitation: it is too strict in regards to the utility it
offers. The most excruciating example of this fact is that in \GSAC, via \Open,
one can only get the actual identity of the signer. Even though the inclusion
of attributes makes the concept of ``identity'' meaningful, we
may not want to fully de-anonymize the signer. Instead, we may be interested in
revealing arbitrary functions of its identity and the signed message. Also, for
the sake of illustation, we restricted to the case of selective disclosure, and
to only one credential per signature. Then, the natural question is whether we
can model an extension to \GSAC that is compatible both with constructions with
that support selective disclosure, and with constructions that support arbitrary
claims on the
credential attributes; and, as it is also common in the anonymous credentials
(or verifiable credentials\footnote{\url{https://www.w3.org/TR/vc-data-model/}.
  Last access, May 5th, 2022.}) domains, to allow credential holders to
combine multiple credentials into one signature/presentation. Similarly, we can
also generalize \GSAC by allowing users to leverage previously obtained
credentials in order to obtain new ones; possibly, applying arbitrary issuance
policies. In a nutshell, our generalization of \GSAC into Universal Anonymous
Signatures (\UAS), enables:

\begin{itemize}
\item Using previously obtained credentials to request new ones, applying
  arbitrary policies to the credentials involved in the request, in
  order to determine whether the new issuance should proceed or not.
\item Modulating the type of utility one wants to get directly from signatures.
  That is, let signatures disclose a subset of the attributes in the credentials
  involved in the signing process, prove arbitrary claims over those attributes,
  or even reveal absolutely nothing at all beyond that that an owner of a valid
  credential created the signature.
\item Defining arbitrary de-anonymization functions, over the signer's
  credentials and signed message. This can range from producing the user public
  key (as in group signatures), to disallowing any kind of de-anonymization,
  passing through any intermediate de-anonymization function, like revealing
  concrete attributes, or functions of them.
\end{itemize}

Furthermore, to make the concept more flexible, we no longer restrict to the
case of one issuer and one opener. Instead, \UAS supports multiple issuers (as
in generalized variants of anonymous credentials), as well as multiple openers,
each one with a predefined open function. Thus, it supports use cases like
having a user produce a signature with credentials obtained from issuers A and
B, which can be inspected by opener C; but later use the same credentials
from issuers A and B, to produce a signature that can be inspected by opener D.
A consequence of this is that the concept of ``group'', which was very present
in \GSAC, is now blurred in \UAS, and we get closer to the context of anonymous
credentials -- while still offering the accountability we inherit from \GSAC
via \Open and \Judge.

\subsection{Model for \UAS Schemes}
\label{ssec:model-uas}

We now define our model for Universal Anonymous Signatures. Property-wise, we
maintain anonymity, non-frameability, and include two extra unforgeability
properties: unforgeability of issuance, which ensures that issuance policies
are not circumvented; and unforgeability of signatures, guaranteeing that
signature evaluation policies are respected, as well as soundness of opening.

From a functional perspective, we require that each issuer and opener fix
the issuance predicate \fissue and opener function \finsp when generating
their public keys. On the other hand, signature evaluation functions \feval can
be defined at signing time. This is in line with the usual practice in anonymous
credentials, that let
users prove arbitrary claims on their credentials, as long as they are met by
the contained attributes. In practice, most probably, the signature evaluation
policy to be
employed in each signature will be defined by the verifier (or jointly between
user and verifier), including the specification of which opener will be
able to post-process signatures\footnote{See, for instance, the concept of
  presentation exchanges, or requests, in the context of Verifiable Credentials:
  \url{https://identity.foundation/presentation-exchange/} (last access, May
  5th, 2022).}. Beyond these general policies, which already give much
flexibility, the syntax for \UAS supports multiple issuers and openers, and
lets users leverage multiple credentials to request issuance of new ones, and
to produce signatures.

\subsubsection{Syntax.} An \UAS scheme is composed by the following algorithms,
where for readability we assume that issuer public keys (\ipk) are extractable
from credentials:

\begin{description}
\item[$\parm \gets \Setup(\secpar)$.] Given a security parameter \secpar,
  returns a global system parameter variable \parm. We assume that \parm are
  available to all the other functions, even if not explicitly listed in their
  input parameters.
\item[$(\ipk,\isk) \gets \IKeyGen(\parm,\fissue)$.] Given global system
  parameters \parm, and the function \fissue to be used to check that credential
  requestors meet the conditions to be issued a credential, an issuer runs
  \IKeyGen to generate its issuing key pair. 
\item[$(\opk,\osk) \gets \OKeyGen(\parm,\finsp)$.] Given global system
  parameters \parm, and function \finsp, an opener runs \OKeyGen to generate
  its opening key pair. The function \finsp defines the type of utility that
  will be extractable from signatures.
\item[$\usk \gets \UKeyGen(\parm)$.] Given global system parameters
  \parm, returns a user's secret key.
\item[$\langle \cred/\bot,\utrans/\bot \rangle \gets
  \langle
  \Obtain(\usk,\scred,\attrs),
  \Issue(\isk,\sipk,\attrs))
  \rangle$.] %
  This interactive protocol lets a user with key \usk running the
  \Obtain process, receive a credential \cred from an issuer in the system, on
  attribute set $\attrs$. The user leverages a set of credentials \scred,
  each with a matching issuer key in \sipk.
  The user outputs the produced credential \cred, while
  the issuer outputs the protocol transcript \utrans for the produced
  credential.
\item[$(\sig,\yeval) \gets \Sign(\usk,\opk,\scred,\msg,\feval)$.] %
  Upon receiving a user secret key \usk, opener public key \opk, a set of
  credentials \scred, a message \msg and evaluation
  function \feval, returns signature \sig and value \yeval.
\item[$1/0 \gets \Verify(\opk, \sipk,\sig,\yeval,\msg,\feval)$.]
  Checks whether $(\sig,\yeval)$ is a valid signature
  over message \msg, from a user with credentials issued by issuers with public
  keys in \sipk, for evaluation function \feval and opener key \opk.
\item[$(\yinsp,\iproof)/\bot \gets \Open(\osk,\sipk,\sig,\yeval,\msg,\feval)$.]
  Executed by the opener with private key \osk. Receives a signature $(\sig,
  \yeval)$ over message \msg and evaluation function \feval, generated using
  credentials issued by the issuers with public keys in \sipk. If \sig is valid,
  the function outputs a value $\yinsp$.
\item[$1/0 \gets \Judge(\opk,\sipk,\yinsp,\iproof,\sig,\yeval,\msg,\feval)$.] %
  Checks if \iproof is a valid opening correctness proof for the value \yinsp,
  obtained by applying \Open to the the signature $(\sig,\yeval)$ over
  message \msg, and for evaluation function \feval. 
\end{description}

\paragraph{Issuance, evaluation, and opening functions.} %
We emphasize that, both in our syntax definition, as well as on the following
modelling, we make use of three different and abstract functions: \fissue,
\feval and \finsp. The three functions are introduced to allow customized
governance of the resulting instantiation of an \UAS scheme. They will be
defined by different parties, but in all cases, they are run by users (maybe,
on user-private data). Also, in all cases, the user has to prove correctness of
their computation. We introduce them next, and give concrete examples for
building specific well-known restrictions of \UAS in
\secref{sec:transformations}.

\begin{description}
\item[$\fissue: (\usk,\scred,\attrs)
  \rightarrow 0/1$.] Chosen by issuers within a family of functions \famfissue,
  the issuance function defines what customized conditions an issuer requires
  in order to issue credentials, when receiving a request from user with secret
  key \usk, for attributes \attrs. \fissue may run checks on a (possibly empty)
  set of additional credentials \scred, all bound to \usk, and possibly
  issued by other issuers. \fissue returns $1$ to accept a request, $0$ to
  reject it.
\item[$\feval: (\usk,\scred,\msg)
  \rightarrow \yeval$.] Signing evaluation functions, from a family of functions
  \famfeval, can be set on a per-signature basis. They receive the user secret
  key \usk, credentials \scred, and message to be signed \msg. \feval can be
  used to control the information related to the signer that will be revealed
  alongside a signature, and to modulate the behaviour of \finsp. Its outputs
  \yeval must belong in a well defined set \rngfeval.
\item[$\finsp: (\yeval,\usk,\scred,\msg) \rightarrow \yinsp$.]
  Chosen by openers from a family of functions \famfinsp. The opening
  functions define what utility value, derived from the user's secret key,
  credentials, and signed message, should be extractable by an opener.
  Note that \finsp also receives as input a value in the range of the \feval
  function, \rngfeval. This allows opening logic to depend on the value
  produced by the evaluation function. The output of \finsp is a value
  \yinsp, which must belong in a well defined set \rngfinsp.
\end{description}

We emphasize that, even though \finsp and \feval seem redundant, they are not.
To see this, observe that \finsp is defined by openers, and will be fixed in
all signatures that can be opened by the opener who defined it. On the other
hand, \feval can be defined by (e.g.) verifiers on a per-signature basis, even
if the signatures use the same \finsp. Thus, \feval can be programmed to contain
the conditions set by (e.g.) verifiers for signatures they receive; whereas
\finsp can be programmed to extract specific utility values when needed, which
may depend on the checks required by the verifier a signature was intended to.

\paragraph{Helper functions \ExtractIssue, \ExtractSign, \IdentifyCred, and
  \IdentifyUK.} In our modelling, we assume the existence of fhese four
functions. They are not
functions available in the actual scheme, but rather to the challenger in the
experiments we use to formalize security of \UAS schemes. Similar techniques
have been used before to prove security in privacy-preserving schemes with some
sort of accountability, but that do not offer conventional opening as vanilla
group signatures. For instance, see related works on DAA \cite{bfg+11,cdl16} and
group signature variants \cite{dl21,fgl21,gl19,lnpy21}. More concretely, these
functions are as follows:

\begin{description}
\item[$\ExtractIssue(\utrans) \rightarrow (\usk,\attrs_{\scred},\scred)$.]
  Receives an $\langle \Obtain, \Issue \rangle$ transcript, and returns the
  credentials (if any) and their attributes that were involved
  in the request. It clearly needs an honest issuer as, otherwise, the
  transcripts won't be available. Consequently, we only use it to define the
  properties that require an honest issuer.
\item[$\ExtractSign(\oid,\siid,\sig,\yeval,\msg,\feval) \rightarrow (\usk,
  \scred,\attrs_{\scred},\yinsp)$.] Receives a signature pair $(\sig,\yeval)$,
  as well as the opener identifier \oid, and the identifiers of all issuers of
  the credentials used to produce the signature over \msg, and for \feval. It
  outputs the user secret key and credentials (with their attributes) used to
  generate the signature, and the value returned by the opening fuction.
\item[$\IdentifyCred(\usk,\attrs_{\cred},\cred)$.] Returns $1$ if \cred has been
  issued over attributes $\attrs_{\cred}$ and for a user with secret key \usk.
  Otherwise, it returns $0$. In order to be meaningful, this requires that,
  for every $(\attrs_{\cred},\cred)$ pair, there is at most one \usk that makes
  \IdentifyCred return $1$.
\item[$\IdentifyUK(\uid,\usk)$.] Returns $1$ is \usk is {\uid}'s secret key.
  This is trivial for honest users, for which there must be a one-to-one
  relationship between {\uid}s and {\usk}s. For corrupt users, \IdentifyUK has
  to iterate through the $\langle\Obtain,\Issue\rangle$ transcripts associated
  to \uid, extract the used secret key, and check if there is a match. Note that
  this does not guarantee that there will be only one \usk per corrupt \uid,
  though. Also, when used for corrupt users, this can only be used when the
  issuer is honest, as transcripts are needed. 
\end{description}

Note that, for all the helper functions, in the case of credentials, transcripts,
and signatures by honest users, it is enough to have access to the corresponding
state information (described below) maintained by the challenger in our
experiments. For credentials and join transcripts of corrupt users, or
dishonestly produced signatures, we do need to perform actual extraction.
Certainly, the challenger
needs special knowledge/power such as decryption trapdoors, the ability to
rewind the game, or program random oracles. The approach needs thus to depend on
the concrete construction. \jdv{Although, for the case of \ExtractIssue and
  \IdentifyUK, online extractability (or alternative requirements, such as
  non-parallel or logarithmic number of joins) is necessary.}

\paragraph{Global Variables.} %
The environment manages several global variables in the games posed to the
adversary. Users are referred to with user identifiers, \uid; for credentials,
we use \cid; for issuers, \iid; and for openers, \oid. In all cases, we use bold
font to denote sets: e.g., \scid and \siid denote sets of credential and issuer
identifiers. All tables/sets are initialized as empty tables/sets.

\begin{description}
\item[Sets for parties]:
  \begin{description}
  \item[\HU and \CU.] Keep track of honest (\HU) and corrupted (\CU) users;
    i.e., they are sets of {\uid}s.
  \item[\HI and \CI.] Keep track of honest (\HI) and corrupted (\CI) issuers;
    i.e., they are sets of {\iid}s.
  \item[\HO and \CO.] Keep track of honest (\HO) and corrupted (\CO) openers;
    i.e., they are sets of {\oid}s.
  \end{description}
\item[Tables for keys]:
  \begin{description}
  \item[\UK.] \UK maintains user keys $\usk$. To refer to the key of a specific
    user, we use $\UK[\uid]$. 
  \item[\IK, \PUBIK and \PRVIK.] \IK maintains issuer key pairs, where
    $\IK[\iid]$ refers to the key pair of the issuer with identifier \iid. We
    use \PUBIK to refer to the public component, which also includes the \fissue
    function; and \PRVIK refers to the private component of the key pair.
  \item[\OK, \PUBOK, \PRVOK.] Same as \IK, but for opener key pairs. Instead
    of \fissue, \OK includes the \finsp function.
  \end{description}
\item[Tables for credentials-related data]:
  \begin{description}
  \item[\CRED.] Stores information related to credentials obtained by users in
    the system. Thus, it is indexable by \cid. More specifically, it stores
    tuples of the form $(\uid,\cred,\iid,\attrs,\scid)$, where \uid is the
    identity of the owner of the credential, \cred (when available) is the
    credential itself, \iid is the identifier of the credential issuer, \attrs
    are the attributes included in \cred, and \scid are the identifiers of the
    credentials (if any) that \uid used to request \cred. For notational
    convenience, we may use $\CRED[\scid]$ to refer to $\CRED[\cid]$ for all
    $\cid \in \scid$. Also, when clear from context, we sometimes use
    $\CRED[\cid]$ (resp. $\CRED[\scid]$ to mean \cred (resp. \scred) in
    $\CRED[\cid] = (\cdot,\cred,\cdot,\cdot,\cdot)$ (resp. $\CRED[\scid]$).
  \item[\OWNR.] For notational convenience, when we write $\OWNR[\cid]$ we mean
    ``\uid such that $\CRED[\cid] = (\uid, \cdot, \cdot, \cdot, \cdot)$''.
  \item[\ATTR.] For notational convenience, when we write $\ATTR[\cid]$ we mean
    ``\attrs such that $\CRED[\cid] = (\cdot, \cdot, \cdot, \attrs, \cdot)$''.
  \item[\ISR.] For notational convenience, when we write $\ISR[\cid]$ we mean
    ``\iid such that $\CRED[\cid] = (\cdot, \iid, \cdot, \cdot, \cdot)$''.
  \end{description}
\item[Tables for signatures]:
  \begin{description}
  \item[\SIG.] Maintains signatures generated via the \SIGN oracle, on behalf
    of honest users. Entries of this table are $(\oid,\scid,\sig,\yeval,\msg,
    \feval)$, where \oid is the opener chosen for the signature, \scid is the
    set of credentials used for signing, \feval is the signing evaluation
    function, and \sig and \msg are the produced signature and signed message.
  \item[\CSIG.] Maintains challenge signatures output to the adversary; i.e.,
    the table is indexable by challenge signatures \csig.
    Each entry contains also $\cuid_b$ and $\scid_b$ the challenge user and
    credential identifiers set used to produce \csig; as well as the
    corresponding challenge user and credential set indexed by the complementary
    $1-b$; the signed message \msg and signing evaluation function \feval, the
    result of \feval, \yeval, and the opener identifier \oid.
  \end{description}
\end{description}

\paragraph{Oracles.} %
In the game-based definitions of our \UAS model, we leverage the following
oracles, which are formally defined in \figref{fig:oracles1} and
\figref{fig:oracles2}. 

\begin{description}
\item[\IGEN.] Adds a new issuer to the game, generating its keypair and setting
  the associated issuance function.
\item[\OGEN.] Adds a new opener to the game, generating its key pair and
  setting the associated evaluation and inspection functions.
\item[\ICORR.] Corrupts an existing (and honest) issuer, by giving its secret
  key to the adversary.
\item[\OCORR.] Corrupts an existing (and honest) opener, by giving its secret
  key to the adversary.  
\item[\HUGEN.] Adds a new honest user to the game, by honestly generating
  the user's key pair.
\item[\CUGEN.] Adds a new corrupt user to the game or, if the specified
  user already exists and is honest, corrupts it, leaking its key and
  credentials.
\item[\RREG.] Reads the given transcript table entry.
\item[\WREG.] Sets a transcript table entry to the given value.
\item[\OBTISS.] Lets the adversary add a new honestly generated credential to
  the game, on behalf of an honest user.
\item[\OBTAIN.] Enables the adversary to play the role of a dishonest issuer
  in games that support it, by letting it interact with honest users who want to
  receive credentials.
\item[\ISSUE.] Allows the adversary to play the role of dishonest users,
  requesting an honest issuer to produce credentials for them.
\item[\SIGN.] Lets the adversary get signatures from credentials belonging
  to honest users.
\item[\OPEN.] Given an honestly produced signature, outputs the result of the
  opening function, along with a correctness proof.
\item[\CHALb.] Upon receiving two challenge users and credential sets, a common
  singing evaluation function and a message, returns a signature produced by one
  of these two user and credential sets, defined by the bit $b$, which is
  established in the anonymity game.
\end{description}

{%\setlength\intextsep{\sep}
  \begin{figure*}[htp!]
    \centering
    \scalebox{0.9}{

      \begin{minipage}[t]{0.55\textwidth}       

        \procedure{$\IGEN(\iid,\fissue)$}{%
          \pcif \iid \in \HI \lor \iid \in \CI: \pcreturn \bot \\
          \pcif \fissue \notin \famfissue: \pcreturn \bot \\
          (\ipk,\isk) \gets \IKeyGen(\parm) \\
          \IK[\iid] \gets ((\ipk,\fissue),\isk) \\
          \HI \gets \HI \cup \lbrace \iid \rbrace \\
          \pcreturn \ipk \\
        }

        \procedure{$\ICORR(\iid)$}{%
          \pcif \iid \in \CI \lor \iid \notin \HI: \pcreturn \bot \\
          \HI \gets \HI \setminus \lbrace \iid \rbrace \\
          \CI \gets \CI \cup \lbrace \gid \rbrace \\
          \pcreturn \isk \\
        }        

        \procedure{$\HUGEN(\uid)$}{%
          \pcif \uid \in \HU \lor \uid \in \CU: \pcreturn \bot \\
          \usk \gets \UKeyGen(\parm) \\
          \UK[\uid] \gets \usk;
          \HU \gets \HU \cup \lbrace  \uid \rbrace \\
          \pcreturn \top \\
        }

        \procedure{$\RREG(i)$}{%
          \pcreturn \trans[i]
        }        
        
      \end{minipage}
    }
    \scalebox{0.9}{
      
      \begin{minipage}[t]{.5\textwidth}

        \procedure{$\OGEN(\oid,\finsp)$}{%
          \pcif \oid \in \HO \lor \oid \in \CO: \pcreturn \bot \\
          \pcif \finsp \notin \famfinsp: \pcreturn \bot \\
          (\opk,\osk) \gets \OKeyGen(\parm) \\
          \OK[\oid] \gets ((\opk,\finsp),\osk) \\
          \HO \gets \HO \cup \lbrace \oid \rbrace \\
          \pcreturn \opk \\
        }

        \procedure{$\OCORR(\oid)$}{%
          \pcif \oid \in \CO \lor \oid \notin \HO: \pcreturn \bot \\
          \HO \gets \HO \setminus \lbrace \oid \rbrace \\
          \CO \gets \CO \cup \lbrace \oid \rbrace \\
          \pcreturn \osk \\
        }        
        
        \procedure{$\CUGEN(\uid)$}{%          
          \pcif \uid \in \CU: \pcreturn \bot \\
          \CU \gets \CU \cup \lbrace \uid \rbrace \\          
          \pcif \uid \in \HU: \\
          \pcind \HU \gets \HU \setminus \lbrace \uid \rbrace; \\
          \pcind \pcreturn (\UK[\uid],\CRED[\uid]) \\
          \pcelse: \UK[\uid] = \bot \\          
          \pcreturn \top \\
        }

        \procedure{$\WREG(i,\rho)$}{%
          \trans[i] \gets \rho
        }        
        
      \end{minipage}
      
    }

    \caption{Detailed oracles available in our model (1/2). Oracles for
      generating key material for users, issuers, and openers.}
    \label{fig:oracles1}
  \end{figure*}
}

{%\setlength\intextsep{\sep}
  \begin{figure*}[htp!]
    \centering
    \scalebox{0.9}{

      \begin{minipage}[t]{0.55\textwidth}

        \procedure{$\ISSUE(\cid,\uid,\iid,\attrs,\siid)$}{%
          \pcif \uid \notin \CU: \pcreturn \bot \\          
          \pcif \iid \notin \HI: \pcreturn \bot \\
          %\pcif \exists \iid' \in \ISR[\scid] \notin \HI: \pcreturn \bot \\
          \pcif \CRED[\cid] \neq \bot: \pcreturn \bot \\
          \langle \cdot, \utrans \rangle \gets
          \langle \adv, 
          \Issue(\PRVIK[\iid],\siid,\attrs) \rangle \\
          \trans[\cid] \gets \utrans \\
          \CRED[\cid] \gets (\uid, \cdot, \iid, \attrs, \cdot, \siid) \\
          \pcreturn \top \\          
        }                

        \procedure{$\OBTAIN(\cid,\uid,\iid,\attrs,\scid)$}{%
          \pcif \uid \notin \HU: \pcreturn \bot \\
          \pcif \iid \notin \CI: \pcreturn \bot \\
          \pcif \CRED[\cid] \neq \bot: \pcreturn \bot \\
          %\pcif \exists \cid' \in \scid~\st~\CRED[\cid'] = \bot: \pcreturn \bot \\
          \langle \cred, \cdot \rangle \gets
          \langle \Obtain(\UK[\uid],\CRED[\scid],\attrs),\adv \rangle \\
          \CRED[\cid] \gets (\uid, \cred, \iid, \attrs, \scid, \siid) \\
          \pcreturn \top \\
        }

        \procedure{$\CHALb(\oid,\cuid_{0,1},\cscid_{0,1},\msg,\feval)$}{%
          \pcif \cuid_0 \notin \HU \lor \cuid_1 \notin \HU: \pcreturn \bot \\
          \pcif \oid \in \CO: \pcreturn \bot \\
          \pcif \yeval = \feval(\UK[\cuid_0],\CRED[\cscid_0],\msg) \neq \\
          \pcind \feval(\UK[\cuid_1],\CRED[\cscid_1],\msg):
          \pcreturn \bot \\
          \pcif \PUBIK[\cscid_0] \neq \PUBIK[\cscid_1]: \pcreturn \bot \\
          (\yeval,\csig_b) \gets \Sign(\UK[\cuid_b],\PUBOK[\oid], \\
          \hspace*{71pt}\CRED[\cscid_b],\msg,\feval) \\
          (\yeval,\csig_{1-b}) \gets \Sign(\UK[\cuid_{1-b}],\PUBOK[\oid], \\
          \hspace*{80pt}\CRED[\cscid_{1-b}],\msg,\feval) \\          
          \CSIG[\csig_b] \gets 
          \lbrace (\oid,\cuid_b,\cscid_b,\msg,\feval,\yeval,\\
          \hspace*{74pt}\cuid_{1-b},\csig_{1-b},\cscid_{1-b})
          \rbrace \\
          \pcreturn (\csig_b,\yeval)
        }        
        
      \end{minipage}
    }
    \scalebox{0.9}{
      
      \begin{minipage}[t]{.5\textwidth}

        \procedure{$\OBTISS(\cid,\uid,\iid,\attrs,\scid)$}{%
          \pcif \uid \notin \HU: \pcreturn \bot \\
          \pcif \iid \notin \HI: \pcreturn \bot \\
          \pcif \CRED[\cid] \neq \bot: \pcreturn \bot \\
          %\pcif \exists \iid' \in \ISR[\scid]~\st~\iid' \notin \HI: \pcreturn \bot \\
          %\pcif \exists \cid' \in \scid~\st~\CRED[\cid'] = \bot:
          \pcreturn \bot \\
          \langle \cred, \utrans \rangle \gets
          \langle \Obtain(\UK[\uid],\CRED[\scid],\attrs), \\
          \hspace*{60pt} \Issue(\PRVIK[\iid],\ISR[\scid],\attrs)
          \rangle \\
          \trans[\cid] \gets \utrans \\
          \CRED[\cid] \gets (\uid, \cred, \iid, \attrs, \scid, \siid) \\
          \pcreturn \top \\
        }

        \procedure{$\SIGN(\oid,\uid,\scid,\msg,\feval)$}{%
          \pcif \uid \notin \HU: \pcreturn \bot \\
          (\sig,\yeval) \gets \Sign(\UK[\uid],\PUBOK[\oid],\CRED[\scid],\msg,
          \feval) \\
          \SIG[\uid] \gets \SIG[\uid] \cup
          \lbrace (\oid,\scid,\sig,\yeval,\msg,\feval) \rbrace \\
          \pcreturn (\sig,\yeval) \\
        }                

        \procedure{$\OPEN(\oid,\sig,\yeval,\msg)$}{%
          \textrm{Let}~\uid~\textrm{be s.t.}~(\oid,\scid,\sig,\yeval,\msg,\feval)
          \in \SIG[\uid] \\
          (\yinsp,\iproof) \gets
          \Open(\PRVOK[\oid],\PUBIK[\scid],\sig,\yeval,\msg,\feval) \\
          \pcif \CSIG[\sig] \neq \bot: \\
          \pcind \textrm{Parse $\CSIG[\sig]$ as $(\oid,\cuid_b,\scid_b,\msg,
            \feval,\yeval$} \\
          \hspace*{83pt}\cuid_{1-b},\csig_{1-b},\scid_{1-b}) \\
          \pcind (\yinsp',\iproof') \gets
          \Open(\PRVOK[\oid],\IK[\siid],\\
          \hspace*{107pt} \sig_{1-b},\msg,\feval) \\
          \pcind \pcif \yinsp' \neq \yinsp: \pcreturn \bot \\
          \pcreturn (\yinsp,\iproof)
        }
        
      \end{minipage}
      
    }

    \caption{Detailed oracles available in our model (2/2). Oracles for
      obtaining credentials, signatures, and processing them.}
    \label{fig:oracles2}
  \end{figure*}
}

\paragraph{Correctness.} %
Correctness of \UAS schemes is formalized through the experiment in
\figref{fig:exp-uas-corr}. It states that a signature over any arbitrary message
and valid function \feval, produced honestly leveraging credential set \scid,
owned by user \uid, is accepted by \Verify. Moreover, all the credentials in
\scid meet the conditions set by the corresponding \fissue defined by the issuer
which issued each credential. Similarly, the output \yeval of \feval matches the
value produced by \Sign alongside with \sig; and the value produced by \Open
is accepted by \Judge, and matches the output of applying \finsp on \yeval, the
credentials, user key, and message.

\begin{definition}{(Correctness of \UAS)}
  \label{def:correctness-uas}
  An \UAS scheme is correct if, for any p.p.t. adversary $\adv$,
  $\ExpCorrect(1^\secpar)$ outputs 1 with negligible probability.
\end{definition}

\begin{figure}[htp!]
  \procedure[linenumbering]{$\ExpCorrect(1^\secpar)$}{%
    \parm \gets \Setup(1^\secpar) \\
    (\uid,\oid,\scid,\msg,\feval)
    \gets \adv^{\IGEN,\OGEN,\HUGEN,\OBTISS,\RREG}(\parm) \\
    \pcif \feval \notin \famfeval: \pcreturn 0 \\
    \pcif \OWNR[\scid] \neq \uid: \pcreturn 0 \\
    (\sig,\yeval) \gets \Sign(\UK[\uid],\PUBOK[\oid],\scid,\msg,\feval) \\
    \pcif \Verify(\PUBOK[\oid],\PUBIK[\scid],\sig,\msg,\feval) = 0: \pcreturn 1 \\
    \pcfor \cid \in \scid \pcdo: \\
    \pcind \textrm{Let}~\scred^{\cid}~\textrm{be the credentials used to obtain}
    ~\cid;~\textrm{Parse}~\PUBIK[\ISR[\cid]]~\textrm{as}~((\cdot,\fissue^{\cid}),\cdot)\\
    \pcind \pcif \fissue^{\cid}(\UK[\uid],\scred^{\cid},\ATTR[\cid]) = 0: \pcreturn 1 \\
    \pcif \feval(\UK[\uid],\CRED[\scid],\msg) \neq \yeval: \pcreturn 1 \\
    (\yinsp,\iproof) \gets \Open(\PRVOK[\gid],\PUBIK[\scid],\sig,\msg,\feval) \\
    \pcif \Judge(\PUBOK[\oid],\PUBIK[\scid],\y,\iproof,\sig,\yeval,\msg,\feval)
    = 0 \lor \yinsp \neq \finsp^\gid(\yeval,\UK[\uid],\CRED[\scid],\msg)): \\
    \pcind \pcreturn 1 \\
    \pcreturn 0
  }  
  \caption{Correctness experiment for \UAS schemes.}
  \label{fig:exp-uas-corr}
\end{figure}

\subsubsection{Security Properties}
\label{sssec:security}

\paragraph{Anonymity.} %
In addition to the considerations made for \GSAC (see \secref{ssec:model-gsac}),
in our notion of anonymity for \UAS,
in order to prevent trivial wins by the adversary, we have to restrict that the
signing evaluation function outputs the same value for both user-credentials
pairs. Very interestingly, since our opening functionality does not
(necessarily) output the identity of the signer, \uline{we can even allow the
  adversary to open challenge signatures, as long as they produce the same
  output}. Other than this, the overall approach is similar: as
in group signatures and anonymous credentials, the adversary picks two pairs of
honest users and credential sets. We also require the adversary to pick an
evaluation function and an opener (hence, an opening function). Then, the
user-credential pair defind by the value of $b$ (unknown to the adversary) is
used to program the \CHALb oracle. The formal specification of the anonymity
game is given in \figref{fig:exp-uas-anonb}, where $\Oanonc \gets (\lbrace\HU,
\CU\rbrace\GEN,\lbrace\II,\OO\rbrace\GEN,\lbrace\II,\OO\rbrace\CORR,\OBTAIN,
\WREG,\SIGN,\OPEN)$ and $\Oanong \gets (\lbrace\HU,\CU\rbrace\GEN,\lbrace\II,
\OO\rbrace\GEN,\lbrace\II,\OO\rbrace\CORR,\OBTAIN,\WREG,\SIGN,\OPEN,\CHALb)$

\begin{figure}[htp!]
  \procedure[linenumbering]{$\ExpAnonb(1^\secpar)$}{%
     \parm \gets \Setup(1^\secpar) \\
     (\cuid_0,\cscid_0,\cuid_1,\cscid_1,\feval,\status) \gets \adv^{\Oanonc}
     (\choose,\parm) \\
     \pcif \feval \notin \famfeval: \pcreturn \bot \\
     b^* \gets \adv^{\Oanong} (\guess,\status) \\
     \pcreturn b^*
  }
  \caption{Anonymity experiment for \UAS schemes.}
  \label{fig:exp-uas-anonb}
\end{figure}

\begin{definition}{(Anonymity of \UAS)}
  \label{def:anonymity-uas}  
  We define the advantage \AdvAnon of $\adv$ against \ExpAnonb as
  $\AdvAnon=|\Pr\lbrack\ExpAnono(1^\secpar)=1\rbrack-
  \Pr\lbrack\ExpAnonz(1^\secpar)=1\rbrack|$.
  %
  An \UAS scheme satisfies anonymity if, for any p.p.t. adversary $\adv$,
  \AdvAnon is a negligible function of $1^\secpar$.
\end{definition}

\paragraph{Discussion on the generality of anonymity in \UAS schemes.} %
This notion of anonymity is more general than that of group signatures in the
sense that calls to the \OPEN oracle reveal an arbitrary function of the
identity of the user -- which can certainly be the member index itself, as it
is frequent in group signatures, or any other function computable from the
user key, its credentials, and signed message. Moreover, note that
our notion is even stronger than the conventional ``CCA-like'' anonymity notion
that gives the adversary access to the open oracle, only restricting it when
trying to open challenge signature. Precisely the introduction of generic
opening functions allows us to let the adversary open challenge signatures
$\csig_b$ as long as the counterpart $\csig_{1-b}$
makes \Open produce the same \y value as output. This is certainly not
possible when using conventional opening, since it directly outputs the identity
of the signer -- which cannot be the same for different challenge users.

Note that we could actually remove the requirement that the evaluation function
has to produce the same output on both challenge user-credentials pairs.
However, by doing so, we would force all constructions to maintain private the
output of \feval. We opt not to require that, though, as it directly allows
our model to cover interesting use cases -- such as restricting to selective
disclosure, or privacy-preserving variants of functional signatures \jdv{check
  the latter}, although we show how to do that in \secref{sec:transformations}.

\paragraph{Unforgeability.} In anonymous credentials, unforgeability requires
that no adversary can succeed in a credential presentation for attributes that
are not contained in a legitimately issued credential (set) it controls. Note
that, for this, assuming honest issuers is essential as, otherwise, the
adversary can get credentials on any attribute set it wants.
%
The somehow equivalent property in group signatures, inasmuch it also requires
honest issuers, is traceability. It captures the security of the
open-related functionality (e.g., \Open and \Judge) over signatures
that produced by potentially malicious signers, in the presence of an honest
issuer. In a nutshell, it ensures that every signature accepted by
the verification algorithm must have been created by a user who joined the
group, and that the result of \Open (and thus, \Judge) over such a valid
signature is consistent with its signer. With conventional opening, this is
essentially checked by requesting the adversary to produce a signature,
obtaining -- via open -- the ``identifier'' of the user who produced the
signature. Schemes
that do not have conventional opening resort to more subtle techniques, like
matching keys extracted during join transcripts, with keys used for signing
(see, e.g., \cite{dl21}). Thus, even though for more subtle reasons (the need
for reliable bookkeeping during joins) than for anonymous credentials, honest
issuers are required for traceability of group signatures too.

In our \UAS scheme we allow users prove claims over the attributes they
own -- attested via obtained credentials -- through the \feval function.
Roughly, we capture this as in unforgeability requirement of anonymous
credentials. Namely, we check that the \yeval value returned by \Sign along with
the signature, matches the expected one from the specified \feval function.
On the other hand, the opening capabilities of \UAS schemes also call for a
traceability-like property. However, note that, as opposed to group signatures,
$\langle \Obtain,\Issue \rangle$ protocols are over credentials rather than
users -- although, ultimately, credentials must be owned by some user.
Furthermore, since \Open
does not return the actual identity of the signer, but a function \finsp of it
(and other arguments), we need to make sure that the output of \Open
matches the output of \finsp. As in group signatures with non-conventional open,
we resort to extraction-based techniques. All this is captured via \ExpForgeSign
in \figref{fig:exp-uas-unfor-issue}. Therein, the adversary is challenged to
produce a signature $(\sig,\yeval)$ over message \msg and for evaluation
function \feval; as well as the identifiers for the opener (\oid) and
credential issuers (\siid) used to compute the signature. As in traditional
group signatures, the adversary wins if the signature is accepted by \Verify,
yet \Open or \Judge fail. Then, the game extracts the user key and
credentials used to produce the signature. From it, the game checks if the
output of \feval matches the \yeval value produced by the adversary -- this
mimics the behaviour of unforgeability in anonymous credentials. Finally, the
game also checks that the output of \Open (even if accepted by \Judge)
matches the output of \finsp. If there is any mismatch in the last two checks,
or some of the credentials used to produce the signature do not correspond to
the secret key that was allegedly used to request them, the adversary wins the
game.

Our \UAS scheme includes yet another generalisation that requires unforgeability-like
security, though. Concretely, the issuance function \fissue. We need to make
sure that no credential is issued unless its corresponding
request meets the defined issuance policy. From the point of view of group
signatures, this may seem redundant. After all, if all valid signatures produce
consistent evaluation and opening results, given the employed user key,
credentials, and signed message, the notion of traceability seems to be
satisfied. However, from the point of view of anonymous credentials, an attacker
being able to obtain a credential, even when it does not meet the required
conditions to have it issued, is clearly problematic: it would allow to prove
claims over attributes it does not really ``own'' (even if the output of \Sign
and \Open are consistent with the values returned by \feval and \finsp, and
the credential was obtained via an $\langle \Obtain,\Issue \rangle$ run). To
capture this, we define the \ExpForgeIssue experiment. In the experiment, the
adversary is challenged to produce a credential identifier that must be
associated to an existing $\langle \Obtain,\Issue \rangle$ interaction -- thus,
the corresponding \trans entry must exist (which we can check, as the issuer is
honest). The adversary wins if, either the extraction process fails, or the
extracted user secret key and credentials make the corresponding issuance
function fail.

For both \ExpForgeIssue and \ExpForgeSign, the adversary is given access to the
oracle set $\Oforgeissue = \Oforgesign \gets \lbrace\HU,\CU\rbrace\GEN,\IGEN,
\OGEN,\OCORR,\OBTISS,\ISSUE,\RREG,\SIGN,\OPEN$.

\begin{figure}[htp!]
    \procedure[linenumbering]{$\ExpForgeIssue(1^\secpar)$}{%
      \parm \gets \Setup(1^\secpar) \\
      \cid \gets \adv^{\Oforgeissue}(\parm) \\
      \pcif \trans[\cid] = \bot \lor \CRED[\cid] = \bot: \pcreturn 0 \\
      \textrm{Parse}~\CRED[\cid]~\textrm{as}~(\cdot,\cdot,\iid,\cdot,\cdot);~
      \IK[\iid]~\textrm{as}~((\ipk,\fissue),\cdot) \\
      (\usk,\scred,\attrs_{\scred}) \gets \ExtractIssue(\trans[\cid]) \\
      \pcif \fissue(\usk,\scred,\ATTR[\cid]) = 0 \lor
      \exists \cred \in \scred~\st~\IdentifyCred(\usk,\attrs_{\cred},\cred) = 0: \\
      \pcind \pcreturn 1 \\
%      \pcif \nexists \uid~\st~\IdentifyUK(\uid,\usk) = 1: \pcreturn 1 \\
      \pcreturn 0
    }
  \caption{Experiment for unforgeability of credential issuance in \UAS schemes.}
  \label{fig:exp-uas-unfor-issue}
\end{figure}    

\begin{figure}[htp!]
    \procedure[linenumbering]{$\ExpForgeSign(1^\secpar)$}{%
      \parm \gets \Setup(1^\secpar) \\
      (\oid,\siid,\sig,\yeval,\msg,\feval) \gets \adv^{\Oforgesign}(\parm) \\
      \pcif \exists \uid~\st~(\cdot,\cdot,\sig,\yeval,\msg,\feval) \in
      \SIG[\uid]: \pcreturn 0 \\
      \pcif \Verify(\PUBOK[\oid],\PUBIK[\siid],\sig,\yeval,\msg,\feval) = 0:
      \pcreturn 0 \\
      (\yinsp,\iproof) \gets \Open(\PRVOK[\oid],\siid,\sig,\yeval,\msg) \\
      \pcif \Judge(\PUBOK[\oid],\PUBIK[\siid],\yinsp,\iproof,\sig,\yeval,\msg,\feval)
      = 0: \pcreturn 1 \\
      (\usk,\scred,\attrs_{\scred},\yinsp',r) \gets \ExtractSign(\oid,\siid,\sig,
      \yeval,\msg,\feval) \\
      \pcif \feval(\usk,\scred,\msg) \neq \yeval: \pcreturn 1 \\
      \pcif \finsp(\yeval,\usk,\scred,\msg) \neq \yinsp \lor \yinsp \neq \yinsp':
      \pcreturn 1 \\
      \pcif \exists \cred \in \scred~\st~\IdentifyCred(\usk,\attrs_{\cred},\cred) = 0:
      \pcreturn 1 \\
      \pcif \nexists \uid~\st~\IdentifyUK(\uid,\usk) = 1: \pcreturn 1 \\
      \pcreturn 0
    }
  \caption{Experiment for unforgeability of signatures in \UAS schemes.}
  \label{fig:exp-uas-unfor-sign}
\end{figure}

\begin{definition}{(Unforgeable issuance of \UAS)}
  \label{def:issue-forge-uas}  
  We define the advantage \AdvForgeIssue of $\adv$ against \ExpForgeIssue as
  $\AdvForgeIssue=\Pr\lbrack\ExpForgeIssue(1^\secpar)=1\rbrack$.
  %
  A \UAS scheme has unforgeable issuance if, for any p.p.t. adversary $\adv$,
  \AdvForgeIssue is a negligible function of $1^\secpar$.
\end{definition}

\begin{definition}{(Unforgeable signing of \UAS)}
  \label{def:sign-forge-uas}  
  We define the advantage \AdvForgeSign of $\adv$ against \ExpForgeSign as
  $\AdvForgeSign=\Pr\lbrack\ExpForgeSign(1^\secpar)=1\rbrack$.
  %
  A \UAS scheme has unforgeable signing if, for any p.p.t. adversary $\adv$,
  \AdvForgeSign is a negligible function of $1^\secpar$.
\end{definition}

For short, we say that an \UAS scheme that has both unforgeable issuance and
signing, is an unforgeable \UAS scheme.

\paragraph{Discussion on the generality of unforgeability in \UAS schemes.} %
The notion of signature unforgeability we present for \UAS is strictly more
general than the corresponding one of traceability for group signatures. This is
again a direct consequence of the fact that \Open can return an arbitrary
function of the signer's key and credentials (and signed message), which is a
strict generalization of the conventional \Open. Although, even in that case, we
need to take into account attributes, and the fact that the same user may obtain
multiple credentials (that is why, even when having \Open return the identity of
the signer, our notion is not exactly the same). In this sense, the sign
unforgeability notion for \UAS is equivalent to that of anonymous
credentials. It would seem, though, that we do not need the traceability part of
group signatures; after all, it is the protection against wrong claims on
attributes what enables meaningful and flexible authentication. However, the
type of protection against misuses of the open functionality that we can
get with an honest issuer (as in traceability) is much higher than without an
honest issuer (as in non-frameability). Specifically, with an honest issuer we
can ensure that the adversary cannot even alter the value returned by \Open on
signatures by corrupt users, nor the output of the signing predicate \feval.
Whereas, with a corrupt issuer, all we can ensure is that
the adversary cannot forge a signature from an honest user for which \Open
returns the same value as a signature by that honest user would produce; and,
certainly, a corrupt issuer can arbitrarily issue credentials meeting any
desired predicate \feval. Signing unforgeability is, therefore, a core property
to ensure accountability.

Similarly, by adding the related notion of issuance unforgeability, we ensure
that no credential can be issued that did not meet the corresponding issuance
policy. This is again something not necessary in group signatures (with
verifiable openings), where \Judge accepting the opening proof implies that
there is a valid join transcript associated to the membership credential used
to produce the signature. However, in \UAS, given that the credentials contain
attributes, even though that transcript exists, we need to make sure that the
issuance policy over these attributes was satisfied -- and this is not something
(easily) extractable from the signature.

\paragraph{Non-frameability.} %
The notion of non-frameability in \UAS schemes is unavoidably more subtle than
in group signatures. To see this, we note that, by allowing arbitrary
evaluation and open functions to be used, it can be perfectly valid to
have a signature produced by a corrupted user output the same \yeval or \yinsp
values than the ones output when evaluating or opening a signature by an honest
user. As a concrete example, imagine an open function that returns the
nationality of the signer. In any country, there will be many users (corrupt or
not) sharing nationality.
%
More generally, since the issuer is dishonest in non-frameability properties
and, in \UAS, the value produced by \Open may depend on the attributes
included in user credentials, the adversary may even be able to just produce
``legitimate'' openings that output any desired value.

Thus, we again need to resort to extraction techniques. In the non-frameability
definition for \UAS, given in experiment \ExpNonframe in
\figref{fig:exp-uas-frame}, the adversary is challenged to produce a signature
and opening proof that is accepted by \Verify and \Judge, respectively. From
the signature, we then extract the secret key of the signer, and match it
against the secret keys of the honest users. The adversary wins if there is a
match and the signature has not been queried to \SIGN, or if the value \yinsp
output by the adversary is accepted by \Judge, yet it is different from the
extracted $\yinsp'$ value. In the game, the
adversary has access to the oracles in $\Oframe \gets \lbrace\HU,\CU\rbrace\GEN,
\lbrace\II,\OO\rbrace\GEN,\lbrace\II,\OO\rbrace\CORR,\WREG,\OBTAIN,\SIGN$.

\begin{figure}[htp!]
  \procedure[linenumbering]{$\ExpNonframe(1^\secpar)$}{%
    \parm \gets \Setup(1^\secpar) \\
    (\oid,\siid,\sig,\yeval,\msg,\feval,\yinsp,\iproof) \gets
    \adv^{\Oframe}(\parm) \\
%    \pcif \exists \uid~\st~(\cdot,\cdot,\sig,\yeval,\msg,\feval) \in \SIG[\uid]:
%    \pcreturn 0 \\
    \pcif \Verify(\PUBOK[\oid],\PUBIK[\siid],\sig,\yeval,\msg,\feval) = 0:
    \pcreturn 0 \\
    \pcif \Judge(\PUBOK[\oid],\PUBIK[\siid],\yinsp,\iproof,\sig,\yeval,\msg) = 0:
    \pcreturn 0 \\
    (\usk,\scred,\attrs_{\scred},\yinsp',r) \gets
    \ExtractSign(\oid,\siid,\sig,\yeval,\msg,\feval) \\
    \pcif \exists \uid \in \HU~\st~\UK[\uid] = \usk~\land \\
    \pcind (\nexists (\cdot,\cdot,\sig,\yeval,\msg,\feval) \in \SIG[\uid]
    \lor \yinsp \neq \yinsp'): \pcreturn 1 \\
    \pcreturn 0
  }
  \caption{Experiment for non-frameability on \UAS schemes.}
  \label{fig:exp-uas-frame}
\end{figure}

\begin{definition}{(Non-frameability of \UAS)}
  \label{def:frame-uas}
  We define the advantage \AdvNonframe of $\adv$ against \ExpNonframe as
  $\AdvNonframe=\Pr\lbrack\ExpNonframe(1^\secpar)=1\rbrack$.
  %
  A \GSAC scheme satisfies non-frameability if, for any p.p.t. adversary $\adv$,
  \AdvNonframe is a negligible function of $1^\secpar$.
\end{definition}

\paragraph{Discussion on the generality of non-frameability in \UAS schemes.} %
Anonymous credentials do not have non-frameability property and, thus, it is
hard to make a comparison. However, we can draw some connections with AC schemes
that support revocation, as revocation is somehow equivalent to linking, which
is a type of inspection available in group signatures. In this sense, note that
basic revocation (without straight deanonymization) can be trivially achieved
through our generic \Open function. For instance, one could set \finsp to
be a pseudorandom number seeded with the user's public key (or credential). In
this sense, \Open could be essentially seen as a Verifiable Random Function.
If we compare with group signatures, our notion is also more general than the
conventional one, again for the same reason as sign unforgeability (i.e., \Open
can return any value, not just the signer's identity). Thus, the need to extract
the signer's data in order to detect if a framing has taken place.
  
%%% Local Variables:
%%% mode: latex
%%% TeX-master: "uas"
%%% End:

\section{\CUASGen: A Generic \UAS Construction}
\label{sec:gen-construction}

In this section, we give a generic construction of an \UAS scheme, based on
generic building blocks. In \secref{sec:instantiation}, we give a concrete
instantiation.

\subsection{Building Blocks}
\label{ssec:bblocks}

\paragraph{Vector Commitment schemes.} %
Defined as a tuple $(\CSetup,\CCommit)$. Algorithm $\Cparm \gets
\CSetup(\Csecpar)$ produces the parameters for committing to values. $\Ccom
\gets \CCommit(\Cparm, \msgset; r)$ produces a commitment \Ccom over a set of
messages \msgset, from which we may omit randomness $r$. \todo{Informally define
  hiding and binding, leaving formal definitions to the appendix.}

\paragraph{Public-Key Encryption.} %
Defined as a tuple $(\ESetup,\EKeyGen,\EEnc,\EDec)$. Algorithm $\Eparm \gets
\ESetup(\Esecpar)$ produces public parameters for the other algorithms.
$(\Eek,\Edk) \gets \EKeyGen(\Eparm)$ generates the encryption-decryption key
pair, algorithm $\Ec \gets \EEnc(\Eek,\msg)$ encrypts message \msg with
encryption key \Eek, producing ciphertext \Ec, and $\msg/\bot \gets \EDec(\Edk,
\Ec)$ decrypts ciphertexts using decryption key \Edk. \todo{Informally define
  security properties we'll need.}

\paragraph{Digital Signatures.} %
Defined as a tuple $(\SSetup,\SKeyGen,\SSign,\SVerify)$. Algorithm $\Sparm \gets
\SSetup(\Ssecpar)$ produces public parameters for the other algorithms.
$(\Svk,\Ssk) \gets \SKeyGen(\Sparm)$ generates the verification-signing key
pair, algorithm $\Ssig \gets \SSign(\Ssk,\msg)$ signs message \msg with
signing key \Ssk, producing signature \Ssig, and $1/0 \gets \SVerify(\Svk,
\Ssig,\msg)$ checks whether \Ssig is a valid signature over \msg, under
verification key \Svk. \todo{Informally define security properties we'll need.}

\paragraph{Non-Interactive Zero-Knowledge.} %
We use non-interactive zero-knowledge proofs of knowledge (NIZK), in the CRS
model \needcite. Informally, a NIZK scheme over an NP relation \NIZKRel is
defined as a tuple $(\NIZKSetup^\NIZKRel,\NIZKProve^\NIZKRel,
\NIZKVerify^\NIZKRel)$. Algorithm $\NIZKcrs \gets \NIZKSetup^\NIZKRel
(\NIZKsecpar)$ produces the common reference string \NIZKcrs. $\NIZKproof/\bot
\gets \NIZKProve^\NIZKRel(\NIZKcrs,\NIZKw,\NIZKx)$ creates a NIZK proof of
knowledge of witness \NIZKw for \NIZKx such that $(\NIZKw,\NIZKx) \in \NIZKRel$.
$1/0 \gets \NIZKVerify^\NIZKRel(\NIZKcrs,\NIZKx,\NIZKproof)$ verifies the proof.
\todo{Informally define security properties we'll need.}

\paragraph{Signatures over Blocks of Committed Messages, with proofs.} %
We use signature schemes that allow signing messages, or commitments to messages,
in blocks, and are compatible with (efficient) proof systems over the produced
signature and signed (commitments to) messages. For this purpose, we define such
schemes as a tuple $(\SBCMSetup,\SBCMKeyGen,\SBCMSign,\SBCMVerify)$. Algorithm
$\SBCMparm \gets \SBCMSetup(\SBCMsecpar)$ produces the public parameters for the
scheme. $(\SBCMvk,\SBCMsk) \gets \SBCMKeyGen(\SBCMparm)$ produces a
verification-signing key
pair. Algorithm $\SBCMsig \gets \SBCMSign(\SBCMsk,\Ccom,\msgset)$ produces a
signature over a set of committed messages \Ccom and a set of messages
\msgset, where either \Ccom or \msgset may be empty. $1/0 \gets
\SBCMVerify(\SBCMvk,\SBCMsig,\overline{\msgset})$ verifies a signature \SBCMsig
over message set $\overline{\msgset}$, which must contain both the messages that
were signed as commitments as well as those signed in ``the clear''. In
addition, the produced signatures must be compatible with (efficient) NIZK
proofs of knowledge of a signature, and of (arbitrary) claims over the signed
(committed) messages.
\todo{Informally define security properties we'll need.}

\subsection{Generic Construction \CUASGen}
\label{ssec:generic-construction-uas}

We use three different NP relations in our generic construction. Namely,
$\NIZKRel_{\Issue}$ for issuance, $\NIZKRel_{\Sign}$ for signing, and
$\NIZKRel_{\Inspect}$ for inspection. We will be defining them in the
corresponding protocol/algorithm.

\todo{Many variables need renaming here.}

\paragraph{$\parm \gets \Setup(\secpar,\AttrSpace)$.} %
The setup process essentially consists on generating the public parameters
for all the building blocks. In detail, it parses \secpar as $(\Csecpar,
\NIZKsecpar,\SBCMsecpar,\Esecpar)$. Then, run $\Cparm \gets
\CSetup(\Csecpar)$, $\SBCMparm \gets  \SBCMSetup(\SBCMsecpar)$, $\Sparm \gets
\SSetup(\Ssecpar)$, $\Eparm \gets \ESetup(\Esecpar)$, $\NIZKcrs_{\Issue} \gets
\NIZKSetup^{\NIZKRel_{\Issue}}(\NIZKsecpar)$, $\NIZKcrs_{\Sign} \gets
\NIZKSetup^{\NIZKRel_{\Sign}}(\NIZKsecpar)$, and $\NIZKcrs_{\Inspect} \gets
\NIZKSetup^{\NIZKRel_{\Inspect}}(\NIZKsecpar)$. Return $(\Cparm,\SBCMparm,
\Sparm,\Eparm,\NIZKcrs_{\Issue},\NIZKcrs_{\Sign},\NIZKcrs_{\Inspect},
\AttrSpace)$

\paragraph{$(\ipk,\isk) \gets \IKeyGen(\parm,\fissue)$.} %
To generate its key pair, each issuer first parses \parm as $(\cdot,\SBCMparm,
\Sparm,\cdot,\cdot,\cdot,\cdot,\cdot)$. Then, runs $(\Svk,\Ssk) \gets
\SKeyGen(\Sparm)$, $(\SBCMvk,\SBCMsk) \gets \SBCMKeyGen(\SBCMparm)$,
$\sig_{\fissue} \gets \SSign(\Ssk,\fissue)$, $\ipk \gets (\Svk,\fissue,
\sig_{\fissue})$, $\isk \gets \Ssk$ and return $(\ipk,\isk)$.

\paragraph{$(\opk,\osk) \gets \OKeyGen(\parm,\finsp)$.} %
To generate its key pair, each inspector first parses \parm as $(\cdot,\cdot,
\cdot,\Eparm,\cdot,\cdot,\cdot,\cdot)$. Then, runs $(\Svk,\Ssk) \gets \SKeyGen
(\Sparm)$, $(\Eek,\Edk) \gets \EKeyGen(\Eparm)$, $\sig_{\finsp} \gets \SSign
(\Ssk,\finsp)$, $\opk \gets (\Svk,\Eek,\finsp,\sig_{\finsp})$, and $\osk \gets
(\Ssk,\Edk)$. Finally, returns $(\opk,\osk)$.

\paragraph{$\usk \gets \UKeyGen(\parm)$.} %
Each user, prior to requesting credentials, generates his secret key by parsing
\parm as $(\cdot,\cdot,\cdot,\cdot,\cdot,\AttrSpace)$, and picking randomly
$\usk \getr \AttrSpace$. Finally, return \usk.

\paragraph{$\langle \cred/\bot,\utrans/\bot \rangle \gets
  \langle\Obtain(\usk,\scred,\attrs),\Issue(\isk,\upk,\sipk,\attrs)\rangle$.} %
The protocol is run between an issuer, and a user with secret key \usk and
credentials \scred, where each $\cred \in \scred$ is issued by an issuer with
public key $\ipk_{\cred}$ (which we assume that the user can easily retrieve,
e.g., from secure storage, given \cred), and attests attributes
$\attrs_{\cred}$. For readability we write $\attrs_{\scred}$ as abbreviation for
$\lbrace \attrs_{\cred} \rbrace_{\cred \in \scred}$, and similarly for
$\sipk_{\scred}$. The user requests a signature on a commitment to the user key,
as well as on the attributes in \attrs. In addition, the user proves that the
issuance function \fissue established by the issuer is satisfied by the
credentials in $\scred$
and its user secret key. For this, we define relation $\NIZKRel_{\Issue} = \lbrace
(\usk,\scred,\attrs_{\scred}), (\Ccom,\attrs,\sipk_{\scred}): \Ccom = \CCommit(usk) \land
\fissue(\usk,\scred,\attrs) = 1 \land \forall \cred \in \scred,
\SBCMVerify(\ipk_{\cred},\cred,
\attrs_{\cred} \cup \lbrace \usk \rbrace) = 1 \rbrace$. The interactive protocol
for a user to obtain a credential from an issuer of the system is as follows:

\begin{itemize}
\item \uline{User}: Commit to the user secret key with $\Ccom \gets
  \CCommit(\usk)$. Compute proof $\NIZKproof \gets
  \NIZKProve^{\NIZKRel_{\Issue}}(\NIZKcrs_{\Issue},(\usk,\scred,\attrs_{\scred}),
  (\Ccom,\attrs))$. Send $(\Ccom,\NIZKproof)$ to Issuer.
\item \uline{Issuer}: Verify \NIZKproof with $\NIZKVerify^{\NIZKRel_{\Issue}}
  (\NIZKcrs_{\Issue},\NIZKproof,(\Ccom,\attrs,\sipk))$, and abort if it fails. Then,
  compute the credential by running $\cred \gets \SBCMSign(\SBCMsk,\Ccom,
  \attrs)$. Send \cred to User. Output $\utrans \gets (\Ccom,\attrs,\sipk,
  \cred,\NIZKproof)$.
\item \uline{User}: Verify the credential with $\SBCMVerify(\SBCMvk,\cred,
  \attrs \cup \lbrace \usk \rbrace)$. Reject if verification fails.
  Otherwise, return \cred.
\end{itemize}

\paragraph{$(\sig,\yeval) \gets \Sign(\usk,\opk,\scred,\msg,\feval)$.} %
In the signing algorithm, we make use of the following relation:
$\NIZKRel_{\Sign} = \lbrace (\usk,\scred,\attrs_{\scred},\yinsp,r),(\msg,\feval,
\yeval,\Ec,\sipk_{\scred},\Eek): \Ec = \EEnc(\Eek,\yinsp;r) \land \yeval =
\feval(\usk,\scred,\msg) \land
\yinsp = \finsp(\yeval,\usk,\scred,\msg) \land \forall \cred \in \scred,
\SBCMVerify(\ipk_{\cred},\cred,\attrs_{\cred} \cup \lbrace \usk \rbrace) = 1)
\rbrace$ where, for each $\cred \in \scred$, $\attrs_{\cred}$ and $\ipk_{\cred}$,
as well as $\attrs_{\scred}$ and $\sipk_{\scred}$ are as in $\langle \Obtain,
\Issue \rangle$.
%
From this, in order to produce a valid signature, the user first evaluates
$\yeval \gets \feval (\usk,\scred,\msg)$, and decides whether or not to continue
with the signing process -- this may depend, e.g., on the inspection policy of
\opk, as the output of \feval may influence whether the user will be
de-anonymizable or not, depending on the \finsp function in \opk.
%
Then, the user parses \opk as $(\Svk,\Eek,\finsp,\sig_{\finsp})$ and checks that
$\Verify(\Svk,\sig_{\finsp},\finsp) = 1$ (note that this step may be cached), to
compute $\yinsp \gets \finsp(\yeval,\usk,\scred,\msg)$, and encrypt it with
\Eek by running $\Ec \gets \EEnc(\Eek,\yeval;r)$ for some fresh randomness $r$.
Finally, the user computes
$\NIZKproof \gets \NIZKProve^{\NIZKRel_{\Sign}}(\NIZKcrs_{\Sign},(\usk,\scred,
\attrs_{\scred},\yinsp,r),(\msg,\feval,\yeval,\Ec,\sipk_{\scred},\Eek))$ and
outputs $(\sig = (\NIZKproof,\Ec),\yeval)$.

\paragraph{$1/0 \gets \Verify(\opk,\sipk,\sig,\yeval,\msg,\feval)$.} %
The ``cryptographic'' side of the verification essentially consists on checking
the NIZK proof. That is, parse \sig as $(\NIZKproof,\Ec)$ and check whether
$\NIZKVerify(\NIZKcrs,\NIZKproof,(\msg,\feval,\yeval,\Ec,\sipk)) = 1$. In
addition, the verifier may further check whether \yeval meets its needs.

\paragraph{$(\yinsp,\NIZKproof)/\bot \gets
  \Inspect(\osk,\sipk,\sig,\yeval,\msg,\feval)$.} %
We first define NIZK relation $\NIZKRel_{\Inspect} = \lbrace (\osk),(\Ec,\yinsp)
: \yinsp = \EDec(\osk,\Ec) \rbrace$.
%
For inspection, the inspector first verifies the signature by running $\Verify(
\opk,\sipk, \sig,\yeval,\msg,\feval)$. If the verification succeeds, it parses
\sig as $(\NIZKproof,\Ec)$, decrypts \Ec by running $\yeval \gets \EDec(\osk,
\Ec)$, and computes $\NIZKproof_{\Inspect} \gets \NIZKProve^{\NIZKRel_{\Inspect}}
(\NIZKcrs_{\Inspect},\osk,(\Ec,\yinsp))$. It returns $(\yinsp,
\NIZKproof_{\Inspect})$.

\paragraph{$1/0 \gets \Judge(\opk,\yinsp,\NIZKproof,\sig,\yeval,\msg)$.} %
To assess the validity of an inspection proof, first check the signature
by running $\Verify(\opk,\sipk, \sig,\yeval,\msg,\feval)$. If the check succeeds,
parse \sig as $((\cdot,\Ec),\cdot)$ and verify \NIZKproof with $\NIZKVerify
(\NIZKcrs_{\Inspect},\NIZKproof,(\Ec,\yinsp))$. Accept it the NIZK verification
passes, and reject otherwise.

%%% Local Variables:
%%% mode: latex
%%% TeX-master: "uas"
%%% End:

\subsection{Correctness and Security of \CUASGen}
\label{ssec:security-uas}

First, we define the \Identify, \ExtractIssue and \ExtractSign functions that
are needed for some of the properties to be meaningful, in
\figref{fig:helper-funcs}.

\begin{figure}[ht!]
  \begin{minipage}[t]{\textwidth}
    \procedure{$\ExtractIssue(\parm,\trans)$}{%
      \textrm{Parse \parm as $(\cdot,\cdot,\cdot,\cdot,\NIZKcrs_{\Issue},\cdot,
        \cdot,\cdot)$; $\NIZKcrs_{\Issue}$ as $(\NIZKcrs,\NIZKtrap)$; and
        \trans as $(\Ccom,\attrs,\sipk,\cred,\NIZKproof)$} \\
      \pcif \NIZKVerify(\NIZKcrs,\NIZKproof,(\Ccom,\attrs,\sipk)): 
      \pcreturn \bot \\
      (\usk,\scred,\attrs_{\scred}) \gets \NIZKExtract(\NIZKcrs,\NIZKtrap,
      (\Ccom,\attrs,\sipk),\NIZKproof) \\
      \pcreturn (\usk,\scred,\attrs_{\scred}) \\
    }
    
    \procedure{$\ExtractSign(\parm,\oid,\siid,\sig,\yeval,\msg,\feval)$}{%
      \textrm{Parse \parm as $(\cdot,\cdot,\cdot,\cdot,\cdot,\NIZKcrs_{\Sign},
        \cdot,\cdot)$; $\NIZKcrs_{\Sign}$ as $(\NIZKcrs,\NIZKtrap)$; and
        \sig as $(\NIZKproof,\Ec)$} \\
      \textrm{Parse $\PUBOK[\oid]$ as $(\opk,\cdot)$ and let $\sipk \gets
        \PUBIK[\siid]$} \\
      \pcif \NIZKVerify(\NIZKcrs,\NIZKproof,(\msg,\feval,\yeval,\Ec,
      \sipk,\opk)): \pcreturn \bot \\
      (\usk,\scred,\attrs_{\scred},\yinsp,r) \gets \NIZKExtract(\NIZKcrs,\NIZKtrap,
      (\msg,\feval,\yeval,\Ec,\sipk,\opk),\NIZKproof) \\
      \pcreturn (\usk,\scred,\attrs_{\scred},\yinsp) \\
    }
    
    \procedure{$\Identify(\usk,\attrs_{\cred},\cred)$}{%
      \pcreturn \SBCMVerify(\ipk_{\cred},\cred,\usk,\attrs_{\cred}) \\
    }    
  \end{minipage}
  \label{fig:helper-funcs}
  \caption{Definition of helper functions \Identify, \ExtractIssue and
    \ExtractSign, for \CUASGen.}
\end{figure}

\begin{theorem}[Correctness of \CUASGen]
  \label{thm:correctness-uas}
  If the underlying schemes for vector commitments, encryption, digital
  signatures, signatures on blocks of committed messages, and NIZKs are
  correct, our generic construction \CUASGen satisfies correctness as
  defined in \defref{def:correctness-uas}.
\end{theorem}

\begin{proof}[\thmref{thm:correctness-uas}]
  \todo{XXX}
\end{proof}

\begin{theorem}[Anonymity of \CUASGen]
  \label{thm:anonymity-uas}
  If the NIZK system used for $\NIZKRel_{\Sign}$ is zero-knowledge and
  simulation-extractable, our \CUASGen construction satisfies anonymity as
  defined in \defref{def:anonymity-uas}.
\end{theorem}

\begin{proof}[\thmref{thm:anonymity-uas}]
  In this proof, we restrict to the case in which the adversary can only make
  one query to the challenge oracle. Note however that the generalization to
  polynomially many queries given in \cite{bsz05} applies here too (with the
  corresponding security loss). Thus, proving security for one query to the
  challenge oracle is enough.

  We start from $G_0=\ExpAnonb$, and define game $G_1$ to be exactly the same
  as $G_0$, except that, within the $\Setup$ algorithm, we replace
  $\NIZKSetup^{\Sign}$ with $\NIZKSimSetup^{\Sign}$. By simulation
  extractability, $G_1$ is indistinguishable from $G_0$.
  
  From $G_1$, we consider $G^0_1$, which we define to be $G_1$, for $b=0$
  (i.e., \ExpAnonz, using $\NIZKSimSetup^{\Sign}$). The challenge sent to the
  adversary is $(\csig_0,\yeval) \gets \Sign(\PRVUK[\cuid_0],\PUBOK[\oid],
  \CRED[\scid_0],\msg,\feval)$, where $\csig_0 = (\pi_0,\Ec_{\yinsp})$, with
  $\pi_0 = \NIZKProve^{\NIZKRel_{\Sign}}(\NIZKcrs_{\Sign},(\msg,\feval,\yeval,
  \Ec_{\yinsp},\PUBIK[\scid_0],\PUBOK[\oid]),(\PRVUK[\cuid_0],
  \CRED[\scid_0],\attrs_{\scid_0},\yinsp,r))$ and $\Ec_{\yinsp} =
  \EEnc(\PUBOK[\oid],\yinsp;r)$.
  %
  Further, we build $G_2^0$ from $G_1^0$ by simulating the proof $\pi_0$. That
  is, in $G_{2,0}$, $\csig_0 = (\pi_0^s,\Ec_{\yinsp})$, where $\pi^s_0 =
  \NIZKSim^{\NIZKRel_{\Sign}}(\NIZKcrs_{\Sign},\NIZKtrap,(\msg,\feval,\yeval,
  \Ec_{\yinsp},\PUBIK[\scid_0],\PUBOK[\oid]))$. By zero-knowledgeness
  of $\NIZK^{\Sign}$, $G_2^0$ is indistinguishable from $G_1^0$.

  Similarly, we consider $G_1^1$ and $G_2^1$. That is, $G_1^1$ is $G_1$
  for $b=1$, where the challenge
  sent to the adversary is $(\csig_1,\yeval) \gets \Sign(\PRVUK[\cuid_1],
  \PUBOK[\oid],\CRED[\scid_1],\msg,\feval)$, where $\csig_1 = (\pi_1,
  \Ec_{\yinsp})$, with $\pi_1 = \NIZKProve^{\NIZKRel_{\Sign}}(\NIZKcrs_{\Sign},
  (\msg,\feval,\yeval,\Ec_{\yinsp},\PUBIK[\scid_1],\PUBOK[\oid]),
  (\PRVUK[\cuid_1],\CRED[\scid_1],\attrs_{\scid_1},\yinsp,r'))$ and $\Ec_{\yinsp}
  = \EEnc(\PUBOK[\oid],\yinsp;r')$. As before, $G_2^1$ is built from $G_1^1$,
  simulating $\pi_1$. That is, in $G_2^1$, $\csig_1 = (\pi_1^s,\Ec_{\yinsp})$,
  where $\pi^s_1 = \NIZKSim^{\NIZKRel_{\Sign}}(\NIZKcrs_{\Sign},\NIZKtrap,(\msg,
  \feval,\yeval,\Ec_{\yinsp},\PUBIK[\scid_1],\PUBOK[\oid]))$. Again, by
  zero-knowledge of $\NIZK^{\Sign}$, $G_2^1$ is indistinguishable from $G_1^1$.
  Note also that $G_2^1=G_2^0$, as in the challenge oracle, used in the
  anonymity game, we restrict to $\PUBIK[\scid_0] = \PUBIK[\scid_1]$.

  Finally, consider the definition of $\AdvAnon=|\Pr\lbrack
  \ExpAnono(1^\secpar)=1\rbrack-\Pr\lbrack\ExpAnonz(1^\secpar)=1\rbrack|$. As
  argued, $G_1$ is indistinguishable from $\ExpAnonb$, thus
  $\AdvAnon \approx |\Pr\lbrack G_1^1(1^\secpar)=1\rbrack-\Pr\lbrack
  G_1^0(1^\secpar)=1\rbrack| \approx
  |\Pr\lbrack G_2^1(1^\secpar)=1\rbrack-\Pr\lbrack
  G_2^0(1^\secpar)=1\rbrack|$. Since $G_2^1=G_2^0$, it follows that
  \AdvAnon is negligible.
  %
  \qed
\end{proof}

\begin{theorem}[Issuance unforgeability of \CUASGen]
  \label{thm:issue-forge-uas}
  If the underlying scheme for signatures on blocks of committed messages is
  existentially unforgeable, and the NIZK used for $\NIZKRel_{\Issue}$ is
  simulation extractable and sound, then our \CUASGen construction satisfies
  issuance unforgeability as defined in \defref{def:issue-forge-uas}, except
  with negligible probability.
\end{theorem}

\todo{\usk belongs to \AttrSpace! I think this can lead to malleability attacks.
  Make them disjoint?}

\begin{proof}[\thmref{thm:issue-forge-uas}]
  We show that the probability that \fissue outputs $0$ is negligible, as well
  as the probability that the extracted \usk is not the one that was used to
  request some of the credentials employed to obtain the credential specified by
  the adversary.
  %
  For this purpose, we define two games, $G_0=\ExpForgeIssue$, and $G_1$, which
  is exactly the same, but where, within the \Setup algorithm, we replace
  $\NIZKSetup^{\Issue}$ with $\NIZKSimSetup^{\Issue}$. As per the definition of
  \NIZK in \appref{sapp:nizk}, both games are indistinguishable.

  Now, observe that the adversary is required to output a credential
  identifier for which associated entries in \trans and \CRED exist; moreover,
  if such a credential was produced by an issuer, we must have access to those
  entries, as issuers are assumed to be honest.
  %
  Then, given that $\NIZKRel_{\Issue}$ is knowledge extractable (which is implied
  by simulation-extractability), in game $G_1$
  we can apply the \NIZKExtract function, which produces a tuple $(\usk,\scred,
  \attrs_{\scred})$ from $\utrans = (\Ccom,\attrs,\sipk,\cred,\NIZKproof)$.
  %
  Since \NIZKproof is accepted by \ExtractIssue, from the soundness of \NIZK and
  existential unforgeability of \SBCM, we know that all $\cred \in \scred$ are
  valid signatures over \usk, and their respective $\attrs_{\cred}$. Thus,
  \Identify returns $1$ for all $(\usk,\attrs_{\cred},\cred)$ tuples.
  Moreover, all the credentials in \scred given to \fissue belong to the same
  user, who is the owner of \usk.
  %
  Finally, since issuers are honest, we know that $\ATTR[\cid] = \attrs$ and,
  consequently, $\fissue(\usk,\scred,\ATTR[\cid]) = \fissue(\usk,\scred,\attrs)
  = 1$, due to the soundness of \NIZK.
  %
  \qed
\end{proof}

\begin{theorem}[Signing unforgeability of \CUASGen]
  \label{thm:sign-forge-uas}
  If the underlying NIZK scheme for $\NIZKRel_{\Sign}$ is sound and simulation
  extractable, the NIZK scheme for $\NIZKRel_{\Inspect}$ is sound and simulation
  extractable, and \SBCM is existentially unforgeable, then our \CUASGen
  construction satisfies signing unforgeability as defined in
  \defref{def:sign-forge-uas}, except with negligible probability.
\end{theorem}

\begin{proof}[\thmref{thm:sign-forge-uas}]
  As for \thmref{thm:issue-forge-uas}, we define two games, $G_0=\ExpForgeSign$,
  and $G_1$, which is exactly the same, but where, within the \Setup algorithm,
  we replace $\NIZKSetup^{\Sign}$ with $\NIZKSimSetup^{\Sign}$. As per the
  definition of \NIZK in \appref{sapp:nizk}, both games are indistinguishable.

  From $G_1$, and in order to define the winning conditions for the adversary
  in the signing unforgeability game, consider the following events:

  \begin{description}
  \item[$V$.] Where $V = \Verify(\opk,\sipk,\sig,\yeval,\msg,\feval) = 1$.
  \item[$J$.] Where $J = \Judge(\opk,\sipk,\yinsp,\iproof,\sig,\yeval,\msg,
    \feval) = 0$.
  \item[$L$.] Where $L = (\msg,\feval,\yeval,\Ec,\sipk,\opk) \in
    \NIZKLang^{\Sign}$.    
  \item[$I$.] Where $I = \exists \cred \in \scred~\st~\Identify(\usk,
    \attrs_{\cred},\msg) = 0$.
  \end{description}

  $\adv$ wins if $V \land (J \lor I) = (\overline{L} \land V \land (J \lor I))
  \lor (L \land V \land (J \lor I))$.
  %
  $V$ implies that $(\msg,\feval,\yeval,\Ec,\sipk,\opk) \in \NIZKRel^{\Sign}$.
  Thus, after soundness of $\NIZK^{\Sign}$, the probability of $\overline{L}
  \land V \land (J \lor I)$ is negligible in the security parameter.
  %
  For $(L \land V \land (J \lor I))$ to be satisfied, there are three cases:
  \begin{enumerate}
  \item $L \land V \land J \land \overline{I}$. The game returns 1 in step 6.
  \item $L \land V \land J \land I$.  The game returns 1 in step 6.
  \item $L \land V \land \overline{J} \land I$. The game returns 1 in step 10. 
  \end{enumerate}

  In case 1, $L \land V$ implies that $(\msg,\feval,\yeval,\Ec,\sipk,\opk) \in
  \NIZKRel^{\Sign}$. More concretely, soundness of $NIZK^{\Sign}$ implies that
  $\Ec = \EEnc(\opk,\yinsp;r)$, for $\yinsp$ and $r$ known to the signer. Since
  $(\yinsp,\iproof)$ is generated honestly by the challenger from
  $(\opk,\sipk,\sig = (\NIZKproof_{\Sign},\Ec),\yeval,\msg)$, correctness of
  public key encryption implies that $\EDec(\osk,\Ec) = \yinsp$. Consequently,
  the probability of $L \land V \land J \land \overline{I}$ = 0, as \Judge
  checks precisely that \Ec is a correct encryption of \yinsp under \opk.

  The analysis for case 2 is the same as for case 1.

  For case 3, $(\msg,\feval,\yeval,\Ec,\sipk,\opk) \in \NIZKRel^{\Sign}$,
  $\Judge(\opk,\sipk,\yinsp,\iproof,\sig,\yeval,\msg,\feval)
  = 1$, but there exists some credential \cred for which $\Identify(\usk,
  \attrs_{\cred},\msg)=0$. Note that \usk, $\attrs_{\cred}$ and \cred (for all
  $\cred \in \scred$) are output by \ExtractSign. Thus, after the
  simulation-extractability property and soundness of $\NIZK^{\Sign}$, $(\msg,
  \feval,\yeval,\Ec,\sipk,\opk) \in \NIZKRel^{\Sign}$, which more concretely
  means that $\SBCMVerify(\ipk_{\cred},\cred,\attrs_{\cred} \cup \lbrace \usk
  \rbrace) = 1 = \Identify(\usk,\attrs_{\cred},\cred)$, for all $\cred \in
  \scred$. The probability of case 3 is therefore $0$.
  %
  Moreover, since \SBCMVerify returns $1$ for all $\cred \in \scred$, it must
  be that all of them were obtained via queries to the \ISSUE or \OBTISS
  oracles. Otherwise, if there exists some \cred that was not obtained via
  a call to these oracles, the pair $(\lbrace \usk \rbrace \cup \attrs_{\cred},
  \cred)$ constitutes an existential forgery of \SBCM.

  Cases 1, 2 and 3 above account for winning conditions at steps 6 and 10.
  % 
  Additionally, $\adv$ wins at step 8 if $\feval(\usk,\scred,\msg) \neq \yeval$,
  where \yeval is the value output by the adversary in step 2. However, since
  $\Verify(\opk,\sipk,\sig,\yeval,\msg,\feval) = 1$, soundness of $NIZK^{\Sign}$
  implies that this has negligible probability.
  %
  Similarly, $\adv$ wins at step 9 if (1) $\finsp(\yeval,\usk,\scred,\msg) \neq
  \yinsp$, where \yinsp is the value output by \Inspect at line 5; or if (2)
  $\yinsp \neq \yinsp'$, where $\yinsp'$ is the value extracted by \ExtractSign
  at step 7. For (1), soundness of $\NIZK^{\Sign}$ ensures that \yinsp is the
  correct evaluation of \finsp, whereas soundness of $\NIZK^{\Inspect}$ and
  correctness of the public key encryption ensure that this is also the value
  output by \Inspect. Thus, the probability of $\adv$ winning the game because
  of (1) is negligible. Finally, for (2), simulation-extractability of
  $\NIZK^{\Inspect}$ and correctness of the public key encryption ensure that
  the $\yinsp'$ value extracted by \NIZKExtract matches the value produced by
  \Inspect.
  %
  \qed
\end{proof}

\begin{theorem}[Non-frameability of \CUASGen]
  \label{thm:frame-uas}
  If the underlying NIZK schemes are zero-knowledge and the scheme used for
  $\NIZK^{\Sign}$ is simulation-extractable, and the commitment scheme is
  hiding, then our \CUASGen construction satisfies non-frameability as defined
  in \defref{def:frame-uas}, except with negligible probability.
\end{theorem}

\begin{proof}[\thmref{thm:frame-uas}]
  We prove that, given an adversary $\adv$ against non-frameability of \CUASGen,
  we can build an adversary \advB that breaks the hiding property of the
  underlying commitment scheme, with non-negligible probability.

  We start from $G_0=\ExpNonframe$. $\adv$ makes queries to the oracles in
  \Oframe.

  For $G_1$, within \Setup, we replace the \Setup algorithms for the three
  NIZKs (\Issue, \Sign and \Inspect) with their corresponding \SimSetup
  variants. Consequently, the corresponding queries to \Prove are also
  simulated via the simulator. By the zero-knowledge property of the NIZK
  systems, $G_1$ is indistinguishable from $G_0$.
  
  We build adversary \advB against hiding of commitments from $G_1$ against
  non-frameability. In the hiding game (see \figref{fig:com-games}), \advB first
  picks two messages $\msg_0$ and $\msg_1$, and then receives a commitment \Ccom
  of $\msg_b$. Let \advB pick both $\msg_0$ and $\msg_1$ from \AttrSpace. Then,
  \advB initializes $G_1$ for $\adv$ against non-frameability, and randomly
  picks a number $u \getr [1,q]$, where $q$ can be as large as \advB wants, but
  will be the maximum number of honest users to let $\adv$ add to the game.
  Then, when $\adv$ asks to create the $u$-th user, \advB ignores the call to
  \UKeyGen. For every call that $\adv$ makes to the \OBTAIN oracle associated to
  the $u$-th user, \advB uses the commitment \Ccom received as challenge in its
  game against the hiding property of commmitments, and uses it as commitment to
  the $u$-th user's \usk. \advB then simulates all the NIZK proofs associated to
  the $u$-th user, in calls to \OBTAIN, \SIGN and \INSPECT. Again, due to the
  zero-knowledge property of the associated NIZK systems, the outputs of the
  modified oracles are indistinguishable from the original outputs (as \Ccom
  is a valid commitment of a user secret key). Eventually, $\adv$ outputs a
  $(\sig,\yeval,\msg,\feval,\yinsp,\iproof)$ tuple that is accepted by \Verify
  and \Judge. Due to simulation extractability of $NIZK^{\Sign}$, \ExtractSign
  must be able to produce a $(\usk,\scred,\attrs_{\scred},\yinsp')$ tuple. Since
  $\adv$ wins the
  non-frameability game with non-neglibible probability, with probability $1/q$,
  the \usk value belongs to the $u$-th honest user, so it must be equal to
  either $\msg_0$ or $\msg_1$; \advB responds accordingly to its challenge in
  the game for the hiding property of the commmitment scheme. By assumption,
  \advB wins with non-negligible probability.
  %
  \qed
\end{proof}


%%% Local Variables:
%%% mode: latex
%%% TeX-master: "uas"
%%% End:

\section{Relating \UAS with Other Schemes, and Variations}
\label{sec:transformations}

In order to justify \UAS's universality, in this section we describe how can
we leverage our \UAS scheme to build other known schemes, only by specifying
different issuance, signature evaluation, and opening functions; as well as,
occasionally, some minor variations in some NIZK relations. Table
\tabref{tab:uas-alt-funcs} summarizes the different functions we use in our
subsequent variations of \UAS. \jdv{The relationships seem quite
  straightforward; thus, we only informally sketch how one could build other
  schemes (or something very similar to them) from \UAS, and leave the formal
  analysis for futher work.}

\begin{table}[ht!]
  \begin{tabular}{c | c | c | c | c | c}
    \bf Target scheme & \bf $\NIZKRel_{\Sign}$ variant & \bf \fissue variant & \bf \feval variant & \bf \finsp variant & \bf  Defined in \\
    \hline
    GS & None & $\fissue^1$ & $\feval^0$ & $\finsp^{\upk}$ & Implied by \GSAC \\
    AC\footnote{We show how to build ACs with selective disclosure, but ACs for
    arbitrary claims are direct too.} & None & $\fissue^1$ & $\feval^{\dattrs}$ & $\finsp^0$ & Implied by \GSAC \\
    \GSAC & None & $\fissue^1$ & $\feval^{\dattrs}$ & $\finsp^{\upk}$
    & \secref{ssec:uas-gsac} \\
    \bf GS-MDO & None & $\fissue^1$ & $\feval^0$ & $\finsp^{\smsg}$
    & \secref{ssec:uas-gsmdo} \\
    \bf Ring sigs. & $\NIZKRel_{\Sign-prv}$ & $\fissue^{\sring}$ & $\feval^{\attrs}$ & $\finsp^0$ & \secref{ssec:uas-ring} \\
    \bf MPS\footnote{Building MPS from \UAS also requires a simple definitional change. See
    \secref{ssec:uas-mps}.} & $\NIZKRel_{\Sign-enc}$ & Any & $\feval^{enc}$ & Any & \secref{ssec:uas-mps} \\
    \bf Delegatable creds & None & $\fissue^n$ & Any & Any & \secref{ssec:uas-delcred} \\
  \end{tabular}
  \caption{How to build related schemes from \UAS.}
  \label{tab:uas-alt-funcs}
\end{table}

\subsection{\GSAC from \UAS}
\label{ssec:uas-gsac}

Take the generic construction for \UAS in \secref{ssec:generic-construction-uas}.
First, consider the constant issuance function $\fissue^1$, the signature
evaluation function $\feval^{\dattrs}$, and the opening function $\finsp^{\upk}$,
as defined in \esref{eq:uas-gsac-funcs}

\begin{align}
  & \fissue^1(\usk,\scred,\attrs) \coloneqq 1 \nonumber \\
  & \feval^{\dattrs}(\usk,\cred,\msg) \coloneqq \dattrs(\cred) \nonumber \\
  & \finsp^{\upk}(\yeval,\usk,\scred,\msg) \coloneqq \CCommit(\usk;0)
    \label{eq:uas-gsac-funcs}
\end{align}

Where $\dattrs(\cred)$ is the function returning the attributes indexed by
\dattrs within \cred (which we can assume to be easy to do with proper
encoding). It is straightforward to see that an \UAS scheme with $\fissue^1$
as issuance function, $\feval^{\dattrs}$ as signing evaluation function, and
$\finsp^{\upk}$ as opening function, is an implementation of the functionality
described for \GSAC --which, in addition, lets the signer use more than one
credential per signature; although this can of course be restricted
implementation-wise.
%
Furthermore, the anonymity, sign unforgeability, and non-frameability properties
of \UAS directly imply the anonymity, traceability, and non-frameability
properties of \GSAC.

\paragraph{Vanilla Group Signatures and Anonymous Credentials from \UAS.} As a
corollary to the previous analysis, recall that, in \secref{ssec:variants-gsac}
we proved that \GSAC implies both group signatures and anonymous credentials.
Thus, since \UAS implies \GSAC, \UAS implies also both group signatures and
anonymous credentials.

\subsection{\UAS with Multiple Openers}
\label{ssec:uas-multiopen}

We briefly mention that we have defined \UAS signatures as being ``openable'' by
only one opener. However, since there is no tight relation between issuer and
opener (as opposed to conventional group signatures), it is trivial to
extend to the case with multiple openers per signature. In that case, the
$\NIZKRel_{\Sign}$ relation should be extended to include proofs of correct
encryption of the \yinsp values produced by each chosen opener.
Similarly, the sign unforgeability and non-frameability definitions should be
extended to check correctness of the corresponding extra open values.

\subsection{Group Signatures with Message Dependent Opening}
\label{ssec:uas-gsmdo}

In \cite{khk+19}, one of the first group signature schemes with a certain
flexibility in the open-related functionality was introduced. Briefly, a group
signature scheme with message dependent opening works as a conventional group
signature scheme, with the addition that it is only possible to open signatures
over specific messages (e.g., text with offensive language). For this, the
authors introduce an extra authority, the admitter, who creates a
``message-specific token'' for each message that is deemed unacceptable.
These message-specific tokens are later needed by the opener, in order to open
signatures over unacceptable messages. \UAS can implement this functionality in
a straightforward manner, by leveraging $\fissue^1$ as defined in
\esref{eq:uas-gsac-funcs}, and $\feval^0$ and $\finsp^{\smsg}$, as defined in
\esref{eq:uas-gsmdo-funcs}.

\begin{align}
  & \feval^0(\usk,\scred,\msg) \coloneqq \pcreturn 0 \nonumber \\
  & \finsp^{\smsg}(\yeval,\usk,\cred,\msg) \coloneqq \lbrace \pcif \msg \in \smsg:
    \pcreturn \CCommit(\usk;0); \pcelse \pcreturn \bot \rbrace
    \label{eq:uas-gsmdo-funcs}
\end{align}

\jdv{In \cite{khk+19}, the authors prove that GS-MDO implies identity-based
  encryption. \UAS implying GS-MDO would thus imply IBE, but we do not use
  IBE in our constructions. Is there some connection between some of the main
  building blocks we use, and IBE?}

\subsection{Ring Signatures}
\label{ssec:uas-ring}
\todo{This subsection needs much refinement.}

In a nutshell, ring signatures \cite{rst06} are like group signatures where
rather than having an issuer accepting users into a group, any user can create
an ad hoc group composed by himself, and any arbitrary set of other users that
have advertised their public keys in some publicly accessible way. Also, in
ring signatures no de-anonymization is possible\footnote{At least, in vanilla
  ring signatures. Some variants allow some sort of linkability; for instance
  \cite{lww04}.}, although the most distinguishing property is probably the lack
of issuer, which has led some relevant systems like Monero%
\footnote{\url{https://www.getmonero.org/resources/moneropedia/ringsignatures.html}.
  Last access on May 8th, 2022.} to opt for ring signatures rather than, e.g.,
group signatures.

\todo{Explain the following better...}
To reach (vanilla) ring signatures from \UAS, we have to slightly alter our
generic construction. This is due to the fact that the NP relation
$\NIZKRel_{\Sign}$ defined in \secref{ssec:generic-construction-uas} reveals the
public key of the issuer of each credential used to produce the signatures.
This is a problem as, intuitively, the signer in a ring
signature is a sort of issuer for the ad hoc group. To hide the issuer, we
replace the NP relation in \secref{ssec:generic-construction-uas} with
$\NIZKRel_{\Sign-prv}$. We also use issuance function $\fissue^{\sring}$,
signature evaluation function $\feval^{\attrs}$, and opening function
$\finsp^0$. All of them are specified in \esref{eq:uas-ring-funcs}.

\begin{align}
  & \NIZKRel_{\Sign-prv} \coloneqq \lbrace (\usk,\scred,\attrs_{\scred},\yinsp,r,
    \sipk_{\scred}),(\msg,\feval,\yeval,\Ec,\Eek): \nonumber \\
  & \hspace*{6.405em}\Ec = \EEnc(\Eek,\yinsp;r) \land \nonumber \\
  & \hspace*{6.40em}\forall \cred \in \scred,\SBCMVerify(\ipk_{\cred},\cred,\usk,
    \attrs_{\cred}) = 1) \land \nonumber \\
  & \hspace*{6.40em}\yeval = \feval(\usk,\scred,\msg) \land
    \yinsp = \finsp(\yeval,\usk,\scred,\msg)
     \rbrace \nonumber \\
  & \fissue^{\sring}(\usk,\scred,\attrs) \coloneqq \lbrace \pcif \attrs
    \subseteq \sring \cup \lbrace \CCommit(\usk;0) \rbrace: \pcreturn 1; \pcelse
    \pcreturn 0 \rbrace \nonumber \\
  & \feval^{\attrs}(\usk,\cred,\msg) \coloneqq \lbrace \pcif \CCommit(\usk;0) \in
    \attrs: \pcreturn \attrs(\cred); \pcelse \pcreturn \bot \rbrace \nonumber \\
  & \finsp^0(\yeval,\usk,\cred,\msg) \coloneqq 0 \label{eq:uas-ring-funcs}
\end{align}

Where $\usk \gets \isk$ for each issuer/user and, consequently 
\sring is some (arbitrary) ad hoc set containing the public keys of the
users (issuers) that the signer wants to include in its ring%
\footnote{Note that, implicitly, this conditions that the space of issuer public
  keys is a subset of \AttrSpace. If this is not the case ``by default'', this
  can be achieved via some appropriate mapping (e.g., an FDH or a XOF
  \needcite). That is, instead of having the ring be a set of {\ipk}s, we
  have the ring be a set of $F({\ipk}s)$, where $F$ is an appropriate mapping.
  For simplicity of exposition, we assume that the space of issuer public keys
  is compatible with \AttrSpace.}.
In $\NIZKRel_{\Sign-prv}$
the public keys of the issuers are now part of the witness in the NP relation,
which means that they will not be revealed in the proofs. Also, observe that, if
we let the attributes in a credential be public keys, $\fissue^{\sring}$ means
that a credential will be issued if all the public keys specified as attributes
to be included in the credential are part of the union of the ring and the
public key of the signer. To build $\feval^{\attrs}$, we define $\attrs(\cred)$
to be the function that returns all the attributes encoded in a credential.
Then, if the public key of the signer is included in the ring, $\feval^{\attrs}$
returns all the attributes in the credential (i.e., all the public keys in the
ring); otherwise, it aborts. Finally, let $\finsp^0$ disallows any
de-anonymization.

Then, an \UAS scheme with $\NIZKRel^{\Sign-prv}$ as NP relation for
$\NIZK^{\Sign}$, and $\fissue^{ring}$, $\feval^{\attrs}$, and $\finsp^0$ is
intuitively a ring signature scheme. To see this, observe that any user can
act as an issuer. Thus, the owner of \usk can issue to himself a credential
for any arbitrary ring it desires. While the $\NIZKRel^{\Issue}$ reveals the
issuer's public key, note that this is irrelevant, as the issuer is the user
himself. Then, the newly defined $\NIZKRel^{\Sign-prv}$ does exactly the same
as in our original \UAS scheme, but without revealing the issuer's public key.
$\feval^{\attrs}$ reveals all the attributes in the credential, which are
the public keys of the ring, including the signer's public key. This is actually
what ring signatures do: the signer, who is the owner of the private key
associated to one of the public keys in the ring, advertises which are the
public keys in the ring, and proves knowledge of one of them. If the NIZK
proof verifies, this means that the signer is indeed the owner of such a private
key. Finally, since $\finsp^0$ returns always $0$, then no de-anonymization is
possible.
%
Interestingly, this construction based on \UAS allows adding extra attributes
(beyond the public keys in the ring) to the produced signatures, which may be
useful for real world use cases.

\subsection{Multimodal Private Signatures}
\label{ssec:uas-mps}

To show that Multimodal Private Signatures (MPS) \needcite can be built from
\UAS, we need to give an alternative, simulation-based, definition of the
anonymity property. In the simulation-based approach, we require the adversary
to guess the bit $b$ defining whether it is interacting with the real world,
where it gets signatures by real users, or with a simulation, where all
signatures are simulated and do not contain information about the signer.
This alternative formulation is given in \figref{fig:exp-uas-simanon}, where
$\Osimanon = (\lbrace\HU,\CU\rbrace\GEN,\lbrace\II,\OO\rbrace\GEN,\lbrace\II,
\OO\rbrace\CORR,\OBTAIN,\WREG)$.

\begin{figure}[htp!]

  \centering
  \procedure[linenumbering]{$\ExpSimAnonb(1^\secpar)$}{%
    \pcif b = 0: \\
    \parm \gets \Setup(1^\secpar) \\    
    \pcind b^* \gets \adv^{\Osimanon,\SIGN,\OPEN}(\parm) \\
    \pcelse: \\
    \parm \gets \SIMSETUP(1^\secpar) \\
    \pcind b^* \gets \adv^{\Osimanon,\SIMSIGN,\SIMOPEN}(\parm) \\    
    \pcreturn b^*
  }
  
  \caption{Simulation-based anonymity experiment for \UAS schemes.}
  \label{fig:exp-uas-simanon}
\end{figure}

\begin{definition}{(Simulatable Anonymity of \UAS)}
  \label{def:sim-anonymity-uas}  
  We define the advantage \AdvSimAnon of $\adv$ against \ExpSimAnonb as
  $\AdvSimAnon=|\Pr\lbrack\ExpSimAnono(1^\secpar)=1\rbrack-
  \Pr\lbrack\ExpSimAnonz(1^\secpar)=1\rbrack|$.
  %
  An \UAS scheme satisfies simulatable anonymity if there exists simulators
  \SIMSETUP, \SIMSIGN and \SIMOPEN such that, for any p.p.t. adversary $\adv$,
  \AdvSimAnon is a negligible function of $1^\secpar$.
\end{definition}

Note that, for our generic construction \CUASGen, \SIMSETUP, \SIMSIGN, and
\SIMOPEN are straightforward, as all we need to do is:

\begin{itemize}
\item To build \SIMSETUP, just replace the $\NIZKSetup^{\Sign}$ and
  $\NIZKSetup^{\Open}$ algorithms within \Setup, by their respective
  $\NIZKSimSetup$ algorithms.
\item To build \SIMSIGN from the \SIGN oracle, just simulate the NIZK proof
  contained in the signatures produced by the \Sign algorithm.
\item To build \SIMOPEN from the \OPEN oracle, just simulate the NIZK proof
  showing opening correctness, instead of producing them as in the \Open
  algorithm.
\end{itemize}

Indistinguishability then follows from the zero-knowledge property of each
corresponding NIZK system. Since both scenarios are indistinguishable, and
all signature-related proofs received by the adversary in the simulation do not
depend on any witness (as they are simulated), then it follows that the
adversary cannot gain any information about signers in the real world.

With this alternative definition, we can achieve a generalisation of MPS
basically by, instead of returning the plaintext \yeval value, returning an
encryption of it under the opener's public key; additionally proving 
that the value used to compute the output of \finsp is the plaintext \yeval
value. In more detail, we define an extended $\NIZKRel_{\Sign-enc}$ relation,
and, given any \feval function, a variant $\feval^{enc}$ operating on an
encryption of \yeval. Both are specified in \esref{eq:uas-mps-funcs}.

\begin{align}
  & \NIZKRel_{\Sign-enc} \coloneqq \lbrace (\usk,\scred,\attrs_{\scred},\yinsp,r),
    (\msg,\feval,\yeval,\Ec,\sipk_{\scred},\Eek): \nonumber \\
  & \hspace*{6.00em}\Ec = \EEnc(\Eek,\yinsp;r) \land \nonumber \\
  & \hspace*{6.00em}\forall \cred \in \scred,\SBCMVerify(\ipk_{\cred},\cred,\usk,
  \attrs_{\cred}) = 1) \land \nonumber \\
  & \hspace*{6.00em}\yeval = \EEnc(\opk,\feval(\usk,\scred,\msg)) \land \nonumber \\
  & \hspace*{6.00em}\yinsp = \finsp(\feval(\usk,\scred,\msg),\usk,\scred,\msg)
    \rbrace \nonumber \\
  & \feval^{enc}(\usk,\scred,\msg) \coloneqq \lbrace \yeval
    \gets \feval(\usk,\scred,\msg); \yeval' \gets \EEnc(\opk,\yeval); \pcreturn
    \yeval' \rbrace \label{eq:uas-mps-funcs}
\end{align}

This would also require that the public-key encryption scheme is IND-CCA.

Given this approach, the need to provide an alternative definition for anonymity
is clearer. Namely, in the original \CHALb oracle, we restrict that the
signatures produced by both challenge users output the same \yeval value.
However, if instead of returning the plaintext \yeval value, we return an
encrypted version of it (as we do in the variant just described), this clearly
becomes unachievable.

\jdv{It would seem that the simulation-based anonymity definition is more
  general. Can we prove that? Concretely, if it is also good for selective
  disclosure, we may just adopt it as default.}

\subsection{Delegatable Credentials}
\label{ssec:uas-delcred}

Building delegatable credentials is also straightforward, thanks to the \fissue
function. For instance, take the simplification of level-based delegation, where
the owner of a credential of level $n$ can only issue credentials of level
$n+1$. Without loss of generality, assume that the credential level is encoded
in the first attribute of the credential (after the \usk), which we denote with
$a_1$. In this context, any owner of a credential of level $n$ can define an
issuance function $\fissue^n$, as in \eref{eq:uas-delcred-func}.

\begin{align}
  \fissue^n \coloneqq \lbrace \pcif a_1 = n+1: \pcreturn 1;
  \pcelse \pcreturn 0 \rbrace \label{eq:uas-delcred-func}
\end{align}

\jdv{\subsection{Functional Signatures}}
\jdv{I think that, by making \yeval explicit, \UAS is essentially a
  privacy-preserving extension to Functional Signatures. Still, I read that paper
  quite some time ago. Re-check.}

%%% Local Variables:
%%% mode: latex
%%% TeX-master: "uas"
%%% End:


%%% Local Variables:
%%% mode: latex
%%% TeX-master: "uas"
%%% End:

%\section{Efficiency and Experimental Analysis}
\label{sec:analysis}

\subsection{Computational and Communication Costs}
\label{ssec:asymptotic-analysis}

\comment{This section is just a placeholder for now. We should have some
  asymptotic analysis or similar.}

\subsection{Benchmarks}
\label{ssec:benchmarks}

\comment{This section is just a placeholder for now. Eventually, we'd like
  to have some PoC to report.}

%%% Local Variables:
%%% mode: latex
%%% TeX-master: "uas"
%%% End:

\section{Conclusion}
\label{sec:conclusion}

\comment{This section is just a placeholder for now...}

%%% Local Variables:
%%% mode: latex
%%% TeX-master: "uas"
%%% End:


\bibliographystyle{splncs04}
\bibliography{uas}

\appendix

\section{Cryptographic Building Blocks}
\label{app:crypto-building-blocks}

\subsection{Digital Signatures}
\label{sapp:digital-signatures}

We rely on digital signatures as a core building block. A digital signature
provides the functionality defined by the following syntax:

\begin{description}
\item[$\Sparm \gets \SSetup(\Ssecpar)$.] It produces public parameters for the
  other algorithms, given an input security parameter \Ssecpar.
\item[$(\Svk,\Ssk) \gets \SKeyGen(\Sparm)$.] Generates a verification-signing
  key pair.
\item[$\Ssig \gets \SSign(\Ssk,\msg)$.] Signs message \msg with signing key
  \Ssk, producing signature \Ssig,  
\item[$1/0 \gets \SVerify(\Svk,\Ssig,\msg)$.] Checks whether \Ssig is a valid
  signature over \msg, under verification key \Svk.
\end{description}

A digital signature scheme is correct if honestly generated signatures, using
honestly generated key pairs, are always accepted by \SVerify. 
%
A digital signature scheme \S has existential unforgeability if, for all p.p.t.
adversaries $\adv$, $\Pr[\ExpEUF(\Ssecpar) = 1]$ is a negligible function of the
security parameter.

\begin{figure}[ht!]
  \begin{minipage}[t]{\textwidth}
    \centering
    \procedure{$\ExpEUF(\Ssecpar)$}{%
      \Sparm \gets \SSetup(\Ssecpar) \\
      (\Svk,\Ssk) \gets \SKeyGen(\Sparm) \\
      (\Ssig,\msg) \gets \adv^{\SSign(\Ssk,\cdot)}(\Svk) \\
      \pcif \SVerify(\Svk,\Ssig,\msg) = 0: \pcreturn 0 \\
      \pcif \msg~\textrm{was not queried to \SSign}: \pcreturn 1 \\
      \pcreturn 0
    }
  \end{minipage}
  \label{fig:euf-game}
  \caption{Existential Unforgeability Game.}
\end{figure}

\subsection{Public-Key Encryption}
\label{sapp:pk-encryption}

A public-key encryption scheme is defined by the following algorithms:

\begin{description}
\item[$\Eparm \gets \ESetup(\Esecpar)$.] Produces public parameters \Eparm given
  a security parameter \Esecpar.
\item[$(\Eek,\Edk) \gets \EKeyGen(\Eparm)$.] Given public parameters \Eparm,
  produces an encryption-decryption keypair $(\Eek,\Edk)$.
\item[$\Ec \gets \EEnc(\Eek,\msg)$.] Encrypts message \msg with encryption key
  \Eek, producing ciphertext \Ec.
\item[$\msg \gets \EDec(\Edk,\Ec)$.] Decrypts ciphertext \Ec with decryption key
  \Edk.
\end{description}

A public-key encryption scheme is correct if, given a honestly generated key
pair $(\Eek,\Edk)$, produced with honestly generated parameters \Eparm,
$\Pr[\EDec(\Edk,\EEnc(\Eek,\msg))=\msg] = 1$ with overwhelming probability.

A public-key encryption scheme has IND-CCA2 security if
$\Pr[\ExpINDCCAiio(\Esecpar) = 1] - \Pr[\ExpINDCCAiiz(\Esecpar) = 1]|$ is
a negligible function of \Esecpar, for any p.p.t. adversary \adv, where
\ExpINDCCAiib is as defined in \figref{fig:indcca2-game}.

\begin{figure}[ht!]
  \begin{minipage}[t]{\textwidth}
    \centering      
    \procedure{$\ExpINDCCAiib(\Esecpar)$}{%
      \Eparm \gets \ESetup(\Esecpar) \\
      (\Eek,\Edk) \gets \EKeyGen(\Eparm) \\
      b^* \gets \adv^{\ELR(b,\cdot,\cdot),\EDEC(\Edk,\cdot)}(\Eek),
      ~\textrm{where:} \\
      \pcind \ELR(b,\msg_0,\msg_1)~\textrm{returns}~\EEnc(\Eek,\msg_b),
      ~\textrm{and} \\
      \pcind \EDEC(\Edk,\Ec)~\textrm{returns}~\EDec(\Edk,\Ec) \\
      \pcif \EDEC~\textrm{has been called with an output of \ELR, abort} \\
      \pcreturn b^*
    }
  \end{minipage}
  \label{fig:indcca2-game}
  \caption{IND-CCA2 Game.}
\end{figure}

\subsection{Commitments}
\label{sapp:commitments}

A commitment scheme is defined by the following algorithms:

\begin{description}
\item[$\Cparm \gets \CSetup(\Csecpar)$.] Given a security parameter \Csecpar,
  returns the public parameters \Cparm to commit messages.
\item[$\Ccom \gets \CCommit(\Cparm,\msg;r)$.] Given the public parameters and
  a message \msg, outputs a commitment \Ccom to \msg, for which randomness $r$
  from some predefined randomness space $\mathcal{R}$ is used.
\end{description}

Opening a commitment \Ccom means revealing the message \msg and randomness $r$
that were used to produce \Ccom. Commitment schemes are required to be binding
and (usually) hiding:

\begin{description}
\item[Binding.] Intuitively, the binding property of commitment schemes means
  that no adversary can change the message that has been committed to. More
  formally, $\Pr[\ExpComBind(\Csecpar) = 1]$ must be a negligible function of
  the security parameter.
\item[Hiding.] The hiding property captures that no adversary should be able to
  learn the message that was committed, when given only the commitment. This is
  formally defined through \ExpComHideb, where $|\Pr[\ExpComHideb(\Csecpar)=1|
  b=1] - \Pr[\ExpComHideb(\Csecpar)=1|b=0]|$ must be a negligible function of
  the security parameter.
\end{description}

\begin{figure}[ht!]
  \begin{minipage}[t]{0.5\textwidth}
    \procedure{$\ExpComBind(\Csecpar)$}{%
      \Cparm \gets \CSetup(\Csecpar) \\
      (\msg_0,r_0,\msg_1,r_1) \gets \adv(\Cparm) \\
      \Ccom_0 \gets \CCommit(\Cparm,\msg_0,r_0) \\
      \Ccom_1 \gets \CCommit(\Cparm,\msg_1,r_1) \\
      \pcif \msg_0 \neq \msg_1 \land \Ccom_0 = \Ccom_1: \pcreturn 1 \\
      \pcreturn 0
    }
  \end{minipage}
  \begin{minipage}[t]{0.5\textwidth}
    \procedure{$\ExpComHideb(\secpar)$}{%
      \Cparm \gets \CSetup(\secpar) \\
      (\msg_0,\msg_1,st) \gets \adv(\Cparm) \\
      r \getr \mathcal{R} \\
      \Ccom \gets \CCommit(\Cparm,\msg_b,r) \\
      b' \gets \adv(st,\Ccom) \\
      \pcreturn b'
    }
  \end{minipage}
  \label{fig:com-games}
  \caption{Games for commitment schemes.}
\end{figure}

\paragraph{Commitments on Blocks of Messages.} We also use an extension
of commitment schemes that allows committing to multiple messages at once. The
properties we need are the same, and their definitions are extended in the
natural way. Namely, $\CCommit$ receives a vector/block of messages, \msgset
instead of a single message. In the games, the adversary returns sets of
messages and, in the binding game, the comparison $\msg_0 \neq \msg_1$ now
compares sets $\msgset_0$ and $\msgset_1$, which must differ in at least one
element. This extension is straight-forward, for instance, from Pedersen
commitments \cite{bcc+15}.

\subsection{Simulation-Extractable Non-Interactive Zero-Knowledge
  Proofs of Knowledge}
\label{sapp:nizk}

Let \NIZKRel be an NP relation defined by pairs of elements $(\NIZKx,\NIZKw)$,
where \NIZKx is a statement and \NIZKw a witness proving that $(\NIZKx,\NIZKw)
\in \NIZKRel$. For concrete relations, we write $\NIZKRel = \lbrace (\NIZKx),
(\NIZKw): f(x,w) \rbrace$, where $f(x,w)$ is a boolean predicate denoting the
concrete conditions that \NIZKx and \NIZKw need to meet. The set of all \NIZKx
such that there exists a \NIZKw for which $(\NIZKx,\NIZKw) \in \NIZKRel$ is the
language, or \NIZKLang, for \NIZKRel. $\NIZKx \notin \NIZKLang$ means that
there is no $\NIZKw$ such that $(\NIZKx,\NIZKw) \in \NIZKRel$.

We use non-interactive zero-knowledge proofs of knowledge (NIZKPoK, or, for
short, NIZK) over NP relations, in the Common Reference String (CRS) model. A
NIZK system is a tuple $(\NIZKSetup,\NIZKProve,\NIZKVerify)$, defined as follows
\cite{gos06}:

\begin{description}
\item[$\NIZKcrs \gets \NIZKSetup(\NIZKsecpar)$.] Generates a CRS \NIZKcrs from
  security parameters \NIZKsecpar.
\item[$\NIZKproof \gets \NIZKProve(\NIZKcrs,\NIZKx,\NIZKw)$.] Given \NIZKcrs,
  statement \NIZKx, and witness \NIZKw, creates a proof \NIZKproof.
\item[$1/0 \gets \NIZKVerify(\NIZKcrs,\NIZKproof,\NIZKx)$.] Checks whether
  \NIZKproof is a valid proof for \NIZKx.
\end{description}

Any zero-knowledge proof of knowledge must meet completeness, soundness and
zero-knowledge,properties. We further need an extra property, called
\emph{simulation-extractability} \cite{cl06}, which amplifies the security
requirements of simulation soundness.
%
To define more formally the properties we need, we have to define three extra
algorithms:

\begin{description}
\item[$(\NIZKcrs,\NIZKtrap) \gets \NIZKSimSetup(\NIZKsecpar)$.] Produces a
  \NIZKcrs as the \NIZKSetup algorithm, along with a trapdoor \NIZKtrap.
\item[$\NIZKproof \gets \NIZKSim(\NIZKcrs,\NIZKtrap,\NIZKx)$.] Given a trapdoor
  \NIZKtrap produced by \NIZKSimSetup, and a statement $\NIZKx \in \NIZKLang$,
  produces a simulated proof \NIZKproof of $\NIZKx \in \NIZKLang$.
\item[$\NIZKw \gets \NIZKExtract(\NIZKcrs,\NIZKtrap,\NIZKx,\NIZKproof)$.] Given
  a trapdoor \NIZKtrap produced by \NIZKSimSetup, and a proof \NIZKproof for
  $\NIZKx \in \NIZKLang$, returns a valid \NIZKw for \NIZKx.
\end{description}

When we want to make explicit the NP relation \NIZKRel to which the previous
algorithms refer to, we use $\NIZKSetup^\NIZKRel,\NIZKProve^\NIZKRel$, 
$\NIZKVerify^\NIZKRel$, etc., and omit the \NIZK prefix and subindex when clear
from context. Altogether, the tuple $(\Setup,\Prove,\Verify,\SimSetup,\Sim,
\Extract)$ needs to meet the following properties:

\paragraph{Completeness.} %
Ensures that, for any $(\NIZKx,\NIZKw) \in \NIZKRel$, any honest prover will be
able to create a proof \NIZKproof that is accepted by any honest verifier, with
overwhelming probability. More precisely, $\Pr\lbrack\ExpNIZKComp(\NIZKsecpar)
\rbrack = 1$ with overwhelming probability, for any p.p.t. \adv, for
\ExpNIZKComp in \figref{fig:nizk-games}.

\paragraph{Soundness.} %
Ensures that no adversary can create proofs accepted by \Verify, for
statements $\NIZKx \notin \NIZKLang$, except with negligible probability. That
is, for \ExpNIZKSound as in \figref{fig:nizk-games}, $\Pr\lbrack\ExpNIZKSound
(\NIZKsecpar)\rbrack=1$ with overwhelming probability. If this holds only
against p.p.t. adversaries, we talk of Non-Interactive \emph{arguments}, while
if soundness holds even against unbounded adversaries, we talks about
Non-Interactive proofs.

\paragraph{Zero-knowledge.} %
Intuitively, captures that no information can be learned from a statement and
proof pair, beyond their validity (or not). This is captured by requiring the
adversary to distinguish between a run in the real world ($b=0$), where the
setup is done with \Setup, and $\adv$ has access to an honest prover
\Prove; and a run in an ideal world ($b=1$), where the setup is replaced by
\SimSetup, and proofs are simulated with the help of the trapdoor produced by
\SimSetup, except when $\adv$ specifies $(\NIZKx,\NIZKw) \notin \NIZKRel$. Note
that, in the context of simulation-extractable NIZK, this property not only
requires that the simulated proofs are indistinguishable to the real ones; it
also requires that \SimSetup is indistinguishable from \Setup. All this is
formalised by requiring that $|\Pr[\ExpNIZKZKb(\NIZKsecpar) = 1 | b = 1] -
\Pr[\ExpNIZKZKb(\NIZKsecpar) = 1 | b = 0]$ be a negligible function of \secpar,
where \ExpNIZKZKb is as defined in \figref{fig:nizk-games}.

\paragraph{Simulation-Extractability.} As stated, simulation-extractability
is an extension to simulation soundness. In a nutshell,
simulation-extractability requires that, even after having received a polynomial
number of simulated proofs of knowledge, no adversary can output a valid proof
of knowledge from which no witness can be extracted. It implies simulation
soundness, which ``just'' requires that no adversary can produce a valid proof
after having seen polynomially many simulated proofs (but does not guarantee
extraction). Formally, for simulation-extractability we require that
$\Pr[\ExpNIZKSimExt(\NIZKsecpar)] = 1$ is a negligible function of \secpar,
where \ExpNIZKSimExt is defined in \figref{fig:nizk-games}.

\begin{figure}[ht!]
  \begin{minipage}[t]{0.5\textwidth}
    \procedure{$\ExpNIZKComp(\NIZKsecpar)$}{%
      \NIZKcrs \gets \Setup(\NIZKsecpar) \\
      (\NIZKx,\NIZKw) \gets \adv(\NIZKcrs) \\
      \pcif (\NIZKx,\NIZKw) \notin \NIZKRel: \pcreturn 0 \\
      \NIZKproof \gets \Prove(\NIZKcrs,\NIZKx,\NIZKw) \\
      b \gets \Verify(\NIZKcrs,\NIZKproof,\NIZKx) \\
      \pcreturn b \\
    }
    \procedure{$\ExpNIZKSimExt(\NIZKsecpar)$}{%
      (\NIZKcrs,\NIZKtrap) \gets \SimSetup(\NIZKsecpar) \\
      (\NIZKx,\NIZKproof) \gets \adv^{\Sim'(\NIZKcrs,\NIZKtrap,\cdot,\cdot)}
      (\NIZKcrs) \\
      \pcind \textrm{Where}~\Sim'(\NIZKcrs,\NIZKtrap,\NIZKx,\NIZKw)~\textrm{returns} \\
      \pcind \pcind
      \Sim(\NIZKcrs,\NIZKtrap,\NIZKx)~\pcif (\NIZKx,\NIZKw) \in \NIZKRel \\
      \pcind \pcind \bot~\pcif (\NIZKx,\NIZKw) \notin \NIZKRel \\
      \NIZKw' \gets \Extract(\NIZKcrs,\NIZKtrap,\NIZKx,\NIZKproof) \\
      \pcif \Verify(\NIZKcrs,\NIZKproof,\NIZKx) = 1 \land
      (\NIZKx,\NIZKw') \notin \NIZKRel~\land \\
      \pcind \NIZKx~\textrm{was not queried to $\Sim$ via $\Sim'$}: \\
      \pcind \pcreturn 1 \\
      \pcreturn 0
    }    
  \end{minipage}
  \begin{minipage}[t]{0.5\textwidth}
    \procedure{$\ExpNIZKSound(\NIZKsecpar)$}{%
      \NIZKcrs \gets \Setup(\NIZKsecpar) \\
      (\NIZKproof,\NIZKx) \gets \adv(\NIZKcrs) \\
      \pcif \NIZKx \notin \NIZKLang \land
      \Verify(\NIZKcrs,\NIZKproof,\NIZKx): \\
      \pcind \pcreturn 0 \\
      \pcreturn 1 \\
    }
    
    \procedure{$\ExpNIZKZKb(\NIZKsecpar)$}{%
      \pcif b = 0: \\
      \pcind \NIZKcrs \gets \Setup(\NIZKsecpar) \\
      \pcind \pcreturn \adv^{\Prove(\NIZKcrs,\cdot,\cdot)}(\NIZKcrs) \\
      \pcif b = 1: \\
      \pcind (\NIZKcrs,\NIZKtrap) \gets \SimSetup(\secpar) \\
      \pcind \pcreturn \adv^{\Sim'(\NIZKcrs,\NIZKtrap,\cdot,\cdot)}(\NIZKcrs),~
      \textrm{where} \\
      \pcind \Sim'(\NIZKcrs,\NIZKtrap,\NIZKx,\NIZKw)~\textrm{returns} \\
      \pcind \pcind \Sim(\NIZKcrs,\NIZKtrap,\NIZKx)~\pcif (\NIZKx,\NIZKw)
      \in \NIZKRel \\
      \pcind \pcind \bot~\pcif (\NIZKx,\NIZKw) \notin \NIZKRel
    }    
  \end{minipage}
  \label{fig:nizk-games}
  \caption{Games for Simulation-Extractable NIZK schemes.}
\end{figure}

As studied in \cite{cl06}, simulation-extractable NIZKPoKs formalize the concept
of ``signatures of knowledge'' (see, e.g., \cite{cs97}). Which basically means
that, given an $(\NIZKx,\NIZKw)$ pair from an NP relation, we can treat \NIZKx
as a public key, and \NIZKw as its corresponding private key, and leverage them
to build digital signature schemes -- with the advantage of being able to do so
while proving arbitrary claims, as long as they can be represented as an NP
relation. We note that, given a simulation-extractable NIZK system, it is
straightforward to build a signature of knowledge by adding the message to be
signed in the statement of the NIZK.

\iffalse
\subsection{Signatures over Blocks of Messages}
\label{sapp:sbm}

A signature scheme on blocks of messages (\SBM) allows a signer to create a
single signature over a set of messages. The resulting signature is typically
more concise than just creating multiple signatures, and typical schemes
\cite{cl02,asm06,ps16} are additionally compatible with efficient proof
protocols over the signed messages. The functionality offered by an \SBM scheme
is as follow:

\begin{description}
\item[$\SBMparm \gets \SBMSetup(\SBMsecpar)$.] It produces public parameters
  for the other algorithms, given an input security parameter \SBMsecpar.
\item[$(\SBMvk,\SBMsk) \gets \SBMKeyGen(\SBMparm)$.] Generates a
  verification-signing key pair.
\item[$\SBMsig \gets \SBMSign(\SBMsk,\smsg)$.] Produces a signature \sig, over
  a block of messages \smsg, using signing key \SBMsk.
\item[$1/0 \gets \SBMVerify(\SBMvk,\SBMsig,\widetilde{\smsg})$.] Checks
  whether \SBMsig is a valid signature over the set of messages \smsg, under
  verification key \SBMvk.
\end{description}

An \SBM scheme must satisfy correctness and unforgeability properties.

\paragraph{Correctness.} %
Informally, an \SBM scheme is correct if signatures generated between an honest
party running running \SBMSign over \smsg, for honestly generated parameters
and key pairs, results in a signature over $\smsg \cup \overline{\smsg}$ that is
accepted by \SBMVerify.

\paragraph{Unforgeability.} %
It must be unfeasible for an adversary to produce signatures over blocks of
messages that have not been signed by the signer. More formally, an \SBM scheme
is unforgeable if, for all p.p.t. adversaries $\adv$, $\Pr[\ExpSBMEUF
(\SBMsecpar) = 1]$, as defined in \figref{fig:sbm-games}, is a negligible
function of the security parameter. 

\begin{figure}[ht!]
  \begin{minipage}[t]{\textwidth}
    \centering    
    \procedure{$\ExpSBMEUF(\SBMsecpar)$}{%
      \SBMparm \gets \SBMSetup(\SBMsecpar) \\
      (\SBMvk,\SBMsk) \gets \SBMKeyGen(\SBMparm) \\
      (\SBMsig,\smsg) \gets \adv^{\SSign(\SBMsk,\cdot)}(\SBMvk) \\
      \pcif \SBMVerify(\SBMvk,\SBMsig,\smsg) = 0: \pcreturn 0 \\
      \pcif \smsg~\textrm{was not queried to \SBMSign}: \pcreturn 1 \\
      \pcreturn 0
    }
  \end{minipage}
  \label{fig:sbm-games}
  \caption{Unforgeability game for \SBM schemes.}
\end{figure}
\fi

\subsection{Signatures over Blocks of Committed Messages}
\label{sapp:sbcm}

For our generic constructions, we use interactive signing protocols between a
user and a signer, where the user has a block of messages to sign blindly, and
both receive a common block of messages to be also included in the resulting
signature. This is precisely the case of partially blind signatures, that
collapse to blind signatures when there is no common message between user
and signer; and to conventional signatures when the user does not input a
message to be blindly signed \cite{ao00}. Partially blind signatures, as
blind signatures \cite{ps96}, cannot be modelled with the conventional security
against existential forgeries. Simply because the user is actually expected to
be able to create signatures on messages unknown to the signer, which formally
translates into the impossibility to check whether a signature output by the
adversary is over a message that has been queried to the signing oracle or not.
Thus, instead of using the conventional existential unforgeability property,
(partially) blind signature schemes move to the ``one-more'' paradigm, which
demands that no adversary can produce $n+1$ distinct signatures after having
interacted with the signing oracle at most $n$ times.

To the best of our knowledge, models of existing schemes for signing blocks of
messages like \cite{cl02,asm06,ps16,cdl16b} target the case of signing blocks
of \emph{plain} messages, and are subsequently informally extended to support
signing commitments to blocks of messages via interactive protocols. However,
they do not support signing both committed and plain messages (although the
extension is trivial); and, more importantly, do not give security models of
the resulting construction, nor of course prove its security. While the
extension seems straightforward, the modelling needs to be changed due to the
mentioned nuance of the conventional EUF notion not being compatible with
interactive signing protocols where the signer does not learn (some of) the
signed message(s). As we aim at using this variant as a generic building block,
we briefly model such a scheme for Signatures over Blocks of Committed Messages
(\SBCM).

The syntax for an \SBCM scheme is as follows:

\begin{description}
\item[$\SBCMparm \gets \SBCMSetup(\SBCMsecpar)$.] It produces public parameters
  for the other algorithms, given an input security parameter \SBCMsecpar.
\item[$(\SBCMvk,\SBCMsk) \gets \SBCMKeyGen(\SBCMparm)$.] Generates a
  verification-signing key pair.
\item[$\SBCMsig/\bot \gets \langle \SBCMCom(\SBCMvk,\overline{\smsg},\smsg),
  \SBCMSign(\SBCMsk,\smsg) \rangle$.] An interactive protocol between two
  parties with shared input a block of messages \smsg. The party running
  \SBCMCom further knows a set of messages $\overline{\smsg}$ to be
  block-committed and then signed along with \smsg. The party running \SBCMSign
  controls the signing key \SBCMsk. As a result, both parties get signature
  \SBCMsig over $\smsg \cup \overline{\smsg}$.
\item[$1/0 \gets \SBCMVerify(\SBCMvk,\SBCMsig,\overline{\smsg},\smsg)$.] Checks
  whether \SBCMsig is a valid signature over the set of messages
  $\overline{\smsg}$ and \smsg, under verification key \SBCMvk.
\end{description}

Note that these functionality is just a generalization of partially blind
signature schemes, where both user and signer can produce signatures over
blocks of (committed) messages. The correctness and security properties are
then defined as follows.

\paragraph{Correctness.} %
Informally, an \SBCM scheme is correct if signatures generated between an honest
party running \SBCMCom with message sets $\overline{\smsg}$ and \smsg, and an
honest signer running \SBCMSign over \smsg, for honestly generated parameters
and key pairs, results in a signature over $\overline{\smsg}$ and \smsg that is
accepted by \SBCMVerify. 

\paragraph{Unforgeability.} %
It must be unfeasible for an adversary to produce signatures over blocks of
messages that have not been signed (in plain, or committed shape) by the
signer. More formally, an \SBCM scheme is unforgeable if, for all p.p.t.
adversaries $\adv$, $\Pr[\ExpSBCMOMF(\SBCMsecpar) = 1]$, as defined in
\figref{fig:sbcm-games}, is a negligible function of the security parameter.
Note that this follows the ``one-more-forgery'' type of definition of blind
signatures \cite{bold02}.

\paragraph{Blindness.} %
Finally, the signer must not learn the plaintext values of the messages that are
signed in committed form. Note that this is a weaker notion than the usual
blindness property of blind signature schemes, where it is required that the
adversary cannot link a signature to the signing process that produced it.
Formally, blindness for \SBCM schemes is defined as in \ExpSBCMBlindb in
\figref{fig:sbcm-games}. An \SBCM scheme is blind if, for all p.p.t.
adversaries $\adv$, $|\Pr[\ExpSBCMBlindo(\SBCMsecpar) = 1] -
\Pr[\ExpSBCMBlindz(\SBCMsecpar) = 1]|$ is a negligible function of the security
parameter.

\begin{figure}[ht!]
  \begin{minipage}[t]{\textwidth}
    \centering
    
    \procedure{$\ExpSBCMOMF(\SBCMsecpar)$}{%
      \SBCMparm \gets \SBCMSetup(\SBCMsecpar) \\
      (\SBCMvk,\SBCMsk) \gets \SBCMKeyGen(\SBCMparm) \\
      ((\sig_1,\overline{\smsg}_1, \smsg_1),\dots,
      (\sig_{n+1},\overline{\smsg}_{n+1},\smsg_{n+1})) \gets
      \adv^{\langle \cdot, \SBCMSign(\SBCMsk,\cdot) \rangle}(\SBCMvk) \\
      \pcif \exists i \in [n+1]~\st~\SBCMVerify(\SBCMvk,\sig_i,
      \overline{\smsg}_i,\smsg_i) = 0: \pcreturn 0 \\
      \pcif \exists i \neq j \in [n+1]~\st~\smsg_i = \smsg_j \land
      \overline{\smsg}_i = \overline{\smsg}_j: \pcreturn 0 \\
      \pcif \langle \cdot, \SBCMSign(\SBCMsk,\cdot)\rangle~
      \textrm{was called more than $k$ times}: \pcreturn 0 \\
      \pcreturn 1 \\
    }
    
    \procedure{$\ExpSBCMBlindb(\SBCMsecpar)$}{%
      \SBCMparm \gets \SBCMSetup(\SBCMsecpar) \\
      (\SBCMvk,\SBCMsk) \gets \SBCMKeyGen(\SBCMparm) \\
      (\overline{\smsg}_0,\overline{\smsg}_1,\smsg) \gets
      \adv^{\langle \cdot, \SBCMSign(\SBCMsk,\cdot) \rangle}(\SBCMvk) \\
      \csig \gets \langle \SBCMCom(\SBCMvk,\overline{\smsg}_b,\smsg),
      \SBCMSign(\SBCMsk,\smsg) \rangle \\
      b^* \gets \adv^{\langle \cdot, \SBCMSign(\SBCMsk,\cdot) \rangle}(\SBCMvk,
      \csig) \\
      \pcif b = b^*: \pcreturn 1 \\
      \pcreturn 0
    }
  \end{minipage}
  \label{fig:sbcm-games}
  \caption{Games for \SBCM schemes.}
\end{figure}

Finally, in addition, we require that the produced signatures must be compatible
with (efficient) NIZK proofs of knowledge of a signature, and of (arbitrary)
claims over the signed (committed) messages.

\subsection{An Instantiation of \SBCM with BBS+}

Next, we give an instantiation of an \SBCM scheme, based on BBS+ signatures.
We emphasize again that this is essentially equivalent to the protocol for
signing committed block of messages in \cite{asm06} and, also, to the equivalent
ones in \cite{cl02,ps16} (although not for BBS+ signatures). The main difference
being that we allow merging committed blocks of messages and blocks of
(plaintext) messages into the same signature.
%
Note also that, in our instantiation, we just include a generic \NIZK, for some
relation over witness $\overline{\smsg}$ (i.e., the messages to be blindly
signed), and statement $(\Ccom,\smsg)$ (i.e., their block commitment, and the
plainly signed messages). This is intentional, as we want to support cases where
proving arbitrary predicates is possible (as opposed to ``just'' proving that
the commitment is over the messages in $\overline{\smsg}$).
%
When we want to make explicit the relation over which the employed \NIZK is
defined, we add a $\NIZKRel$ superscript to the algorithms.

\paragraph{$\SBCMparm \gets \SBCMSetup(\SBCMsecpar,\nattrs,\tnattrs)$.} %
Generates a bilinear group $\BB = (p,\GG_1,\GG_2,\GG_T,\gen{g}_1,\gen{g}_2,e)
\gets \PGen(\SBCMsecpar)$, and $\nattrs+\tnattrs+1$ additional generators
$\gen{g}$, $\gen{h}_1,...,\gen{h}_{\nattrs}$,$\gen{\th}_1,...,\gen{\th}_{\tnattrs}$
of $\GG_1$. Returns $\SBCMparm \gets (\SBCMsecpar,\nattrs,\tnattrs,\BB,
\gen{g},\gen{h}_1,...,\gen{h}_{\nattrs},\gen{\th}_1,...,\gen{\th}_{\tnattrs})$.
We assume that \SBCMparm is available to all other algorithms, even when not
explicitly passed as an argument.

\paragraph{$(\SBCMvk,\SBCMsk) \gets \SBCMKeyGen(\SBCMparm)$.} %
Parses \SBCMparm as $(\cdot,\cdot,(p,\GG_1,\GG_2,\GG_T,\gen{g}_1,\gen{g}_2,e),
\dots$ $\NIZKcrs \gets \NIZKSetup(\secpar)$. Outputs $\SBCMsk \gets \ZZ^*_p$,
and $\SBCMvk \gets (\NIZKcrs,\gen{g}_2^{\SBCMsk})$.

\paragraph{$\SBCMsig/\bot \gets \langle \SBCMCom(\SBCMvk,\overline{\smsg},
  \smsg), \SBCMSign(\SBCMsk,\smsg) \rangle$.} %

\begin{itemize}
\item \underline{User}: If $|\smsg|>\nattrs$ or $|\overline{\smsg}|>\tnattrs$,
  abort. Else, fetch fresh randomness $r \getr \ZZ^*_p$, compute $\Ccom \gets
  \gen{g}^r\prod_{i \in [|\overline{\smsg}|]}\gen{\th}_i^{\overline{\smsg}_i}$,
  and $\NIZKproof \gets \NIZKProve(\NIZKcrs,(r,\overline{\smsg}),\Ccom)$.
  Send $(\Ccom,\NIZKproof)$ to Issuer.
\item \underline{Signer}: If $|\smsg|>\nattrs$, abort. Else, run $\NIZKVerify
  (\NIZKcrs,\Ccom,\NIZKproof)$ and return $0$ if it fails. Else, compute
  $x,\tilde{s} \getr \ZZ^*_p, A \gets (\gen{g}_1\Ccom \gen{g}^{\tilde{s}}
  \prod_{i \in |\smsg|}\gen{h}_i^{\smsg_i})^{1/(\SBCMsk+x)}$. Send
  $(A,x,\tilde{s})$ to User.
\item \underline{User}: If $A = 1_{\GG_1}$: return $0$. Else, compute
  $s \gets r + \tilde{s}$. If $e(A,\gen{g}_2)^xe(A,\SBCMvk) \neq
  e(\gen{g}_1\Ccom\gen{g}^{\tilde{s}}\prod_{i \in |\smsg|}\gen{h}_i^{\smsg_i},
  \gen{g}_2)$: return $0$. Else, return $(A,x,s)$.
\end{itemize}

\paragraph{$1/0 \gets \SBCMVerify(\SBCMvk,\SBCMsig,\overline{\smsg},\smsg)$.} %
To verify a signature \SBCMsig, for message set $\overline{\smsg}$ that was
signed as a block commitment, and message set \smsg, signed as plaintext, parse
\SBCMsig as $(A,x,s)$ and check if $e(A,\gen{g}_2^x\SBCMvk) =
e(\gen{g}_1\gen{g}^s\prod_{i \in |\overline{\smsg}|}\gen{h}^{\overline{\smsg}_i}
\prod_{i \in |\smsg|}\gen{h}^{\smsg_i},\gen{g}_2)$

\paragraph{Proving Knowledge of Signature.} %
Note that proving knowledge of a BBS+ signature as produced in our \SBCM variant
is essentially the same as in \cite{asm06,cdl16b}, only needing to account for
the different basis for messages signed in committed and plain form.

\subsubsection{Security.} Next, we sketch that the given BBS+ interactive
signing protocol is correct and OMF-secure.

\paragraph{Correctness.} \todo{XXX}

\paragraph{OMF security.} \todo{YYY}



%%% Local Variables:
%%% mode: latex
%%% TeX-master: "uas"
%%% End:

\section{GSAC Detailed Formalisation}
\label{app:gsac-formal}

\subsection{Detailed Oracles}
\label{sapp:gsac-oracles}

{%\setlength\intextsep{\sep}
  \begin{figure*}[ht!]
    \centering
    \scalebox{0.9}{

      \begin{minipage}[t]{0.55\textwidth}

        \procedure{$\RREG(i)$}{%
          \pcreturn \trans[i] \\
        }

        \procedure{$\HUGEN(\uid)$}{%
          \pcif \uid \in \HU \lor \uid \in \CU: \pcreturn \bot \\
          (\upk,\usk) \gets \UKeyGen(\parm) \\
          \UK[\uid] \gets (\upk,\usk);
          \HU \gets \HU \cup \lbrace  \uid \rbrace \\
          \pcreturn \top \\
        }        
        
        \procedure{$\CUGEN(\uid,\upk)$}{%
          \pcif \uid \in \CU: \pcreturn \bot \\
          \CU \gets \CU \cup \lbrace \uid \rbrace \\          
          \pcif \uid \in \HU: \\
          \pcind \HU \gets \HU \setminus \lbrace \uid \rbrace; \\
          \pcind \pcreturn (\UK[\uid],\CRED[\uid]) \\
          \pcelse: \UK[\uid] = (\upk,\bot) \\          
          \pcreturn \top \\
        }

        \procedure{$\OBTISS(\uid,\cid,\attrs)$}{%
          \pcif \uid \in \CU \lor \uid \notin \HU: \pcreturn \bot \\
          \pcif \CRED[\cid] \neq \bot: \pcreturn \bot \\
          \langle \cred, \utrans \rangle \gets
          \langle \Obtain(\gpk,\PRVUK[\uid],\attrs), \\
          \pcind \pcind \pcind \pcind \pcind \pcind
          \Issue(\gpk,\isk,\attrs) \rangle \\
          \trans[\cid] \gets \utrans;~\CRED[\cid] \gets (\uid,\cred,\attrs) \\
          \pcreturn \top \\
        }        

        \procedure{$\OBTAIN(\uid,\cid,\attrs)$}{%
          \pcif \uid \in \CU \lor \uid \notin \HU: \pcreturn \bot \\
          \pcif \CRED[\cid] \neq \bot: \pcreturn \bot \\
          \langle \cred, \cdot \rangle \gets
          \langle \Obtain(\gpk,\PRVUK[\uid],\attrs),\adv \rangle \\
          \CRED[\cid] \gets (\uid,\cred,\attrs) \\
          \pcreturn \top \\
        }
        
      \end{minipage}
    }
    \scalebox{0.9}{
      
      \begin{minipage}[t]{.5\textwidth}

        \procedure{$\WREG(i,\rho)$}{%
          \trans[i] \gets \rho \\
        }        

        \procedure{$\ISSUE(\uid,\cid,\attrs)$}{%
          \pcif \uid \notin \CU: \pcreturn \bot \\
          \pcif \CRED[\cid] \neq \bot: \pcreturn \bot \\
          \langle \cdot, \utrans \rangle \gets
          \langle \adv, \Issue(\gpk,\isk,\attrs) \rangle \\
          \trans[\cid] \gets \utrans;~
          \CRED[\cid] \gets (\uid,\cdot,\attrs) \\
          \pcreturn \top \\          
        }        

        \procedure{$\SIGN(\cid,\dattrs,\msg)$}{%
          \uid \gets \OWNR[\cid] \\
          \pcif \uid \notin \HU: \pcreturn \bot \\
          \cred \gets \CRED[\cid] \\
          \sig \gets \Sign(\gpk,\PRVUK[\uid],\cred,\dattrs,\msg) \\
          \SIG[\cid] \gets \SIG[\cid] \cup \lbrace (\sig,\dattrs,\msg) \rbrace \\
          \pcreturn \sig \\
        }

        \procedure{$\OPEN(\sig)$}{%
          \textrm{Let}~\cid~\textrm{be s.t.}~(\sig,\dattrs,\msg) \in \SIG[\cid] \\
          \pcif \textrm{no such \cid exists, or there is more than one}: \\
          \pcind \pcreturn \bot \\
          \pcif \sig \in \CSIG: \pcreturn \bot \\
          (\upk,\oproof) \gets \Open(\gpk,\osk,\sig,\dattrs,\msg) \\
%          \OSIG \gets \OSIG \cup \lbrace (\sig,\upk,\cred) \rbrace \\
%          \CCRED \gets \CCRED \cup \lbrace \cid \rbrace \\
          \pcreturn (\upk,\oproof) \\
        }

        \procedure{$\CHALb(\ccid_0,\ccid_1,\dattrs,\msg)$}{%
          \pcif \dattrs \nsubseteq \ATTR[\ccid_0] \cap \ATTR[\ccid_1]:
          \pcreturn \bot \\
          \pcif \cuid_0 \neq \bot \lor \cuid_1 \neq \bot: \pcreturn \bot \\
          \cuid_0 = \OWNR[\ccid_0];~\cuid_1 = \OWNR[\ccid_1] \\
          \pcif \cuid_0 \notin \HU \lor \cuid_1 \notin \HU: \pcreturn \bot \\
 %         \pcif \ccid_0 \in \CCRED \lor \ccid_1 \in \CCRED: \pcreturn \bot \\
          \csig \gets \Sign(\gpk,\PRVUK[\cuid_b],\CRED[\ccid_b],
          \dattrs,\msg) \\
          \CSIG \gets \CSIG \cup \lbrace \csig \rbrace \\
          \pcreturn \csig
        }
        
      \end{minipage}
      
    }

    \caption{Detailed oracles available in our model for \GSAC schemes.}
    \label{fig:oracles}
  \end{figure*}
}

\subsection{Detailed Proofs}
\label{sapp:gsac-proofs}

\commentwho{Jesus}{Strictly, in our generic construction we require IND-CCA (is
  that a strict requirement? or can it be downgraded to IND-CPA?), but in the
  following concrete instantiation, we use ElGamal, which is IND-CPA. However,
  we actually ``lift'' ElGamal with NIZKs of the encrypted plaintexts, which is
  essentially IND-CCA as far as I know. Still, make sure to mention that.}

\begin{proof}[\thmref{thm:correctness-gsac}]
  \todo{XXX}
\end{proof}

\begin{proof}[\thmref{thm:anonymity-gsac}]
  We prove that the probability of an adversary distinguishing signatures by
  challenge users is negligible. For the proof, we restrict to the case of
  allowing only one query to the challenge oracle. The extension to a polynomial
  number of queries given in \cite{bsz05} applies here, with the corresponding
  security loss.

  We begin with game $G_0=\ExpGSACAnonb$. From it, we build $G_1$, where we
  simply replace the \NIZKSetup functions for $\NIZKRel_{\Sign}$ and
  $\NIZKRel_{\Open}$ within the \Setup algorithm. $G_1$ is indistinguishable
  from $G_0$ due to the zero-knowledge property of the NIZK systems.

  From $G_1$, we build $G_2$, where within the challenge signature produced by
  \CHALb, we replace the \Ccom value encrypted within \Sign with a random
  string of the appropriate length (e.g., the same length of the encrypted
  message in $G_1$). The corresponding proof for $\NIZKRel_{\Sign}$ is thus
  also simulated. The zero-knowledge and simulation-extractability properties
  of the NIZKs for $\NIZKRel_{\Sign}$, and IND-CCA of the encryption
  scheme, ensure that the result is indistinguishable to the adversary.

  Now, observe that in $G_2$, the challenge signature received by the adversary
  is completely independent from the challenge bit $b$. Moreover, as argued,
  $G_2$ is indistinguishable from $G_0$, where the adversary plays the original
  \ExpGSACAnonb game. Thus, our generic construction of \GSAC satisfies
  anonymity, except with negligible probability.
  %
  \qed
\end{proof}

\begin{proof}[\thmref{thm:trace-gsac}]
  Assuming the NIZK systems are sound, we build an adversary against existential
  unforgeability of the signature scheme for blocks of committed messages.
  %
  Consider the following events:

  \begin{description}
  \item[$O$.] The \Open algorithm returns $\bot$.
  \item[$J$.] The \Judge algorithm returns $0$.
  \item[$D$.] No credential owned by a user in \CU contains \dattrs.
  \end{description}

  Given these events, the adversary wins if:

  \begin{description}
  \item[$W_1 = O$.] \Open returns $\bot$ (line 5 of \ExpGSACTrace).
  \item[$W_2 = \overline{O} \land J$.] \Open does not fail, but \Judge rejects
    the output of \Open (line 7 of \ExpGSACTrace).
  \item[$W_3 = \overline{O} \land \overline{J} \land D$.] \Open does not fail,
    \Judge accepts the output by \Open, but no corrupt user has a credential
    containing \dattrs (line 8 of \ExpGSACTrace).
  \end{description}

  Clearly, since the group signature is verified at step 4 of the game, if
  verification succeeds, soundness of $\NIZK_{\Sign}$ implies that $(\Ccom_m,
  \Ec,\dattrs) \in \NIZKLang_{\Sign}$, except with negligible probability
  $2^{-\NIZKsecpar}$. Also, observe that each of the winning events for the
  adversary described above are disjoint. Thus:

  \begin{equation}
    \AdvGSACTrace(\secpar) = \Pr[W_1]+\Pr[W_2]+\Pr[W_3]+\negl(\NIZKsecpar)
  \end{equation}

  $\Pr[W_1]=0$ is directly deduced from the fact that $(\Ccom_m,\Ec,\dattrs)
  \in \NIZKLang_{\Sign}$ due to soundness of $\NIZKRel_{\Sign}$. Concretely,
  we know that \Open does not abort due to rejection of the signature by
  \Verify (as this is checked in line 4 of \ExpGSACTrace too) and, thus,
  $\Ec = \EEnc(\opk,\Ccom)$ which, by correctness of the encryption algorithm,
  implies that $\Ccom = \EDec(\osk,\Ec)$, where $\upk=\Ccom=\CCommit(\usk;0)$.
  Completeness of $\NIZK_{\Open}$ thus implies that the \Open algorithm can
  compute the opening proof \oproof, and return $(\upk,\oproof)$.
  
  $\Pr[W_2] = 0$ is similarly deduced from soundness of $\NIZK_{\Sign}$ and
  completeness of $\NIZK_{\Open}$, since the opening proof is computed
  honestly in line 6 of \ExpGSACTrace.

  Finally, assume that there is an adversary $\adv$ against \ExpGSACTrace that
  wins by event $W_3$. Then, it is straight forward to build an adversary \advB
  against unforgeability of the signature scheme on blocks of committed
  messages. Namely, \advB creates the parameters with \Setup and the opening
  key pair with \OKeyGen, but instead of running \IKeyGen, sets the verification
  key to the one received in the \ExpSBCMOMF game. In calls by $\adv$ to \OBTISS
  and \ISSUE oracles, \advB replaces the \Issue algorithm with corresponding
  calls to its $\langle \cdot,\SBCMSign \rangle$ oracle. The other oracles are
  unchanged. Without loss of generality, suppose that $\adv$ makes a total of
  $n$ queries to \OBTISS and \ISSUE, combined; which implies that \advB makes
  a total of $n$ queries to its own oracle. If, in such an execution, $\adv$
  wins by event $W_3$, it means that it has produced a signature \sig, revealing
  attribute set \dattrs, that is accepted by \Verify, for which \Open does not
  fail, and for which an honestly generated open proof is accepted by \Judge,
  but such that no issued credential contains \dattrs. Soundness of
  $\NIZK_{\Sign}$ implies that \sig is a valid signature over $\attrs \cup
  \lbrace \usk \rbrace$, and $\dattrs \subseteq \attrs$; but \sig has not been
  produced via a call to the $\langle \cdot,\SBCMSign \rangle$ oracle. Thus,
  \advB can output at least $n+1$ valid signatures, which contradicts security
  of \SBCM.
  %
  By assumption of security of \SBCM, $\Pr[W_3]$ must therefore be negligible in
  $\SBCMsecpar$ and, consequently: $\AdvGSACTrace(\secpar) = \negl(\SBCMsecpar)+
  \negl(\NIZKsecpar)$, which is negligible.  
  \qed
\end{proof}

\begin{proof}[\thmref{thm:frame-gsac}]
  We build an adversary \advB who, given an adversary $\adv$ against
  \ExpGSACNonframe, breaks the hiding property of \C.

  To do so, we start with $G_0=\ExpGSACNonframe$. From $G_0$, we build $G_1$
  by replacing $\NIZKSetup^{\NIZKRel_{\Issue}}$ and  $\NIZKSetup^{\NIZKRel_{\Sign}}$
  with $\NIZKSimSetup^{\NIZKRel_{\Issue}}$ and $\NIZKSimSetup^{\NIZKRel_{\Sign}}$,
  within the \Setup algorithm. Due to the zero-knowledge property of both
  $\NIZK^{\NIZKRel_{\Issue}}$ and $\NIZK^{\NIZKRel_{\Sign}}$, $G_1$ is
  indistinguishable from $G_0$.

  We now build \advB from $G_1$. Recall that \advB is an adversary against
  hiding of \C and, thus, receives a commitment \Ccom, committing to one out
  of two possible messages, $\msg_0$ and $\msg_1$, which \advB picks at random
  from $\AttrSpace$. \advB initializes everything as $G_1$ and, in addition,
  chooses a random integer $u \in [q]$, where $q$ will be the maximim number of
  users to let $\adv$ create via calls to the \HUGEN oracle (note though that
  $q$ can be as large as desired, as it remains a polynomial function of the
  security parameter). In the $u$-th call to the \HUGEN oracle, \advB ignores
  sets $(\upk,\usk)$ to $(\Ccom,\bot)$. Calls to the \OBTAIN and \SIGN oracles
  are dealt with as usual, except when $\adv$ requests credentials or signatures
  on behalf of the $u$-th user. In that cases, \advB simulates the corresponding
  NIZK proofs by calling $\NIZKSim^{\NIZKRel_{\Issue}}$ or
  $\NIZKSim^{\NIZKRel_{\Sign}}$, respectively. Due to the
  simulation-extractability property of both NIZK systems, the output produced
  by the oracles is indistinguishable from that in $G_1$ and, consequently, from
  $G_0=\ExpGSACNonframe$.

  Finally, assume that $\adv$ wins \ExpGSACNonframe with non-negligible
  probability. This means that it outputs a $(\sig = (\Ec,\pi_\sig),\dattrs,
  \msg,\upk,\pi)$, where $(\sig,\dattrs,\msg)$ are accepted by \Verify, and
  $(\upk,\pi,\sig,\dattrs,\msg)$ are accepted by \Judge. Thus, after soundness
  of $\NIZK^{\NIZKRel_{\Sign}}$ and $\NIZK^{\NIZKRel_{\Open}}$, we know that
  \upk is the correct public key associated to an honest user \uid, but as per
  the winning condition, \sig has not been produced via a call to the \SIGN
  oracle for a \cid owned by \uid. Therefore, we can extract the \usk
  corresponding to \upk from the $\pi_{\sig}$ proof. Since executions in the
  environment prepared by \advB are indistinguishable to $\adv$ from executions
  of $G_0$, we can assume that the probability that the \uid associated to the
  \sig output by $\adv$ equals $1/q$. In that case, \usk precisely matches
  $\msg_b$, and \advB wins the hiding game against \C. Since we assume \C
  to be hiding, this probability must therefore be negligible.   
  \qed
\end{proof}

\subsubsection{\GSAC with Interactive Presentations}
\label{ssap:interactive-gsac}

Next, we sketch proofs of security for the case of \GSAC with interactive
presentations like in the case of anonymous credentials.

\paragraph{Anonymity.} To see why the interactive variant is anonymous, assume
an adversary $\adv$ against anonymity in that case. We build \advB against the
non-interactive \ExpGSACAnonb as follows. \advB initializes everything as in the
\ExpGSACAnonb game. When $\adv$ initiates a call to its interactive \SIGN (resp.
\CHALb) oracle, \advB picks a random number, and sends it to the adversary,
along with the response of its own (non-interactive) \SIGN (resp. \CHALb)
oracle where, to the message passed as parameter to its oracle, concatenates the
produced random number. After such simulation, \advB just outputs whatever
$\adv$ outputs. Clearly, the simulation is perfect and, if $\adv$ wins with
non-negligible probability in the interactive case, then so does \advB in the
non-interactive case.

\paragraph{Traceability.} The simulation described for the anonymity case (i.e.,
\advB choosing the random numbers, and concatenating them to the message to be
signed) applies here too. Thus, an adversary against traceability in the
interactive case, can be used to build an adversary that breaks traceability in
the non-interactive counterpart.

\paragraph{Non-frameability.} The simulation described for the anonymity case
(i.e., \advB choosing the random numbers, and concatenating them to the message
to be signed) applies here too. Thus, an adversary against non-frameability in
the interactive case, can be used to build an adversary that breaks
non-frameability in the non-interactive counterpart. 

\subsection{Concrete Instantiation with BBS+}
\label{sapp:gsac-instantiation}

Intuitively, the credentials that we generate for our \GSAC construction are
Pedersen commitments to $(\usk,\credid,\attrs)$ tuples; i.e., they have
the following structure: $\gen{h}_0^r\gen{h}_1^{\usk}\prod_{i \in [2,\nattrs+1]}
\gen{h}_i^{\attrs_i}$, where the exponent in $\gen{h_0}$ is a fresh random value
that ensures hiding, the exponent in $\gen{h_1}$ encodes the user private
key, and the remaining attributes are encoded in different
exponentiations. This will actually be part of a BBS+ signature, which makes it
easy to add subsequent and efficient zero-knowledge proofs. Concretely, we will
be using several \NIZK proof systems, one for the $\langle \Obtain,\Issue
\rangle$ interactive protocol, another for signing, and a third one for opening.
Each will have its own relation, that we define in the corresponding algorithm.
The concrete algorithms are as follows.

\paragraph{$\Setup(\secpar,\nattrs) \rightarrow \parm$.} %
Generates a bilinear group $\BB = (p,\GG_1,\GG_2,\GG_T,\gen{g}_1,\gen{g}_2,e) \gets
\PGen(\secpar)$, $\nattrs+4$ additional generators $\gen{g},\gen{h},\gen{h}_0,...,
\gen{h}_{\nattrs+1}$ of $\GG_1$. Returns $\parm \gets
(\secpar,\nattrs,\BB,\gen{g},\gen{h},\gen{h}_0,...,\gen{h}_{\nattrs+1})$.

\paragraph{$\IKeyGen(\parm) \rightarrow (\ipk,\isk)$.} %
Parses \parm as $(\secpar,\dots)$ and runs $\NIZKcrs \gets \NIZKSetup(\secpar)$.
Outputs $\isk \gets \ZZ^*_p$, and $\ipk \gets (\NIZKcrs,\gen{g}_2^{\isk})$.

\paragraph{$\OKeyGen(\parm) \rightarrow (\opk,\osk)$.} %
Outputs $\osk \gets \ZZ^*_p$, and $\opk \gets \gen{g}^{\osk}$. This is the
opener's ElGamal encryption key pair \needcite.

\paragraph{$\UKeyGen(\parm) \rightarrow (\upk,\usk)$.} %
Outputs $\usk \gets \ZZ^*_p$, and $\upk \gets \gen{h}_1^{\usk}$. \upk will
simply be used to compute Pedersen commitments \needcite to \usk. However,
it is useful to precompute it, and we treat it as a sort of ``public key''.

\paragraph{$\langle \Obtain(\gpk,\usk,\attrs),\Issue(\gpk,\isk,\attrs) \rangle
  \rightarrow \langle \cred/\bot,\utrans/\bot \rangle$.} %
This interactive protocol is essentially a BBS+ signing  process. The user
commits to its private key, proves knowledge of the corresponding private
key, and asks the issuer to sign the commitment, along with any other arbitraty
set of revealed attributes \attrs. For proving knowledge of the private key, we
define relation $\NIZKRel_{\Issue} = \lbrace (r, \usk), \Ccom: \Ccom =
\gen{h}_0^r\gen{h}_1^{\usk}\rbrace$.

\begin{itemize}
\item \underline{User}: Fetch fresh randomness $r \getr \ZZ^*_p,
  \Ccom \gets \gen{h}_0^r\gen{h}_1^{\usk}$, and compute $\NIZKproof \gets
  \NIZKProve^{\NIZKRel_{\Issue}}(\NIZKcrs,(r,\usk),
  \Ccom)$. Send $(\Ccom,\NIZKproof)$ to Issuer.
\item \underline{Issuer}: Run $\NIZKVerify^{\NIZKRel_{\Issue}}(\NIZKcrs,\Ccom,
  \NIZKproof)$ and return $0$ if it fails. Else, compute
  $x,\tilde{s} \getr \ZZ^*_p, A \gets
  (\gen{g}_1\Ccom h_0^{\tilde{s}} \prod_{i \in \attrs}
  \gen{h}_i^{\attrs_i})^{1/(\isk+x)}$.
  Send $(A,x,\tilde{s})$ to User, and output $\utrans \gets
  (\Ccom,(A,x,\tilde{s}),\attrs,\NIZKproof)$.
\item \underline{User}: If $A = 1_{\GG_1}$: return $0$. Else, compute
  $s \gets r + \tilde{s}$. If $e(A,\gen{g}_2)^xe(A,\ipk) \neq
  e(\gen{g}_1\Ccom\gen{h}_0^{\tilde{s}}\prod_{i \in \attrs}\gen{h}_i^{\attrs_i},
  \gen{g}_2)$: return $0$. Else, return
  $(A,x,s)$.
\end{itemize}

\paragraph{$\Sign(\gpk,\usk,\cred,\dattrs,\msg) \rightarrow \sig$.} %
To create a signature, the user first randomizes its BBS+ credential \cred and
encrypts its public key \upk (note that $\upk=\gen{h}_1^{\usk}$ with the
opener's encryption key. Then, we extend the usual NIZK protocol for proving
knowledge of BBS+ signatures \cite{cdl16b} to, in addition to proving knowledge
of a signature (which, for us is the user credential), also prove that the
attribute encoding \usk within the credential match the encrypted value.
Finally, the user commits to \msg, and includes this commitment within the
proof. For this, we define relation
$\NIZKRel_{\Sign}= \lbrace (r,\usk,\attrs,r_2,r_3,s',\msg),(\dattrs,c_1,c_2,
\Cmsg): \Cmsg = \gen{h}^\msg \land c_1 = g^r \land c_2 = \opk^r\gen{h}_1^{\usk}
\land \hat{A}/d = (A')^{-x}\gen{h}_0^{r_2} \land
\gen{g}_1 \prod_{i \in \dattrs} \gen{h}_i^{\attrs_i} =
d^{r_3}\gen{h}_0^{-s'}\gen{h}_1^{-\usk}
\prod_{i \notin \dattrs} \gen{h}_i^{-\attrs_i} \rbrace$.

\begin{itemize}
\item Parse \gpk as $(\ipk,\opk)$, and \cred as $(A,x,s)$.
\item Re-randomize \cred as $r_1,r_2 \getr
  \ZZ^*_p, r_3 \gets r_1^{-1}, s' \gets s - r_2r_3$, $A' \gets A^{r_1},
  \hat{A} \gets (A')^{-x}(\gen{g}_1\gen{h}_0^s\gen{h}_1^{\usk}
  \prod_{i \in \attrs}\gen{h}_i^{\attrs_i})^{r_1}$,
  $d \gets (\gen{g}_1\gen{h}_0^s\gen{h}_1^{\usk}
  \prod_{i \in \attrs}\gen{h}_i^{\attrs_i})^{r_1}\gen{h}_0^{-r_2}$.
\item Encrypt $\upk=\gen{h}_1^{\usk}$ with ElGamal as $r \getr \ZZ^*_p,
  c \gets (c_1 = \gen{g}^r,c_2 = \opk^r\gen{h}_1^{\usk})$.
\item Compute $\Cmsg \gets \gen{h}^\msg$ and
  $\NIZKproof \gets \NIZKProve^{\NIZKRel_{\Sign}}(\NIZKcrs,
  (r,\usk,\attrs,r_2,r_3,s',\msg), (\dattrs,c_1,c_2,\Cmsg))$.
\item Return $\sig \gets (c=(c_1,c_2),(A',\hat{A},d),\NIZKproof)$.
\end{itemize}

\paragraph{$\Verify(\gpk,\sig,\dattrs,\msg) \rightarrow 1/0$.} %
Parse \gpk as $(\ipk,\opk)$ and $\sig$ as $(c=(c_1,c_2),
(A',\hat{A},d), \NIZKproof)$. Check that $A' \neq 1_{\GG_1}$ and $e(A',\ipk) =
e(\hat{A},\gen{g}_2)$. Compute $\Cmsg \gets \gen{h}^\msg,$ and return
$\NIZKVerify^{\NIZKRel_{\Sign}}(\NIZKcrs,(\dattrs,c_1,c_2,\Cmsg),
\NIZKproof)$.

\paragraph{$\Open(\gpk,\osk,\sig,\dattrs,\msg)
  \rightarrow (\upk,\oproof)/\bot$.} %
For open algorithms, we define the following relation for correct decryption of
ElGamal ciphertexts: $\NIZKRel_{\Open} = \lbrace \osk,(c_1,c_2,\msg):
c_2/c_1^{\osk} = \msg \rbrace$. Given $\NIZKRel_{\Open}$, the opener first,
checks that $\Verify$ accepts the signature, and aborts otherwise. If \sig
is accepted, then parses \sig as $(c=(c_1,c_2),\cdot,\cdot)$.
Sets $\upk \gets c_2/c_1^{\osk}$. Finally, computes the proof of correct
decryption by running $\oproof \gets \NIZKProve^{\NIZKRel_{\Open}}(\NIZKcrs,
\osk,(\upk,c_1,c_2))$, and returns $(\upk,\oproof)$.

\paragraph{$\Judge(\gpk,\upk,\oproof,\sig,\dattrs,\msg)
  \rightarrow 1/0$.} %
First, check that $\Verify$ accepts \sig, and abort otherwise. If the signature
is accepted, then parse \sig as $(c=(c_1,c_2),\cdot,\cdot)$ and return
$\NIZKVerify^{\NIZKRel_{\Open}}(\NIZKcrs,(c_1,c_2,\upk),\oproof)$.

% \commentwho{Jesus}{For consistency with \UAS, maybe remove the need to return
%   \attrs in \Open, and return \dattrs instead. Then, we do not need to pass
%   \trans as a parameter to \Open, as \dattrs is already attested for in the
%   signature being opened. It also makes sense from the point of view that we
%   are already revealing \upk, which would allow tracing; but otherwise don't
%   reveal anything else about the attributes of \upk -- attributes that whoever
%   requested the signature (the verifier) didn't seem to consider necessary, as
%   s/he only requested \dattrs. Still, if we make this modification, mention the
%   possibility to return \attrs, and the option to do it via adding \trans as a
%   parameter to \Open (or including an encryption of all attributes in the
%   signature, which is probably unrealistic.)}

%%% Local Variables:
%%% mode: latex
%%% TeX-master: "uas"
%%% End:


\end{document}

%%% Local Variables:
%%% mode: latex
%%% TeX-master: "uas"
%%% End:
