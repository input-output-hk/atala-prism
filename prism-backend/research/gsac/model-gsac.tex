\subsection{Model for \GSAC}
\label{ssec:model-gsac}

While group signatures and anonymous credentials share quite common syntax and
security properties, which are even frequently formalised in a similar way,
there are aspects that need to be considered with care.

The first choice that has to be made is whether to make sign/show and
verification a non-interactive or an interactive process -- in group signatures,
we are in the former case, whereas in anonymous credentials, we are in the
latter. While both options are of course valid, this has direct impact in the
modelling. We choose the non-interactive approach, and then prove how to
generically translate it into an interactive process \todo{do we?}. The reason
is that this results in a more flexible building block, suitable for a wider set
of scenarios. And, syntactically, it seems more natural to combine two
non-interactive processes into an interactive one, than the other way around.

We also need to take into account that, in group signature schemes, users only
get one membership credential -- typically bound to their personal secret
key --, which they then use to create as many group signatures as they want. On
the other hand, in anonymous credentials, users can get as many credentials as
they desire, which can (typically) then be showed also arbitrarily many times.
In this regard, we follow a somehow intermediate approach: users own a single
personal secret key, which they can use to get as many credentials as they wish.
Subsequently, they can use any of those credentials to create \GSAC signatures.
This has, as expected, some nuanced impact in the modelling. For instance, in
group signatures, tracing and non-frameability are frequently dependent on the
open function, which (among other things) returns an index pointing to the join
transcript associated with the user who produced the signature given by the
adversary as a response to the game's challenge. In our approach, though, each
user may have many such transcripts -- one per credential that the user
obtains. However, note that what actually provides meaning to these notions
is not the index; but the associated user key pair, which is moreover necessary
to prove that the opening has been performed honestly. Thus, with some
restructuring of the corresponding games, we can still capture the main
meaning of these traceability and non-frameability notions from group
signatures. This also impacts anonymity, since in group signatures, this is
modelled by challenging the adversary to guess which user (which implicitly
translates to ``which user credential'') out of two challenge users was
leveraged to produce the challenge signature. But again, in our case, users may
have multiple credentials. However, it seems sufficient to adopt a similar
anonymity flavour to that of anonymous credentials, where the adversary has to
distinguish between presentations (signatures) produced from two arbitrary
credentials, independently on whether these credentials were obtained by the
same user, or not.

Getting closer to the anonymity definition of anonymous credentials also has the
side effect to introduce the notions of attributes for free into the domain of
group signatures -- which, up to now, had considered membership credentials
\emph{without} attributes. Yet another side effect of introducing multiple
attributes (if we think from a group signature perspective) is that, in order
to avoid trivial wins, the anonymity game has to enforce that both challenge
credentials reveal the same predicate on their attributes.
\commentwho{Jesus}{Again, taking into account ABGS, this is no longer true.
  Cross-check and re-write.}

In the model described below, we follow closely the definitions from
\cite{bsz05} for group signatures, and \cite{fhs19} for anonymous credentials
(which are in turn influenced by \cite{bsz05}).

\subsection{Syntax}
\label{ssec:syntax}

\begin{description}
\item[$\parm \gets \Setup(\secpar)$.] Given a security parameter \secpar,
  returns a global system parameter variable \parm.
\item[$(\ipk,\isk) \gets \IKeyGen(\parm)$.] Given global system parameters
  \parm, returns the issuer's key pair. Hereafter, we assume that the public
  part \ipk is added to the group public key \gpk.
\item[$(\opk,\osk) \gets \OKeyGen(\parm)$.] Given global system parameters
  \parm, returns an opener's key pair. Hereafter, we assume that the public part
  \opk is added to the group public key \gpk.
\item[$(\upk,\usk) \gets \UKeyGen(\parm)$.] Given global system parameters
  \parm, returns a user's key pair.
\item[$\langle \cred/\bot,\utrans/\bot \rangle \gets
  \langle \Obtain(\gpk,\usk,\Attrs),\Issue(\gpk,\isk,\Attrs) \rangle$.]
  This interactive protocol lets a user with key pair (\upk,\usk) running the
  \Obtain process, obtain a credential \cred with attributes \Attrs, by
  communicating with an issuer, with private key \isk, running the \Issue
  counterpart. The user outputs the produced credential \cred, while the issuer
  outputs the protocol transcript \utrans for the produced credential.
\item[$\sig \gets \Sign(\gpk,\usk,\cred,\DAttrs,\msg)$.] The user with
  with secret key \usk, who obtained a credential \cred, produces a signature
  \sig over message \msg, revealing subset of attributes \DAttrs in \cred.
\item[$1/0 \gets \Verify(\gpk,\sig,\DAttrs,\msg)$.] Checks whether \sig is a
  valid signature revealing the attribute set \DAttrs of its signing credential,
  over message \msg.
\item[$(\upk,\Attrs,\oproof)/\bot \gets
  \Open(\gpk,\osk,\trans,\sig,\DAttrs,\msg)$.]
  Given a signature \sig over message \msg, produced by a credential with
  attributes containing the set \DAttrs owned by a user with key \upk, returns
  the attribute set \Attrs encoded in the credential used to create the
  signature, and a proof of opening correctness \oproof; or $\bot$ if the
  opening process fails.
\item[$1/0 \gets \Judge(\gpk,\upk,\Attrs,\oproof,\sig,\DAttrs,
  \msg)$.] Checks if \oproof is a valid opening correctness proof stating that
  the user with key \upk and attributes \Attrs generated signature \sig over
  message \msg, revealing attributes $\DAttrs \subseteq \Attrs$. 
\end{description}

The correctness and security properties are defined with the help of the
following sets of oracles, and global variables that help oracles and games
keep consistent state.

\paragraph{Global variables.} %
All the honestly generated users (i.e., all honestly generated user key pairs,
since we assume a one-to-one relationship between user and user key pair), as
well as all honestly generated credentials, are assigned identifiers. Wex
typically write as \uid for users' identifiers, and \cid for credentials'. The
adversary can
refer to any individual user or credential using the corresponding identifier --
even though he may not know the actual contents of the key pairs or credentials.
In games that involve challenge user or credential identifiers, we use \cuid and
\ccid to refer to these challenge user/credential.
%
The games, usually through the oracles, keep track of the users (through
set \UK) and credentials (through table \CRED) that are created as a result of
oracle calls by the adversary, as well as the attributes that each credential
is associated (through table \ATTR), and the user to which each credential has
been issued to (through table \OWNR). We use the user and credential identifiers
to reference specific user key pairs, credentials, and credential attributes.
For instance, $\UK[\uid]$ refers to the user key pair corresponding to user
with identifier \uid, $\PUBUK[\uid]$ refers to the public key of that key pair,
and $\PRVUK[\uid]$ to the private key; credential $\CRED[\cid]$ refers to the
credential data associated to the credential with identifier \cid; $\ATTR[\cid]$
is the set of attributes that was assigned to the credential with identifier
\cid; and $\OWNR[\cid]$ is set to the \uid corresponding to the user that
credential \cid was issued to.
%
Additionally, the games keep track of all honest and corrupt users that have
been generated, through sets \HU and \CU, respectively. They also keep track of
the signatures that have been honestly produced, through the \SIG table. Since
signatures are produced through credentials, the \SIG table is indexed with
{\cid}s. For the anonymity game, we also need to keep track of the challenge
signatures that the adversary has obtained, in order to prevent trivial wins
by allowing any of them to be opened. 
%
All global variables are initially set by the games to empty values (denoted
with $\bot$), and all tables/sets are initialized as empty tables/sets, denoted
with $\emptyset$. Also, for readability, we abuse the syntax as follows: we
write $\CRED[\uid]$ to mean $\CRED[\cid]$ for all $\cid$ such that
$\OWNR[\cid] = \uid$; we also sometimes use \upk as a ``synonym'' of \uid, as it
is possible to do a reverse lookup from the \UK table (i.e., $\CRED[\upk]$ can
be replaced by $\CRED[\uid]~\st~\UK[\uid]=(\upk,\cdot)$); \todo{Something else?}

\paragraph{Oracles.} %
Oracles are the interface of the adversary with the corresponding games. In
other words: through these oracles, the game enviroment exposes to the adversary
functionality that could otherwise be executed only by honest parties with
private knowledge -- knowledge that would make the adversary capable of
trivially breaking the security properties formalized in the experiments.
In the game-based definitions of our \GSAC model, we leverage the following
oracles, which are formally defined in \appref{app:gsac-formal}.

\begin{description}
\item[\RREG.] Reads the given transcript table entry.
\item[\WREG.] Sets a transcript table entry to the given value.
\item[\HUGEN.] Adds a new honest user to the game, by honestly generating
  the user's key pair.
\item[\CUGEN.] If the specified user identifier is not already in the game,
  sets the user public key to a value given by the adversary. If it already
  exists, and is associated to an honest user, then it reveals to the adversary
  the user's secret key.
\item[\OBTISS.] Lets the adversary add a new honestly generated credential to
  the game, on behalf of an honest user.
\item[\OBTAIN.] Enables the adversary to play the role of a dishonest issuer
  in games that support it, by letting it interact with honest users who want to
  receive credentials.
\item[\ISSUE.] Allows the adversary to play the role of dishonest users,
  requesting an honest issuer to produce credentials fo them.
\item[\SIGN.] Lets the adversary get signatures from credentials belonging
  to honest users.
\item[\OPEN.] Given an honestly produced signature, lets the adversary learn
  which credential was used to produce it.
\item[\CHALb.] Upon receiving two challenge credentials, a common intersecting
  set of attributes, and a message, returns a signature produced by one of these
  two credentials, defined by the bit $b$, which is established in the anonymity
  game.
\end{description}

\paragraph{Correctness.} %
Correctness of \GSAC schemes is formalized through the experiment in
\figref{fig:exp-gsac-corr}. It states that a signature for some message \msg,
revealing attributes \DAttrs, which was honestly produced though a credential
that was obtained by an honest user interacting with an honest issuer, with a
set of attributes \Attrs such that $\DAttrs \subseteq \Attrs$, must be accepted
by \Verify. Moreover, an honestly produced correctness proof of opening for such
signature, revealing the public key pair of the user, must also be accepted by
\Judge.

\begin{figure}[htp!]
  \procedure{$\ExpGSACCorrect(1^\secpar)$}{%
     \parm \gets \Setup(1^\secpar) \\
     (\ipk,\isk) \gets \IKeyGen(\parm);~(\opk,\osk) \gets \OKeyGen(\parm);~ 
     \gpk \gets (\ipk,\opk) \\
    (\cid,\DAttrs,\msg) \gets \adv^{\HUGEN,\OBTISS,\RREG}(\gpk) \\
    \pcif \DAttrs \nsubseteq \ATTR[\cid] :
    \pcreturn \bot \\
    \sig \gets \Sign(\gpk,\PRVUK[\OWNR[\cid]],\CRED[\cid],\DAttrs,\msg) \\
    (\upk,\Attrs,\oproof) \gets \Open(\gpk,\osk,\trans,\sig,\DAttrs,\msg) \\
    \pcif \Verify(\gpk, \sig, \DAttrs,\msg) = 0 \lor
    \Judge(\gpk,\upk,\Attrs,\oproof,\sig,\DAttrs,\msg) = 0~\lor\\
     \pcind \DAttrs \nsubseteq \Attrs \lor \Attrs \neq \ATTR[\cid]: \\
     \pcind \pcreturn 1 \\
    \pcreturn 0
  }
  \caption{Correctness experiment for \GSAC schemes.}
  \label{fig:exp-gsac-corr}
\end{figure}

\subsection{Security Properties}
\label{ssec:security}

\paragraph{Anonymity.} %
In group signatures, anonymity captures that no adversary must be able to learn,
from any group signature, anything about its signer. In anonymous credentials,
it requires that no adversary should learn anything about the holder of a
credential that has been successfully shown, beyond that he owns a credential
containing the revealed attributes. In both GS and AC, it is also typically
required that
multiple signatures/presentations by the users are unlinkable. The approach to
formally state this property is in both cases frequently the same: the adversary
picks two (honest) users (or credentials in the AC case), the game randomly
chooses one of them, and lets the adversary request challenge
signatures/presentations from it. The adversary wins if it succeeds in guessing
which was the chosen user/credential better than guessing at random. In group
signatures, the game must also restrict the adversary from opening challenge
signatures. In anonymous credentials, the game must further constraint the
adversary to pick credentials that have some common subset of attributes, and
to use that common subset to request the challenge presentations. Here, we need
to take into account both. Furthermore, a key difference with group signatures
is that the game requires the adversary to output credential identifiers, rather
than user identifiers. Specifically, this means that the adversary may actually
output two credentials that belong to the same user. Therefore, in some sense,
the anonymity we get is more general than that of group signatures. The formal
specification of the anonymity game is given in \figref{fig:exp-gsac-anonb}.
%
Similarly, this notion of anonymity is also stronger than the notion of
anonymity in both group signatures and anonymous credentials, in the sense
that calls to the \OPEN oracle reveal the public key and credential of the users
that (via the \SIGN and \CHALb oracles) create the signatures that the adversary
sees. Therefore, all security relies \emph{only} on the user secret key.

\begin{figure}[htp!]
  \procedure{$\ExpGSACAnonb(1^\secpar)$}{%
     \parm \gets \Setup(1^\secpar) \\
     (\ipk,\isk) \gets \IKeyGen(\parm);~(\opk,\osk) \gets \OKeyGen(\parm);~ 
     \gpk \gets (\ipk,\opk) \\
     (\cid_0,\cid_1,\status) \gets
     \adv^{\HUGEN,\CUGEN,\WREG,\OBTAIN,\SIGN,\OPEN}(\choose,\gpk,\isk) \\
     b^* \gets
     \adv^{\HUGEN,\CUGEN,\WREG,\OBTAIN,\SIGN,\OPEN,\CHALb}(\guess,\gpk,\isk,\status) \\
     \pcreturn b^*
  }
  \caption{Anonymity experiment for \GSAC schemes.}
  \label{fig:exp-gsac-anonb}
\end{figure}

\begin{definition}{(Anonymity of \GSAC)}
  We define the advantage \AdvGSACAnon of $\adv$ against \ExpGSACAnonb as
  $\AdvGSACAnon=|\Pr\lbrack\ExpGSACAnono(1^\secpar)=1\rbrack-
  \Pr\lbrack\ExpGSACAnonz(1^\secpar)=1\rbrack|$.
  %
  A \GSAC scheme satisfies anonymity if, for any p.p.t. adversary $\adv$,
  \AdvAnon is a negligible function of $1^\secpar$.
\end{definition}

\paragraph{Traceability.} %
Traceability is one of the unforgeability-related properties in group
signatures. It captures that any signature accepted by \Verify needs to open
to one of the users that joined the group. While there is no traceability notion
in anonymous credentials, it is natural to map it to their unforgeability
property; if only because both require the issuer to be honest. Unforgeability
in anonymous credentials typically ensures that no adversary can get a verifier
to accept a credential presentation requiring a set of attributes that is not
contained in one of the credentials controlled by the adversary.
%
Our notion of traceability for \GSAC combines both requirements. It assumes an
honest issuer, as otherwise the adversary can create untraceable credentials at
will. The game then lets the adversary add honest and corrupt users, create
honest signatures, and open them. The adversary wins if, after this interaction,
is able to produce a $(\sig,\DAttrs,\msg)$ tuple that is accepted by \Verify,
but either cannot be opened, it can be opened but the proof is rejected by
\Judge, or even though it is accepted by \Judge, there is no credential issued
to the user pointed to by \Open that contains the set of attributes \DAttrs. We
formally define traceability in the \ExpTrace experiment in
\figref{fig:exp-gsac-trace}.

\begin{figure}[htp!]
  \procedure{$\ExpGSACTrace(1^\secpar)$}{%
    \parm \gets \Setup(1^\secpar) \\
    (\ipk,\isk) \gets \IKeyGen(\parm);~(\opk,\osk) \gets \OKeyGen(\parm);~ 
    \gpk \gets (\ipk,\opk) \\
    (\sig,\DAttrs,\msg) \gets
    \adv^{\HUGEN,\CUGEN,\RREG,\OBTISS,\ISSUE,\SIGN,\OPEN}(\gpk,\osk) \\
    \pcreturn 1~\pcif \Verify(\gpk,\sig,\DAttrs,\msg) = 1~\land~( \\
    \pcind \pccomment{\Open fails} \\
    \pcind \Open(\gpk,\osk,\trans,\sig,\DAttrs,\msg) = \bot~\lor \\
    \pcind \pccomment{\Open succeeds, but ...} \\
    \pcind \Open(\gpk,\osk,\utrans,\sig,\DAttrs,\msg) =
    (\upk,\Attrs,\oproof)~\land\\
    \pcind \pcind \pccomment{... the proof is rejected by \Judge, or
      it is accepted but \DAttrs is not contained in \Attrs} \\
    \pcind \pcind (\Judge(\gpk,\upk,\Attrs,\oproof,\sig,\DAttrs,\msg) = 0
    \lor \DAttrs \nsubseteq \Attrs) \\
    ) \\
    \pcreturn 0
  }
  \caption{Traceability experiment for \GSAC schemes.}
  \label{fig:exp-gsac-trace}
\end{figure}

\begin{definition}{(Traceability of \GSAC)}
  We define the advantage \AdvGSACTrace of $\adv$ against \ExpGSACTrace as
  $\AdvGSACTrace=\Pr\lbrack\ExpGSACTrace(1^\secpar)=1\rbrack$.
  %
  A \GSAC scheme satisfies traceability if, for any p.p.t. adversary $\adv$,
  \AdvGSACTrace is a negligible function of $1^\secpar$.
\end{definition}

\paragraph{Non-frameability.} %
Non-frameability variants are a core unforgeability-type property in group
signatures. However, no
similar property is modelled for anonymous credentials (\todo{see the discussion
  in \secref{sec:introduction} for further detail}). It is a quite strong
property, as it must be ensured even in the presence of dishonest issuer and
opener. Intuitively, it prevents the adversary from creating a signature that
frames an honest user. Depending on the inspection capabilities of the scheme,
this framing could be done in different ways; i.e., by convincing third parties
that signatures by different (possibly corrupt) users are linked, or directly
by having open proofs output the identity of a user who did not create the
signature being opened.
%
In \GSAC schemes, in order for a user to be framed, the adversary first needs to
create a $(\sig,\DAttrs,\msg,\upk,\Attrs,\oproof)$ such that the signature,
attributes, and message are accepted by \Verify, and which, along with the user
public key, and open correctness proof, are accepted by \Judge.
Then, the adversary wins the game if the owner of the credential is honest, but
the signature was not produced via the \SIGN oracle, or if the set of attribute
disclosed in the signature are not contained in the set of attributes revealed
during \Open.

\begin{figure}[htp!]
  \procedure{$\ExpGSACNonframe(1^\secpar)$}{%
    \parm \gets \Setup(1^\secpar) \\
     (\ipk,\isk) \gets \IKeyGen(\parm);~(\opk,\osk) \gets \OKeyGen(\parm);~ 
     \gpk \gets (\ipk,\opk) \\
     (\sig,\DAttrs,\msg,\upk,\Attrs,\oproof) \gets
     \adv^{\HUGEN,\CUGEN,\WREG,\OBTAIN,\SIGN}(\gpk,\isk,\osk) \\
     \textrm{Let}~\uid~\textrm{be st}~\PUBUK[\uid] = (\upk,\cdot) \\
     \pcreturn 1~\pcif \Verify(\gpk,\sig,\DAttrs,\msg) = 1 \land 
     \Judge(\gpk,\upk,\Attrs,\oproof,\sig,\DAttrs,\msg) = 1~\land \\
     \pcind \uid \in \HU \land (\sig \notin \SIG[\uid] \lor
     \DAttrs \nsubseteq \Attrs) \\
     \pcreturn 0
  }
  \caption{Non-frameability experiment for \GSAC schemes.}
  \label{fig:exp-gsac-frame}
\end{figure}

\begin{definition}{(Non-frameability of \GSAC)}
  We define the advantage \AdvGSACNonframe of $\adv$ against \ExpGSACNonframe as
  $\AdvGSACNonframe=\Pr\lbrack\ExpGSACNonframe(1^\secpar)=1\rbrack$.
  %
  A \GSAC scheme satisfies non-frameability if, for any p.p.t. adversary $\adv$,
  \AdvNonframe is a negligible function of $1^\secpar$.
\end{definition}

\paragraph{Discussion.} %
\todo{Modelling choices: HU and CU lists are typically updated in join protocols
  in the GS literature. Here, it is best to follow the AC literature, as a same
  user (key) may be associated to multiple join protocols. So there are HUGEN
  and CUGEN oracles.}

\todo{Argue that our properties cover (and augment) a direct combination
  (even if possible) of their GS and AC counterparts. Specifically, anonymity
  of \GSAC is stronger than anonymity in GS and AC, and the same for trace and
  non-frame.}

\todo{It would be nice to prove that a \GSAC scheme to which we remove the
  \Open/\Judge functions becomes an AC scheme. And conversely, a \GSAC scheme
  where all credentials have no attributes, and where we restrict to only
  one credential per user, becomes a conventional GS scheme.}

%%% Local Variables:
%%% mode: latex
%%% TeX-master: "uas"
%%% End:
