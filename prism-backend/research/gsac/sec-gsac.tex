\section{\GSAC: Merging Group Signatures and Anonymous
  Credentials}
\label{sec:gsac}

Before moving into our full-fledged scheme for universal anonymous signatures,
we take a first step which, albeit somehow obvious and simple, helps understand
our final goal. Namely, we give a scheme that combines in a quite direct manner
the functionality of group signature schemes, and that of anonymous credential
schemes.

\paragraph{Why is this merge necessary in the real world?} %
Despite being an \emph{a priori} simple combination of both schemes we argue
that such a combination:

\begin{itemize}
\item Enables a natural way to incorporate accountability within 
  privacy-preserving identities. Furthermore, it does so in a formal way,
  via the traceability notion and, crucially, the non-frameability one
  (which is not present in anonymous credentials).
\item Provides meaning to the conventional \Open functionality of group
  signatures, which traditionally has only required that an abstract concept of
  ``identity'' be returned. Concretely, assuming honest issuance (which is
  already a requirement in anonymous credentials; and in group signatures for
  traceability), we can make specific claims about the attributes associated to
  de-anonymized signers. Otherwise, without integrating attributes within the
  issued credentials, the value returned by \Open serves barely for blacklisting
  or similar functionalities -- but otherwise gives no information straight away
  about the signer's identity, forcing implementors to rely to non-cryptographic
  means to retrieve such information.
\end{itemize}

\paragraph{Intuition for our design and construction.} %
Functionally, we combine both worlds. Namely, a user can get as many credentials
as s/he wants -- although all are bound to the same user key --, and signatures
(or credential presentations) of the same credential must be unlinkable. In this
simplified first step, we restrict to selective attribute disclosure, meaning
that signatures produced by a credential reveal only a subset of the attributes
in that credential. Given such a signature, there is a trusted opener who can
de-anonymize its signer, producing back its public key, as well as a proof of
correct opening.

With respect to the model, privacy-wise this direct combination requires that
any two signatures (credential presentations) reveal nothing about the
credential used to produce them, beyond the revealed attributes.
Security-wise we stick to the conventional separation of
unforgeability-related properties in group signatures. That is, traceability
requires that, in the presence of an honest issuer, all valid signatures must
originate from a credential issued by the issuer; and, from the AC domain, we
import the requirement that no user can claim having attributes that have not
been issued to a single credential. As in group signatures, if the issuer is
corrupt, all we can hope for is that honest users do not get undeserved blame;
but, other than that, the adversary can obviously obtain credentials with
arbitrary attributes. Still, note that this is a clear gain with respect to
conventional anonymous credential systems, which offer no security against a
malicious issuer.
%
Finally, in this first step, we restrict ourselves to signatures/presentations
involving only one credential, to a system with only one issuer and opener  and,
as stated, to selective attribute disclosure.

\subsection{Model for \GSAC}
\label{ssec:model-gsac}

While group signatures and anonymous credentials share quite common syntax and
security properties, which are even frequently formalised in a similar way,
there are aspects that need to be considered with care.

The first choice that has to be made is whether to make sign/show and
verification a non-interactive or an interactive process -- in group signatures,
we are in the former case, whereas in anonymous credentials, we are in the
latter. While both options are of course valid, this has direct impact in the
modelling. We choose the non-interactive approach, and then sketch how to
generically (and trivially) translate it into an interactive process. The reason
is that this results in a more flexible building block, suitable for a wider set
of scenarios. And, syntactically, it seems more natural to combine two
non-interactive processes into an interactive one, than the other way around.

We also need to take into account that, in group signature schemes, users only
get one membership credential -- typically bound to their personal secret
key --, which they then use to create as many group signatures as they want. On
the other hand, in anonymous credentials, users can get as many credentials as
they desire, which can (typically) then be showed also arbitrarily many times.
In this regard, we follow a somehow intermediate approach: users own a single
personal secret key, which they can use to get as many credentials as they wish.
Subsequently, they can use any of those credentials to create \GSAC signatures.
This has, as expected, some nuanced impact in the modelling. For instance, in
group signatures, tracing and non-frameability are frequently dependent on the
open function, which (among other things) returns an member index. In our
approach, though, each user may have many such transcripts -- one per credential
that the user obtains. However, note that what actually provides meaning to
these notions is not the index; but the attributes in the credential, which is
moreover necessary
to prove that the opening has been performed honestly. Thus, with some
restructuring of the corresponding games, we can still capture the main
meaning of these traceability and non-frameability notions from group
signatures. This also impacts anonymity, since in group signatures, this is
modelled by challenging the adversary to guess which user (which implicitly
translates to ``which user credential'') out of two challenge users was
leveraged to produce the challenge signature. But again, in our case, users may
have multiple credentials. However, it seems sufficient to adopt a similar
anonymity flavour to that of anonymous credentials, where the adversary has to
distinguish between presentations (signatures) produced from two arbitrary
credentials, independently on whether these credentials were obtained by the
same user, or not.
%
Getting closer to the anonymity definition of anonymous credentials also has the
side effect to introduce the notions of attributes for free into the domain of
group signatures -- which, up to now, had considered membership credentials
\emph{without} attributes. Yet another side effect of introducing multiple
attributes (if we think from a group signature perspective) is that, in order
to avoid trivial wins, the anonymity game has to enforce that both challenge
credentials reveal the same predicate on their attributes.

\subsubsection{Syntax for \GSAC schemes}
\label{sssec:syntax-gsac}

\begin{description}
\item[$\parm \gets \Setup(\secpar)$.] Given a security parameter \secpar,
  returns a global system parameter variable \parm.
\item[$(\ipk,\isk) \gets \IKeyGen(\parm)$.] Given global system parameters
  \parm, returns the issuer's key pair. Hereafter, we assume that the public
  part \ipk is added to the group public key \gpk.
\item[$(\opk,\osk) \gets \OKeyGen(\parm)$.] Given global system parameters
  \parm, returns an opener's key pair. Hereafter, we assume that the public part
  \opk is added to the group public key \gpk.
\item[$(\upk,\usk) \gets \UKeyGen(\parm)$.] Given global system parameters
  \parm, returns a user's key pair.
\item[$\langle \cred/\bot,\utrans/\bot \rangle \gets
  \langle \Obtain(\gpk,\usk,\attrs),\Issue(\gpk,\isk,\attrs) \rangle$.]
  This interactive protocol lets a user with key pair (\upk,\usk) running the
  \Obtain process, obtain a credential \cred with attributes \attrs, by
  communicating with an issuer, with private key \isk, running the \Issue
  counterpart. The user outputs the produced credential \cred, while the issuer
  outputs the protocol transcript \utrans for the produced credential.
\item[$\sig \gets \Sign(\gpk,\usk,\cred,\dattrs,\msg)$.] The user with
  with secret key \usk, who obtained a credential \cred, produces a signature
  \sig over message \msg, revealing subset of attributes \dattrs in \cred.
\item[$1/0 \gets \Verify(\gpk,\sig,\dattrs,\msg)$.] Checks whether \sig is a
  valid signature revealing the attribute set \dattrs of its signing credential,
  over message \msg.
\item[$(\upk,\attrs,\oproof)/\bot \gets
  \Open(\gpk,\osk,\sig,\dattrs,\msg)$.]
  Given a signature \sig over message \msg, produced by a credential with
  attributes containing the set \dattrs, returns the public key \upk of the
  signer, and the attribute set \attrs contained in the credential used to
  generate the signature, along with a proof of opening correctness \oproof; or
  $\bot$ if the opening process fails.
\item[$1/0 \gets \Judge(\gpk,\upk,\attrs,\oproof,\sig,\dattrs,
  \msg)$.] Checks if \oproof is a valid opening proof for the
  statement: ``The owner of a credential over attribute set \attrs, and owner
  of public key \upk, created signature \sig, which reveals attribute set
  $\dattrs \subseteq \attrs$''.
\end{description}

The correctness and security properties are defined with the help of the
following sets of oracles and global variables that help oracles and games
keep consistent state.

\paragraph{Global variables.} %
All the honestly generated users (i.e., all honestly generated user key pairs,
since we assume a one-to-one relationship between user and user key pair), as
well as all honestly generated credentials, are assigned identifiers. We
typically write \uid for users' identifiers, and \cid for credentials'. The
adversary can
refer to any individual user or credential using the corresponding identifier --
even though he may not know the actual contents of the key pairs or credentials.
In games that involve challenge user or credential identifiers, we use \cuid and
\ccid to refer to these challenge user/credential.
%
The games keep track of the user keys (through table \UK) and credentials (through
table \CRED) that are created as a result of oracle calls by the adversary. We
use the user and credential identifiers to reference specific user key pairs,
credentials, and credential attributes.
For instance, $\UK[\uid]$ refers to the user key pair corresponding to user
with identifier \uid, $\PUBUK[\uid]$ refers to the public key of that key pair,
and $\PRVUK[\uid]$ to the private key; credential $\CRED[\cid]$ refers to the
credential data associated to the credential with identifier \cid; $\ATTR[\cid]$
is the set of attributes that was assigned to the credential with identifier
\cid; and $\OWNR[\cid]$ is set to the \uid corresponding to the user that
credential \cid was issued to.
%
Additionally, the games keep track of all honest and corrupt users that have
been generated through sets \HU and \CU, respectively. They also keep track of
the signatures that have been honestly produced, through the \SIG table. Since
signatures are produced through credentials, the \SIG table is indexed with
{\cid}s. For the anonymity game, we also need to keep track of the challenge
signatures that the adversary has obtained, in order to prevent trivial wins
by allowing any of them to be opened. 
%
All global variables are assumed to be initially set by the games to empty
values, and all tables/sets are initialized as empty tables/sets. Also, for
readability, we abuse the syntax and
write $\CRED[\uid]$ to mean $\CRED[\cid]$ for all $\cid$ such that
$\OWNR[\cid] = \uid$.% ; we also sometimes use \upk as a ``synonym'' of \uid, as it
% is possible to do a reverse lookup from the \UK table (i.e., $\CRED[\upk]$ can
% be replaced by $\CRED[\uid]~\st~\UK[\uid]=(\upk,\cdot)$); \todo{Something else?}

\paragraph{Oracles.} %
Oracles are the interface of the adversary with the corresponding games. In
other words: through these oracles, the game enviroment exposes to the adversary
functionality that could otherwise be executed only by honest parties with
private knowledge -- knowledge that would make the adversary capable of
trivially breaking the security properties formalized in the experiments.
In the game-based definitions of our \GSAC model, we leverage the following
oracles, which are formally defined in \appref{app:gsac-formal}.

\begin{description}
\item[\HUGEN.] Adds a new honest user to the game, by honestly generating
  the user's key pair.
\item[\CUGEN.] If the specified user identifier is not already in the game,
  sets the user public key to a value given by the adversary. If it already
  exists, and is associated to an honest user, then it reveals to the adversary
  the user's secret key.
\item[\OBTISS.] Lets the adversary add a new honestly generated credential to
  the game, on behalf of an honest user.
\item[\OBTAIN.] Enables the adversary to play the role of a dishonest issuer
  in games that support it, by letting it interact with honest users who want to
  receive credentials.
\item[\ISSUE.] Allows the adversary to play the role of dishonest users,
  requesting an honest issuer to produce credentials for them.
\item[\RREG.] Reads the given transcript table entry.
\item[\WREG.] Sets a transcript table entry to the given value.  
\item[\SIGN.] Lets the adversary get signatures from credentials belonging
  to honest users.
\item[\OPEN.] Given an honestly produced signature, lets the adversary learn
  the public key of the user who produced it.
\item[\CHALb.] Upon receiving two challenge credentials, a common intersecting
  set of attributes, and a message, returns a signature produced by one of these
  two credentials, defined by the bit $b$, which is established in the anonymity
  game.
\end{description}

\paragraph{Correctness.} %
Correctness of \GSAC schemes is formalized through the experiment in
\figref{fig:exp-gsac-corr}. It states that a signature for some message \msg,
revealing attributes \dattrs, which was honestly produced through a credential
that was obtained by an honest user interacting with an honest issuer, with a
set of attributes \attrs such that $\dattrs \subseteq \attrs$, must be accepted
by \Verify. Moreover, an honestly produced correctness proof of opening for such
signature, revealing the public key pair of the user and credential attributes,
must also be accepted by \Judge.

\begin{definition}{(Correctness of \GSAC)}
  \label{def:correctness-gsac}
  A \GSAC scheme is correct if, for any p.p.t. adversary $\adv$,
  $\Pr[\ExpGSACCorrect(1^\secpar) = 1]$ is a negligible function of \secpar.
\end{definition}

\begin{figure}[htp!]
  \procedure[linenumbering]{$\ExpGSACCorrect(1^\secpar)$}{%
     \parm \gets \Setup(1^\secpar) \\
     (\ipk,\isk) \gets \IKeyGen(\parm);~(\opk,\osk) \gets \OKeyGen(\parm);~ 
     \gpk \gets (\ipk,\opk) \\
    (\cid,\dattrs,\msg) \gets \adv^{\HUGEN,\OBTISS,\RREG}(\gpk) \\
    \pcif \dattrs \nsubseteq \ATTR[\cid] :
    \pcreturn \bot \\
    \sig \gets \Sign(\gpk,\PRVUK[\OWNR[\cid]],\CRED[\cid],\dattrs,\msg) \\
    (\upk,\attrs,\oproof) \gets \Open(\gpk,\osk,\sig,\dattrs,\msg) \\
    \pcif \Verify(\gpk, \sig, \dattrs,\msg) = 0 \lor
    \Judge(\gpk,\upk,\attrs,\oproof,\sig,\dattrs,\msg) = 0: \\
     \pcind \pcreturn 1 \\
    \pcreturn 0
  }
  \caption{Correctness experiment for \GSAC schemes.}
  \label{fig:exp-gsac-corr}
\end{figure}

\subsubsection{Security Properties of \GSAC Schemes}
\label{sssec:security-gsac}

\paragraph{Anonymity.} %
In group signatures, anonymity captures that no adversary must be able to learn,
from any group signature, anything about its signer. In anonymous credentials
(with selective disclosure), it requires that no adversary should learn anything
about the holder of a credential that has been successfully shown, beyond that
he owns a credential containing the revealed attributes. In both GS and AC, it
is also typically required that
multiple signatures/presentations by the users are unlinkable. The approach to
formally state this property is in both cases frequently the same: the adversary
requests signatures from one out of two challenge users/credentials, where the
one actually producing the signatures is defined by a bit $b$. The adversary
wins if it succeeds in guessing which was the chosen user/credential better than
guessing at random. In group signatures, the game must also restrict the
adversary from opening challenge
signatures. In anonymous credentials, the game must constraint the
adversary to pick credentials that have some common subset of attributes, and
to use that common subset to request the challenge presentations. Here, we need
to take into account both. Furthermore, a key difference with group signatures
is that the game requires the adversary to output credential identifiers, rather
than user identifiers. Specifically, this means that the adversary may actually
output two credentials that belong to the same user. Therefore, in some sense,
the anonymity we get is more general than that of group signatures, as in \GSAC,
like in anonymous credentials, the adversary may request signatures from
different credentials, but belonging to the same user. The formal
specification of the anonymity game is given in \figref{fig:exp-gsac-anonb}.

\begin{figure}[htp!]
  \procedure[linenumbering]{$\ExpGSACAnonb(1^\secpar)$}{%
     \parm \gets \Setup(1^\secpar) \\
     (\ipk,\isk) \gets \IKeyGen(\parm);~(\opk,\osk) \gets \OKeyGen(\parm);~ 
     \gpk \gets (\ipk,\opk) \\
     b^* \gets
     \adv^{\HUGEN,\CUGEN,\WREG,\OBTAIN,\SIGN,\OPEN,\CHALb}(\gpk,\isk,\status) \\
     \pcreturn b^*
  }
  \caption{Anonymity experiment for \GSAC schemes.}
  \label{fig:exp-gsac-anonb}
\end{figure}

\begin{definition}{(Anonymity of \GSAC)}
  \label{def:anonymity-gsac}
  We define the advantage \AdvGSACAnon of $\adv$ against \ExpGSACAnonb as
  $\AdvGSACAnon=|\Pr\lbrack\ExpGSACAnono(1^\secpar)=1\rbrack-
  \Pr\lbrack\ExpGSACAnonz(1^\secpar)=1\rbrack|$.
  %
  A \GSAC scheme satisfies anonymity if, for any p.p.t. adversary $\adv$,
  \AdvAnon is a negligible function of $1^\secpar$.
\end{definition}

\paragraph{Traceability.} %
Traceability is one of the unforgeability-related properties in group
signatures. It captures that any signature accepted by \Verify needs to open
to one of the users that joined the group. While there is no traceability notion
in anonymous credentials, it is natural to map it to their unforgeability
property; if only because both require the issuer to be honest. Unforgeability
in anonymous credentials typically ensures that no adversary can get a verifier
to accept a credential presentation requiring a set of attributes that is not
contained in one of the credentials controlled by the adversary.
%
Our notion of traceability for \GSAC combines both requirements. It assumes an
honest issuer, as otherwise the adversary can create untraceable credentials at
will. The game then lets the adversary add honest and corrupt users, create
honest signatures, and open them. The adversary wins if, after this interaction,
is able to produce a $(\sig,\dattrs,\msg)$ tuple that is accepted by \Verify,
but either cannot be opened, it can be opened but the proof is rejected by
\Judge, or even though it is accepted by \Judge, there is no credential
that contains the set of attributes \dattrs. We formally define traceability in
the \ExpTrace experiment in \figref{fig:exp-gsac-trace}.

\begin{figure}[htp!]
  \procedure[linenumbering]{$\ExpGSACTrace(1^\secpar)$}{%
    \parm \gets \Setup(1^\secpar) \\
    (\ipk,\isk) \gets \IKeyGen(\parm);~(\opk,\osk) \gets \OKeyGen(\parm);~ 
    \gpk \gets (\ipk,\opk) \\
    (\sig,\dattrs,\msg) \gets
    \adv^{\HUGEN,\CUGEN,\RREG,\OBTISS,\ISSUE,\SIGN}(\gpk,\osk) \\
    \pcif \Verify(\gpk,\sig,\dattrs,\msg) = 0: \pcreturn 0 \\
    \pcif \Open(\gpk,\osk,\trans,\sig,\dattrs,\msg) = \bot: \pcreturn 1 \\
    \textrm{Let}~(\upk,\attrs,\oproof) \gets \Open(\gpk,\osk,\trans,\sig,
    \dattrs,\msg) \\
    \pcif \Judge(\gpk,\upk,\attrs,\oproof,\sig,\dattrs,\msg) = 0: \pcreturn 1 \\
    \pcelse \pcif \forall \uid \in \CU \cup \HU, \nexists (\uid,\cdot,\attrs)
    \in \CRED \lor \dattrs \not\subseteq \attrs: \pcreturn 1 \\
    \pcreturn 0
  }
  \caption{Traceability experiment for \GSAC schemes.}
  \label{fig:exp-gsac-trace}
\end{figure}

\begin{definition}{(Traceability of \GSAC)}
  \label{def:trace-gsac}
  We define the advantage \AdvGSACTrace of $\adv$ against \ExpGSACTrace as
  $\AdvGSACTrace=\Pr\lbrack\ExpGSACTrace(1^\secpar)=1\rbrack$.
  %
  A \GSAC scheme satisfies traceability if, for any p.p.t. adversary $\adv$,
  \AdvGSACTrace is a negligible function of $1^\secpar$.
\end{definition}

\paragraph{Non-frameability.} %
Non-frameability variants are a core unforgeability-type property in group
signatures. However, no similar property is modelled for anonymous credentials.
It is a quite strong
property, as it must be ensured even in the presence of dishonest issuer and
opener. Intuitively, it prevents the adversary from creating a signature that
frames an honest user. Depending on the inspection capabilities of the scheme,
this framing could be done in different ways; i.e., by convincing third parties
that signatures by different (possibly corrupt) users are linked, or directly
by having open proofs output the identity of a user who did not create the
signature being opened.
%
In \GSAC schemes, in order for a user to be framed, the adversary first needs to
create a $(\sig,\dattrs,\msg,\upk,\attrs,\oproof)$ tuple such that the signature,
attributes, and message are accepted by \Verify, and which, along with the user
public key, and open correctness proof, are accepted by \Judge; or, alternatively,
produce a signature revealing \dattrs, which is accepted by \Verify, and \Judge
accepts \attrs as the attribute set in the used credential owned by \upk, but
$\dattrs \not\subseteq \attrs$.
Then, the adversary wins the game if the owner of the credential is honest, but
the signature was not produced via the \SIGN oracle.

%, or if the set of attribute disclosed in the signature are not contained in the
% set of attributes revealed during \Open.

\begin{figure}[htp!]
  \procedure[linenumbering]{$\ExpGSACNonframe(1^\secpar)$}{%
    \parm \gets \Setup(1^\secpar) \\
     (\ipk,\isk) \gets \IKeyGen(\parm);~(\opk,\osk) \gets \OKeyGen(\parm);~ 
     \gpk \gets (\ipk,\opk) \\
     (\sig,\dattrs,\msg,\upk,\attrs,\oproof) \gets
     \adv^{\HUGEN,\CUGEN,\WREG,\OBTAIN,\SIGN}(\gpk,\isk,\osk) \\
     \pcif \Verify(\gpk,\sig,\dattrs,\msg) = 0: \pcreturn 0 \\
     \pcif \Judge(\gpk,\upk,\attrs,\oproof,\sig,\dattrs,\msg) = 0: \pcreturn 0 \\
     \textrm{Let}~\uid~\textrm{be st}~\PUBUK[\uid] = (\upk,\cdot) \\
     \pcif \textrm{no such \uid exists}: \pcreturn 0 \\
     \pcif \uid \in \HU \land (\sig \notin \SIG[\uid]
     \lor \dattrs \not\subseteq \attrs): \pcreturn 1 \\
     \pcreturn 0
  }
  \caption{Non-frameability experiment for \GSAC schemes.}
  \label{fig:exp-gsac-frame}
\end{figure}

\begin{definition}{(Non-frameability of \GSAC)}
  \label{def:frame-gsac}
  We define the advantage \AdvGSACNonframe of $\adv$ against \ExpGSACNonframe as
  $\AdvGSACNonframe=\Pr\lbrack\ExpGSACNonframe(1^\secpar)=1\rbrack$.
  %
  A \GSAC scheme satisfies non-frameability if, for any p.p.t. adversary $\adv$,
  \AdvNonframe is a negligible function of $1^\secpar$.
\end{definition}

%%% Local Variables:
%%% mode: latex
%%% TeX-master: "uas"
%%% End:

\subsection{\GSACGen: A Generic Construction for \GSAC}
\label{ssec:generic-gsac}

In a nutshell, our generic \GSAC construction leverages a signature scheme
on blocks of (committed) messages, where each message to be signed is an
attribute in the produced credential; with the exception of the user secret key,
which is treated as a special attribute. Moreover, the user secret key is also
the only ``attribute'' signed in committed form, to prevent the issuer from
learning it. On top of that, we make use of NIZKs to: (1) during credential
issuance, have users prove knowledge of the secret key; (2) during signing,
have users prove knowledge of the user secret key, unrevealed attributes, and
a credential including all (revealed and unrevealed) attributes; and (3), during
open, to have the opener prove correctness of the decryption of the user
``public'' key. Concerning the user public keys, they are mere commitments to
the user secret key (with $0$ as randomness -- although could be any fixed
value), that serve to univocally identify a user, without revealing its secret
key.

In more detail, in the following algorithms, we make use of three different NP
relations for \NIZK proof systems:

\begin{description}
\item[$\NIZKRel_{\Issue}$:] Produced by users requesting a credential. It is
  defined as $\NIZKRel_{\Issue} = \lbrace \usk, \Ccom :
  \Ccom = \CCommit(\usk; r) \rbrace$.
\item[$\NIZKRel_{\Sign}$:] Produced by users when signing a message. It is
  defined as $\NIZKRel_{\Sign} = \lbrace (\usk,\Ccom,\attrs,\msg,\SBCMsig),
  (\Cmsg,\Ec,\dattrs) : \Cmsg = \CCommit(\msg) \land \Ccom =
  \CCommit(\usk; 0) \land \Ec = \EEnc(\opk,\Ccom)
  \land \SBCMVerify(\ipk,\SBCMsig,\usk,\attrs) = 1
  \land \dattrs \subseteq \attrs \rbrace$.
\item[$\NIZKRel_{\Open}$:] Used by the opener when opening a signature. It
  is defined as $\NIZKRel_{\Open} = \lbrace (\osk), (\Ec,\msg) :
  \msg = \EDec(\osk,\Ec) \rbrace$.
\end{description}

With the help of those \NIZK proof systems, we build the algorithms for
\GSACGen as follows:

\paragraph{$\Setup(\secpar,\nattrs) \rightarrow \parm$.} %
Sets up the public parameters. Namely: $\Cparm \gets \CSetup(\secpar)$, $\Eparm
\gets \ESetup(\secpar)$, $\SBCMparm \gets \SBCMSetup(\secpar)$,
$\NIZKcrs_{\Issue} \gets \NIZKSetup^{\Issue}(\secpar)$, $\NIZKcrs_{\Sign} \gets
\NIZKSetup^{\Sign} (\secpar)$, $\NIZKcrs_{\Open}\gets \NIZKSetup^{\Open}(\secpar)$.
Outputs $\parm \gets (\Cparm,\SBCMparm,\Eparm,\NIZKcrs_{\Issue},\NIZKcrs_{\Sign},
\NIZKcrs_{\Open})$. Note that parts of this process can be left to individual
parties (e.g., \SBCMSetup, $\NIZKSetup^{\Issue}$ to the issuer, or \ESetup and
$\NIZKSetup^{\Open}$ to the opener) who later publish the output, but we
concentrate them here in \Setup for readability.

\paragraph{$\IKeyGen(\parm) \rightarrow (\ipk,\isk)$.} %
Generates the signing key pair for the issuer, by parsing \parm as
$(\cdot,\SBCMparm,\cdot,\cdot)$ and running $(\ipk,\isk) \gets
\SBCMKeyGen(\SBCMparm)$.

\paragraph{$\OKeyGen(\parm) \rightarrow (\opk,\osk)$.} %
Generates the encryption key pair for the opener, by parsing \parm as
$(\cdot,\SBCMparm,\Eparm,\cdot)$ and running $(\opk,\osk) \gets
\EKeyGen(\Eparm)$.

\paragraph{$\UKeyGen(\parm) \rightarrow (\upk,\usk)$.} %
Generates users' key pairs by choosing a random value within the attribute space
\AttrSpace, and committing to it. Concretely: $\usk \getr \AttrSpace$, $\upk
\gets \CCommit(\usk; 0)$.

\paragraph{$\langle \Obtain(\gpk,\usk,\attrs),\Issue(\gpk,\isk,\attrs) \rangle
  \rightarrow \langle \cred/\bot,\utrans/\bot \rangle$.} %
In a nutshell, the user requests a signature over a commitment of its user
secret key, as well as the attributes in \attrs. The user proves knowledge of
the committed value, and the issuer sends the signature (credential) in return.
This is directly an execution of the interactive signing protocol of an \SBCM
scheme, in which the user parses \gpk as $(\ipk,\opk)$ and runs $\SBCMCom(\ipk,
\usk,\attrs)$, and the issuer runs $\SBCMSign(\isk,\attrs)$; in both cases,
using $\NIZKRel_{\Issue}$ as \NIZK relation. The credential \cred output to the
user is the signature produced by the interactive signing protocol, and the
\utrans entry obtained by the issuer is the transcript of the protocol, namely
a $(\Ccom,\attrs,\cred,\pi)$ tuple.

\iffalse

\todo{This is not consistent with the $\langle \SBCMCom,\SBCMSign \rangle$
  definition...}
  
\begin{itemize}
\item \underline{User}: Commit to the user secret key by running $\Ccom \gets
  \CCommit(\usk)$. Generate proof $\NIZKproof \gets
  \NIZKProve^{\NIZKRel_{\Issue}}(\NIZKcrs,\usk,\Ccom)$. Send $(\Ccom,
  \NIZKproof)$ to the issuer.
\item \underline{Issuer}: Run $\NIZKVerify^{\NIZKRel_{\Issue}}(\NIZKcrs,
  \C,\NIZKproof)$, and return $\bot$ if it fails. Else, create the credential
  by computing $\SBCMsig \gets \SBCMSign(\isk,\C,\attrs)$. Send \SBCMsig to the
  user, and output $\utrans \gets (\C,\SBCMsig,\attrs,\NIZKproof)$.
\item \underline{User}: Check the signature by running $\SBCMVerify(\ipk,
  \SBCMsig,\attrs \cup \lbrace \usk \rbrace)$, and return $\bot$ if
  verification fails. Otherwise, return $\cred \gets \SBCMsig$.
\end{itemize}
\fi

\paragraph{$\Sign(\gpk,\usk,\cred,\dattrs,\msg) \rightarrow \sig$.} %
Commit to the message and recompute the user public key by running
$\Cmsg \gets \CCommit(\msg), \Ccom \gets \CCommit(\usk; 0)$.
Encrypt \Ccom as $\Ec \gets \EEnc(\opk,\Ccom)$ and create a NIZK proof using
$\NIZKRel_{\Sign}$ as $\NIZKproof \gets \NIZKProve^{\NIZKRel_{\Sign}}
(\NIZKcrs,(\usk,\Ccom,\attrs,\msg,\cred),
(\Cmsg,\Ec,\dattrs))$. Output $\sig \gets (\Ec,\NIZKproof)$.

\paragraph{$\Verify(\gpk,\sig,\dattrs,\msg) \rightarrow 1/0$.} %
Parse \sig as $(\Ec,\NIZKproof)$, compute $\Cmsg \gets \CCommit(\msg)$ and
return $\NIZKVerify^{\NIZKRel_{\Sign}}(\NIZKcrs,(\Cmsg,\Ec,\dattrs),
\NIZKproof)$.

\paragraph{$\Open(\gpk,\osk,\trans,\sig,\dattrs,\msg)
  \rightarrow (\upk,\oproof)/\bot$.} %
First, verify the signature with $\Verify(\gpk,\sig,\dattrs,\msg)$ and
return $\bot$ if verification fails. Else, parse \sig as $(\Ec,\NIZKproof)$
and run $\upk \gets \EDec(\osk,\Ec)$. Compute proof of correct decryption
as $\oproof_{\Open} \gets \NIZKProve^{\NIZKRel_{\Open}}(\NIZKcrs,\osk,(\Ec,
\upk))$. Return $(\upk,\oproof_{\Open})$.

\paragraph{$\Judge(\gpk,\upk,\attrs,\oproof,\sig,\dattrs,\msg)
  \rightarrow 1/0$.} %
First, verify the signature with $\Verify(\gpk,\sig,\dattrs,\msg)$ and
return $\bot$ if verification fails. If verification suceeds, parse
\sig as $(\Ec,\cdot)$ and return $\NIZKVerify^{\NIZKRel_{\Open}}(\NIZKcrs,(\Ec,
\upk))$.


%%% Local Variables:
%%% mode: latex
%%% TeX-master: "uas"
%%% End:

\subsection{Correctness and Security of \GSACGen}
\label{ssec:security-gsac}

\begin{theorem}[Correctness of \GSACGen]
  \label{thm:correctness-gsac}
  If the underlying schemes for \todo{xxx}, our generic construction \GSACGen
  satisfies correctness as defined in \defref{def:correctness-gsac}.
\end{theorem}

\begin{proof}[\thmref{thm:correctness-gsac}]
  \todo{XXX}
\end{proof}

\begin{theorem}[Anonymity of \GSACGen]
  \label{thm:anonymity-gsac}
  If the NIZK system used for $\NIZKRel_{\Sign}$ is zero-knowledge and
  simulation-extractable, our \GSACGen construction satisfies anonymity as
  defined in \defref{def:anonymity-gsac}.
\end{theorem}

\begin{proof}[\thmref{thm:anonymity-gsac}]
  %
  \qed
\end{proof}

\begin{theorem}[Traceability of \GSACGen]
  \label{thm:trace-gsac}
  If the underlying scheme for \todo{xxx}, then our \GSACGen construction
  satisfies traceability as defined in \defref{def:trace-gsac}, except
  with negligible probability.
\end{theorem}

\todo{\usk belongs to \AttrSpace! I think this can lead to malleability attacks.
  Make them disjoint?}

\begin{proof}[\thmref{thm:trace-gsac}]
  \qed
\end{proof}

\begin{theorem}[Non-frameability of \GSACGen]
  \label{thm:frame-gsac}
  If the underlying schemes for \todo{xxx}, then our \GSACGen construction
  satisfies non-frameability as defined in \defref{def:frame-gsac}, except with
  negligible probability.
\end{theorem}

\begin{proof}[\thmref{thm:frame-gsac}]
  \qed
\end{proof}

%%% Local Variables:
%%% mode: latex
%%% TeX-master: "gsac"
%%% End:

\subsection{\GSAC Variants}
\label{ssec:variants-gsac}

\paragraph{\GSAC with interactive presentations.} %
In group signatures, the signing and verification process is non-interactive.
That is, a group member produces a signature and eventually sends it to the
verifier, who can check it without further interaction. Nevertheless, in
anonymous credentials, this process is typically interactive, and consists of
at least one round-trip. This can be useful, for instance, when some notion of
freshness is necessary.

The approach to turn our non-interactive singing-verification into an
interactive protocol that ensures freshness is evident: prior to signing the
message, we require that the honest verifier sends to the user a number picked
uniformly at random from an appropriate domain, which must be concatenated to
the signed message. We sketch a proof for why does this result in a secure
\GSAC scheme with interactive signing and verification in
\appref{ssap:interactive-gsac}

\paragraph{Comparing \GSAC with group signatures and anonymous
  credentials.} %

\GSAC is a direct combination of group signatures and anonymous credentials.
%
On the one hand, by restricting to credentials without attributes (i.e., where
the only attribute is the user secret key), we get a vanilla group signature
scheme. Although, strictly, our scheme allows users to fetch more than one
credential -- which goes beyond vanilla group signatures -- this can be
trivially ``compensated'' by requiring users to prove ownership of a
conventional digital identity during the $\langle \Obtain,\Issue \rangle$
protocol\footnote{This is indeed common practice, but also frequently ommitted
  in research papers.}.
%
On the other hand, by instantiating a \GSAC scheme and simply
throwing out the opener's key pair (thus, making \Open and \Judge useless), one
gets a vanilla anonymous credential system. Or, keeping \osk, one can build an
anonymous credential system with blacklisting, which can be directly built via
the \Open function.

\todo{Can we do formal proofs here of $GSAC \implies$ GS and $GSAC \implies AC$?}

\iffalse
\todo{Modelling choices: HU and CU lists are typically updated in join protocols
  in the GS literature. Here, it is best to follow the AC literature, as a same
  user (key) may be associated to multiple join protocols. So there are HUGEN
  and CUGEN oracles.}

\todo{Other modelling choice: We could return the full set of attributes
  contained in the credential used for creating an opened signature, but that
  would require either complicating the proof, or access to the transcripts log
  at open time, which I opted to avoid for better compatibility with \UAS.}

\todo{It would be nice to prove that a \GSAC scheme to which we remove the
  \Open/\Judge functions becomes an AC scheme. And conversely, a \GSAC scheme
  where all credentials have no attributes, and where we restrict to only
  one credential per user, becomes a conventional GS scheme.}
\fi

%%% Local Variables:
%%% mode: latex
%%% TeX-master: "uas"
%%% End:


%%% Local Variables:
%%% mode: latex
%%% TeX-master: "uas"
%%% End:
