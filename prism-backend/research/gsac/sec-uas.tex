\section{From \GSAC to \UAS}
\label{sec:uas}

\GSAC is a very interesting exercise that showcases the advantages of direct
combination of group signatures and anonymous credentials. However, it also
inherits a main limitation: it is too strict in regards to the utility it
offers. The most excruciating example of this fact is that in \GSAC, via \Open,
one can only get the actual identity of the signer. Even though the inclusion
of attributes makes the concept of ``identity'' meaningful, we
may not want to fully de-anonymize the signer. Instead, we may be interested in
revealing arbitrary functions of its identity and the signed message. Also, for
the sake of illustation, we restricted to the case of selective disclosure, and
to only one credential per signature. Then, the natural question is whether we
can model an extension to \GSAC that is compatible both with constructions
that support selective disclosure, and with constructions that support arbitrary
claims on the
credential attributes; and, as it is also common in the anonymous credentials
(or verifiable credentials\footnote{\url{https://www.w3.org/TR/vc-data-model/}.
  Last access, May 5th, 2022.}) domains, to allow credential holders to
combine multiple credentials into one signature/presentation. Similarly, we can
also generalize \GSAC by allowing users to leverage previously obtained
credentials in order to obtain new ones; possibly, applying arbitrary issuance
policies. In a nutshell, our generalization of \GSAC into Universal Anonymous
Signatures (\UAS), captured in \figref{fig:proof-blocks-uas}, enables:

\begin{itemize}
\item Using previously obtained credentials to request new ones, applying a
  configurable policy defined via an issuance function \fissue, which determines
  whether the new issuance should proceed or not.
\item Modulating the type of utility one wants to get directly from signatures,
  via a signature evaluation function \feval. \feval can be programmed to reveal
  nothing, selective disclosure, arbitrary claims over the credentials'
  attributes, etc.
\item Defining arbitrary accountability information that will be extractable by
  openers through an opening function \finsp. This can range from no information
  at all, to functions on subsets of the attributes, to full de-anonymization.
\end{itemize}

\begin{figure}[ht!]
  \begin{tikzpicture}

  \pgfdeclarelayer{bg}    % declare background layer
  \pgfsetlayers{bg,main}  % set the order of the layers (main is the standard layer)

  \node (issue) at (0.00,2.50) { \bf Issue };
  \node (sign) at (0.00,1.50) { \bf Sign };  
  \node (open) at (0.00,0.50) { \bf Open };

  \node (issue-uas) at (4.00,2.50)
  { $K(\usk) \land K(\scred) \land \usk \in \scred \land \fissue(\cdot) = 1$ };
  \node[align=center] (sign-uas) at (4.00,1.50)
  { $K(\usk) \land K(\scred) \land \usk \in \scred \land E(\yinsp)~\land$ \\
    $\Yeval = \feval(\cdot) \land \yinsp = \finsp(\Yeval,\cdot)$
  };
  \node (open-uas) at (4.00,0.50) { $D(\yinsp)$ };

  % \draw[draw=black,thick] (5.65,2.29) rectangle ++(1.43,0.44);
  % \draw[draw=black,dashed,thick] (1.80,1.09) rectangle ++(1.75,0.44);
  % \draw[draw=black,dotted,thick] (3.85,1.09) rectangle ++(2.30,0.44);
    
\end{tikzpicture}

%%% Local Variables:
%%% mode: latex
%%% TeX-master: t
%%% End:

  \caption{Comparison of statements proved per operation in \GSAC and \UAS.
    $K(x)$ means proving knowledge of $x$; $E(y)$ means proving correct
    encryption of $y$; $D(z)$ means proving correct decryption of $z$. In
    \UAS, solid boxes span configurable statements that control issuance; dashed
    boxes span configurable statements that control utility; dotted boxes span
    configurable statements that control accountability. Unboxed statements are
    fixed, and are thus common to all \GSAC/\UAS schemes.}
  \label{fig:proof-blocks-uas}
\end{figure}

Furthermore, to make the concept more flexible, we no longer restrict to the
case of one issuer and one opener. Instead, \UAS supports multiple issuers (as
in generalized variants of anonymous credentials), as well as multiple openers,
each one with a predefined open function. 
A consequence of this is that the concept of ``group'', which was very present
in \GSAC, is now blurred in \UAS, and we get closer to the context of anonymous
credentials -- while still offering the accountability we inherit from \GSAC
via \Open and \Judge.
%
Although this may seem an innocent change, it brings much flexibility in
practice.

\section{Model for \UAS Schemes}
\label{sec:model-uas}

We now define our model for Universal Anonymous Signatures. At a very high
level, it is a combination of the main models for group signatures and
anonymous credentials -- we try to extract the most useful features from both.
Instead of considering just one group, as in group signatures, we directly and
explicitly include support for multiple issuers as in anonymous credentials.
Moreover, instead of letting each user own a single credential, we allow
multiple credentials per user, even within the same group. This of course
makes us put forward refined notions of anonymity, traceability and
non-frameability. Note that, property-wise, we keep the three main properties
from group signatures, rather than sticking to the two usual properties of
anonymous credentials (anonymity and unforgeability). This is a consequence
of keeping the inspection capability from group signatures, which we consider
crucial in order to ensure some sort of accountability even in very adversarial
contexts where the issuer may be corrupt.
%
But, as stated in the introduction, we go beyond just combining anonymous
credentials and group signatures, and include support for arbitrary policies
during issuance, signing, and inspection. Consequently, our model needs to
capture the flexibility that these policies provide. In short, manipulations of
the issuance policy are only considered forgeries if the adversary manages to
create a valid signature using a credential that was obtained by dishonestly
manipulating the issuance policy (we discuss this further in the traceability
definition). Signing policies are taken into account both in the anonymity
property, where we need to restrict that both credentials used to produce the
challenge signatures always satisfy the policy, and in traceability.
\commentwho{Jesus}{Think a bit more about the following claim, and update if
  needed.}
We cannot give any assurance on the signing policies when the issuers are
corrupt, and thus they are not considered in the non-frameability property.
Inspection policies are considered both in traceability and non-frameability,
where the adversary wins if it manages to somehow alter the correct output of
\Inspect. What is even suprising, is that given the possible non-injectivity of
the inspection function lets us reach an even stronger notion of anonymity.
Namely, \uline{we can allow the adversary to open challenge signatures, as
  long as the inspection function outputs the same value for both}. We give more
details on these aspects in the actual definitions of the security properties.
%
Finally, from a functional perspective, we require that both issuer and
inspector fix the issuance predicate \fissue and inspection function \finsp when
generating their public keys. Indeed, both will be part of the common group key.
This also makes our definitions easier, as supporting multiple issuance
predicate and inspection functions per group would require to consider many
special cases -- and, anyway, that would probably end up being equivalent to
just requiring that a new group needs to be created per each $(\fissue,\finsp)$
pair. On the other hand, signing policies can be defined at signing time (in
practice, most probably by the verifiers). This is in line with the usual
practice in anonymous credentials, that let users prove arbitrary claims on
their credentials, as long as they are met by the contained attributes.

\subsection{Syntax}
\label{ssec:syntax}

In the following, we assume a setting with multiple groups. For simplicity,
we assume that each has its own issuer and inspector.

\todo{Are generalizations where different groups can share issuer or inspector
  straight forward?}

\todo{The notation for multisets is not strictly correct. Still, leaving it as
  is for now.} 

\begin{description}
\item[$\parm \gets \Setup(\secpar)$.] Given a security parameter \secpar,
  returns a global system parameter variable \parm.
\item[$(\ipk,\isk) \gets \IKeyGen(\parm,\fissue)$.] Given global system
  parameters \parm, and the function \fissue to be used to check that credential
  requestors meet the conditions to be issued a credential, an issuer runs
  \IKeyGen to generate its issuing key pair. Hereafter, we assume that the
  public part \ipk is added to the group public key \gpk, as well as \fissue.
\item[$(\opk,\osk) \gets \OKeyGen(\parm,\finsp)$.] Given global system
  parameters \parm, and function \finsp, an inspector runs \OKeyGen to generate
  its inspecting key pair. The function \finsp defines the type of utility that
  will be extractable from those pairs that do not meet conditions defined
  at signing time (typically, by verifiers). Hereafter, we assume that the
  public part \opk is added to the group public key \gpk, along with \finsp.
\item[$(\upk,\usk) \gets \UKeyGen(\parm)$.] Given global system parameters
  \parm, returns a user's key pair.
\item[$\langle \cred/\bot,\utrans/\bot \rangle \gets
  \langle
  \Obtain(\usk,\attrs,\ldblbrace (gpk_i,\cred_i)\rdblbrace_{i \in \Issuers}),
  \Issue(\isk,\upk,\attrs,\ldblbrace \gpk_i) \rdblbrace_{i \in \Issuers})
  \rangle$.] %
  This interactive protocol lets a user with key pair (\upk,\usk) running the
  \Obtain process, obtain a credential \cred from an issuer in the system, on
  attribute set $\attrs$ plus, possibly, blinded attributes attested by
  previously obtained credentials $\cred_i$, from a multiset of other issuers
  (which may include the one to whom a new credential is being requested) with
  public keys $\gpk_i$. The user outputs the produced credential \cred, while
  the issuer outputs the protocol transcript \utrans for the produced
  credential. To ease notation, we assume that the group to which each
  credential belongs is available from the credential itself, and therefore
  omit the $\gpk_i$ values from the syntax, unless necessary to avoid ambiguity.
\item[$\sig \gets \Sign(\gpk,\usk,\cred,\msg,\feval)$.]
  The user with with secret key \usk and credential \cred, produces a signature
  \sig over message \msg, meeting the conditions defined by \feval.
\item[$1/0 \gets \Verify(\gpk,\sig,\msg,\feval)$.] Checks whether \sig is a
  valid signature, over message \msg, from a user in the group with public key
  \gpk, for signing function \feval.
\item[$(\y,\iproof)/\bot \gets \Inspect(\gpk,\osk,\trans,\sig,\msg,\feval)$.] %
  Executed by the inspector with private key \osk. Receives a signature \sig
  over message \msg and evaluation function \feval. If \trans contains a set of
  valid transcripts corresponding
  to the $\langle\Obtain,\Issue\rangle$ interactive protocol execution that
  issued the credential used to produce \sig, the function outputs a value $\y$
  derived from the signer's public key, credential, message, and signature, as
  well as a proof of correct inspection. We sometimes abuse notation, and write
  $\trans[\uid]$ to mean that the \Inspect function is operating only for the
  obtain transcripts related to user \uid.
\item[$1/0 \gets \Judge(\gpk,\y,\iproof,\sig,\msg,\feval)$.] %
  Checks if \iproof is a valid inspection correctness proof for the value \y,
  obtained by applying \Inspect to the the signature \sig over message \msg. 
\end{description}

The correctness and security properties are defined with the help of the
following sets of oracles, and global variables that help oracles and games
keep consistent state.

\paragraph{Issuance, evaluation, and inspection functions.} %
We emphasize that, both in our syntax definition, as well as on the following
modelling, we make use of three different and abstract functions: \fissue,
\feval and \finsp. The three functions are introduced to allow customized
governance of the resulting instantiation of an \UAS scheme. They will be
defined by different parties, but in all cases, they are run by users (maybe,
on user-private data). Also, in all cases, the user has to prove correctness of
their computation. We introduce them next, \todo{and will give concrete examples
  in \secref{sec:uac-instantiation}.}

\begin{description}
\item[$\fissue: (\upk, \attrs,\scred) \rightarrow 0/1$.] Defined by issuers,
  governs what customized conditions an issuer requires in order to issue
  credentials, when receiving a request from user with public key \upk, for
  attributes \attrs. \fissue may run checks on a (possibly empty) set of
  additional credentials \scred, which may further be used for blind issuance
  \commentwho{Jesus}{can it?}. \fissue returns $1$ to accept a request, $0$ to
  reject it.  
\item[$\feval: (\upk,\cred,\msg) \rightarrow \varrngfeval$.] Can be defined by any
  party, although we anticipate that this will typically be done by either
  signers, verifiers, or some governance organization. It defines customized
  conditions to be met by the owner of a user key \upk and credential \cred,
  in order to sign a concrete message \msg. Its outputs  $\varrngfeval$ must
  belong in a well defined set \rngfeval, with the restriction that it must
  return $0$ when the conditions for signing are not met.
\item[$\finsp: (\varrngfeval,\upk,\cred,\msg) \rightarrow \varrngfinsp$.]
  Defines what utility value, derived from the user's public key, credential,
  and message should be extractable by the group inspector. Note that \finsp
  also receives as input a value in the range of the \feval function, \rngfeval.
  This will allow inspection logic to depend on the value produced by the
  evaluation function. The output of \finsp is a value \varrngfinsp, which must
  belong in a well defined set \rngfinsp.
\end{description}

\paragraph{Helper Function \Identify.} In addition, we make use of a helper
function \Identify for some of our definitions. This function receives
a signature and a user secret key, and determines whether the latter was used
to produce the former. This is in line with previous works on DAA
\cite{bfg+11,cdl16} and group signatures without traditional open functions
\cite{dl21,fgl21,gl19}. More concretely, the \Identify function, defined as $1/0
\gets \Identify(\usk,\sig)$, conveys meaning to our definitions conditioned on
the fact that, given a signature \sig accepted by \Verify, \Identify returns $1$
for exactly one \usk.

\paragraph{Global Variables.} %
The environment manages several global variables in the games posed to the
adversary. Users are referred to with user identifiers, \uid; for credentials,
we use \cid; for groups, \gid. For credentials and groups, we use bold font to
denote sets: i.e., \scid and \sgid denote sets of credential and group
identifiers. All tables/sets are initialized as empty tables/sets, denoted
with $\emptyset$.

\begin{description}
\item[Tables for parties]:
  \begin{description}
  \item[\HU and \CU.] Keep track of honest (\HU) and corrupted (\CU) users;
    i.e., they are sets of {\uid}s.
  \item[\HI and \CI.] Keep track of honest (\HI) and corrupted (\CI) issuers.
    Since we assume only one issuer per group, they are sets of {\gid}s.
  \item[\HO and \CO.] Keep track of honest (\HO) and corrupted (\CO) inspectors.
    Since we assume only one inspector per group, they are sets of {\gid}s.
  \end{description}
\item[Tables for keys]:
  \begin{description}
  \item[\UK, \PUBUK and \PRVUK.] \UK maintains user key pairs $(\upk,\usk)$.
    To refer to the key pair of a specific user, we use $\UK[\uid]$. \PUBUK
    is a shorthand to refer to the public part, and \PRVUK refers to the
    private part -- we may also index both by \uid.
  \item[\IK, \PUBIK and \PRVIK.] Same as \UK, but for issuer key pairs. In
    addition, \IK also includes the \fissue function, which is part of the
    public key.
  \item[\OK, \PUBOK, \PRVOK.] Same as \UK, but for inspector key pairs. In
    addition, \OK also includes the \finsp function, which is part of the
    public key.
  \item[\GK.] We bundle public keys for issuers and inspectors into a common
    public key for each group. We use the table \GK for that purpose which,
    consequently, can also be indexed by \gid.
  \end{description}
\item[Tables for credentials-related data]:
  \begin{description}
  \item[\CRED.] Stores information related to credentials obtained by honest
    users. Thus, it is indexable by \cid. More specifically, it stores tuples of
    the form $(\uid,\cred,\gid,\attrs,\scid)$, where \uid is the identity of the
    owner of the credential, \cred (when available) is the credential itself,
    \gid is the identifier of the group to which the credential belongs, \attrs
    are the attributes included in \cred, and \scid are the identifiers of any
    additional credential that \uid used to request \cred.
  \item[\OWNR.] For notational convenience, when we write $\OWNR[\cid]$ we mean
    ``\uid such that $\CRED[\cid] = (\uid, \cdot, \cdot, \cdot, \cdot)$''.
  \item[\ATTR.] For notational convenience, when we write $\ATTR[\cid]$ we mean
    ``\attrs such that $\CRED[\cid] = (\cdot, \cdot, \cdot, \attrs, \cdot)$''.
  \item[\GRP.] For notational convenience, when we write $\GRP[\cid]$ we mean
    ``\gid such that $\CRED[\cid] = (\cdot, \gid, \cdot, \cdot, \cdot)$''.
  \end{description}
\item[Tables for signatures]:
  \begin{description}
  \item[\SIG.] Maintains signatures generated via the \SIGN oracle, on behalf
    of honest users.
  \item[\CSIG.] Maintains challenge signatures, generated via the \CHALb oracle,
    by one of the challenge users in the anonymity game.
  \end{description}
\end{description}

\paragraph{Oracles.} %
Oracles are the interface of the adversary with the corresponding games. In
other words: through these oracles, the game environment exposes to the adversary
functionality that could otherwise be executed only by honest parties with
private knowledge -- knowledge that would make the adversary capable of
trivially breaking the security properties formalized in the experiments.
In the game-based definitions of our \UAS model, we leverage the following
oracles, which are formally defined in \figref{fig:oracles1} and
\figref{fig:oracles2}. 

\begin{description}
\item[\IGEN.] Adds a new issuer to the game, generating its keypair and setting
  the associated issuance function.
\item[\OGEN.] Adds a new inspector to the game, generating its key pair and
  setting the associated evaluation and inspection functions.
\item[\ICORR.] Corrupts an existing (and honest) issuer, by giving its secret
  key to the adversary.
\item[\OCORR.] Corrupts an existing (and honest) inspector, by giving its secret
  key to the adversary.  
\item[\HUGEN.] Adds a new honest user to the game, by honestly generating
  the user's key pair.
\item[\CUGEN.] Adds a new corrupt user to the game or, if the specified
  user already exists and is honest, corrupts it.
\item[\OBTISS.] Lets the adversary add a new honestly generated credential to
  the game, on behalf of an honest user.
\item[\OBTAIN.] Enables the adversary to play the role of a dishonest issuer
  in games that support it, by letting it interact with honest users who want to
  receive credentials.
\item[\ISSUE.] Allows the adversary to play the role of dishonest users,
  requesting an honest issuer to produce credentials for them.
\item[\SIGN.] Lets the adversary get signatures from credentials belonging
  to honest users.
\item[\OPEN.] Given an honestly produced signature, lets the adversary learn
  which credential was used to produce it.
\item[\CHALb.] Upon receiving two challenge credentials, a common intersecting
  set of attributes, and a message, returns a signature produced by one of these
  two credentials, defined by the bit $b$, which is established in the anonymity
  game.
\end{description}

{%\setlength\intextsep{\sep}
  \begin{figure*}[htp!]
    \centering
    \scalebox{0.9}{

      \begin{minipage}[t]{0.55\textwidth}

        \procedure{$\IGEN(\gid,\fissue)$}{%
          \pcif \gid \in \HI \lor \gid \in \CI: \pcreturn \bot \\
          (\ipk,\isk) \gets \IKeyGen(\parm) \\
          \IK[\gid] \gets ((\ipk,\fissue),\isk) \\
          \HI \gets \HI \cup \lbrace \gid \rbrace \\
          \GK[\gid] \gets ((\ipk,\fissue),\cdot) \\
          \pcreturn \ipk \\
        }

        \procedure{$\ICORR(\gid)$}{%
          \pcif \gid \in \CI \lor \gid \notin \HI: \pcreturn \bot \\
          \HI \gets \HI \setminus \lbrace \gid \rbrace \\
          \CI \gets \CI \cup \lbrace \gid \rbrace \\
          \pcreturn \isk \\
        }        

        \procedure{$\HUGEN(\uid)$}{%
          \pcif \uid \in \HU \lor \uid \in \CU: \pcreturn \bot \\
          (\upk,\usk) \gets \UKeyGen(\parm) \\
          \UK[\uid] \gets (\upk,\usk);
          \HU \gets \HU \cup \lbrace  \uid \rbrace \\
          \pcreturn \top
        }        
        
      \end{minipage}
    }
    \scalebox{0.9}{
      
      \begin{minipage}[t]{.5\textwidth}

        \procedure{$\OGEN(\gid,\feval,\finsp)$}{%
          \pcif \gid \in \HO \lor \gid \in \CO: \pcreturn \bot \\
          (\opk,\osk) \gets \OKeyGen(\parm) \\
          \OK[\gid] \gets ((\opk,\finsp),\osk) \\
          \HO \gets \HO \cup \lbrace \gid \rbrace \\
          \GK[\gid] \gets (\cdot,(\opk,\finsp)) \\
          \pcreturn \opk \\
        }

        \procedure{$\OCORR(\gid)$}{%
          \pcif \gid \in \CO \lor \gid \notin \HO: \pcreturn \bot \\
          \HO \gets \HO \setminus \lbrace \gid \rbrace \\
          \CO \gets \CO \cup \lbrace \gid \rbrace \\
          \pcreturn \osk \\
        }        
        
        \procedure{$\CUGEN(\uid,\upk)$}{%          
          \pcif \uid \in \CU: \pcreturn \bot \\
          \CU \gets \CU \cup \lbrace \uid \rbrace \\          
          \pcif \uid \in \HU: \\
          \pcind \HU \gets \HU \setminus \lbrace \uid \rbrace; \\
          \pcind \pcreturn (\UK[\uid],\CRED[\uid]) \\
          \pcelse: \UK[\uid] = (\upk,\bot) \\          
          \pcreturn \top
        }
        
      \end{minipage}
      
    }

    \caption{Detailed oracles available in our model (1/2). Oracles for
      generating key material for users, issuers, and inspectors.}
    \label{fig:oracles1}
  \end{figure*}
}

{%\setlength\intextsep{\sep}
  \begin{figure*}[htp!]
    \centering
    \scalebox{0.9}{

      \begin{minipage}[t]{0.55\textwidth}

        \procedure{$\ISSUE(\cid,\uid,\gid,\attrs,\ldblbrace\scid\rdblbrace)$}{%
          \pcif \uid \notin \CU: \pcreturn \bot \\          
          \pcif \gid \notin \HI: \pcreturn \bot \\
          \pcif \exists \gid' \in \sgid~\st~\gid' \notin \HI: \pcreturn \bot \\
          \pcif \CRED[\cid] \neq \bot: \pcreturn \bot \\
          \langle \cdot, \utrans \rangle \gets
          \langle \adv, \\
          \hspace*{45pt}
          \Issue(\PRVIK[\gid],\PUBUK[\uid], \attrs,
          \ldblbrace \CRED[\scid] \rdblbrace) \rangle \\
          \trans[\cid] \gets \utrans \\
          \CRED[\cid] \gets (\uid, \cdot, \gid, \attrs, \scid) \\
          \pcreturn \top \\          
        }                

        \procedure{$\OBTAIN(\cid,\uid,\gid,\attrs,\ldblbrace\scid\rdblbrace)$}{%
          \pcif \uid \notin \HU: \pcreturn \bot \\
          \pcif \exists \cid' \in \scid~\st~\CRED[\cid'] \neq \bot: \pcreturn \bot \\
          \langle \cred, \cdot \rangle \gets
          \langle \Obtain(\PRVUK[\uid],\attrs, 
          \ldblbrace \CRED[\scid] \rdblbrace), \\
          \hspace*{50pt} \adv \rangle \\
          \CRED[\cid] \gets (\uid, \cdot, \gid, \attrs, \scid) \\          
          \pcreturn \top \\
        }

        \procedure{$\INSPECT(\gid,\sig,\msg)$}{%
          \textrm{Let}~\uid~\textrm{be s.t.}~(\cid,\sig,\msg,\feval)
          \in \SIG[\uid] \\
          (\y,\iproof) \gets
          \Inspect(\GK[\gid],\PRVOK[\gid],\trans,\sig,\msg,\feval) \\
          \pcif \CSIG[\sig] \neq \bot: \\
          \pcind \textrm{Parse $\CSIG[\sig]$ as $(\sig_{1-b},\msg,\feval)$} \\
          \pcind (\y_{1-b},\iproof_{1-b}) \gets
          \Inspect(\GK[\gid],\PRVOK[\gid],\\
          \hspace*{107pt} \trans,\sig_{1-b},\msg,\feval) \\
          \pcind \pcif \y_{1-b} \neq \y: \pcreturn \bot \\
          \pcreturn (\y,\iproof)
        }
        
      \end{minipage}
    }
    \scalebox{0.9}{
      
      \begin{minipage}[t]{.5\textwidth}

        \procedure{$\OBTISS(\cid,\uid,\gid,\attrs,\ldblbrace\scid\rdblbrace)$}{%
          \pcif \uid \notin \HU: \pcreturn \bot \\
          \pcif \gid \notin \HI: \pcreturn \bot \\
          \pcif \exists \gid' \in \sgid~\st~\gid' \notin \HI: \pcreturn \bot \\
          \pcif \exists \cid' \in \scid~\st~\CRED[\cid'] \neq \bot:
          \pcreturn \bot \\
          \langle \cred, \utrans \rangle \gets
          \langle \Obtain(\PRVUK[\uid],\attrs,
          \ldblbrace \CRED[\scid] \rdblbrace), \\
          \hspace*{60pt} \Issue(\PRVIK[\gid],\PUBUK[\uid],\attrs,
          \ldblbrace \CRED[\scid] \rdblbrace ) \rangle \\
          \trans[\cid] \gets \utrans \\
          \CRED[\cid] \gets (\uid, \cred, \gid, \attrs, \scid) \\
          \pcreturn \top \\
        }

        \procedure{$\SIGN(\uid,\cid,\msg,\feval)$}{%
          \pcif \uid \notin \HU \lor \CRED[\cid] = \bot: \pcreturn \bot \\
          \sig \gets \Sign(\GK[\GRP[\cid]],\PRVUK[\uid],\msg,\CRED[\cid],\feval) \\
          \SIG[\uid] \gets \SIG[\uid] \cup
          \lbrace (\cid,\sig,\msg,\feval) \rbrace \\
          \pcreturn \sig \\
        }                

        \procedure{$\CHALb(\cuid_{0,1},\ccid_{0,1},\msg,\feval)$}{%
          \pcif \cuid_0 \notin \HU \lor \cuid_1 \notin \HU: \pcreturn \bot \\
          \pcif \gid = \GRP[\ccid_0] \neq \GRP[\ccid_1]: \pcreturn \bot \\
          \pcif \gid \in \CO: \pcreturn \bot \\
          \pcif \feval(\PUBUK[\uid_0],\CRED[\scid_0],\msg) \neq \\
          \pcind \feval(\PUBUK[\uid_1],\CRED[\scid_1],\msg):
          \pcreturn \bot \\
          \csig_b \gets \Sign(\GK[\gid],\PRVUK[\cuid_b],\CRED[\ccid_b],
          \msg,\feval) \\
          \csig_{1-b} \gets \Sign(\GK[\gid],\PRVUK[\cuid_{1-b}],\CRED[\ccid_{1-b}],
          \msg,\feval) \\          
          \CSIG[\csig_b] \gets 
          \lbrace (\csig_{1-b},\msg,\feval) \rbrace \\
          \pcreturn \csig_b
        }
        
      \end{minipage}
      
    }

    \caption{Detailed oracles available in our model (2/2). Oracles for
      obtaining credentials, signatures, and processing them.}
    \label{fig:oracles2}
  \end{figure*}
}

\paragraph{Correctness.} %
Correctness of \UAS schemes is formalized through the experiment in
\figref{fig:exp-uas-corr}. It states that a signature over any arbitrary message
and valid function \feval \todo{Should we explicitly check \feval in the game?
  If so, how?},
produced honestly by an honest user \uid with credential \cid, is accepted by
\Verify. Moreover, upon issuance of the credential \cid, it must have met the
conditions set by \fissue. Similarly, the value returnd by \Inspect must be
accepted by \Judge, and must match the output of applying \finsp on \feval, the
credential, user key, and message.

\begin{figure}[htp!]
  \procedure{$\ExpCorrect(1^\secpar)$}{%
    \parm \gets \Setup(1^\secpar) \\
    (\uid,\cid,\msg,\feval)
    \gets \adv^{\IGEN,\OGEN,\HUGEN,\OBTISS}(\parm) \\
    \sig \gets \Sign(\GK[\cid],\PRVUK[\uid],\cid,\msg,\feval) \\
    \pcif \Verify(\GK[\cid],\sig,\msg,\feval) = 0: \pcreturn 1~
    \pccomment{\sig fails to verify} \\
    \pcif \CorrectIssue(\uid,\cid) = 0: \pcreturn 1~
    \pccomment{\cid should not have been issued} \\
    % \pcif \CorrectEval(\uid,\cid,\msg,\feval)
    % = 0: \pcreturn 1~
    % \pccomment{\msg should not have been signed} \\
    \pcif \CorrectEvalInspect(\uid,\cid,\msg,\sig,\feval) = 0: \pcreturn 1~
    \pccomment{The combination of evaluation and inspection is wrong} \\
    \pcreturn 0 \\
  }

  \procedure{$\CorrectIssue(\uid,\cid)$}{
    \gid \gets \GRP[\cid] \\
    \textrm{Parse}~\ATTR[\cid]~\textrm{as}~(\attrs^\cid,\scid^\cid);~
    \textrm{Parse}~\GK[\gid]~\textrm{as}~
    ((\cdot,\fissue^\gid),\cdot) \\        
    \pcif \fissue^\gid(\PUBUK[\uid],\attrs^\cid,\CRED[\scid]^\cid) = 0: 
    \pcreturn 0 \\
    \pcreturn 1 \\   
  }

  % \procedure{$\CorrectEval(\uid,\cid,\msg,\feval)$}{
  %   \pcif \feval(\PUBUK[\uid],\CRED[\cid],\msg) = 0: \pcreturn 0 \\
  %   \pcreturn 1 \\
  % }

  \procedure{$\CorrectEvalInspect(\uid,\cid,\msg,\sig,\feval)$}{
    \textrm{Let}~\gid \gets \GRP[\cid];~
    \textrm{Parse}~\GK[\gid]~\textrm{as}~(\cdot,(\cdot,\finsp^\gid)) \\
    \pccomment{\sig cannot be inspected} \\
    \pcif \Inspect(\GK[\gid],\PRVOK[\gid],\trans,\sig,\msg,\feval) = \bot: \pcreturn 0 \\
    \pccomment{The produced $(\y,\iproof)$ pair is rejected by \Judge
      % \todo{This must be conditioned on \feval = 0!}
    } \\
    \pcif (\y,\iproof) \gets \Inspect(\GK[\gid],\PRVOK[\gid],\trans,\sig,\msg,\feval)
    \land
    \Judge(\GK[\gid],\y,\iproof,\sig,\msg) = 0: \\
    \pcind \pcreturn 0 \\
    \pccomment{\y is the wrong value} \\
    \pcif \y \neq \finsp^\gid(\feval(\PUBUK[\uid],\CRED[\cid],\msg),
    \PUBUK[\uid],\CRED[\cid],\msg)):
    \pcreturn 0 \\
    \pcreturn 1
  }
  
  \caption{Correctness experiment for \UAS schemes.}
  \label{fig:exp-uas-corr}
\end{figure}

\subsection{Security Properties}
\label{ssec:security}

\paragraph{Anonymity.} %
In group signatures, anonymity captures that no adversary must be able to learn,
from any group signature, the identity (e.g., member index) of its signer. In 
anonymous credentials, it requires that no adversary should learn anything about
the holder of a successfully shown credential, beyond that he owns a credential
attesting for the claimed predicate of the attributes it contains. In both GS
and AC, it is also typically required that
multiple signatures/presentations by the users are unlinkable. The approach to
formally state this property is in both cases frequently the same: the adversary
picks two (honest) users (or credentials in the AC case), the game randomly
chooses one of them, and lets the adversary request challenge
signatures/presentations from it. The adversary wins if it succeeds in guessing
which was the chosen user/credential better than guessing at random. In group
signatures, the game must also restrict the adversary from opening challenge
signatures. In anonymous credentials, the game must further constraint the
adversary to output credentials that both meet the same (challenge) predicate
of their attributes.

In our notion of anonymity for \UAS, we need to merge the previous constraints.
Furthermore, a key difference with group signatures is that the game requires
the adversary to output credential identifiers, rather than user identifiers.
Specifically, this means that the adversary may actually output two credentials
that belong to the same user. Therefore, in some sense, the anonymity we get is
more general than that of group signatures. Moreover, in order to prevent
trivial wins by the adversary, we have to restrict that the challenge signatures
belong to a group that has been ``programmed'' with the same evaluation and
inspection functions, which is again a generalization over group signatures,
as discussed in the sequel. Other than this, the overall approach is the same:
the adversary picks two honestly obtained credentials, and one of them (set
by the $b$ value defining the $anon-b$ game) is used to program the \CHALb
oracle. The formal specification of the anonymity game is
given in \figref{fig:exp-uas-anonb}, where $\Oanonc \gets (\lbrace\HU,\CU\rbrace
\GEN,\lbrace\II,\OO\rbrace\GEN,\lbrace\II,\OO\rbrace\CORR,\OBTAIN,\SIGN,
\INSPECT)$ and $\Oanong \gets (\lbrace\HU,\CU\rbrace\GEN,\lbrace\II,\OO\rbrace
\GEN,\lbrace\II,\OO\rbrace\CORR,\OBTAIN,\SIGN,\INSPECT,\CHALb)$

\begin{figure}[htp!]
  \procedure{$\ExpAnonb(1^\secpar)$}{%
     \parm \gets \Setup(1^\secpar) \\
     (\ccid_0,\ccid_1,\status) \gets \adv^{\Oanonc}(\choose,\parm) \\
     b^* \gets \adv^{\Oanong} (\guess,\status) \\
     \pcreturn b^*
  }
  \caption{Anonymity experiment for \UAS schemes.}
  \label{fig:exp-uas-anonb}
\end{figure}

\begin{definition}{(Anonymity of \UAS)}
  We define the advantage \AdvAnon of $\adv$ against \ExpAnonb as
  $\AdvAnon=|\Pr\lbrack\ExpAnono(1^\secpar)=1\rbrack-
  \Pr\lbrack\ExpAnonz(1^\secpar)=1\rbrack|$.
  %
  An \UAS scheme satisfies anonymity if, for any p.p.t. adversary $\adv$,
  \AdvAnon is a negligible function of $1^\secpar$.
\end{definition}

\paragraph{Discussion on the generality of anonymity in \UAS schemes.} %
This notion of anonymity is more general than that of group signatures in the
sense that calls to the \INSPECT oracle reveal an arbitrary function of the
identity of the user -- which can certainly be the member index itself, as it
is frequent in group signatures, or any other function computable from the
user public key, its credentials, and produced signature. Moreover, note that
our notion is even stronger than the conventional ``CCA-like'' anonymity notion
that gives the adversary access to the open oracle, only restricting it when
trying to open challenge signature. Precisely the introduction of generic
inspection functions allows us to \uline{\emph{let the adversary open
    challenge signatures $\csig_b$}} as long as the counterpart $\csig_{1-b}$
makes \Inspect produce the same \y value as output. This is certainly not
possible when using conventional opening, since it directly outputs the identity
of the signer -- which cannot be the same for different challenge users.

\paragraph{Traceability.} %
% Traceability is one of the unforgeability-related properties in group
% signatures. It captures that any signature accepted by \Verify needs to open
% to one of the users that joined the group. While there is no traceability notion
% in anonymous credentials, it is natural to map it to their unforgeability
% property; if only because both require the issuer to be honest. Unforgeability
% in anonymous credentials typically ensures that no adversary can get a verifier
% to accept a credential presentation requiring a set of attributes that is not
% contained in one of the credentials controlled by the adversary.

% Our notion of traceability for \UAS is inspired by both. First, we restrict to
% signatures that are accepted by \Verify. For any such signature, \Inspect has to
% create a valid output (i.e., must not abort). The output of \Inspect has to be
% accepted by \Judge too. Moreover, there must exist one user (either honest or
% corrupt) who used his public key to obtain one or more credentials that allowed
% him to produce a signature which, once inspected, returns \y.
In group signatures, traceability ensures that, in the presence of an honest
issuer, every signature accepted by the verify algorithm must have been created
by a user who joined the group.  With conventional opening, this is essentially
checked by requesting the adversary to produce a signature, opening it, and
checking whether or not it comes from a group member; i.e., this ensures that,
as long as there are verifiable inspections, inspection cannot be forged in the
presence of an honest issuer.
%
The somehow equivalent property in anonymous credentials is unforgeability. It
requires that, again in the presence of an honest issuer, no adversary can
succeed in a credential presentation for attributes (or predicates of them) that
are not contained in a legitimately issued credential it controls. In other
words, that the evaluation of the credential attributes cannot be forged.
%
Both notions have an equivalent in \UAS. That is, we need to ensure that, in the
presence of an honest issuer, no signature accepted by \Verify can result in a
wrong inspection, nor it can have been obtained by a wrong evaluation of some
credential. Note, however, that \UAS offer the adversary yet another way to
break traceability/unforgeability. Namely, an adversary can attempt to
illegitimately obtain a credential that later enables it to produce untraceable
signatures; i.e., signatures that evaluate correctly, or for which \Inspect
returns the appropiate value, yet the adversary should not have been able to
obtain a credential that allowed it to produce such signatures. Our notion of
traceability/unforgeability thus challenges the adversary to produce a signature
that breaks any of those conditions. The formal
definition of the traceability experiment is given in \figref{fig:exp-uas-trace},
where $\Otrace \gets \lbrace\HU,\CU\rbrace\GEN,\IGEN,\OGEN,\OCORR,\OBTISS,
\ISSUE,\SIGN,\INSPECT$.

\begin{figure}[htp!]
    \procedure{$\ExpTrace(1^\secpar)$}{%
      \parm \gets \Setup(1^\secpar) \\
      (\gid,\sig,\msg,\feval) \gets \adv^{\Otrace}(\parm) \\
      \pcif \Verify(\GK[\gid],\sig,\msg,\feval) = 0: \pcreturn 0 \\
      \pcif \EvalInspectForge(\gid,\sig,\msg,\feval) = 1 \\%\land
%      \InspectForge(\gid,\sig,\msg,\feval) = 1: \\
      \pcind \pcreturn 1 \\
      \pcreturn 0 \\
    }
        
    % \procedure{$\EvalForge(\gid,\sig,\msg,\feval)$}{%
    %   \pcif \exists \uid~\st~\Identify(\uid,\sig) = 1~\land \\
    %   \pcind \pccomment{\uid does not own a \cred that satisfies \feval...} \\
    %   \pcind (\nexists \cred \in \CRED[\uid]~\st~
    %   \feval(\PUBUK[\uid],\cred,\msg) \neq 0 ~\lor \\
    %   \hspace*{13pt} \pccomment{... or \uid does, but \cred was itself obtained
    %     fraudulently} \\
    %   \hspace*{13pt} (\exists \cred \in \CRED[\uid]~\st~
    %   \feval(\PUBUK[\uid],\cred,\msg) \neq 0~\land \\
    %   \pcind \pcind \fissue(\PUBUK[\uid],\attrs^\cid,\scred^\cid) = 0, \\
    %   \pcind \pcind \textrm{where $\CRED[\cid]=\cred$ and
    %     $(\attrs^\cid,\scred^\cid) \gets \ATTR[\cid]$})): \\
    %   \pcind \pcreturn 1 \\
    %   \pcelse \pcreturn 0 \\
    % }
    
    \procedure{$\EvalInspectForge(\gid,\sig,\msg,\feval)$}{%
      \pccomment{All \Obtain transcripts fail to inspect \sig} \\
      \pcif \Inspect(\PRVOK[\gid],\trans,\sig,\msg) \neq \bot~\lor \\
      \pcind \pccomment{A valid transcript exists, but ...} \\
      \pcind \exists \uid~\st~\Inspect(\PRVOK[\gid],\trans[\uid],\sig,\msg) =
      (\y,\iproof)~\land (\\
      \pcind \pcind \pccomment{... the proof is rejected by \Judge ...} \\
      \pcind \pcind \Judge(\y,\iproof,\sig,\msg,\GK[\sgid]) = 0~\lor \\
      \pcind \pcind \pccomment{... or \uid does not own a \cred for which
        \finsp outputs \y} \\
      \pcind \pcind \nexists \cred \in \CRED[\uid]~\st~
      \y = \finsp(\feval(\PUBUK[\uid],\cred,\msg),
      \PUBUK[\uid],\cred,\msg)~\lor \\
      \pcind \pcind \pccomment{... or \uid does, but \cred was itself obtained
        fraudulently} \\
      \pcind \pcind (\exists \cred \in \CRED[\uid]~\st~
      y = \finsp(\feval(\PUBUK[\uid],\cred,\msg),
      \PUBUK[\uid],\cred,\msg)~\land \\
      \pcind \pcind \pcind \fissue(\PUBUK[\uid],\attrs^\cid,\scred^\cid) = 0, \\
      \pcind \pcind \pcind \textrm{where $\CRED[\cid]=\cred$ and
        $(\attrs^\cid,\scred^\cid) \gets \ATTR[\cid]$})): \\      
      \pcind \pcreturn 1 \\      
      \pcreturn 0
    }
  \caption{Traceability experiment for \UAS schemes.}
  \label{fig:exp-uas-trace}
\end{figure}

\begin{definition}{(Traceability of \UAS)}
  We define the advantage \AdvTrace of $\adv$ against \ExpTrace as
  $\AdvTrace=\Pr\lbrack\ExpTrace(1^\secpar)=1\rbrack$.
  %
  A \UAS scheme satisfies traceability if, for any p.p.t. adversary $\adv$,
  \AdvTrace is a negligible function of $1^\secpar$.
\end{definition}

\paragraph{Discussion on the generality of traceability in \UAS schemes.} %
The notion of traceability we present for \UAS is strictly more general than
the corresponding one for group signatures. This is again a direct
consequence of the fact that \Inspect, which can return an arbitrary function
of the signer's credential (and signed message), is a generalization of the
conventional \Open -- if we make \Inspect return just the identity of the
signer, then we get something similar to the usual notion of traceability.
Although, even in that case, we need to take into account attributes, and the
fact that the same user may obtain multiple credentials (that is why, even when
having \Inspect return the identity of the signer, our notion is not exactly
the same). And, in this sense, the unforgeability notion that \UAS have is
equivalent to that of anonymous credentials. It would seem, though, that we do
not need the traceability part of group signatures; after all, it is the
protection against wrong claims on attributes what enables meaningful and
flexible authentication. However, the type of protection against misuses of
the inspection functionality that we can get with an honest issuer (as in
traceability) is much higher than without an honest user (as in
non-frameability). Specifically, with an honest issuer we can ensure that
the adversary cannot even alter the value returned by \Inspect on signatures
by corrupt users, nor the output of the signing predicate \feval. Whereas, with
a corrupt issuer, all we can ensure is that
the adversary cannot forge a signature from an honest user for which \Inspect
returns the same value as a signature by that honest user would produce; and,
certainly, a corrupt issuer can arbitrarily issue credentials meeting any
desired predicate \feval .Traceability is, therefore, a core property to ensure
accountability.

Note also that we have implicitly assumed an unforgeability relaxation in the
sense that we do not consider a forgery the fact that an adversary can obtain
credentials that do not meet the \fissue predicate \emph{as long as they are
  not used to produced untraceable signatures}. Indeed, this is in line with
previous work \cite[Section 3.3.3]{ckl+15}, and it does make sense as, in the
real world, being able to fraudulently obtain credentials that cannot be later
used to authenticate, does not pose a risk. However, an adversary breaking
the issue policy defined by \fissue to get a credential and later using it to
create a valid signature (i.e., meeting \feval or \finsp) is considered a
forgery by our definition.

\paragraph{Non-frameability.} %
Non-frameability variants are a core unforgeability-type property in group
signatures. However, no
similar property is modeled for anonymous credentials (\todo{see the discussion
  in \secref{sec:introduction} for further detail}). It is a quite strong
property, as it must be ensured even in the presence of dishonest issuer and
opener. Intuitively, it prevents the adversary from creating a signature that
frames an honest user. Depending on the inspection capabilities of the scheme,
this framing could be done in different ways; i.e., by convincing third parties
that signatures by different (possibly corrupt) users are linked, or directly
by having inspection proofs output the identity of a user who did not create the
signature being inspected.

The notion of non-frameability in \UAS schemes is unavoidably more subtle than
in group signatures, though. To see this, we note that, by allowing any
arbitrary inspection function \finsp to be used, it can be perfectly valid to
have a signature produced by a corrupted user output the same \y value upon
inspection than the one that \finsp outputs when inspecting a signature by an
honest user. As a concrete example, imagine an inspection function that returns
the banking account number (or cryptocurrency address) to be used to request
payment of a fine. In this example, a corrupted user and an honest user could
both have delegated this to some agency\footnote{\todo{Maybe, come up with a
    better example. How about spam filters that apply some sort of classification
    engine? \finsp could be this engine, and emails by many different users could
    be classified just as ``Spam'' or ``Not spam''}}, so it makes perfect sense
that the inspection function returns the same value for both users.
%
Thus, in \UAS schemes, we must be more subtle, by necessity. Note that we cannot
just require that no adversary can produce a signature for which inspection
returns the same value as it would for a signature honestly generated by an
uncorrupted user. This is again the case since inspection may be non-injective,
which means that inspection of a signature by some corrupted user may
legitimately produce the same value as that of an honest user. Thus, we need
to further refine the definition by requiring that (i) no adversary can produce
a signature that is recognized (via \Identify) as originating from an honest
user \uid, (ii) who owns a credential that makes \finsp return a valid \y value
(i.e., generated by \Inspect along with a proof, and accepted by \Judge), (iii)
without having queried the \SIGN oracle before. This non-frameability property
for \UAS schemes is formally defined in \figref{fig:exp-uas-frame}, where
$\Oframe \gets \lbrace\HU,\CU\rbrace\GEN,
\lbrace\II,\OO\rbrace\GEN,\lbrace\II,\OO\rbrace\CORR,\OBTAIN,\SIGN$

\begin{figure}[htp!]
  \procedure{$\ExpNonframe(1^\secpar)$}{%
    \parm \gets \Setup(1^\secpar) \\
     (\gid,\sig,\msg,\feval,\y,\iproof) \gets \adv^{\Oframe}(\parm) \\
     \pcreturn 1~\pcif \Verify(\GK[\gid],\sig,\msg,\feval) = 1 \land
     \Judge(\GK[\gid],\y,\iproof,\sig,\msg) = 1~\land \\
     \pcind \pccomment{\adv~did not query \SIGN to get \sig, yet \sig originates
       from an honest \uid, owning a credential that makes \finsp
       evaluate to \y} \\
     \pcind \exists \uid \in \HU~\st~\Identify(\PRVUK[\uid],\sig) = 1 \land
     (\cdot,\sig,\msg,\cdot) \notin \SIG[\uid]~\land\\
     \pcind \exists \cred \in \CRED[\uid]~\st~
     \finsp(\feval(\PUBUK[\uid],\cred,\msg),\PUBUK[\uid],\cred,\msg) = \y \\
%     \pccomment{\todo{I don't think we can give any assurance about \feval here,
%         given that issuers are corrupt. But still, think about it.}} \\
     \pcreturn 0
  }
  \caption{Non-frameability experiment for \UAS schemes.}
  \label{fig:exp-uas-frame}
\end{figure}

\begin{definition}{(Non-frameability of \GSAC)}
  We define the advantage \AdvNonframe of $\adv$ against \ExpNonframe as
  $\AdvNonframe=\Pr\lbrack\ExpNonframe(1^\secpar)=1\rbrack$.
  %
  A \GSAC scheme satisfies non-frameability if, for any p.p.t. adversary $\adv$,
  \AdvNonframe is a negligible function of $1^\secpar$.
\end{definition}

\paragraph{Discussion on the generality of non-frameability in \UAS schemes.} %
Anonymous credentials do not have non-frameability property and, thus, it is
hard to make a comparison. However, we can draw some connections with AC schemes
that support revocation, as revocation is somehow equivalent to linking, which
is a type of inspection available in group signatures. In this sense, note that
basic revocation (without straight deanonymization) can be trivially achieved
through our generic \Inspect function. For instance, one could set \finsp to
be a pseudorandom number seeded with the user's public key (or credential). In
this sense, \Inspect could be essentially seen as a Verifiable Random Function.
If we compare with group signatures, our notion is again more general than the
conventional one. Indeed, in typical group signatures, \Open returns the
identity of the signer, and thus non-frameability does not allow the same value
to be returned when opening signatures by different users. Otherwise, a
corrupted user controlled by the adversary could be able to create a signature
opening to some honest user (hence, framing him). But this is again possible
with our definition which, in case of making \Inspect equal \Open, becomes
equivalent to conventional non-frameability. However, it also allows more
generic situations in which signatures by different users may return the same
value. 

\subsection{Functionality ``Transformations'' and Special Cases}
\label{ssec:transformations}

\commentwho{Jesus}{This section is aimed at showing the ``universality'' of
  our model. If I'm not mistaken, ideally, it should support the following
  variations (except the transformation to interactive presentations,
  probably).}

\subsubsection{Transformation to Interactive Presentations}
\label{sssec:interactivetransform}

\todo{Is it possible to give it in some generic sense?}

\subsubsection{Combining Multiple Credentials in One Presentation}

\todo{I think this might be possible if we can use \fissue so that any user can
  create arbitrary groups, where he is the issuer, an can thus run
  \Obtain,\Issue protocols to combine credentials from other groups. This is
  probably inefficient (or, at least, more than doing so wihtout having to
  create ad hoc groups), but at least allows the enhanced functionality...}

\subsubsection{Restricting to Conventional GS and AC Schemes}

\todo{It would be nice to prove that a \GSAC scheme to which we remove the
  \Open/\Judge functions becomes an AC scheme. And conversely, a \GSAC scheme
  where all credentials have no attributes, and where we restrict to only
  one credential per user, becomes a conventional GS scheme.}


\subsubsection{Group Signatures with Message Dependent Opening}

\todo{Describe what type of \finsp would be needed to achieve a functionality
  similar to that of GS-MDO. Does this fit our model straight away? Maybe also
  show how other types of GS could be ``mimicked'' (e.g., GS only with linking).
  This would also be good to test if the model is as generic as I think it is}

\subsubsection{Ring Signatures}

\todo{Can we come up with a (\fissue,\feval,\finsp) tuple that allows us to
  somehow mimic ring signatures (i.e., ``a group signature without issuer nor
  opening''? Does the model support this?}
  
%%% Local Variables:
%%% mode: latex
%%% TeX-master: "uas"
%%% End:

\section{A Generic \UAS Construction}
\label{sec:gen-construction}

In this section, we give a generic construction of an \UAS scheme, based on
generic building blocks. In \secref{sec:instantiation}, we give a concrete
instantiation.

\subsection{Building Blocks}
\label{ssec:bblocks}

\subsubsection{Commitments}

\todo{Introduce, and summarise correctness and security.}

\begin{description}
  \item[$\Cparm \gets \CSetup(\Csecpar).$]
  \item[$(\Ccom,\Copn) \gets \CCommit(\Cparm,\msg).$] 
  \item[$\Cproof \gets \CProve(\Ccom,\Copn,\msg).$]    
  \item[$1/0 \gets \CVerify(\Cproof,\Ccom).$] 
\end{description}

\subsubsection{Signatures with Proofs of Knowledge}

As put forward in \cite{cl02}, a generic way to build anonymous credentials
requires a digital signature scheme with efficient protocols for proving
knowledge of a digital signature, and signing committed values. This approach
also has much in common with the Sign-Randomize-Prove (SRP) approach to build
group signatures \needcite. Thus, even a priori, it seems a good core component
to build something that is actually in betwween GS and AC. In the sequel, we use
\RS to refer to schemes of this type, which must provide the functionality
specified by the following syntax:

\begin{description}
\item[$(\RSvk,\RSsk) \gets \RSKeyGen(\RSsecpar,\RSlen)$.] Generates a key pair,
  consisting on the signig key \RSsk and the verification key \RSvk, which
  support sigining blocks of up to \RSlen messages.
\item[$\RSsig \gets \RSSign(\RSsk,\msgset,\cmsgset)$.] Given signing key \RSsk,
  a set of messages \msgset, and a set of commitments to messages, where
  $|\msgset + \cmsgset| \le \RSlen$, returns a signature \RSsig.
\item[$1/0 \gets \RSVerify(\RSvk,\RSsig,\amsgset)$.] Given a verification key
  \RSvk, a signature \RSsig, supposedly over set of messages \amsgset which may
  contain commitments to messages, returns $1$ or $0$. \todo{Assume that, within
    \amsgset, it is possible to know which are commitments and wich plain
    messages?}
\item[$\RSproof/\bot \gets \RSProve(\RSsig,\amsgset,\D)$.] Given signature
  \RSsig, over set of messages (and commitments to messages) \amsgset, out of
  which those indexed by set \D are to be revealed, creates a proof of knowledge
  \RSproof of \RSsig.
\item[$1/0 \gets \RSProveVer(\RSproof,\msgset)$.] Given a proof of knowledge of
  a signature over a set of messages, which includes \msgset, returns $1$ or
  $0$.
\end{description}

\todo{Summarise correctness and security.}

\subsubsection{Digital Signatures}

\todo{Introduce, and summarise correctness and security.}

\begin{description}
\item[$(\Svk,\Ssk) \gets \SKeyGen(\Ssecpar)$.]
\item[$\Ssig \gets \SSign(\Ssk,\msg)$.]
\item[$1/0 \gets \SVerify(\Svk,\Ssig,\msg)$.]  
\end{description}

\subsubsection{Encryption}

\todo{Introduce, and summarise correctness and security.}

\begin{description}
\item[$(\Eek,\Edk) \gets \EKeyGen(\Esecpar)$.]
\item[$\Ec \gets \EEnc(\Eek,\msg)$.]
\item[$\msg \gets \EDec(\Edk,\Ec)$.]
\end{description}

\subsubsection{Non-Interactive ZK Proofs of Knowledge}

\todo{Introduce, and summarise correctness and security.}

\begin{description}
\item[$\NIZKcrs\gets \NIZKSetup(\NIZKsecpar)$.]
\item[$\NIZKproof \gets \NIZKProve(\NIZKcrs,\NIZKx,\NIZKw)$.]
\item[$1/0 \gets \NIZKVerify(\NIZKcrs,\NIZKproof,\NIZKx)$.]
\end{description}

We use $\NIZK^\Lang$ to denote a \NIZK which is used to prove statements on a
given NP language \Lang.

\subsubsection{Signature Proofs of Knowledge}

\todo{Introduce, and summarise correctness and security.}

\begin{description}
\item[$\SPKproof \gets \SPKProve(\SPKmsg,\SPKx,\SPKw)$.]
\item[$1/0 \gets \SPKVerify(\SPKproof,\SPKmsg,\NIZKx)$.]
\end{description}

We use $\SPK^\Lang$ to denote an \SPK which is used to prove statements on a
given NP language \Lang.

\subsection{Generic Construction \CUASGen}

\todo{Many variables need renaming here.}

\paragraph{$\parm \gets \Setup(\secpar)$.} %
\begin{itemize}
  \item Parse \secpar as $(\Csecpar,\NIZKsecpar,\RSsecpar,\Ssecpar,\Esecpar)$
  \item $\Cparm \gets \CSetup(\Csecpar)$
  \item $\NIZKparm \gets \NIZKSetup(\NIZKsecpar)$
  \item Return $(\Cparm,\NIZKsecpar,\RSsecpar,\Ssecpar,\Esecpar)$
\end{itemize}

\paragraph{$(\ipk,\isk) \gets \IKeyGen(\parm,\fissue)$.} %
\begin{itemize}
\item Parse \parm as $(\Cparm,\NIZKsecpar,\RSsecpar,\Ssecpar,\Esecpar)$
\item $(\RSvk,\RSsk) \gets \RSKeyGen(\RSsecpar)$  
\item $\sig_{\fissue} \gets \RSSign(\RSsk,\fissue,\bot)$
\item $\ipk \gets (\RSvk,\fissue,\sig_{\fissue})$
\item $\isk \gets \RSsk$
\item Return $(\ipk,\isk)$
\end{itemize}

\paragraph{$(\opk,\osk) \gets \OKeyGen(\parm,\finsp)$.} %
\begin{itemize}
\item Parse \parm as $(\Cparm,\NIZKsecpar,\RSsecpar,\Ssecpar,\Esecpar)$
\item $(\Svk,\Ssk) \gets \SKeyGen(\Ssecpar)$
\item $(\Eek,\Edk) \gets \EKeyGen(\Esecpar)$
\item $\sig_{\finsp} \gets \SSign(\Ssk,\finsp)$
\item $\opk \gets (\Svk,\Eek,\finsp,\sig_{\finsp})$
\item $\osk \gets (\Ssk,\Edk)$
\item Return $(\opk,\osk)$
\end{itemize}

\paragraph{$(\upk,\usk) \gets \UKeyGen(\parm)$.} %
\begin{itemize}
\item Parse \parm as $(\Cparm,\NIZKsecpar,\RSsecpar,\Ssecpar,\Esecpar)$
\item $(\Svk,\Ssk) \gets \SKeyGen(\Ssecpar)$
\item Return $(\Svk,\Ssk)$
\end{itemize}

\paragraph{$\langle \cred/\bot,\utrans/\bot \rangle \gets
  \langle
  \Obtain(\usk,\attrs,\ldblbrace (gpk_i,\cred_i)\rdblbrace_{i \in \Issuers}),
  \Issue(\isk,\upk,\attrs,\ldblbrace \gpk_i \rdblbrace_{i \in \Issuers})
  \rangle$.} %
We define the NP language $\LangIss = \lbrace (\usk,
\ldblbrace \cred_i \rdblbrace_{i \in \Issuers}): \fissue(\usk,\attrs,\ldblbrace
(\gpk_i,\cred_i) \rdblbrace_{i \in \Issuers}) = 1 \land \Ccom = \CCommit(\usk)
\rbrace$ \todo{\fissue needs to ensure that all {\cred}s are linked to \usk. How
  to capture this?} The interactive protocol from a user to obtain a credential
from an issuer of the system is as follows:

\begin{itemize}
\item Issuer: $\ch \gets \bin^*$; Send \ch to User.
\item User:
  \begin{itemize}    
  \item $\Ccom_{\usk} \gets \CCommit(\usk)$
  % \item $\SPKproof^{\Lang} \gets
  %   \SPK \lbrace (\usk): \Ccom_{\usk} = \CCommit(\Cparm,\usk) \rbrace (\ch)$
  \item $\SPKproof^{\LangIss} \gets \SPKProve^{\LangIss}
    (\ch,(\attrs,\Ccom_{\usk}, \ldblbrace \gpk_i \rdblbrace_{i \in \Issuers}),
    (\usk,\ldblbrace \cred_i \rdblbrace_{i \in \Issuers}))$
  \item Send $(\attrs,\Ccom_{\usk},\SPKproof^{\LangIss})$ to Issuer
  \end{itemize}
\item Issuer:
  \begin{itemize}
%  \item $1 \stackrel{?}{=} \SPKVerify(\SPKproof^{\Lang},\Ccom_{\upk})$
  \item $1 \stackrel{?}{=} \SPKVerify(\SPKproof^{\LangIss},\ch,
    (\attrs,\Ccom_{\usk}, \ldblbrace \gpk_i \rdblbrace_{i \in \Issuers}))$
  \item $\cred \gets \RSSign(\isk,\attrs,\Ccom_{\usk})$
  \item $\utrans \gets (\Ccom_{\usk},\cred,\attrs)$
  \item Send \cred to User and output \utrans.
  \end{itemize}
\item User:
  \begin{itemize}
  \item $1 \stackrel{?}{=} \RSVerify(\RSvk,\cred,(\attrs,\Ccom_{\usk}))$
  \item Output \cred
  \end{itemize}   
\end{itemize}

\paragraph{$\sig \gets \Sign(\gpk,\usk,\cred,\msg,\feval)$.} %
We define the NP language $\LangSig = \lbrace (\upk,\cred,\y):
\SVerify(\sig,\upk) = 1~\land~\Ec = \EEnc(\Eek,\y)~\land~
\y = \finsp(\feval(\upk,\cred,\msg),\upk,\cred,\msg)\rbrace$. The
signing algorithm is defined as follows:
\todo{Argue that the composition of \finsp and \feval is not redundant even
  though both depend on the same arguments. Namely, because \feval is defined
  on a per-signature basis (akin to the policies of ACs), and \finsp is defined
  once per group (akin to the open function in GSs). Hence, \finsp cannot just
  be defined to ``contain'' \feval at setup time.}

\begin{itemize}
\item Parse \gpk as $(\ipk,\opk)$
\item Parse \ipk as $(\RSvk,\fissue,\sig_{\fissue})$; Verify $\sig_{\fissue}$
\item Parse \opk as $(\Svk,\Eek,\finsp,\sig_{\finsp})$; Verify $\sig_{\finsp}$
\item $\sig' \gets \SSign(\usk,\msg)$
%\item $b \gets \feval(\upk,\cred,\msg)$
\item % If $b=0$: 
  $\y \gets \finsp(\feval(\upk,\cred,\msg),
  \upk,\cred,\msg)$%; else $\y = \bot$.
\item $\Ec \gets \EEnc(\Eek,\y)$
\item $\SPKproof^{\LangSig} \gets \SPKProve^{\LangSig}(\msg,(\Ec,\sig'),
  (\upk,\cred,\y))$
\item Return $(\Ec,\sig',\SPKproof^{\LangSig})$
\end{itemize}

\paragraph{$1/0 \gets \Verify(\gpk,\sig,\msg,\feval)$.} %
\begin{itemize}
\item Parse \sig as $(\Ec,\sig',\SPKproof^{\LangSig})$
\item Return $\SPKVerify^{\LangSig}(\SPKproof^{\LangSig},\msg,(\Ec,\sig'))$ 
\end{itemize}

\paragraph{$(\y,\iproof)/\bot \gets \Inspect(\gpk,\osk,\trans,\sig,\msg,\feval)$.} %

We follow the approach of \cite{bsz05} for verifiable openings. Namely, we
define an NP language $\LangVerIns = \lbrace (\Eek,\Edk,r):
(\Eek,\Edk) = \EKeyGen(\Esecpar;r) \land \y = \EDec(\Edk,\c) \rbrace$.

\begin{itemize}
\item Parse \osk as $(\Ssk,\Edk)$; \gpk as
  $(\cdot,(\Svk,\Eek,\finsp,\sig_{\finsp}))$
\item  $1 \stackrel{?}{=} \Verify(\gpk,\sig,\msg,\feval)$  
\item $\y \gets \EDec(\Edk,\c)$
\item $\NIZKproof^{\LangVerIns} \gets \NIZKProve(\NIZKcrs,(\c,\y),(\Eek,\Edk,r))$
\item Return $\NIZKproof^{\LangVerIns}$
\end{itemize}

\paragraph{$1/0 \gets \Judge(\gpk,\y,\iproof,\sig,\msg)$.} %

\begin{itemize}
\item  $1 \stackrel{?}{=} \Verify(\gpk,\sig,\msg,\feval)$
\item Parse \sig as $(\c,\cdot,\cdot)$  
\item Return $\NIZKVerify(\NIZKcrs,\iproof,(\c,\y))$
\end{itemize}

%%% Local Variables:
%%% mode: latex
%%% TeX-master: "uas"
%%% End:

\subsection{Correctness and Security of \CUASGen}
\label{ssec:security-uas}

First, we define the \Identify, \ExtractIssue and \ExtractSign functions that
are needed for some of the properties to be meaningful, in
\figref{fig:helper-funcs}.

\begin{figure}[ht!]
  \begin{minipage}[t]{\textwidth}
    \procedure{$\ExtractIssue(\parm,\trans)$}{%
      \textrm{Parse \parm as $(\cdot,\cdot,\cdot,\cdot,\NIZKcrs_{\Issue},\cdot,
        \cdot,\cdot)$; $\NIZKcrs_{\Issue}$ as $(\NIZKcrs,\NIZKtrap)$; and
        \trans as $(\Ccom,\attrs,\sipk,\cred,\NIZKproof)$} \\
      \pcif \NIZKVerify(\NIZKcrs,\NIZKproof,(\Ccom,\attrs,\sipk)): 
      \pcreturn \bot \\
      (\usk,\scred,\attrs_{\scred}) \gets \NIZKExtract(\NIZKcrs,\NIZKtrap,
      (\Ccom,\attrs,\sipk),\NIZKproof) \\
      \pcreturn (\usk,\scred,\attrs_{\scred}) \\
    }
    
    \procedure{$\ExtractSign(\parm,\oid,\siid,\sig,\yeval,\msg,\feval)$}{%
      \textrm{Parse \parm as $(\cdot,\cdot,\cdot,\cdot,\cdot,\NIZKcrs_{\Sign},
        \cdot,\cdot)$; $\NIZKcrs_{\Sign}$ as $(\NIZKcrs,\NIZKtrap)$; and
        \sig as $(\NIZKproof,\Ec)$} \\
      \textrm{Parse $\PUBOK[\oid]$ as $(\opk,\cdot)$ and let $\sipk \gets
        \PUBIK[\siid]$} \\
      \pcif \NIZKVerify(\NIZKcrs,\NIZKproof,(\msg,\feval,\yeval,\Ec,
      \sipk,\opk)): \pcreturn \bot \\
      (\usk,\scred,\attrs_{\scred},\yinsp,r) \gets \NIZKExtract(\NIZKcrs,\NIZKtrap,
      (\msg,\feval,\yeval,\Ec,\sipk,\opk),\NIZKproof) \\
      \pcreturn (\usk,\scred,\attrs_{\scred},\yinsp) \\
    }
    
    \procedure{$\Identify(\usk,\attrs_{\cred},\cred)$}{%
      \pcreturn \SBCMVerify(\ipk_{\cred},\cred,\attrs_{\cred} \cup
      \lbrace \usk \rbrace) \\
    }    
  \end{minipage}
  \label{fig:helper-funcs}
  \caption{Definition of helper functions \Identify, \ExtractIssue and
    \ExtractSign, for \CUASGen.}
\end{figure}

\begin{theorem}[Correctness of \CUASGen]
  \label{thm:correctness-uas}
  If the underlying schemes for vector commitments, encryption, digital
  signatures, signatures on blocks of committed messages, and NIZKs are
  correct, our generic construction \CUASGen satisfies correctness as
  defined in \defref{def:correctness-uas}.
\end{theorem}

\begin{proof}[\thmref{thm:correctness-uas}]
  \todo{XXX}
\end{proof}

\begin{theorem}[Anonymity of \CUASGen]
  \label{thm:anonymity-uas}
  If the underlying encryption scheme is \todo{IND-CCA}, the vector commitment
  scheme is \todo{binding}, the scheme for signatures on blocks of committed
  messages is \todo{XXX}, and the NIZKs used for $\NIZKRel_{\Issue},
  \NIZKRel_{\Sign}$, and $\NIZKRel_{\Inspect}$ are \todo{zero-knowledge} and
  \todo{simulation-sound?}, our \CUASGen construction satisfies anonymity as
  defined in \defref{def:anonymity-uas}.
\end{theorem}

\begin{proof}[\thmref{thm:anonymity-uas}]
\end{proof}

\begin{theorem}[Issuance unforgeability of \CUASGen]
  \label{thm:issue-forge-uas}
  If the underlying scheme for signatures on blocks of committed messages is
  existentially unforgeable, and the NIZK used for $\NIZKRel_{\Issue}$ is
  simulation extractable and sound, then our \CUASGen construction satisfies
  issuance unforgeability as defined in \defref{def:issue-forge-uas}, except
  with negligible probability. \todo{EUF of \SBCM?}
\end{theorem}

\todo{\usk belongs to \AttrSpace! I think this can lead to malleability attacks.
  Make them disjoint?}

\begin{proof}[\thmref{thm:issue-forge-uas}]
  We show that the probability that \fissue outputs $0$ is negligible, as well
  as the probability that the extracted \usk is not the one that was used to
  request some of the credentials employed to obtain the credential specified by
  the adversary.
  %
  For this purpose, we define two games, $G_0=\ExpForgeIssue$, and $G_1$, which
  is exactly the same, but where, within the \Setup algorithm, we replace
  $\NIZKSetup^{\Issue}$ with $\NIZKSimSetup^{\Issue}$. As per the definition of
  \NIZK in \appref{sapp:nizk}, both games are indistinguishable.

  Now, observe that the adversary is required to output a credential
  identifier for which associated entries in \trans and \CRED exist; moreover,
  if such a credential was produced by an issuer, we must have access to those
  entries, as issuers are assumed to be honest.
  %
  Then, given that $\NIZKRel_{\Issue}$ is knowledge extractable, in game $G_1$
  we can apply the \NIZKExtract function, which produces a tuple $(\usk,\scred,
  \attrs_{\scred})$ from $\utrans = (\Ccom,\attrs,\sipk,\cred,\NIZKproof)$.
  %
  Since \NIZKproof is accepted by \ExtractIssue, from the soundness of \NIZK and
  existential unforgeability of \SBCM, we know that all $\cred \in \scred$ are
  valid signatures over \usk, and their respective $\attrs_{\cred}$. Thus,
  \Identify returns $1$ for all $(\usk,\attrs_{\cred},\cred)$ tuples.
  Moreover, all the credentials in \scred given to \fissue belong to the same
  user, who is the owner of \usk.
  %
  Finally, since issuers are honest, we know that $\ATTR[\cid] = \attrs$ and,
  consequently, $\fissue(\usk,\scred,\ATTR[\cid]) = \fissue(\usk,\scred,\attrs)
  = 1$, due to the soundness of \NIZK.
  %
  \qed
\end{proof}

\begin{theorem}[Signing unforgeability of \CUASGen]
  \label{thm:sign-forge-uas}
  If the underlying NIZK scheme for $\NIZKRel_{\Sign}$ is sound and simulation
  extractable, the NIZK scheme for $\NIZKRel_{\Inspect}$ is sound and simulation
  extractable, and \SBCM is existentially unforgeable, then our \CUASGen
  construction satisfies signing unforgeability as defined in
  \defref{def:sign-forge-uas}, except with negligible probability.
\end{theorem}

\begin{proof}[\thmref{thm:sign-forge-uas}]
  As for \thmref{thm:issue-forge-uas}, we define two games, $G_0=\ExpForgeSign$,
  and $G_1$, which is exactly the same, but where, within the \Setup algorithm,
  we replace $\NIZKSetup^{\Sign}$ with $\NIZKSimSetup^{\Sign}$. As per the
  definition of \NIZK in \appref{sapp:nizk}, both games are indistinguishable.

  From $G_1$, and in order to define the winning conditions for the adversary
  in the signing unforgeability game, consider the following events:

  \begin{description}
  \item[$V$.] Where $V = \Verify(\opk,\sipk,\sig,\yeval,\msg,\feval) = 1$.
  \item[$J$.] Where $J = \Judge(\opk,\sipk,\yinsp,\iproof,\sig,\yeval,\msg,
    \feval) = 0$.
  \item[$L$.] Where $L = (\msg,\feval,\yeval,\Ec,\sipk,\opk) \in
    \NIZKLang^{\Sign}$.    
  \item[$I$.] Where $I = \exists \cred \in \scred~\st~\Identify(\usk,
    \attrs_{\cred},\msg) = 0$.
  \end{description}

  $\adv$ wins if $V \land (J \lor I) = (\overline{L} \land V \land (J \lor I))
  \lor (L \land V \land (J \lor I))$.
  %
  $V$ implies that $(\msg,\feval,\yeval,\Ec,\sipk,\opk) \in \NIZKRel^{\Sign}$.
  Thus, after soundness of $\NIZK^{\Sign}$, the probability of $\overline{L}
  \land V \land (J \lor I)$ is negligible in the security parameter.
  %
  For $(L \land V \land (J \lor I))$ to be satisfied, there are three cases:
  \begin{enumerate}
  \item $L \land V \land J \land \overline{I}$. The game returns 1 in step 6.
  \item $L \land V \land J \land I$.  The game returns 1 in step 6.
  \item $L \land V \land \overline{J} \land I$. The game returns 1 in step 10. 
  \end{enumerate}

  In case 1, $L \land V$ implies that $(\msg,\feval,\yeval,\Ec,\sipk,\opk) \in
  \NIZKRel^{\Sign}$. More concretely, soundness of $NIZK^{\Sign}$ implies that
  $\Ec = \EEnc(\opk,\yinsp;r)$, for $\yinsp$ and $r$ known to the signer. Since
  $(\yinsp,\iproof)$ is generated honestly by the challenger from
  $(\opk,\sipk,\sig = (\NIZKproof_{\Sign},\Ec),\yeval,\msg)$, correctness of
  public key encryption implies that $\EDec(\osk,\Ec) = \yinsp$. Consequently,
  the probability of $L \land V \land J \land \overline{I}$ = 0, as \Judge
  checks precisely that \Ec is a correct encryption of \yinsp under \opk.

  The analysis for case 2 is the same as for case 1.

  For case 3, $(\msg,\feval,\yeval,\Ec,\sipk,\opk) \in \NIZKRel^{\Sign}$,
  $\Judge(\opk,\sipk,\yinsp,\iproof,\sig,\yeval,\msg,\feval)
  = 1$, but there exists some credential \cred for which $\Identify(\usk,
  \attrs_{\cred},\msg)=0$. Note that \usk, $\attrs_{\cred}$ and \cred (for all
  $\cred \in \scred$) are output by \ExtractSign. Thus, after the
  simulation-extractability property and soudness of $\NIZK^{\Sign}$, $(\msg,
  \feval,\yeval,\Ec,\sipk,\opk) \in \NIZKRel^{\Sign}$, which more concretely
  means that $\SBCMVerify(\ipk_{\cred},\cred,\attrs_{\cred} \cup \lbrace \usk
  \rbrace) = 1 = \Identify(\usk,\attrs_{\cred},\cred)$, for all $\cred \in
  \scred$. The probability of case 3 is therefore $0$.
  %
  Moreover, since \SBCMVerify returns $1$ for all $\cred \in \scred$, it must
  be that all of them were obtained via queries to the \ISSUE or \OBTISS
  oracles. Otherwise, if there exists some \cred that was not obtained via
  a call to these oracles, the pair $(\lbrace \usk \rbrace \cup \attrs_{\cred},
  \cred)$ constitutes an existential forgery of \SBCM.

  Cases 1, 2 and 3 above account for winning conditions at steps 6 and 10.
  % 
  Additionally, $\adv$ wins at step 8 if $\feval(\usk,\scred,\msg) \neq \yeval$,
  where \yeval is the value output by the adversary in step 2. However, since
  $\Verify(\opk,\sipk,\sig,\yeval,\msg,\feval) = 1$, soundness of $NIZK^{\Sign}$
  implies that this has negligible probability.
  %
  Similarly, $\adv$ wins at step 9 if (1) $\finsp(\yeval,\usk,\scred,\msg) \neq
  \yinsp$, where \yinsp is the value output by \Inspect at line 5; or if (2)
  $\yinsp \neq \yinsp'$, where $\yinsp'$ is the value extracted by \ExtractSign
  at step 7. For (1), soundness of $\NIZK^{\Sign}$ ensures that \yinsp is the
  correct evaluation of \finsp, whereas soundness of $\NIZK^{\Inspect}$ and
  correctness of the public key encryption ensure that this is also the value
  output by \Inspect. Thus, the probability of $\adv$ winning the game because
  of (1) is negligible. Finally, for (2), simulation-extractability of
  $\NIZK^{\Inspect}$ and correctness of the public key encryption ensure that
  the $\yinsp'$ value extracted by \NIZKExtract matches the value produced by
  \Inspect.
  %
  \qed
\end{proof}

\begin{theorem}[Non-frameability of \CUASGen]
  \label{thm:frame-uas}
  If the underlying NIZK schemes are zero-knowledge and the scheme used for
  $\NIZK^{\Sign}$ is simulation-extractable, and the commitment scheme is
  hiding, then our \CUASGen construction satisfies non-frameability as defined
  in \defref{def:frame-uas}, except with negligible probability.
\end{theorem}

\begin{proof}[\thmref{thm:frame-uas}]
  We prove that, given an adversary $\adv$ against non-frameability of \CUASGen,
  we can build an adversary \advB that breaks the hiding property of the
  underlying commitment scheme, with non-negligible probability.

  We start from $G_0=\ExpNonframe$. $\adv$ makes queries to the oracles in
  \Oframe.

  For $G_1$, within \Setup, we replace the \Setup algorithms for the three
  NIZKs (\Issue, \Sign and \Inspect) with their corresponding \SimSetup
  variants. Consequently, the corresponding queries to \Prove are also
  simulated via the simulator. By the zero-knowledge property of the NIZK
  systems, $G_1$ is indistinguishable from $G_0$.
  
  We build adversary \advB against hiding of commitments from $G_1$ against
  non-frameability. In the hiding game (see \figref{fig:com-games}), \advB first
  picks two messages $\msg_0$ and $\msg_1$, and then receives a commitment \Ccom
  of $\msg_b$. Let \advB pick both $\msg_0$ and $\msg_1$ from \AttrSpace. Then,
  \advB initializes $G_1$ for $\adv$ against non-frameability, and randomly
  picks a number $u \getr [1,q]$, where $q$ can be as large as \advB wants, but
  will be the maximum number of honest users to let $\adv$ add to the game.
  Then, when $\adv$ asks to create the $u$-th user, \advB ignores the call to
  \UKeyGen. For every call that $\adv$ makes to the \OBTAIN oracle associated to
  the $u$-th user, \advB uses the commitment \Ccom received as challenge in its
  game against the hiding property of commmitments, and uses it as commitment to
  the $u$-th user's \usk. \advB then simulates all the NIZK proofs associated to
  the $u$-th user, in calls to \OBTAIN, \SIGN and \INSPECT. Again, due to the
  zero-knowledge property of the associated NIZK systems, the outputs of the
  moddified oracles are indistinguishable from the original outputs (as \Ccom
  is a valid commitment of a user secret key). Eventually, $\adv$ outputs a
  $(\sig,\yeval,\msg,\feval,\yinsp,\iproof)$ tuple that is accepted by \Verify
  and \Judge. Due to simulation extractability of $NIZK^{\Sign}$, \ExtractSign
  must be able to produce a $(\usk,\scred,\attrs_{\scred},\yinsp')$ tuple. Since
  $\adv$ wins the
  non-frameability game with non-neglibible probability, with probability $1/q$,
  the \usk value belongs to the $u$-th honest user, so it must be equal to
  either $\msg_0$ or $\msg_1$; \advB responds accordingly to its challenge in
  the game for the hiding property of the commmitment scheme. By assumption,
  \advB wins with non-negligible probability.
  %
  \qed
\end{proof}


%%% Local Variables:
%%% mode: latex
%%% TeX-master: "uas"
%%% End:

\section{Relating \UAS with Other Schemes, and Variations}
\label{sec:transformations}

In order to justify \UAS's universality, in this section we describe how do
\UAS schemes relate with other similar systems. \jdv{The relationships seem
  quite straightforward; thus, we only informally sketch how one could build
  other schemes (or something very similar to them) from \UAS, and leave the
  formal analysis for futher work.}
%
As a first (informal) observation, we note that the same transformation to
interactive presentations as described in \GSAC applies here.

\subsection{\GSAC from \UAS}
Take the generic construction for \UAS in \secref{ssec:generic-construction-uas}.
First, consider the constant issuance function $\fissue^1(\usk,\scred,\attrs) =
1$. Next, consider $\finsp^{\upk}(\yeval,\usk,\scred,\msg) \coloneqq
\CCommit(\usk;0)$, and $\feval^{\dattrs}(\usk,\cred,\msg) \coloneqq \dattrs(\cred)$,
where $\dattrs(\cred)$ is the function returning the attributes indexed by
\dattrs within \cred (which we can assume to be easy to do with proper
encoding). It is straightforward to see that an \UAS scheme with $\fissue^1$
as issuance function, $\feval^{\dattrs}$ as signing evaluation function, and
$\finsp^{\upk}$ as opening function, is an implementation of the functionality
described for \GSAC --which, in addition, lets the signer use more than one
credential per signature; although this can of course be restricted
implementation-wise.
%
Furthermore, the anonymity, sign unforgeability, and non-frameability properties
of \UAS directly imply the anonymity, traceability, and non-frameability
properties of \GSAC.

\paragraph{Vanilla Group Signatures and Anonymous Credentials from \UAS.} As a
corollary to the previous analysis, recall that, in \secref{ssec:variants-gsac}
we proved that \GSAC implies both group signatures and anonymous credentials.
Thus, since \UAS implies \GSAC, \UAS implies also both group signatures and
anonymous credentials.

\subsection{\UAS with Multiple Openers}
We briefly mention that we have defined \UAS signatures as being ``openable'' by
only one opener. However, since there is no tight relation between issuer and
opener (as opposed to conventional group signatures), it is trivial to
extend to the case with multiple openers per signature. In that case, the
$\NIZKRel_{\Sign}$ relation should be extended to include proofs of correct
encryption of the \yinsp values produced by each chosen opener.
Similarly, the sign unforgeability and non-frameability definitions should be
extended to check correctness of the corresponding extra open values.

\subsection{Group Signatures with Message Dependent Opening}

In \cite{khk+19}, one of the first group signature schemes with a certain
flexibility in the open-related functionality was introduced. Briefly, a group
signature scheme with message dependent opening works as a conventional group
signature scheme, with the addition that it is only possible to open signatures
over specific messages (e.g., text with offensive language). For this, the
authors introduce an extra authority, the admitter, who creates a
``message-specific token'' for each message that is deemed unacceptable.
These message-specific tokens are later needed by the opener, in order to open
signatures over unacceptable messages.

\UAS can implement this functionality in a straightforward manner. Define
$\finsp^{\smsg}(\yeval,\usk,\cred,\msg) \coloneqq \lbrace \pcif \msg \in \smsg:
\pcreturn \CCommit(\usk;0); \pcelse \pcreturn \bot \rbrace$. Also, let
$\feval^0(\usk,\scred,\msg) \coloneqq \pcreturn 0$. Then, \UAS with $\fissue^1$
as issuance function, $\feval^0$ as signing evaluation function, and
$\finsp^{\smsg}$ as opening function, implements a message-dependent opening
group signature scheme -- again, restricting to one credential per user and
signature, as before. Moreover, it implements the same functionality without
an extra authority.

\jdv{In \cite{khk+19}, the authors prove that GS-MDO implies identity-based
  encryption. \UAS implying GS-MDO would thus imply IBE, but we do not use
  IBE in our constructions. Is there some connection between some of the main
  building blocks we use, and IBE?}

\subsection{Ring Signatures}

In a nutshell, ring signatures \cite{rst06} are like group signatures where
rather than having an issuer accepting users into a group, any user can create
an ad hoc group composed by himself, and any arbitrary set of other users that
have advertised their public keys in some publicly accessible way. Also, in
ring signatures no de-anonymization is possible\footnote{At least, in vanilla
  ring signatures. Some variants allow some sort of linkability; for instance
  \cite{lww04}.}, although the most distinguishing property is probably the lack
of issuer, which has led some relevant systems like Monero%
\footnote{\url{https://www.getmonero.org/resources/moneropedia/ringsignatures.html}.
  Last access on May 8th, 2022.} to opt for ring signatures rather than, e.g.,
group signatures.

To reach (vanilla) ring signatures from \UAS, we have to slightly alter our
generic construction. This is due to the fact that the NP relation
$\NIZKRel_{\Sign}$ defined in \secref{ssec:generic-construction-uas} reveals the
issuers' public keys. This is a problem as, intuitively, the signer in a ring
signature is a sort of issuer for the ad hoc group. To hide the issuer, we
replace the NP relation in \secref{ssec:generic-construction-uas} with
$\NIZKRel^{\Sign-prv} \coloneqq \lbrace (\usk,\scred,\attrs_{\scred},\yinsp,r,
\sipk_{\scred}),(\msg,\feval,\yeval,\Ec,\Eek): \Ec = \EEnc(\Eek,\yinsp;r) \land
\yeval = \feval(\usk,\scred,\msg) \land \yinsp = \finsp(\yeval,\usk,\scred,\msg)
\land \forall \cred \in \scred,\SBCMVerify(\ipk_{\cred},\cred,\usk,
\attrs_{\cred}) = 1) \rbrace$. That is, the public keys of the issuers are now
part of the witness in the NP relation, which means that they will not
be revealed in the proofs.
%
Besides $\NIZKRel^{\Sign-prv}$, let \sring be an ad hoc set containing the
public keys of the users that the signer wants to include in its ad hoc group.
We define $\fissue^{\sring}(\usk,\scred,\attrs) \coloneqq \lbrace \pcif \attrs
\subseteq \sring \cup \lbrace \CCommit(\usk;0) \rbrace: \pcreturn 1; \pcelse
\pcreturn 0 \rbrace$. That is, if we let the attributes in a credential be
public keys, this function means that a credential will be issued if all the
public keys specified as attributes to be included in the credential are part of
the union of the ad hoc ring and the public key of the signer. We also define
$\feval^{\attrs}(\usk,\cred,\msg) \coloneqq \lbrace \pcif \CCommit(\usk;0) \in
\attrs: \pcreturn \attrs(\cred); \pcelse \pcreturn \bot \rbrace$, where
$\attrs(\cred)$ is the function that returns all the attributes encoded in a
credential. That is, if the public key of the signer is included in the ring,
return all the attributes in the credential (i.e., all the public keys in the
ring); otherwise, abort. Finally, let $\finsp^0(\yeval,\usk,\cred,\msg) = 0$.

Then, an \UAS scheme with $\NIZKRel^{\Sign-prv}$ as NP relation for
$\NIZK^{\Sign}$, and $\fissue^{ring}$, $\feval^{\attrs}$, and $\finsp^0$ is
intuitively a ring signature scheme. To see this, observe that any user can
act as an issuer. Thus, the owner of \usk can issue to himself a credential
for any arbitrary ring it desires. While the $\NIZKRel^{\Issue}$ reveals the
issuer's public key, note that this is irrelevant, as the issuer is the user
himself. Then, the newly defined $\NIZKRel^{\Sign-prv}$ does exactly the same
as in our original \UAS scheme, but without revealing the issuer's public key.
$\feval^{\attrs}$ reveals all the attributes in the credential, which are
the public keys of the ring, including the signer's public key. This is actually
what ring signatures do: the signer, who is the owner of the private key
associated to one of the public keys in the ring, advertises which are the
public keys in the ring, and proves knowledge of one of them. If the NIZK
proof verifies, this means that the signer is indeed the owner of such a private
key. Finally, since $\finsp^0$ returns always $0$, then no de-anonymization is
possible.
%
Interestingly, this construction based on \UAS allows adding extra attributes
(beyond the public keys in the ring) to the produced signatures, which may be
useful for real world use cases.

\subsection{Multimodal Private Signatures}

To show that Multimodal Private Signatures (MPS) \needcite can be built from
\UAS, we need to give an alternative, simulation-based, definition of the
anonymity property. In the simulation-based approach, we require the adversary
to guess the bit $b$ defining whether it is interacting with the real world,
where it gets signatures by real users, or with a simulation, where all
signatures are simulated and do not contain information about the signer.
This alternative formulation is given in \figref{fig:exp-uas-simanon}, where
$\Osimanon = (\lbrace\HU,\CU\rbrace\GEN,\lbrace\II,\OO\rbrace\GEN,\lbrace\II,
\OO\rbrace\CORR,\OBTAIN,\WREG)$.

\begin{figure}[htp!]

  \centering
  \procedure[linenumbering]{$\ExpSimAnonb(1^\secpar)$}{%
    \pcif b = 0: \\
    \parm \gets \Setup(1^\secpar) \\    
    \pcind b^* \gets \adv^{\Osimanon,\SIGN,\OPEN}(\parm) \\
    \pcelse: \\
    \parm \gets \SIMSETUP(1^\secpar) \\
    \pcind b^* \gets \adv^{\Osimanon,\SIMSIGN,\SIMOPEN}(\parm) \\    
    \pcreturn b^*
  }
  
  \caption{Simulation-based anonymity experiment for \UAS schemes.}
  \label{fig:exp-uas-simanon}
\end{figure}

\begin{definition}{(Simulatable Anonymity of \UAS)}
  \label{def:sim-anonymity-uas}  
  We define the advantage \AdvSimAnon of $\adv$ against \ExpSimAnonb as
  $\AdvSimAnon=|\Pr\lbrack\ExpSimAnono(1^\secpar)=1\rbrack-
  \Pr\lbrack\ExpSimAnonz(1^\secpar)=1\rbrack|$.
  %
  An \UAS scheme satisfies simulatable anonymity if there exists simulators
  \SIMSETUP, \SIMSIGN and \SIMOPEN such that, for any p.p.t. adversary $\adv$,
  \AdvSimAnon is a negligible function of $1^\secpar$.
\end{definition}

Note that, for our generic construction \CUASGen, \SIMSETUP, \SIMSIGN, and
\SIMOPEN are straightforward, as all we need to do is:

\begin{itemize}
\item To build \SIMSETUP, just replace the $\NIZKSetup^{\Sign}$ and
  $\NIZKSetup^{\Open}$ algorithms within \Setup, by their respective
  $\NIZKSimSetup$ algorithms.
\item To build \SIMSIGN from the \SIGN oracle, just simulate the NIZK proof
  contained in the signatures produced by the \Sign algorithm.
\item To build \SIMOPEN from the \OPEN oracle, just simulate the NIZK proof
  showing opening correctness, instead of producing them as in the \Open
  algorithm.
\end{itemize}

Indistinguishability then follows from the zero-knowledge property of each
corresponding NIZK system. Since both scenarios are indistinguishable, and
all signature-related proofs received by the adversary in the simulation do not
depend on any witness (as they are simulated), then it follows that the
adversary cannot gain any information about signers in the real world.

With this alternative definition, we can achieve a generalisation of MPS
basically by, insted of returning the plaintext \yeval value, returning an
encryption of it under the opener's public key; additionally proving 
that the value used to compute the output of \finsp is the plaintext \yeval
value. In more detail, given any \feval function, we first define an
encrypted variant $\feval^{enc}(\usk,\scred,\msg) \coloneqq \lbrace \yeval
\gets \feval(\usk,\scred,\msg); \yeval' \gets \EEnc(\opk,\yeval); \pcreturn
\yeval' \rbrace$.
%
Naturally, the $\NIZKRel_{\Sign}$ has to be correspondigly extended, as
$\NIZKRel_{\Sign}^{enc} = \lbrace (\usk,\scred,\attrs_{\scred},\yinsp,r),(\msg,
\feval,\yeval,\Ec,\sipk_{\scred},\Eek): \Ec = \EEnc(\Eek,\yinsp;r) \land
\yeval = \EEnc(\opk,\feval(\usk,\scred,\msg)) \land \yinsp = \finsp(\feval(\usk,
\scred,\msg),\usk,\scred,\msg) \land \forall \cred \in \scred,
\SBCMVerify(\ipk_{\cred},\cred,\usk,\attrs_{\cred}) = 1) \rbrace$. This would
also seem to require that the used public-key encryption scheme is IND-CCA.

Given this approach, the need to provide an alternative definition for anonymity
is clearer. Namely, in the original \CHALb oracle, we restrict that the
signatures produced by both challenge users output the same \yeval value.
However, if instead of returning the plaintext \yeval value, we return an
encrypted version of it (as we do in the variant just described), this clearly
becomes unachievable.

\jdv{It would seem that the simulation-based anonymity definition is more
  general. Can we prove that? Concretely, if it is also good for selective
  disclosure, we may just adopt it as default.}

\subsection{Delegatable Credentials}

Building delegatable credentials is also straightforward, thanks to the \fissue
function. For instance, take the simplification of level-based delegation, where
the owner of a credential of level $n$ can only issue credentials of level
$n+1$. Without loss of generality, assume that the credential level is encoded
in the first attribute of the credential (after the \usk), which we denote with
$a_1$. In this context, any owner of a credential of level $n$ can define an
issuance function $\fissue^n \coloneqq \lbrace \pcif a_1 = n+1: \pcreturn 1;
\pcelse \pcreturn 0 \rbrace$.

\jdv{\subsection{Functional Signatures}}
\jdv{I think that, by making \yeval explicit, \UAS is essentially a
  privacy-preserving extension to Functional Signatures. Still, I read that paper
  quite some time ago. Re-check.}

%%% Local Variables:
%%% mode: latex
%%% TeX-master: "uas"
%%% End:


%%% Local Variables:
%%% mode: latex
%%% TeX-master: "uas"
%%% End:
