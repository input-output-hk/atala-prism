\section{From \GSAC to \UAS}
\label{sec:uas}

\GSAC is a very interesting exercise that showcases the advantages of direct
combination of group signatures and anonymous credentials. However, it also
inherits a main limitation: it is too strict in regards to the utility it
offers. The most excruciating example of this fact is that in \GSAC, via \Open,
one can only get the actual identity of the signer. Even though the inclusion
of attributes makes the concept of ``identity'' meaningful, we
may not want to fully de-anonymize the signer. Instead, we may be interested in
revealing arbitrary functions of its identity and the signed message. Also, for
the sake of illustation, we restricted to the case of selective disclosure, and
to only one credential per signature. Then, the natural question is whether we
can model an extension to \GSAC that is compatible both with constructions
that support selective disclosure, and with constructions that support arbitrary
claims on the
credential attributes; and, as it is also common in the anonymous credentials
(or verifiable credentials\footnote{\url{https://www.w3.org/TR/vc-data-model/}.
  Last access, May 5th, 2022.}) domains, to allow credential holders to
combine multiple credentials into one signature/presentation. Similarly, we can
also generalize \GSAC by allowing users to leverage previously obtained
credentials in order to obtain new ones; possibly, applying arbitrary issuance
policies. In a nutshell, our generalization of \GSAC into Universal Anonymous
Signatures (\UAS), captured in \figref{fig:proof-blocks-uas}, enables:

\begin{itemize}
\item Using previously obtained credentials to request new ones, applying a
  configurable policy defined via an issuance function \fissue, which determines
  whether the new issuance should proceed or not.
\item Modulating the type of utility one wants to get directly from signatures,
  via a signature evaluation function \feval. \feval can be programmed to reveal
  nothing, selective disclosure, arbitrary claims over the credentials'
  attributes, etc.
\item Defining arbitrary accountability information that will be extractable by
  openers through an opening function \finsp. This can range from no information
  at all, to functions on subsets of the attributes, to full de-anonymization.
\end{itemize}

\begin{figure}[ht!]
  \begin{tikzpicture}

  \pgfdeclarelayer{bg}    % declare background layer
  \pgfsetlayers{bg,main}  % set the order of the layers (main is the standard layer)

  \node (GSAC) at (3.00,3.50) { \GSAC };
  \node (UAS) at (9.00,3.50) { \UAS };

  \node (issue) at (0.00,2.50) { \bf Issue };
  \node (sign) at (0.00,1.50) { \bf Sign };  
  \node (open) at (0.00,0.50) { \bf Open };

  \node (issue-gsac) at (3.00,2.50) { $K(\usk)$ };
  \node[align=center] (sign-gsac) at (3.00,1.50)
  { $K(\usk) \land K(\cred) ~\land$ \\
    $\lbrace \usk,\attrs \rbrace \in \cred \land E(\upk)~\land$ \\
    $E(\attrs) \land \dattrs \subseteq \attrs$ };
  \node (open-gsac) at (3.00,0.50) { $D(\upk) \land D(\attrs)$ };

  \node (issue-uas) at (9.00,2.50)
  { $K(\usk) \land K(\scred) \land \usk \in \scred \land \fissue(\cdot) = 1$ };
  \node[align=center] (sign-uas) at (9.00,1.50)
  { $K(\usk) \land K(\scred) \land \usk \in \scred \land E(\yinsp)~\land$ \\
    $\yeval = \feval(\cdot) \land \yinsp = \finsp(\yeval,\cdot)$
  };
  \node (open-uas) at (9.00,0.50) { $D(\yinsp)$ };

  \draw[draw=black,thick] (10.80,2.29) rectangle ++(1.43,0.44);
  \draw[draw=black,dashed,thick] (6.80,1.09) rectangle ++(1.75,0.44);
  \draw[draw=black,dotted,thick] (8.85,1.09) rectangle ++(2.30,0.44);
    
\end{tikzpicture}

%%% Local Variables:
%%% mode: latex
%%% TeX-master: t
%%% End:

  \caption{Comparison of statements proved per operation in \GSAC and \UAS.
    $K(x)$ means proving knowledge of $x$; $E(y)$ means proving correct
    encryption of $y$; $D(z)$ means proving correct decryption of (an encryption
    of) $z$. In \UAS, solid boxes span configurable statements that control
    issuance; dashed boxes span configurable statements that control utility;
    dotted boxes span configurable statements that control accountability.
    Unboxed statements are fixed, and are thus common to all \GSAC/\UAS
    schemes.}
  \label{fig:proof-blocks-uas}
\end{figure}

Furthermore, to make the concept more flexible, we no longer restrict to the
case of one issuer and one opener. Instead, \UAS supports multiple issuers (as
in generalized variants of anonymous credentials), as well as multiple openers,
each one with a predefined open function. 
A consequence of this is that the concept of ``group'', which was very present
in \GSAC, is now blurred in \UAS, and we get closer to the context of anonymous
credentials -- while still offering the accountability we inherit from \GSAC
via \Open and \Judge.
%
Although this may seem an innocent change, it brings much flexibility in
practice.

\subsection{Model for \UAS Schemes}
\label{ssec:model-uas}

We now define our model for Universal Anonymous Signatures. Property-wise, we
maintain anonymity, non-frameability, and include two extra unforgeability
properties: unforgeability of issuance, which ensures that issuance policies
are not circumvented; and unforgeability of signatures, guaranteeing that
signature evaluation policies are respected, as well as soundness of opening.

From a functional perspective, we require that each issuer and opener fix
the issuance predicate \fissue and opener function \finsp when generating
their public keys. On the other hand, signature evaluation functions \feval can
be defined at signing time. This is in line with the usual practice in anonymous
credentials, that let
users prove arbitrary claims on their credentials, as long as they are met by
the contained attributes. In practice, most probably, the signature evaluation
policy to be
employed in each signature will be defined by the verifier (or jointly between
user and verifier), including the specification of which opener will be
able to post-process signatures\footnote{See, for instance, the concept of
  presentation exchanges, or requests, in the context of Verifiable Credentials:
  \url{https://identity.foundation/presentation-exchange/} (last access, May
  5th, 2022).}. Beyond these general policies, which already give much
flexibility, the syntax for \UAS supports multiple issuers and openers, and
lets users leverage multiple credentials to request issuance of new ones, and
to produce signatures.

\subsubsection{Syntax.} An \UAS scheme is composed by the following algorithms,
where for readability we assume that issuer public keys (\ipk) are extractable
from credentials:

\begin{description}
\item[$\parm \gets \Setup(\secpar)$.] Given a security parameter \secpar,
  returns a global system parameter variable \parm. We assume that \parm are
  available to all the other functions, even if not explicitly listed in their
  input parameters.
\item[$(\ipk,\isk) \gets \IKeyGen(\parm,\fissue)$.] Given global system
  parameters \parm, and the function \fissue to be used to check that credential
  requestors meet the conditions to be issued a credential, an issuer runs
  \IKeyGen to generate its issuing key pair. 
\item[$(\opk,\osk) \gets \OKeyGen(\parm,\finsp)$.] Given global system
  parameters \parm, and function \finsp, an opener runs \OKeyGen to generate
  its opening key pair. The function \finsp defines the type of utility that
  will be extractable from signatures.
\item[$\usk \gets \UKeyGen(\parm)$.] Given global system parameters
  \parm, returns a user's secret key.
\item[$\langle \cred/\bot,\utrans/\bot \rangle \gets
  \langle
  \Obtain(\usk,\scred,\attrs),
  \Issue(\isk,\sipk,\attrs))
  \rangle$.] %
  This interactive protocol lets a user with key \usk running the
  \Obtain process, receive a credential \cred from an issuer in the system, on
  attribute set $\attrs$. The user leverages a set of credentials \scred,
  each with a matching issuer key in \sipk.
  The user outputs the produced credential \cred, while
  the issuer outputs the protocol transcript \utrans for the produced
  credential.
\item[$(\sig,\yeval) \gets \Sign(\usk,\opk,\scred,\msg,\feval)$.] %
  Upon receiving a user secret key \usk, opener public key \opk, a set of
  credentials \scred, a message \msg and evaluation
  function \feval, returns signature \sig and value \yeval.
\item[$1/0 \gets \Verify(\opk, \sipk,\sig,\yeval,\msg,\feval)$.]
  Checks whether $(\sig,\yeval)$ is a valid signature
  over message \msg, from a user with credentials issued by issuers with public
  keys in \sipk, for evaluation function \feval and opener key \opk.
\item[$(\yinsp,\iproof)/\bot \gets \Open(\osk,\sipk,\sig,\yeval,\msg,\feval)$.]
  Executed by the opener with private key \osk. Receives a signature $(\sig,
  \yeval)$ over message \msg and evaluation function \feval, generated using
  credentials issued by the issuers with public keys in \sipk. If \sig is valid,
  the function outputs a value $\yinsp$.
\item[$1/0 \gets \Judge(\opk,\sipk,\yinsp,\iproof,\sig,\yeval,\msg,\feval)$.] %
  Checks if \iproof is a valid opening correctness proof for the value \yinsp,
  obtained by applying \Open to the the signature $(\sig,\yeval)$ over
  message \msg, and for evaluation function \feval. 
\end{description}

\paragraph{Issuance, evaluation, and opening functions.} %
We emphasize that, both in our syntax definition, as well as on the following
modelling, we make use of three different and abstract functions: \fissue,
\feval and \finsp. The three functions are introduced to allow customized
governance of the resulting instantiation of an \UAS scheme. They will be
defined by different parties, but in all cases, they are run by users (maybe,
on user-private data). Also, in all cases, the user has to prove correctness of
their computation. We introduce them next, and give concrete examples for
building specific well-known restrictions of \UAS in
\secref{sec:transformations}.

\begin{description}
\item[$\fissue: (\usk,\scred,\attrs)
  \rightarrow 0/1$.] Chosen by issuers within a family of functions \famfissue,
  the issuance function defines what customized conditions an issuer requires
  in order to issue credentials, when receiving a request from user with secret
  key \usk, for attributes \attrs. \fissue may run checks on a (possibly empty)
  set of additional credentials \scred, all bound to \usk, and possibly
  issued by other issuers. \fissue returns $1$ to accept a request, $0$ to
  reject it.
\item[$\feval: (\usk,\scred,\msg)
  \rightarrow \yeval$.] Signing evaluation functions, from a family of functions
  \famfeval, can be set on a per-signature basis. They receive the user secret
  key \usk, credentials \scred, and message to be signed \msg. \feval can be
  used to control the information related to the signer that will be revealed
  alongside a signature, and to modulate the behaviour of \finsp. Its outputs
  \yeval must belong in a well defined set \rngfeval.
\item[$\finsp: (\yeval,\usk,\scred,\msg) \rightarrow \yinsp$.]
  Chosen by openers from a family of functions \famfinsp. The opening
  functions define what utility value, derived from the user's secret key,
  credentials, and signed message, should be extractable by an opener.
  Note that \finsp also receives as input a value in the range of the \feval
  function, \rngfeval. This allows opening logic to depend on the value
  produced by the evaluation function. The output of \finsp is a value
  \yinsp, which must belong in a well defined set \rngfinsp.
\end{description}

We emphasize that, even though \finsp and \feval seem redundant, they are not.
To see this, observe that \finsp is defined by openers, and will be fixed in
all signatures that can be opened by the opener who defined it. On the other
hand, \feval can be defined by (e.g.) verifiers on a per-signature basis, even
if the signatures use the same \finsp. Thus, \feval can be programmed to contain
the conditions set by (e.g.) verifiers for signatures they receive; whereas
\finsp can be programmed to extract specific utility values when needed, which
may depend on the checks required by the verifier a signature was intended to.

\paragraph{Helper functions \ExtractIssue, \ExtractSign, \IdentifyCred, and
  \IdentifyUK.} In our modelling, we assume the existence of fhese four
functions. They are not
functions available in the actual scheme, but rather to the challenger in the
experiments we use to formalize security of \UAS schemes. Similar techniques
have been used before to prove security in privacy-preserving schemes with some
sort of accountability, but that do not offer conventional opening as vanilla
group signatures. For instance, see related works on DAA \cite{bfg+11,cdl16} and
group signature variants \cite{dl21,fgl21,gl19,lnpy21}. More concretely, these
functions are as follows:

\begin{description}
\item[$\ExtractIssue(\utrans) \rightarrow (\usk,\attrs_{\scred},\scred)$.]
  Receives an $\langle \Obtain, \Issue \rangle$ transcript, and returns the
  credentials (if any) and their attributes that were involved
  in the request. It clearly needs an honest issuer as, otherwise, the
  transcripts won't be available. Consequently, we only use it to define the
  properties that require an honest issuer.
\item[$\ExtractSign(\oid,\siid,\sig,\yeval,\msg,\feval) \rightarrow (\usk,
  \scred,\attrs_{\scred},\yinsp)$.] Receives a signature pair $(\sig,\yeval)$,
  as well as the opener identifier \oid, and the identifiers of all issuers of
  the credentials used to produce the signature over \msg, and for \feval. It
  outputs the user secret key and credentials (with their attributes) used to
  generate the signature, and the value returned by the opening fuction.
\item[$\IdentifyCred(\usk,\attrs_{\cred},\cred)$.] Returns $1$ if \cred has been
  issued over attributes $\attrs_{\cred}$ and for a user with secret key \usk.
  Otherwise, it returns $0$. In order to be meaningful, this requires that,
  for every $(\attrs_{\cred},\cred)$ pair, there is at most one \usk that makes
  \IdentifyCred return $1$.
\item[$\IdentifyUK(\uid,\usk)$.] Returns $1$ is \usk is {\uid}'s secret key.
  This is trivial for honest users, for which there must be a one-to-one
  relationship between {\uid}s and {\usk}s. For corrupt users, \IdentifyUK has
  to iterate through the $\langle\Obtain,\Issue\rangle$ transcripts associated
  to \uid, extract the used secret key, and check if there is a match. Note that
  this does not guarantee that there will be only one \usk per corrupt \uid,
  though. Also, when used for corrupt users, this can only be used when the
  issuer is honest, as transcripts are needed. 
\end{description}

Note that, for all the helper functions, in the case of credentials, transcripts,
and signatures by honest users, it is enough to have access to the corresponding
state information (described below) maintained by the challenger in our
experiments. For credentials and join transcripts of corrupt users, or
dishonestly produced signatures, we do need to perform actual extraction.
Certainly, the challenger
needs special knowledge/power such as decryption trapdoors, the ability to
rewind the game, or program random oracles. The approach needs thus to depend on
the concrete construction. \jdv{Although, for the case of \ExtractIssue and
  \IdentifyUK, online extractability (or alternative requirements, such as
  non-parallel or logarithmic number of joins) is necessary.}

\paragraph{Global Variables.} %
The environment manages several global variables in the games posed to the
adversary. Users are referred to with user identifiers, \uid; for credentials,
we use \cid; for issuers, \iid; and for openers, \oid. In all cases, we use bold
font to denote sets: e.g., \scid and \siid denote sets of credential and issuer
identifiers. All tables/sets are initialized as empty tables/sets.

\begin{description}
\item[Sets for parties]:
  \begin{description}
  \item[\HU and \CU.] Keep track of honest (\HU) and corrupted (\CU) users;
    i.e., they are sets of {\uid}s.
  \item[\HI and \CI.] Keep track of honest (\HI) and corrupted (\CI) issuers;
    i.e., they are sets of {\iid}s.
  \item[\HO and \CO.] Keep track of honest (\HO) and corrupted (\CO) openers;
    i.e., they are sets of {\oid}s.
  \end{description}
\item[Tables for keys]:
  \begin{description}
  \item[\UK.] \UK maintains user keys $\usk$. To refer to the key of a specific
    user, we use $\UK[\uid]$. 
  \item[\IK, \PUBIK and \PRVIK.] \IK maintains issuer key pairs, where
    $\IK[\iid]$ refers to the key pair of the issuer with identifier \iid. We
    use \PUBIK to refer to the public component, which also includes the \fissue
    function; and \PRVIK refers to the private component of the key pair.
  \item[\OK, \PUBOK, \PRVOK.] Same as \IK, but for opener key pairs. Instead
    of \fissue, \OK includes the \finsp function.
  \end{description}
\item[Tables for credentials-related data]:
  \begin{description}
  \item[\CRED.] Stores information related to credentials obtained by users in
    the system. Thus, it is indexable by \cid. More specifically, it stores
    tuples of the form $(\uid,\cred,\iid,\attrs,\scid)$, where \uid is the
    identity of the owner of the credential, \cred (when available) is the
    credential itself, \iid is the identifier of the credential issuer, \attrs
    are the attributes included in \cred, and \scid are the identifiers of the
    credentials (if any) that \uid used to request \cred. For notational
    convenience, we may use $\CRED[\scid]$ to refer to $\CRED[\cid]$ for all
    $\cid \in \scid$. Also, when clear from context, we sometimes use
    $\CRED[\cid]$ (resp. $\CRED[\scid]$ to mean \cred (resp. \scred) in
    $\CRED[\cid] = (\cdot,\cred,\cdot,\cdot,\cdot)$ (resp. $\CRED[\scid]$).
  \item[\OWNR.] For notational convenience, when we write $\OWNR[\cid]$ we mean
    ``\uid such that $\CRED[\cid] = (\uid, \cdot, \cdot, \cdot, \cdot)$''.
  \item[\ATTR.] For notational convenience, when we write $\ATTR[\cid]$ we mean
    ``\attrs such that $\CRED[\cid] = (\cdot, \cdot, \cdot, \attrs, \cdot)$''.
  \item[\ISR.] For notational convenience, when we write $\ISR[\cid]$ we mean
    ``\iid such that $\CRED[\cid] = (\cdot, \iid, \cdot, \cdot, \cdot)$''.
  \end{description}
\item[Tables for signatures]:
  \begin{description}
  \item[\SIG.] Maintains signatures generated via the \SIGN oracle, on behalf
    of honest users. Entries of this table are $(\oid,\scid,\sig,\yeval,\msg,
    \feval)$, where \oid is the opener chosen for the signature, \scid is the
    set of credentials used for signing, \feval is the signing evaluation
    function, and \sig and \msg are the produced signature and signed message.
  \item[\CSIG.] Maintains challenge signatures output to the adversary; i.e.,
    the table is indexable by challenge signatures \csig.
    Each entry contains also $\cuid_b$ and $\scid_b$ the challenge user and
    credential identifiers set used to produce \csig; as well as the
    corresponding challenge user and credential set indexed by the complementary
    $1-b$; the signed message \msg and signing evaluation function \feval, the
    result of \feval, \yeval, and the opener identifier \oid.
  \end{description}
\end{description}

\paragraph{Oracles.} %
In the game-based definitions of our \UAS model, we leverage the following
oracles, which are formally defined in \figref{fig:oracles1} and
\figref{fig:oracles2}. 

\begin{description}
\item[\IGEN.] Adds a new issuer to the game, generating its keypair and setting
  the associated issuance function.
\item[\OGEN.] Adds a new opener to the game, generating its key pair and
  setting the associated evaluation and inspection functions.
\item[\ICORR.] Corrupts an existing (and honest) issuer, by giving its secret
  key to the adversary.
\item[\OCORR.] Corrupts an existing (and honest) opener, by giving its secret
  key to the adversary.  
\item[\HUGEN.] Adds a new honest user to the game, by honestly generating
  the user's key pair.
\item[\CUGEN.] Adds a new corrupt user to the game or, if the specified
  user already exists and is honest, corrupts it, leaking its key and
  credentials.
\item[\RREG.] Reads the given transcript table entry.
\item[\WREG.] Sets a transcript table entry to the given value.
\item[\OBTISS.] Lets the adversary add a new honestly generated credential to
  the game, on behalf of an honest user.
\item[\OBTAIN.] Enables the adversary to play the role of a dishonest issuer
  in games that support it, by letting it interact with honest users who want to
  receive credentials.
\item[\ISSUE.] Allows the adversary to play the role of dishonest users,
  requesting an honest issuer to produce credentials for them.
\item[\SIGN.] Lets the adversary get signatures from credentials belonging
  to honest users.
\item[\OPEN.] Given an honestly produced signature, outputs the result of the
  opening function, along with a correctness proof.
\item[\CHALb.] Upon receiving two challenge users and credential sets, a common
  singing evaluation function and a message, returns a signature produced by one
  of these two user and credential sets, defined by the bit $b$, which is
  established in the anonymity game.
\end{description}

{%\setlength\intextsep{\sep}
  \begin{figure*}[htp!]
    \centering
    \scalebox{0.9}{

      \begin{minipage}[t]{0.55\textwidth}       

        \procedure{$\IGEN(\iid,\fissue)$}{%
          \pcif \iid \in \HI \lor \iid \in \CI: \pcreturn \bot \\
          \pcif \fissue \notin \famfissue: \pcreturn \bot \\
          (\ipk,\isk) \gets \IKeyGen(\parm) \\
          \IK[\iid] \gets ((\ipk,\fissue),\isk) \\
          \HI \gets \HI \cup \lbrace \iid \rbrace \\
          \pcreturn \ipk \\
        }

        \procedure{$\ICORR(\iid)$}{%
          \pcif \iid \in \CI \lor \iid \notin \HI: \pcreturn \bot \\
          \HI \gets \HI \setminus \lbrace \iid \rbrace \\
          \CI \gets \CI \cup \lbrace \gid \rbrace \\
          \pcreturn \isk \\
        }        

        \procedure{$\HUGEN(\uid)$}{%
          \pcif \uid \in \HU \lor \uid \in \CU: \pcreturn \bot \\
          \usk \gets \UKeyGen(\parm) \\
          \UK[\uid] \gets \usk;
          \HU \gets \HU \cup \lbrace  \uid \rbrace \\
          \pcreturn \top \\
        }

        \procedure{$\RREG(i)$}{%
          \pcreturn \trans[i]
        }        
        
      \end{minipage}
    }
    \scalebox{0.9}{
      
      \begin{minipage}[t]{.5\textwidth}

        \procedure{$\OGEN(\oid,\finsp)$}{%
          \pcif \oid \in \HO \lor \oid \in \CO: \pcreturn \bot \\
          \pcif \finsp \notin \famfinsp: \pcreturn \bot \\
          (\opk,\osk) \gets \OKeyGen(\parm) \\
          \OK[\oid] \gets ((\opk,\finsp),\osk) \\
          \HO \gets \HO \cup \lbrace \oid \rbrace \\
          \pcreturn \opk \\
        }

        \procedure{$\OCORR(\oid)$}{%
          \pcif \oid \in \CO \lor \oid \notin \HO: \pcreturn \bot \\
          \HO \gets \HO \setminus \lbrace \oid \rbrace \\
          \CO \gets \CO \cup \lbrace \oid \rbrace \\
          \pcreturn \osk \\
        }        
        
        \procedure{$\CUGEN(\uid)$}{%          
          \pcif \uid \in \CU: \pcreturn \bot \\
          \CU \gets \CU \cup \lbrace \uid \rbrace \\          
          \pcif \uid \in \HU: \\
          \pcind \HU \gets \HU \setminus \lbrace \uid \rbrace; \\
          \pcind \pcreturn (\UK[\uid],\CRED[\uid]) \\
          \pcelse: \UK[\uid] = \bot \\          
          \pcreturn \top \\
        }

        \procedure{$\WREG(i,\rho)$}{%
          \trans[i] \gets \rho
        }        
        
      \end{minipage}
      
    }

    \caption{Detailed oracles available in our model (1/2). Oracles for
      generating key material for users, issuers, and openers.}
    \label{fig:oracles1}
  \end{figure*}
}

{%\setlength\intextsep{\sep}
  \begin{figure*}[htp!]
    \centering
    \scalebox{0.9}{

      \begin{minipage}[t]{0.55\textwidth}

        \procedure{$\ISSUE(\cid,\uid,\iid,\attrs,\siid)$}{%
          \pcif \uid \notin \CU: \pcreturn \bot \\          
          \pcif \iid \notin \HI: \pcreturn \bot \\
          %\pcif \exists \iid' \in \ISR[\scid] \notin \HI: \pcreturn \bot \\
          \pcif \CRED[\cid] \neq \bot: \pcreturn \bot \\
          \langle \cdot, \utrans \rangle \gets
          \langle \adv, 
          \Issue(\PRVIK[\iid],\siid,\attrs) \rangle \\
          \trans[\cid] \gets \utrans \\
          \CRED[\cid] \gets (\uid, \cdot, \iid, \attrs, \cdot, \siid) \\
          \pcreturn \top \\          
        }                

        \procedure{$\OBTAIN(\cid,\uid,\iid,\attrs,\scid)$}{%
          \pcif \uid \notin \HU: \pcreturn \bot \\
          \pcif \iid \notin \CI: \pcreturn \bot \\
          \pcif \CRED[\cid] \neq \bot: \pcreturn \bot \\
          %\pcif \exists \cid' \in \scid~\st~\CRED[\cid'] = \bot: \pcreturn \bot \\
          \langle \cred, \cdot \rangle \gets
          \langle \Obtain(\UK[\uid],\CRED[\scid],\attrs),\adv \rangle \\
          \CRED[\cid] \gets (\uid, \cred, \iid, \attrs, \scid, \siid) \\
          \pcreturn \top \\
        }

        \procedure{$\CHALb(\oid,\cuid_{0,1},\cscid_{0,1},\msg,\feval)$}{%
          \pcif \cuid_0 \notin \HU \lor \cuid_1 \notin \HU: \pcreturn \bot \\
          \pcif \oid \in \CO: \pcreturn \bot \\
          \pcif \yeval = \feval(\UK[\cuid_0],\CRED[\cscid_0],\msg) \neq \\
          \pcind \feval(\UK[\cuid_1],\CRED[\cscid_1],\msg):
          \pcreturn \bot \\
          \pcif \PUBIK[\cscid_0] \neq \PUBIK[\cscid_1]: \pcreturn \bot \\
          (\yeval,\csig_b) \gets \Sign(\UK[\cuid_b],\PUBOK[\oid], \\
          \hspace*{71pt}\CRED[\cscid_b],\msg,\feval) \\
          (\yeval,\csig_{1-b}) \gets \Sign(\UK[\cuid_{1-b}],\PUBOK[\oid], \\
          \hspace*{80pt}\CRED[\cscid_{1-b}],\msg,\feval) \\          
          \CSIG[\csig_b] \gets 
          \lbrace (\oid,\cuid_b,\cscid_b,\msg,\feval,\yeval,\\
          \hspace*{74pt}\cuid_{1-b},\csig_{1-b},\cscid_{1-b})
          \rbrace \\
          \pcreturn (\csig_b,\yeval)
        }        
        
      \end{minipage}
    }
    \scalebox{0.9}{
      
      \begin{minipage}[t]{.5\textwidth}

        \procedure{$\OBTISS(\cid,\uid,\iid,\attrs,\scid)$}{%
          \pcif \uid \notin \HU: \pcreturn \bot \\
          \pcif \iid \notin \HI: \pcreturn \bot \\
          \pcif \CRED[\cid] \neq \bot: \pcreturn \bot \\
          %\pcif \exists \iid' \in \ISR[\scid]~\st~\iid' \notin \HI: \pcreturn \bot \\
          %\pcif \exists \cid' \in \scid~\st~\CRED[\cid'] = \bot:
          \pcreturn \bot \\
          \langle \cred, \utrans \rangle \gets
          \langle \Obtain(\UK[\uid],\CRED[\scid],\attrs), \\
          \hspace*{60pt} \Issue(\PRVIK[\iid],\ISR[\scid],\attrs)
          \rangle \\
          \trans[\cid] \gets \utrans \\
          \CRED[\cid] \gets (\uid, \cred, \iid, \attrs, \scid, \siid) \\
          \pcreturn \top \\
        }

        \procedure{$\SIGN(\oid,\uid,\scid,\msg,\feval)$}{%
          \pcif \uid \notin \HU: \pcreturn \bot \\
          (\sig,\yeval) \gets \Sign(\UK[\uid],\PUBOK[\oid],\CRED[\scid],\msg,
          \feval) \\
          \SIG[\uid] \gets \SIG[\uid] \cup
          \lbrace (\oid,\scid,\sig,\yeval,\msg,\feval) \rbrace \\
          \pcreturn (\sig,\yeval) \\
        }                

        \procedure{$\OPEN(\oid,\sig,\yeval,\msg)$}{%
          \textrm{Let}~\uid~\textrm{be s.t.}~(\oid,\scid,\sig,\yeval,\msg,\feval)
          \in \SIG[\uid] \\
          (\yinsp,\iproof) \gets
          \Open(\PRVOK[\oid],\PUBIK[\scid],\sig,\yeval,\msg,\feval) \\
          \pcif \CSIG[\sig] \neq \bot: \\
          \pcind \textrm{Parse $\CSIG[\sig]$ as $(\oid,\cuid_b,\scid_b,\msg,
            \feval,\yeval$} \\
          \hspace*{83pt}\cuid_{1-b},\csig_{1-b},\scid_{1-b}) \\
          \pcind (\yinsp',\iproof') \gets
          \Open(\PRVOK[\oid],\IK[\siid],\\
          \hspace*{107pt} \sig_{1-b},\msg,\feval) \\
          \pcind \pcif \yinsp' \neq \yinsp: \pcreturn \bot \\
          \pcreturn (\yinsp,\iproof)
        }
        
      \end{minipage}
      
    }

    \caption{Detailed oracles available in our model (2/2). Oracles for
      obtaining credentials, signatures, and processing them.}
    \label{fig:oracles2}
  \end{figure*}
}

\paragraph{Correctness.} %
Correctness of \UAS schemes is formalized through the experiment in
\figref{fig:exp-uas-corr}. It states that a signature over any arbitrary message
and valid function \feval, produced honestly leveraging credential set \scid,
owned by user \uid, is accepted by \Verify. Moreover, all the credentials in
\scid meet the conditions set by the corresponding \fissue defined by the issuer
which issued each credential. Similarly, the output \yeval of \feval matches the
value produced by \Sign alongside with \sig; and the value produced by \Open
is accepted by \Judge, and matches the output of applying \finsp on \yeval, the
credentials, user key, and message.

\begin{definition}{(Correctness of \UAS)}
  \label{def:correctness-uas}
  An \UAS scheme is correct if, for any p.p.t. adversary $\adv$,
  $\ExpCorrect(1^\secpar)$ outputs 1 with negligible probability.
\end{definition}

\begin{figure}[htp!]
  \procedure[linenumbering]{$\ExpCorrect(1^\secpar)$}{%
    \parm \gets \Setup(1^\secpar) \\
    (\uid,\oid,\scid,\msg,\feval)
    \gets \adv^{\IGEN,\OGEN,\HUGEN,\OBTISS,\RREG}(\parm) \\
    \pcif \feval \notin \famfeval: \pcreturn 0 \\
    \pcif \OWNR[\scid] \neq \uid: \pcreturn 0 \\
    (\sig,\yeval) \gets \Sign(\UK[\uid],\PUBOK[\oid],\scid,\msg,\feval) \\
    \pcif \Verify(\PUBOK[\oid],\PUBIK[\scid],\sig,\msg,\feval) = 0: \pcreturn 1 \\
    \pcfor \cid \in \scid \pcdo: \\
    \pcind \textrm{Let}~\scred^{\cid}~\textrm{be the credentials used to obtain}
    ~\cid;~\textrm{Parse}~\PUBIK[\ISR[\cid]]~\textrm{as}~((\cdot,\fissue^{\cid}),\cdot)\\
    \pcind \pcif \fissue^{\cid}(\UK[\uid],\scred^{\cid},\ATTR[\cid]) = 0: \pcreturn 1 \\
    \pcif \feval(\UK[\uid],\CRED[\scid],\msg) \neq \yeval: \pcreturn 1 \\
    (\yinsp,\iproof) \gets \Open(\PRVOK[\gid],\PUBIK[\scid],\sig,\msg,\feval) \\
    \pcif \Judge(\PUBOK[\oid],\PUBIK[\scid],\y,\iproof,\sig,\yeval,\msg,\feval)
    = 0 \lor \yinsp \neq \finsp^\gid(\yeval,\UK[\uid],\CRED[\scid],\msg)): \\
    \pcind \pcreturn 1 \\
    \pcreturn 0
  }  
  \caption{Correctness experiment for \UAS schemes.}
  \label{fig:exp-uas-corr}
\end{figure}

\subsubsection{Security Properties}
\label{sssec:security}

\paragraph{Anonymity.} %
In addition to the considerations made for \GSAC (see \secref{ssec:model-gsac}),
in our notion of anonymity for \UAS,
in order to prevent trivial wins by the adversary, we have to restrict that the
signing evaluation function outputs the same value for both user-credentials
pairs. Very interestingly, since our opening functionality does not
(necessarily) output the identity of the signer, \uline{we can even allow the
  adversary to open challenge signatures, as long as they produce the same
  output}. Other than this, the overall approach is similar: as
in group signatures and anonymous credentials, the adversary picks two pairs of
honest users and credential sets. We also require the adversary to pick an
evaluation function and an opener (hence, an opening function). Then, the
user-credential pair defind by the value of $b$ (unknown to the adversary) is
used to program the \CHALb oracle. The formal specification of the anonymity
game is given in \figref{fig:exp-uas-anonb}, where $\Oanonc \gets (\lbrace\HU,
\CU\rbrace\GEN,\lbrace\II,\OO\rbrace\GEN,\lbrace\II,\OO\rbrace\CORR,\OBTAIN,
\WREG,\SIGN,\OPEN)$ and $\Oanong \gets (\lbrace\HU,\CU\rbrace\GEN,\lbrace\II,
\OO\rbrace\GEN,\lbrace\II,\OO\rbrace\CORR,\OBTAIN,\WREG,\SIGN,\OPEN,\CHALb)$

\begin{figure}[htp!]
  \procedure[linenumbering]{$\ExpAnonb(1^\secpar)$}{%
     \parm \gets \Setup(1^\secpar) \\
     (\cuid_0,\cscid_0,\cuid_1,\cscid_1,\feval,\status) \gets \adv^{\Oanonc}
     (\choose,\parm) \\
     \pcif \feval \notin \famfeval: \pcreturn \bot \\
     b^* \gets \adv^{\Oanong} (\guess,\status) \\
     \pcreturn b^*
  }
  \caption{Anonymity experiment for \UAS schemes.}
  \label{fig:exp-uas-anonb}
\end{figure}

\begin{definition}{(Anonymity of \UAS)}
  \label{def:anonymity-uas}  
  We define the advantage \AdvAnon of $\adv$ against \ExpAnonb as
  $\AdvAnon=|\Pr\lbrack\ExpAnono(1^\secpar)=1\rbrack-
  \Pr\lbrack\ExpAnonz(1^\secpar)=1\rbrack|$.
  %
  An \UAS scheme satisfies anonymity if, for any p.p.t. adversary $\adv$,
  \AdvAnon is a negligible function of $1^\secpar$.
\end{definition}

\paragraph{Discussion on the generality of anonymity in \UAS schemes.} %
This notion of anonymity is more general than that of group signatures in the
sense that calls to the \OPEN oracle reveal an arbitrary function of the
identity of the user -- which can certainly be the member index itself, as it
is frequent in group signatures, or any other function computable from the
user key, its credentials, and signed message. Moreover, note that
our notion is even stronger than the conventional ``CCA-like'' anonymity notion
that gives the adversary access to the open oracle, only restricting it when
trying to open challenge signature. Precisely the introduction of generic
opening functions allows us to let the adversary open challenge signatures
$\csig_b$ as long as the counterpart $\csig_{1-b}$
makes \Open produce the same \y value as output. This is certainly not
possible when using conventional opening, since it directly outputs the identity
of the signer -- which cannot be the same for different challenge users.

Note that we could actually remove the requirement that the evaluation function
has to produce the same output on both challenge user-credentials pairs.
However, by doing so, we would force all constructions to maintain private the
output of \feval. We opt not to require that, though, as it directly allows
our model to cover interesting use cases -- such as restricting to selective
disclosure, or privacy-preserving variants of functional signatures \jdv{check
  the latter}, although we show how to do that in \secref{sec:transformations}.

\paragraph{Unforgeability.} In anonymous credentials, unforgeability requires
that no adversary can succeed in a credential presentation for attributes that
are not contained in a legitimately issued credential (set) it controls. Note
that, for this, assuming honest issuers is essential as, otherwise, the
adversary can get credentials on any attribute set it wants.
%
The somehow equivalent property in group signatures, inasmuch it also requires
honest issuers, is traceability. It captures the security of the
open-related functionality (e.g., \Open and \Judge) over signatures
that produced by potentially malicious signers, in the presence of an honest
issuer. In a nutshell, it ensures that every signature accepted by
the verification algorithm must have been created by a user who joined the
group, and that the result of \Open (and thus, \Judge) over such a valid
signature is consistent with its signer. With conventional opening, this is
essentially checked by requesting the adversary to produce a signature,
obtaining -- via open -- the ``identifier'' of the user who produced the
signature. Schemes
that do not have conventional opening resort to more subtle techniques, like
matching keys extracted during join transcripts, with keys used for signing
(see, e.g., \cite{dl21}). Thus, even though for more subtle reasons (the need
for reliable bookkeeping during joins) than for anonymous credentials, honest
issuers are required for traceability of group signatures too.

In our \UAS scheme we allow users prove claims over the attributes they
own -- attested via obtained credentials -- through the \feval function.
Roughly, we capture this as in unforgeability requirement of anonymous
credentials. Namely, we check that the \yeval value returned by \Sign along with
the signature, matches the expected one from the specified \feval function.
On the other hand, the opening capabilities of \UAS schemes also call for a
traceability-like property. However, note that, as opposed to group signatures,
$\langle \Obtain,\Issue \rangle$ protocols are over credentials rather than
users -- although, ultimately, credentials must be owned by some user.
Furthermore, since \Open
does not return the actual identity of the signer, but a function \finsp of it
(and other arguments), we need to make sure that the output of \Open
matches the output of \finsp. As in group signatures with non-conventional open,
we resort to extraction-based techniques. All this is captured via \ExpForgeSign
in \figref{fig:exp-uas-unfor-issue}. Therein, the adversary is challenged to
produce a signature $(\sig,\yeval)$ over message \msg and for evaluation
function \feval; as well as the identifiers for the opener (\oid) and
credential issuers (\siid) used to compute the signature. As in traditional
group signatures, the adversary wins if the signature is accepted by \Verify,
yet \Open or \Judge fail. Then, the game extracts the user key and
credentials used to produce the signature. From it, the game checks if the
output of \feval matches the \yeval value produced by the adversary -- this
mimics the behaviour of unforgeability in anonymous credentials. Finally, the
game also checks that the output of \Open (even if accepted by \Judge)
matches the output of \finsp. If there is any mismatch in the last two checks,
or some of the credentials used to produce the signature do not correspond to
the secret key that was allegedly used to request them, the adversary wins the
game.

Our \UAS scheme includes yet another generalisation that requires unforgeability-like
security, though. Concretely, the issuance function \fissue. We need to make
sure that no credential is issued unless its corresponding
request meets the defined issuance policy. From the point of view of group
signatures, this may seem redundant. After all, if all valid signatures produce
consistent evaluation and opening results, given the employed user key,
credentials, and signed message, the notion of traceability seems to be
satisfied. However, from the point of view of anonymous credentials, an attacker
being able to obtain a credential, even when it does not meet the required
conditions to have it issued, is clearly problematic: it would allow to prove
claims over attributes it does not really ``own'' (even if the output of \Sign
and \Open are consistent with the values returned by \feval and \finsp, and
the credential was obtained via an $\langle \Obtain,\Issue \rangle$ run). To
capture this, we define the \ExpForgeIssue experiment. In the experiment, the
adversary is challenged to produce a credential identifier that must be
associated to an existing $\langle \Obtain,\Issue \rangle$ interaction -- thus,
the corresponding \trans entry must exist (which we can check, as the issuer is
honest). The adversary wins if, either the extraction process fails, or the
extracted user secret key and credentials make the corresponding issuance
function fail.

For both \ExpForgeIssue and \ExpForgeSign, the adversary is given access to the
oracle set $\Oforgeissue = \Oforgesign \gets \lbrace\HU,\CU\rbrace\GEN,\IGEN,
\OGEN,\OCORR,\OBTISS,\ISSUE,\RREG,\SIGN,\OPEN$.

\begin{figure}[htp!]
    \procedure[linenumbering]{$\ExpForgeIssue(1^\secpar)$}{%
      \parm \gets \Setup(1^\secpar) \\
      \cid \gets \adv^{\Oforgeissue}(\parm) \\
      \pcif \trans[\cid] = \bot \lor \CRED[\cid] = \bot: \pcreturn 0 \\
      \textrm{Parse}~\CRED[\cid]~\textrm{as}~(\cdot,\cdot,\iid,\cdot,\cdot);~
      \IK[\iid]~\textrm{as}~((\ipk,\fissue),\cdot) \\
      (\usk,\scred,\attrs_{\scred}) \gets \ExtractIssue(\trans[\cid]) \\
      \pcif \fissue(\usk,\scred,\ATTR[\cid]) = 0 \lor
      \exists \cred \in \scred~\st~\IdentifyCred(\usk,\attrs_{\cred},\cred) = 0: \\
      \pcind \pcreturn 1 \\
%      \pcif \nexists \uid~\st~\IdentifyUK(\uid,\usk) = 1: \pcreturn 1 \\
      \pcreturn 0
    }
  \caption{Experiment for unforgeability of credential issuance in \UAS schemes.}
  \label{fig:exp-uas-unfor-issue}
\end{figure}    

\begin{figure}[htp!]
    \procedure[linenumbering]{$\ExpForgeSign(1^\secpar)$}{%
      \parm \gets \Setup(1^\secpar) \\
      (\oid,\siid,\sig,\yeval,\msg,\feval) \gets \adv^{\Oforgesign}(\parm) \\
      \pcif \exists \uid~\st~(\cdot,\cdot,\sig,\yeval,\msg,\feval) \in
      \SIG[\uid]: \pcreturn 0 \\
      \pcif \Verify(\PUBOK[\oid],\PUBIK[\siid],\sig,\yeval,\msg,\feval) = 0:
      \pcreturn 0 \\
      (\yinsp,\iproof) \gets \Open(\PRVOK[\oid],\siid,\sig,\yeval,\msg) \\
      \pcif \Judge(\PUBOK[\oid],\PUBIK[\siid],\yinsp,\iproof,\sig,\yeval,\msg,\feval)
      = 0: \pcreturn 1 \\
      (\usk,\scred,\attrs_{\scred},\yinsp',r) \gets \ExtractSign(\oid,\siid,\sig,
      \yeval,\msg,\feval) \\
      \pcif \feval(\usk,\scred,\msg) \neq \yeval: \pcreturn 1 \\
      \pcif \finsp(\yeval,\usk,\scred,\msg) \neq \yinsp \lor \yinsp \neq \yinsp':
      \pcreturn 1 \\
      \pcif \exists \cred \in \scred~\st~\IdentifyCred(\usk,\attrs_{\cred},\cred) = 0:
      \pcreturn 1 \\
      \pcif \nexists \uid~\st~\IdentifyUK(\uid,\usk) = 1: \pcreturn 1 \\
      \pcreturn 0
    }
  \caption{Experiment for unforgeability of signatures in \UAS schemes.}
  \label{fig:exp-uas-unfor-sign}
\end{figure}

\begin{definition}{(Unforgeable issuance of \UAS)}
  \label{def:issue-forge-uas}  
  We define the advantage \AdvForgeIssue of $\adv$ against \ExpForgeIssue as
  $\AdvForgeIssue=\Pr\lbrack\ExpForgeIssue(1^\secpar)=1\rbrack$.
  %
  A \UAS scheme has unforgeable issuance if, for any p.p.t. adversary $\adv$,
  \AdvForgeIssue is a negligible function of $1^\secpar$.
\end{definition}

\begin{definition}{(Unforgeable signing of \UAS)}
  \label{def:sign-forge-uas}  
  We define the advantage \AdvForgeSign of $\adv$ against \ExpForgeSign as
  $\AdvForgeSign=\Pr\lbrack\ExpForgeSign(1^\secpar)=1\rbrack$.
  %
  A \UAS scheme has unforgeable signing if, for any p.p.t. adversary $\adv$,
  \AdvForgeSign is a negligible function of $1^\secpar$.
\end{definition}

For short, we say that an \UAS scheme that has both unforgeable issuance and
signing, is an unforgeable \UAS scheme.

\paragraph{Discussion on the generality of unforgeability in \UAS schemes.} %
The notion of signature unforgeability we present for \UAS is strictly more
general than the corresponding one of traceability for group signatures. This is
again a direct consequence of the fact that \Open can return an arbitrary
function of the signer's key and credentials (and signed message), which is a
strict generalization of the conventional \Open. Although, even in that case, we
need to take into account attributes, and the fact that the same user may obtain
multiple credentials (that is why, even when having \Open return the identity of
the signer, our notion is not exactly the same). In this sense, the sign
unforgeability notion for \UAS is equivalent to that of anonymous
credentials. It would seem, though, that we do not need the traceability part of
group signatures; after all, it is the protection against wrong claims on
attributes what enables meaningful and flexible authentication. However, the
type of protection against misuses of the open functionality that we can
get with an honest issuer (as in traceability) is much higher than without an
honest issuer (as in non-frameability). Specifically, with an honest issuer we
can ensure that the adversary cannot even alter the value returned by \Open on
signatures by corrupt users, nor the output of the signing predicate \feval.
Whereas, with a corrupt issuer, all we can ensure is that
the adversary cannot forge a signature from an honest user for which \Open
returns the same value as a signature by that honest user would produce; and,
certainly, a corrupt issuer can arbitrarily issue credentials meeting any
desired predicate \feval. Signing unforgeability is, therefore, a core property
to ensure accountability.

Similarly, by adding the related notion of issuance unforgeability, we ensure
that no credential can be issued that did not meet the corresponding issuance
policy. This is again something not necessary in group signatures (with
verifiable openings), where \Judge accepting the opening proof implies that
there is a valid join transcript associated to the membership credential used
to produce the signature. However, in \UAS, given that the credentials contain
attributes, even though that transcript exists, we need to make sure that the
issuance policy over these attributes was satisfied -- and this is not something
(easily) extractable from the signature.

\paragraph{Non-frameability.} %
The notion of non-frameability in \UAS schemes is unavoidably more subtle than
in group signatures. To see this, we note that, by allowing arbitrary
evaluation and open functions to be used, it can be perfectly valid to
have a signature produced by a corrupted user output the same \yeval or \yinsp
values than the ones output when evaluating or opening a signature by an honest
user. As a concrete example, imagine an open function that returns the
nationality of the signer. In any country, there will be many users (corrupt or
not) sharing nationality.
%
More generally, since the issuer is dishonest in non-frameability properties
and, in \UAS, the value produced by \Open may depend on the attributes
included in user credentials, the adversary may even be able to just produce
``legitimate'' openings that output any desired value.

Thus, we again need to resort to extraction techniques. In the non-frameability
definition for \UAS, given in experiment \ExpNonframe in
\figref{fig:exp-uas-frame}, the adversary is challenged to produce a signature
and opening proof that is accepted by \Verify and \Judge, respectively. From
the signature, we then extract the secret key of the signer, and match it
against the secret keys of the honest users. The adversary wins if there is a
match and the signature has not been queried to \SIGN, or if the value \yinsp
output by the adversary is accepted by \Judge, yet it is different from the
extracted $\yinsp'$ value. In the game, the
adversary has access to the oracles in $\Oframe \gets \lbrace\HU,\CU\rbrace\GEN,
\lbrace\II,\OO\rbrace\GEN,\lbrace\II,\OO\rbrace\CORR,\WREG,\OBTAIN,\SIGN$.

\begin{figure}[htp!]
  \procedure[linenumbering]{$\ExpNonframe(1^\secpar)$}{%
    \parm \gets \Setup(1^\secpar) \\
    (\oid,\siid,\sig,\yeval,\msg,\feval,\yinsp,\iproof) \gets
    \adv^{\Oframe}(\parm) \\
%    \pcif \exists \uid~\st~(\cdot,\cdot,\sig,\yeval,\msg,\feval) \in \SIG[\uid]:
%    \pcreturn 0 \\
    \pcif \Verify(\PUBOK[\oid],\PUBIK[\siid],\sig,\yeval,\msg,\feval) = 0:
    \pcreturn 0 \\
    \pcif \Judge(\PUBOK[\oid],\PUBIK[\siid],\yinsp,\iproof,\sig,\yeval,\msg) = 0:
    \pcreturn 0 \\
    (\usk,\scred,\attrs_{\scred},\yinsp',r) \gets
    \ExtractSign(\oid,\siid,\sig,\yeval,\msg,\feval) \\
    \pcif \exists \uid \in \HU~\st~\UK[\uid] = \usk~\land \\
    \pcind (\nexists (\cdot,\cdot,\sig,\yeval,\msg,\feval) \in \SIG[\uid]
    \lor \yinsp \neq \yinsp'): \pcreturn 1 \\
    \pcreturn 0
  }
  \caption{Experiment for non-frameability on \UAS schemes.}
  \label{fig:exp-uas-frame}
\end{figure}

\begin{definition}{(Non-frameability of \UAS)}
  \label{def:frame-uas}
  We define the advantage \AdvNonframe of $\adv$ against \ExpNonframe as
  $\AdvNonframe=\Pr\lbrack\ExpNonframe(1^\secpar)=1\rbrack$.
  %
  A \GSAC scheme satisfies non-frameability if, for any p.p.t. adversary $\adv$,
  \AdvNonframe is a negligible function of $1^\secpar$.
\end{definition}

\paragraph{Discussion on the generality of non-frameability in \UAS schemes.} %
Anonymous credentials do not have non-frameability property and, thus, it is
hard to make a comparison. However, we can draw some connections with AC schemes
that support revocation, as revocation is somehow equivalent to linking, which
is a type of inspection available in group signatures. In this sense, note that
basic revocation (without straight deanonymization) can be trivially achieved
through our generic \Open function. For instance, one could set \finsp to
be a pseudorandom number seeded with the user's public key (or credential). In
this sense, \Open could be essentially seen as a Verifiable Random Function.
If we compare with group signatures, our notion is also more general than the
conventional one, again for the same reason as sign unforgeability (i.e., \Open
can return any value, not just the signer's identity). Thus, the need to extract
the signer's data in order to detect if a framing has taken place.
  
%%% Local Variables:
%%% mode: latex
%%% TeX-master: "uas"
%%% End:

\section{\CUASGen: A Generic \UAS Construction}
\label{sec:gen-construction}

In this section, we give a generic construction of an \UAS scheme, based on
generic building blocks. In \secref{sec:instantiation}, we give a concrete
instantiation.

\subsection{Building Blocks}
\label{ssec:bblocks}

\paragraph{Vector Commitment schemes.} %
Defined as a tuple $(\CSetup,\CCommit)$. Algorithm $\Cparm \gets
\CSetup(\Csecpar)$ produces the parameters for committing to values. $\Ccom
\gets \CCommit(\Cparm, \msgset; r)$ produces a commitment \Ccom over a set of
messages \msgset, from which we may omit randomness $r$. \todo{Informally define
  hiding and binding, leaving formal definitions to the appendix.}

\paragraph{Public-Key Encryption.} %
Defined as a tuple $(\ESetup,\EKeyGen,\EEnc,\EDec)$. Algorithm $\Eparm \gets
\ESetup(\Esecpar)$ produces public parameters for the other algorithms.
$(\Eek,\Edk) \gets \EKeyGen(\Eparm)$ generates the encryption-decryption key
pair, algorithm $\Ec \gets \EEnc(\Eek,\msg)$ encrypts message \msg with
encryption key \Eek, producing ciphertext \Ec, and $\msg/\bot \gets \EDec(\Edk,
\Ec)$ decrypts ciphertexts using decryption key \Edk. \todo{Informally define
  security properties we'll need.}

\paragraph{Digital Signatures.} %
Defined as a tuple $(\SSetup,\SKeyGen,\SSign,\SVerify)$. Algorithm $\Sparm \gets
\SSetup(\Ssecpar)$ produces public parameters for the other algorithms.
$(\Svk,\Ssk) \gets \SKeyGen(\Sparm)$ generates the verification-signing key
pair, algorithm $\Ssig \gets \SSign(\Ssk,\msg)$ signs message \msg with
signing key \Ssk, producing signature \Ssig, and $1/0 \gets \SVerify(\Svk,
\Ssig,\msg)$ checks whether \Ssig is a valid signature over \msg, under
verification key \Svk. \todo{Informally define security properties we'll need.}

\paragraph{Non-Interactive Zero-Knowledge.} %
We use non-interactive zero-knowledge proofs of knowledge (NIZK), in the CRS
model \needcite. Informally, a NIZK scheme over an NP relation \NIZKRel is
defined as a tuple $(\NIZKSetup^\NIZKRel,\NIZKProve^\NIZKRel,
\NIZKVerify^\NIZKRel)$. Algorithm $\NIZKcrs \gets \NIZKSetup^\NIZKRel
(\NIZKsecpar)$ produces the common reference string \NIZKcrs. $\NIZKproof/\bot
\gets \NIZKProve^\NIZKRel(\NIZKcrs,\NIZKw,\NIZKx)$ creates a NIZK proof of
knowledge of witness \NIZKw for \NIZKx such that $(\NIZKw,\NIZKx) \in \NIZKRel$.
$1/0 \gets \NIZKVerify^\NIZKRel(\NIZKcrs,\NIZKx,\NIZKproof)$ verifies the proof.
\todo{Informally define security properties we'll need.}

\paragraph{Signatures over Blocks of Committed Messages, with proofs.} %
We use signature schemes that allow signing messages, or commitments to messages,
in blocks, and are compatible with (efficient) proof systems over the produced
signature and signed (commitments to) messages. For this purpose, we define such
schemes as a tuple $(\SBCMSetup,\SBCMKeyGen,\SBCMSign,\SBCMVerify)$. Algorithm
$\SBCMparm \gets \SBCMSetup(\SBCMsecpar)$ produces the public parameters for the
scheme. $(\SBCMvk,\SBCMsk) \gets \SBCMKeyGen(\SBCMparm)$ produces a
verification-signing key
pair. Algorithm $\SBCMsig \gets \SBCMSign(\SBCMsk,\Ccom,\msgset)$ produces a
signature over a set of committed messages \Ccom and a set of messages
\msgset, where either \Ccom or \msgset may be empty. $1/0 \gets
\SBCMVerify(\SBCMvk,\SBCMsig,\overline{\msgset})$ verifies a signature \SBCMsig
over message set $\overline{\msgset}$, which must contain both the messages that
were signed as commitments as well as those signed in ``the clear''. In
addition, the produced signatures must be compatible with (efficient) NIZK
proofs of knowledge of a signature, and of (arbitrary) claims over the signed
(committed) messages.
\todo{Informally define security properties we'll need.}

\subsection{Generic Construction \CUASGen}
\label{ssec:generic-construction-uas}

We use three different NP relations in our generic construction. Namely,
$\NIZKRel_{\Issue}$ for issuance, $\NIZKRel_{\Sign}$ for signing, and
$\NIZKRel_{\Inspect}$ for inspection. We will be defining them in the
corresponding protocol/algorithm.

\todo{Many variables need renaming here.}

\paragraph{$\parm \gets \Setup(\secpar,\AttrSpace)$.} %
The setup process essentially consists on generating the public parameters
for all the building blocks. In detail, it parses \secpar as $(\Csecpar,
\NIZKsecpar,\SBCMsecpar,\Esecpar)$. Then, run $\Cparm \gets
\CSetup(\Csecpar)$, $\SBCMparm \gets  \SBCMSetup(\SBCMsecpar)$, $\Sparm \gets
\SSetup(\Ssecpar)$, $\Eparm \gets \ESetup(\Esecpar)$, $\NIZKcrs_{\Issue} \gets
\NIZKSetup^{\NIZKRel_{\Issue}}(\NIZKsecpar)$, $\NIZKcrs_{\Sign} \gets
\NIZKSetup^{\NIZKRel_{\Sign}}(\NIZKsecpar)$, and $\NIZKcrs_{\Inspect} \gets
\NIZKSetup^{\NIZKRel_{\Inspect}}(\NIZKsecpar)$. Return $(\Cparm,\SBCMparm,
\Sparm,\Eparm,\NIZKcrs_{\Issue},\NIZKcrs_{\Sign},\NIZKcrs_{\Inspect},
\AttrSpace)$

\paragraph{$(\ipk,\isk) \gets \IKeyGen(\parm,\fissue)$.} %
To generate its key pair, each issuer first parses \parm as $(\cdot,\SBCMparm,
\Sparm,\cdot,\cdot,\cdot,\cdot,\cdot)$. Then, runs $(\Svk,\Ssk) \gets
\SKeyGen(\Sparm)$, $(\SBCMvk,\SBCMsk) \gets \SBCMKeyGen(\SBCMparm)$,
$\sig_{\fissue} \gets \SSign(\Ssk,\fissue)$, $\ipk \gets (\Svk,\fissue,
\sig_{\fissue})$, $\isk \gets \Ssk$ and return $(\ipk,\isk)$.

\paragraph{$(\opk,\osk) \gets \OKeyGen(\parm,\finsp)$.} %
To generate its key pair, each inspector first parses \parm as $(\cdot,\cdot,
\cdot,\Eparm,\cdot,\cdot,\cdot,\cdot)$. Then, runs $(\Svk,\Ssk) \gets \SKeyGen
(\Sparm)$, $(\Eek,\Edk) \gets \EKeyGen(\Eparm)$, $\sig_{\finsp} \gets \SSign
(\Ssk,\finsp)$, $\opk \gets (\Svk,\Eek,\finsp,\sig_{\finsp})$, and $\osk \gets
(\Ssk,\Edk)$. Finally, returns $(\opk,\osk)$.

\paragraph{$\usk \gets \UKeyGen(\parm)$.} %
Each user, prior to requesting credentials, generates his secret key by parsing
\parm as $(\cdot,\cdot,\cdot,\cdot,\cdot,\AttrSpace)$, and picking randomly
$\usk \getr \AttrSpace$. Finally, return \usk.

\paragraph{$\langle \cred/\bot,\utrans/\bot \rangle \gets
  \langle\Obtain(\usk,\scred,\attrs),\Issue(\isk,\upk,\sipk,\attrs)\rangle$.} %
The protocol is run between an issuer, and a user with secret key \usk and
credentials \scred, where each $\cred \in \scred$ is issued by an issuer with
public key $\ipk_{\cred}$ (which we assume that the user can easily retrieve,
e.g., from secure storage, given \cred), and attests attributes
$\attrs_{\cred}$. For readability we write $\attrs_{\scred}$ as abbreviation for
$\lbrace \attrs_{\cred} \rbrace_{\cred \in \scred}$, and similarly for
$\sipk_{\scred}$. The user requests a signature on a commitment to the user key,
as well as on the attributes in \attrs. In addition, the user proves that the
issuance function \fissue established by the issuer is satisfied by the
credentials in $\scred$
and its user secret key. For this, we define relation $\NIZKRel_{\Issue} = \lbrace
(\usk,\scred,\attrs_{\scred}), (\Ccom,\attrs,\sipk_{\scred}): \Ccom = \CCommit(usk) \land
\fissue(\usk,\scred,\attrs) = 1 \land \forall \cred \in \scred,
\SBCMVerify(\ipk_{\cred},\cred,
\attrs_{\cred} \cup \lbrace \usk \rbrace) = 1 \rbrace$. The interactive protocol
for a user to obtain a credential from an issuer of the system is as follows:

\begin{itemize}
\item \uline{User}: Commit to the user secret key with $\Ccom \gets
  \CCommit(\usk)$. Compute proof $\NIZKproof \gets
  \NIZKProve^{\NIZKRel_{\Issue}}(\NIZKcrs_{\Issue},(\usk,\scred,\attrs_{\scred}),
  (\Ccom,\attrs))$. Send $(\Ccom,\NIZKproof)$ to Issuer.
\item \uline{Issuer}: Verify \NIZKproof with $\NIZKVerify^{\NIZKRel_{\Issue}}
  (\NIZKcrs_{\Issue},\NIZKproof,(\Ccom,\attrs,\sipk))$, and abort if it fails. Then,
  compute the credential by running $\cred \gets \SBCMSign(\SBCMsk,\Ccom,
  \attrs)$. Send \cred to User. Output $\utrans \gets (\Ccom,\attrs,\sipk,
  \cred,\NIZKproof)$.
\item \uline{User}: Verify the credential with $\SBCMVerify(\SBCMvk,\cred,
  \attrs \cup \lbrace \usk \rbrace)$. Reject if verification fails.
  Otherwise, return \cred.
\end{itemize}

\paragraph{$(\sig,\yeval) \gets \Sign(\usk,\opk,\scred,\msg,\feval)$.} %
In the signing algorithm, we make use of the following relation:
$\NIZKRel_{\Sign} = \lbrace (\usk,\scred,\attrs_{\scred},\yinsp,r),(\msg,\feval,
\yeval,\Ec,\sipk_{\scred},\Eek): \Ec = \EEnc(\Eek,\yinsp;r) \land \yeval =
\feval(\usk,\scred,\msg) \land
\yinsp = \finsp(\yeval,\usk,\scred,\msg) \land \forall \cred \in \scred,
\SBCMVerify(\ipk_{\cred},\cred,\attrs_{\cred} \cup \lbrace \usk \rbrace) = 1)
\rbrace$ where, for each $\cred \in \scred$, $\attrs_{\cred}$ and $\ipk_{\cred}$,
as well as $\attrs_{\scred}$ and $\sipk_{\scred}$ are as in $\langle \Obtain,
\Issue \rangle$.
%
From this, in order to produce a valid signature, the user first evaluates
$\yeval \gets \feval (\usk,\scred,\msg)$, and decides whether or not to continue
with the signing process -- this may depend, e.g., on the inspection policy of
\opk, as the output of \feval may influence whether the user will be
de-anonymizable or not, depending on the \finsp function in \opk.
%
Then, the user parses \opk as $(\Svk,\Eek,\finsp,\sig_{\finsp})$ and checks that
$\Verify(\Svk,\sig_{\finsp},\finsp) = 1$ (note that this step may be cached), to
compute $\yinsp \gets \finsp(\yeval,\usk,\scred,\msg)$, and encrypt it with
\Eek by running $\Ec \gets \EEnc(\Eek,\yeval;r)$ for some fresh randomness $r$.
Finally, the user computes
$\NIZKproof \gets \NIZKProve^{\NIZKRel_{\Sign}}(\NIZKcrs_{\Sign},(\usk,\scred,
\attrs_{\scred},\yinsp,r),(\msg,\feval,\yeval,\Ec,\sipk_{\scred},\Eek))$ and
outputs $(\sig = (\NIZKproof,\Ec),\yeval)$.

\paragraph{$1/0 \gets \Verify(\opk,\sipk,\sig,\yeval,\msg,\feval)$.} %
The ``cryptographic'' side of the verification essentially consists on checking
the NIZK proof. That is, parse \sig as $(\NIZKproof,\Ec)$ and check whether
$\NIZKVerify(\NIZKcrs,\NIZKproof,(\msg,\feval,\yeval,\Ec,\sipk)) = 1$. In
addition, the verifier may further check whether \yeval meets its needs.

\paragraph{$(\yinsp,\NIZKproof)/\bot \gets
  \Inspect(\osk,\sipk,\sig,\yeval,\msg,\feval)$.} %
We first define NIZK relation $\NIZKRel_{\Inspect} = \lbrace (\osk),(\Ec,\yinsp)
: \yinsp = \EDec(\osk,\Ec) \rbrace$.
%
For inspection, the inspector first verifies the signature by running $\Verify(
\opk,\sipk, \sig,\yeval,\msg,\feval)$. If the verification succeeds, it parses
\sig as $(\NIZKproof,\Ec)$, decrypts \Ec by running $\yeval \gets \EDec(\osk,
\Ec)$, and computes $\NIZKproof_{\Inspect} \gets \NIZKProve^{\NIZKRel_{\Inspect}}
(\NIZKcrs_{\Inspect},\osk,(\Ec,\yinsp))$. It returns $(\yinsp,
\NIZKproof_{\Inspect})$.

\paragraph{$1/0 \gets \Judge(\opk,\yinsp,\NIZKproof,\sig,\yeval,\msg)$.} %
To assess the validity of an inspection proof, first check the signature
by running $\Verify(\opk,\sipk, \sig,\yeval,\msg,\feval)$. If the check succeeds,
parse \sig as $((\cdot,\Ec),\cdot)$ and verify \NIZKproof with $\NIZKVerify
(\NIZKcrs_{\Inspect},\NIZKproof,(\Ec,\yinsp))$. Accept it the NIZK verification
passes, and reject otherwise.

%%% Local Variables:
%%% mode: latex
%%% TeX-master: "uas"
%%% End:

\subsection{Correctness and Security of \CUASGen}
\label{ssec:security-uas}

First, we define the \Identify, \ExtractIssue and \ExtractSign functions that
are needed for some of the properties to be meaningful, in
\figref{fig:helper-funcs}.

\begin{figure}[ht!]
  \begin{minipage}[t]{\textwidth}
    \procedure{$\ExtractIssue(\parm,\trans)$}{%
      \textrm{Parse \parm as $(\cdot,\cdot,\cdot,\cdot,\NIZKcrs_{\Issue},\cdot,
        \cdot,\cdot)$; $\NIZKcrs_{\Issue}$ as $(\NIZKcrs,\NIZKtrap)$; and
        \trans as $(\Ccom,\attrs,\sipk,\cred,\NIZKproof)$} \\
      \pcif \NIZKVerify(\NIZKcrs,\NIZKproof,(\Ccom,\attrs,\sipk)): 
      \pcreturn \bot \\
      (\usk,\scred,\attrs_{\scred}) \gets \NIZKExtract(\NIZKcrs,\NIZKtrap,
      (\Ccom,\attrs,\sipk),\NIZKproof) \\
      \pcreturn (\usk,\scred,\attrs_{\scred}) \\
    }
    
    \procedure{$\ExtractSign(\parm,\oid,\siid,\sig,\yeval,\msg,\feval)$}{%
      \textrm{Parse \parm as $(\cdot,\cdot,\cdot,\cdot,\cdot,\NIZKcrs_{\Sign},
        \cdot,\cdot)$; $\NIZKcrs_{\Sign}$ as $(\NIZKcrs,\NIZKtrap)$; and
        \sig as $(\NIZKproof,\Ec)$} \\
      \textrm{Parse $\PUBOK[\oid]$ as $(\opk,\cdot)$ and let $\sipk \gets
        \PUBIK[\siid]$} \\
      \pcif \NIZKVerify(\NIZKcrs,\NIZKproof,(\msg,\feval,\yeval,\Ec,
      \sipk,\opk)): \pcreturn \bot \\
      (\usk,\scred,\attrs_{\scred},\yinsp,r) \gets \NIZKExtract(\NIZKcrs,\NIZKtrap,
      (\msg,\feval,\yeval,\Ec,\sipk,\opk),\NIZKproof) \\
      \pcreturn (\usk,\scred,\attrs_{\scred},\yinsp) \\
    }
    
    \procedure{$\Identify(\usk,\attrs_{\cred},\cred)$}{%
      \pcreturn \SBCMVerify(\ipk_{\cred},\cred,\usk,\attrs_{\cred}) \\
    }    
  \end{minipage}
  \label{fig:helper-funcs}
  \caption{Definition of helper functions \Identify, \ExtractIssue and
    \ExtractSign, for \CUASGen.}
\end{figure}

\begin{theorem}[Correctness of \CUASGen]
  \label{thm:correctness-uas}
  If the underlying schemes for vector commitments, encryption, digital
  signatures, signatures on blocks of committed messages, and NIZKs are
  correct, our generic construction \CUASGen satisfies correctness as
  defined in \defref{def:correctness-uas}.
\end{theorem}

\begin{proof}[\thmref{thm:correctness-uas}]
  \todo{XXX}
\end{proof}

\begin{theorem}[Anonymity of \CUASGen]
  \label{thm:anonymity-uas}
  If the NIZK system used for $\NIZKRel_{\Sign}$ is zero-knowledge and
  simulation-extractable, our \CUASGen construction satisfies anonymity as
  defined in \defref{def:anonymity-uas}.
\end{theorem}

\begin{proof}[\thmref{thm:anonymity-uas}]
  In this proof, we restrict to the case in which the adversary can only make
  one query to the challenge oracle. Note however that the generalization to
  polynomially many queries given in \cite{bsz05} applies here too (with the
  corresponding security loss). Thus, proving security for one query to the
  challenge oracle is enough.

  We start from $G_0=\ExpAnonb$, and define game $G_1$ to be exactly the same
  as $G_0$, except that, within the $\Setup$ algorithm, we replace
  $\NIZKSetup^{\Sign}$ with $\NIZKSimSetup^{\Sign}$. By simulation
  extractability, $G_1$ is indistinguishable from $G_0$.
  
  From $G_1$, we consider $G^0_1$, which we define to be $G_1$, for $b=0$
  (i.e., \ExpAnonz, using $\NIZKSimSetup^{\Sign}$). The challenge sent to the
  adversary is $(\csig_0,\yeval) \gets \Sign(\PRVUK[\cuid_0],\PUBOK[\oid],
  \CRED[\scid_0],\msg,\feval)$, where $\csig_0 = (\pi_0,\Ec_{\yinsp})$, with
  $\pi_0 = \NIZKProve^{\NIZKRel_{\Sign}}(\NIZKcrs_{\Sign},(\msg,\feval,\yeval,
  \Ec_{\yinsp},\PUBIK[\scid_0],\PUBOK[\oid]),(\PRVUK[\cuid_0],
  \CRED[\scid_0],\attrs_{\scid_0},\yinsp,r))$ and $\Ec_{\yinsp} =
  \EEnc(\PUBOK[\oid],\yinsp;r)$.
  %
  Further, we build $G_2^0$ from $G_1^0$ by simulating the proof $\pi_0$. That
  is, in $G_{2,0}$, $\csig_0 = (\pi_0^s,\Ec_{\yinsp})$, where $\pi^s_0 =
  \NIZKSim^{\NIZKRel_{\Sign}}(\NIZKcrs_{\Sign},\NIZKtrap,(\msg,\feval,\yeval,
  \Ec_{\yinsp},\PUBIK[\scid_0],\PUBOK[\oid]))$. By zero-knowledgeness
  of $\NIZK^{\Sign}$, $G_2^0$ is indistinguishable from $G_1^0$.

  Similarly, we consider $G_1^1$ and $G_2^1$. That is, $G_1^1$ is $G_1$
  for $b=1$, where the challenge
  sent to the adversary is $(\csig_1,\yeval) \gets \Sign(\PRVUK[\cuid_1],
  \PUBOK[\oid],\CRED[\scid_1],\msg,\feval)$, where $\csig_1 = (\pi_1,
  \Ec_{\yinsp})$, with $\pi_1 = \NIZKProve^{\NIZKRel_{\Sign}}(\NIZKcrs_{\Sign},
  (\msg,\feval,\yeval,\Ec_{\yinsp},\PUBIK[\scid_1],\PUBOK[\oid]),
  (\PRVUK[\cuid_1],\CRED[\scid_1],\attrs_{\scid_1},\yinsp,r'))$ and $\Ec_{\yinsp}
  = \EEnc(\PUBOK[\oid],\yinsp;r')$. As before, $G_2^1$ is built from $G_1^1$,
  simulating $\pi_1$. That is, in $G_2^1$, $\csig_1 = (\pi_1^s,\Ec_{\yinsp})$,
  where $\pi^s_1 = \NIZKSim^{\NIZKRel_{\Sign}}(\NIZKcrs_{\Sign},\NIZKtrap,(\msg,
  \feval,\yeval,\Ec_{\yinsp},\PUBIK[\scid_1],\PUBOK[\oid]))$. Again, by
  zero-knowledge of $\NIZK^{\Sign}$, $G_2^1$ is indistinguishable from $G_1^1$.
  Note also that $G_2^1=G_2^0$, as in the challenge oracle, used in the
  anonymity game, we restrict to $\PUBIK[\scid_0] = \PUBIK[\scid_1]$.

  Finally, consider the definition of $\AdvAnon=|\Pr\lbrack
  \ExpAnono(1^\secpar)=1\rbrack-\Pr\lbrack\ExpAnonz(1^\secpar)=1\rbrack|$. As
  argued, $G_1$ is indistinguishable from $\ExpAnonb$, thus
  $\AdvAnon \approx |\Pr\lbrack G_1^1(1^\secpar)=1\rbrack-\Pr\lbrack
  G_1^0(1^\secpar)=1\rbrack| \approx
  |\Pr\lbrack G_2^1(1^\secpar)=1\rbrack-\Pr\lbrack
  G_2^0(1^\secpar)=1\rbrack|$. Since $G_2^1=G_2^0$, it follows that
  \AdvAnon is negligible.
  %
  \qed
\end{proof}

\begin{theorem}[Issuance unforgeability of \CUASGen]
  \label{thm:issue-forge-uas}
  If the underlying scheme for signatures on blocks of committed messages is
  existentially unforgeable, and the NIZK used for $\NIZKRel_{\Issue}$ is
  simulation extractable and sound, then our \CUASGen construction satisfies
  issuance unforgeability as defined in \defref{def:issue-forge-uas}, except
  with negligible probability.
\end{theorem}

\todo{\usk belongs to \AttrSpace! I think this can lead to malleability attacks.
  Make them disjoint?}

\begin{proof}[\thmref{thm:issue-forge-uas}]
  We show that the probability that \fissue outputs $0$ is negligible, as well
  as the probability that the extracted \usk is not the one that was used to
  request some of the credentials employed to obtain the credential specified by
  the adversary.
  %
  For this purpose, we define two games, $G_0=\ExpForgeIssue$, and $G_1$, which
  is exactly the same, but where, within the \Setup algorithm, we replace
  $\NIZKSetup^{\Issue}$ with $\NIZKSimSetup^{\Issue}$. As per the definition of
  \NIZK in \appref{sapp:nizk}, both games are indistinguishable.

  Now, observe that the adversary is required to output a credential
  identifier for which associated entries in \trans and \CRED exist; moreover,
  if such a credential was produced by an issuer, we must have access to those
  entries, as issuers are assumed to be honest.
  %
  Then, given that $\NIZKRel_{\Issue}$ is knowledge extractable (which is implied
  by simulation-extractability), in game $G_1$
  we can apply the \NIZKExtract function, which produces a tuple $(\usk,\scred,
  \attrs_{\scred})$ from $\utrans = (\Ccom,\attrs,\sipk,\cred,\NIZKproof)$.
  %
  Since \NIZKproof is accepted by \ExtractIssue, from the soundness of \NIZK and
  existential unforgeability of \SBCM, we know that all $\cred \in \scred$ are
  valid signatures over \usk, and their respective $\attrs_{\cred}$. Thus,
  \Identify returns $1$ for all $(\usk,\attrs_{\cred},\cred)$ tuples.
  Moreover, all the credentials in \scred given to \fissue belong to the same
  user, who is the owner of \usk.
  %
  Finally, since issuers are honest, we know that $\ATTR[\cid] = \attrs$ and,
  consequently, $\fissue(\usk,\scred,\ATTR[\cid]) = \fissue(\usk,\scred,\attrs)
  = 1$, due to the soundness of \NIZK.
  %
  \qed
\end{proof}

\begin{theorem}[Signing unforgeability of \CUASGen]
  \label{thm:sign-forge-uas}
  If the underlying NIZK scheme for $\NIZKRel_{\Sign}$ is sound and simulation
  extractable, the NIZK scheme for $\NIZKRel_{\Inspect}$ is sound and simulation
  extractable, and \SBCM is existentially unforgeable, then our \CUASGen
  construction satisfies signing unforgeability as defined in
  \defref{def:sign-forge-uas}, except with negligible probability.
\end{theorem}

\begin{proof}[\thmref{thm:sign-forge-uas}]
  As for \thmref{thm:issue-forge-uas}, we define two games, $G_0=\ExpForgeSign$,
  and $G_1$, which is exactly the same, but where, within the \Setup algorithm,
  we replace $\NIZKSetup^{\Sign}$ with $\NIZKSimSetup^{\Sign}$. As per the
  definition of \NIZK in \appref{sapp:nizk}, both games are indistinguishable.

  From $G_1$, and in order to define the winning conditions for the adversary
  in the signing unforgeability game, consider the following events:

  \begin{description}
  \item[$V$.] Where $V = \Verify(\opk,\sipk,\sig,\yeval,\msg,\feval) = 1$.
  \item[$J$.] Where $J = \Judge(\opk,\sipk,\yinsp,\iproof,\sig,\yeval,\msg,
    \feval) = 0$.
  \item[$L$.] Where $L = (\msg,\feval,\yeval,\Ec,\sipk,\opk) \in
    \NIZKLang^{\Sign}$.    
  \item[$I$.] Where $I = \exists \cred \in \scred~\st~\Identify(\usk,
    \attrs_{\cred},\msg) = 0$.
  \end{description}

  $\adv$ wins if $V \land (J \lor I) = (\overline{L} \land V \land (J \lor I))
  \lor (L \land V \land (J \lor I))$.
  %
  $V$ implies that $(\msg,\feval,\yeval,\Ec,\sipk,\opk) \in \NIZKRel^{\Sign}$.
  Thus, after soundness of $\NIZK^{\Sign}$, the probability of $\overline{L}
  \land V \land (J \lor I)$ is negligible in the security parameter.
  %
  For $(L \land V \land (J \lor I))$ to be satisfied, there are three cases:
  \begin{enumerate}
  \item $L \land V \land J \land \overline{I}$. The game returns 1 in step 6.
  \item $L \land V \land J \land I$.  The game returns 1 in step 6.
  \item $L \land V \land \overline{J} \land I$. The game returns 1 in step 10. 
  \end{enumerate}

  In case 1, $L \land V$ implies that $(\msg,\feval,\yeval,\Ec,\sipk,\opk) \in
  \NIZKRel^{\Sign}$. More concretely, soundness of $NIZK^{\Sign}$ implies that
  $\Ec = \EEnc(\opk,\yinsp;r)$, for $\yinsp$ and $r$ known to the signer. Since
  $(\yinsp,\iproof)$ is generated honestly by the challenger from
  $(\opk,\sipk,\sig = (\NIZKproof_{\Sign},\Ec),\yeval,\msg)$, correctness of
  public key encryption implies that $\EDec(\osk,\Ec) = \yinsp$. Consequently,
  the probability of $L \land V \land J \land \overline{I}$ = 0, as \Judge
  checks precisely that \Ec is a correct encryption of \yinsp under \opk.

  The analysis for case 2 is the same as for case 1.

  For case 3, $(\msg,\feval,\yeval,\Ec,\sipk,\opk) \in \NIZKRel^{\Sign}$,
  $\Judge(\opk,\sipk,\yinsp,\iproof,\sig,\yeval,\msg,\feval)
  = 1$, but there exists some credential \cred for which $\Identify(\usk,
  \attrs_{\cred},\msg)=0$. Note that \usk, $\attrs_{\cred}$ and \cred (for all
  $\cred \in \scred$) are output by \ExtractSign. Thus, after the
  simulation-extractability property and soundness of $\NIZK^{\Sign}$, $(\msg,
  \feval,\yeval,\Ec,\sipk,\opk) \in \NIZKRel^{\Sign}$, which more concretely
  means that $\SBCMVerify(\ipk_{\cred},\cred,\attrs_{\cred} \cup \lbrace \usk
  \rbrace) = 1 = \Identify(\usk,\attrs_{\cred},\cred)$, for all $\cred \in
  \scred$. The probability of case 3 is therefore $0$.
  %
  Moreover, since \SBCMVerify returns $1$ for all $\cred \in \scred$, it must
  be that all of them were obtained via queries to the \ISSUE or \OBTISS
  oracles. Otherwise, if there exists some \cred that was not obtained via
  a call to these oracles, the pair $(\lbrace \usk \rbrace \cup \attrs_{\cred},
  \cred)$ constitutes an existential forgery of \SBCM.

  Cases 1, 2 and 3 above account for winning conditions at steps 6 and 10.
  % 
  Additionally, $\adv$ wins at step 8 if $\feval(\usk,\scred,\msg) \neq \yeval$,
  where \yeval is the value output by the adversary in step 2. However, since
  $\Verify(\opk,\sipk,\sig,\yeval,\msg,\feval) = 1$, soundness of $NIZK^{\Sign}$
  implies that this has negligible probability.
  %
  Similarly, $\adv$ wins at step 9 if (1) $\finsp(\yeval,\usk,\scred,\msg) \neq
  \yinsp$, where \yinsp is the value output by \Inspect at line 5; or if (2)
  $\yinsp \neq \yinsp'$, where $\yinsp'$ is the value extracted by \ExtractSign
  at step 7. For (1), soundness of $\NIZK^{\Sign}$ ensures that \yinsp is the
  correct evaluation of \finsp, whereas soundness of $\NIZK^{\Inspect}$ and
  correctness of the public key encryption ensure that this is also the value
  output by \Inspect. Thus, the probability of $\adv$ winning the game because
  of (1) is negligible. Finally, for (2), simulation-extractability of
  $\NIZK^{\Inspect}$ and correctness of the public key encryption ensure that
  the $\yinsp'$ value extracted by \NIZKExtract matches the value produced by
  \Inspect.
  %
  \qed
\end{proof}

\begin{theorem}[Non-frameability of \CUASGen]
  \label{thm:frame-uas}
  If the underlying NIZK schemes are zero-knowledge and the scheme used for
  $\NIZK^{\Sign}$ is simulation-extractable, and the commitment scheme is
  hiding, then our \CUASGen construction satisfies non-frameability as defined
  in \defref{def:frame-uas}, except with negligible probability.
\end{theorem}

\begin{proof}[\thmref{thm:frame-uas}]
  We prove that, given an adversary $\adv$ against non-frameability of \CUASGen,
  we can build an adversary \advB that breaks the hiding property of the
  underlying commitment scheme, with non-negligible probability.

  We start from $G_0=\ExpNonframe$. $\adv$ makes queries to the oracles in
  \Oframe.

  For $G_1$, within \Setup, we replace the \Setup algorithms for the three
  NIZKs (\Issue, \Sign and \Inspect) with their corresponding \SimSetup
  variants. Consequently, the corresponding queries to \Prove are also
  simulated via the simulator. By the zero-knowledge property of the NIZK
  systems, $G_1$ is indistinguishable from $G_0$.
  
  We build adversary \advB against hiding of commitments from $G_1$ against
  non-frameability. In the hiding game (see \figref{fig:com-games}), \advB first
  picks two messages $\msg_0$ and $\msg_1$, and then receives a commitment \Ccom
  of $\msg_b$. Let \advB pick both $\msg_0$ and $\msg_1$ from \AttrSpace. Then,
  \advB initializes $G_1$ for $\adv$ against non-frameability, and randomly
  picks a number $u \getr [1,q]$, where $q$ can be as large as \advB wants, but
  will be the maximum number of honest users to let $\adv$ add to the game.
  Then, when $\adv$ asks to create the $u$-th user, \advB ignores the call to
  \UKeyGen. For every call that $\adv$ makes to the \OBTAIN oracle associated to
  the $u$-th user, \advB uses the commitment \Ccom received as challenge in its
  game against the hiding property of commmitments, and uses it as commitment to
  the $u$-th user's \usk. \advB then simulates all the NIZK proofs associated to
  the $u$-th user, in calls to \OBTAIN, \SIGN and \INSPECT. Again, due to the
  zero-knowledge property of the associated NIZK systems, the outputs of the
  modified oracles are indistinguishable from the original outputs (as \Ccom
  is a valid commitment of a user secret key). Eventually, $\adv$ outputs a
  $(\sig,\yeval,\msg,\feval,\yinsp,\iproof)$ tuple that is accepted by \Verify
  and \Judge. Due to simulation extractability of $NIZK^{\Sign}$, \ExtractSign
  must be able to produce a $(\usk,\scred,\attrs_{\scred},\yinsp')$ tuple. Since
  $\adv$ wins the
  non-frameability game with non-neglibible probability, with probability $1/q$,
  the \usk value belongs to the $u$-th honest user, so it must be equal to
  either $\msg_0$ or $\msg_1$; \advB responds accordingly to its challenge in
  the game for the hiding property of the commmitment scheme. By assumption,
  \advB wins with non-negligible probability.
  %
  \qed
\end{proof}


%%% Local Variables:
%%% mode: latex
%%% TeX-master: "uas"
%%% End:

\section{Relating \UAS with Other Schemes, and Variations}
\label{sec:transformations}

In order to justify \UAS's universality, in this section we describe how can
we leverage our \UAS scheme to build other known schemes, only by specifying
different issuance, signature evaluation, and opening functions; as well as,
occasionally, some minor variations in some NIZK relations. Table
\tabref{tab:uas-alt-funcs} summarizes the different functions we use in our
subsequent variations of \UAS. \jdv{The relationships seem quite
  straightforward; thus, we only informally sketch how one could build other
  schemes (or something very similar to them) from \UAS, and leave the formal
  analysis for futher work.}

\begin{table}[ht!]
  \begin{tabular}{c | c | c | c | c | c}
    \bf Target scheme & \bf $\NIZKRel_{\Sign}$ variant & \bf \fissue variant & \bf \feval variant & \bf \finsp variant & \bf  Defined in \\
    \hline
    GS & None & $\fissue^1$ & $\feval^0$ & $\finsp^{\upk}$ & Implied by \GSAC \\
    AC\footnote{We show how to build ACs with selective disclosure, but ACs for
    arbitrary claims are direct too.} & None & $\fissue^1$ & $\feval^{\dattrs}$ & $\finsp^0$ & Implied by \GSAC \\
    \GSAC & None & $\fissue^1$ & $\feval^{\dattrs}$ & $\finsp^{\upk}$
    & \secref{ssec:uas-gsac} \\
    \bf GS-MDO & None & $\fissue^1$ & $\feval^0$ & $\finsp^{\smsg}$
    & \secref{ssec:uas-gsmdo} \\
    \bf Ring sigs. & $\NIZKRel_{\Sign-prv}$ & $\fissue^{\sring}$ & $\feval^{\attrs}$ & $\finsp^0$ & \secref{ssec:uas-ring} \\
    \bf MPS\footnote{Building MPS from \UAS also requires a simple definitional change. See
    \secref{ssec:uas-mps}.} & $\NIZKRel_{\Sign-enc}$ & Any & $\feval^{enc}$ & Any & \secref{ssec:uas-mps} \\
    \bf Delegatable creds & None & $\fissue^n$ & Any & Any & \secref{ssec:uas-delcred} \\
  \end{tabular}
  \caption{How to build related schemes from \UAS.}
  \label{tab:uas-alt-funcs}
\end{table}

\subsection{\GSAC from \UAS}
\label{ssec:uas-gsac}

Take the generic construction for \UAS in \secref{ssec:generic-construction-uas}.
First, consider the constant issuance function $\fissue^1$, the signature
evaluation function $\feval^{\dattrs}$, and the opening function $\finsp^{\upk}$,
as defined in \esref{eq:uas-gsac-funcs}

\begin{align}
  & \fissue^1(\usk,\scred,\attrs) \coloneqq 1 \nonumber \\
  & \feval^{\dattrs}(\usk,\cred,\msg) \coloneqq \dattrs(\cred) \nonumber \\
  & \finsp^{\upk}(\yeval,\usk,\scred,\msg) \coloneqq \CCommit(\usk;0)
    \label{eq:uas-gsac-funcs}
\end{align}

Where $\dattrs(\cred)$ is the function returning the attributes indexed by
\dattrs within \cred (which we can assume to be easy to do with proper
encoding). It is straightforward to see that an \UAS scheme with $\fissue^1$
as issuance function, $\feval^{\dattrs}$ as signing evaluation function, and
$\finsp^{\upk}$ as opening function, is an implementation of the functionality
described for \GSAC --which, in addition, lets the signer use more than one
credential per signature; although this can of course be restricted
implementation-wise.
%
Furthermore, the anonymity, sign unforgeability, and non-frameability properties
of \UAS directly imply the anonymity, traceability, and non-frameability
properties of \GSAC.

\paragraph{Vanilla Group Signatures and Anonymous Credentials from \UAS.} As a
corollary to the previous analysis, recall that, in \secref{ssec:variants-gsac}
we proved that \GSAC implies both group signatures and anonymous credentials.
Thus, since \UAS implies \GSAC, \UAS implies also both group signatures and
anonymous credentials.

\subsection{\UAS with Multiple Openers}
\label{ssec:uas-multiopen}

We briefly mention that we have defined \UAS signatures as being ``openable'' by
only one opener. However, since there is no tight relation between issuer and
opener (as opposed to conventional group signatures), it is trivial to
extend to the case with multiple openers per signature. In that case, the
$\NIZKRel_{\Sign}$ relation should be extended to include proofs of correct
encryption of the \yinsp values produced by each chosen opener.
Similarly, the sign unforgeability and non-frameability definitions should be
extended to check correctness of the corresponding extra open values.

\subsection{Group Signatures with Message Dependent Opening}
\label{ssec:uas-gsmdo}

In \cite{khk+19}, one of the first group signature schemes with a certain
flexibility in the open-related functionality was introduced. Briefly, a group
signature scheme with message dependent opening works as a conventional group
signature scheme, with the addition that it is only possible to open signatures
over specific messages (e.g., text with offensive language). For this, the
authors introduce an extra authority, the admitter, who creates a
``message-specific token'' for each message that is deemed unacceptable.
These message-specific tokens are later needed by the opener, in order to open
signatures over unacceptable messages. \UAS can implement this functionality in
a straightforward manner, by leveraging $\fissue^1$ as defined in
\esref{eq:uas-gsac-funcs}, and $\feval^0$ and $\finsp^{\smsg}$, as defined in
\esref{eq:uas-gsmdo-funcs}.

\begin{align}
  & \feval^0(\usk,\scred,\msg) \coloneqq \pcreturn 0 \nonumber \\
  & \finsp^{\smsg}(\yeval,\usk,\cred,\msg) \coloneqq \lbrace \pcif \msg \in \smsg:
    \pcreturn \CCommit(\usk;0); \pcelse \pcreturn \bot \rbrace
    \label{eq:uas-gsmdo-funcs}
\end{align}

\jdv{In \cite{khk+19}, the authors prove that GS-MDO implies identity-based
  encryption. \UAS implying GS-MDO would thus imply IBE, but we do not use
  IBE in our constructions. Is there some connection between some of the main
  building blocks we use, and IBE?}

\subsection{Ring Signatures}
\label{ssec:uas-ring}
\todo{This subsection needs much refinement.}

In a nutshell, ring signatures \cite{rst06} are like group signatures where
rather than having an issuer accepting users into a group, any user can create
an ad hoc group composed by himself, and any arbitrary set of other users that
have advertised their public keys in some publicly accessible way. Also, in
ring signatures no de-anonymization is possible\footnote{At least, in vanilla
  ring signatures. Some variants allow some sort of linkability; for instance
  \cite{lww04}.}, although the most distinguishing property is probably the lack
of issuer, which has led some relevant systems like Monero%
\footnote{\url{https://www.getmonero.org/resources/moneropedia/ringsignatures.html}.
  Last access on May 8th, 2022.} to opt for ring signatures rather than, e.g.,
group signatures.

\todo{Explain the following better...}
To reach (vanilla) ring signatures from \UAS, we have to slightly alter our
generic construction. This is due to the fact that the NP relation
$\NIZKRel_{\Sign}$ defined in \secref{ssec:generic-construction-uas} reveals the
public key of the issuer of each credential used to produce the signatures.
This is a problem as, intuitively, the signer in a ring
signature is a sort of issuer for the ad hoc group. To hide the issuer, we
replace the NP relation in \secref{ssec:generic-construction-uas} with
$\NIZKRel_{\Sign-prv}$. We also use issuance function $\fissue^{\sring}$,
signature evaluation function $\feval^{\attrs}$, and opening function
$\finsp^0$. All of them are specified in \esref{eq:uas-ring-funcs}.

\begin{align}
  & \NIZKRel_{\Sign-prv} \coloneqq \lbrace (\usk,\scred,\attrs_{\scred},\yinsp,r,
    \sipk_{\scred}),(\msg,\feval,\yeval,\Ec,\Eek): \nonumber \\
  & \hspace*{6.405em}\Ec = \EEnc(\Eek,\yinsp;r) \land \nonumber \\
  & \hspace*{6.40em}\forall \cred \in \scred,\SBCMVerify(\ipk_{\cred},\cred,\usk,
    \attrs_{\cred}) = 1) \land \nonumber \\
  & \hspace*{6.40em}\yeval = \feval(\usk,\scred,\msg) \land
    \yinsp = \finsp(\yeval,\usk,\scred,\msg)
     \rbrace \nonumber \\
  & \fissue^{\sring}(\usk,\scred,\attrs) \coloneqq \lbrace \pcif \attrs
    \subseteq \sring \cup \lbrace \CCommit(\usk;0) \rbrace: \pcreturn 1; \pcelse
    \pcreturn 0 \rbrace \nonumber \\
  & \feval^{\attrs}(\usk,\cred,\msg) \coloneqq \lbrace \pcif \CCommit(\usk;0) \in
    \attrs: \pcreturn \attrs(\cred); \pcelse \pcreturn \bot \rbrace \nonumber \\
  & \finsp^0(\yeval,\usk,\cred,\msg) \coloneqq 0 \label{eq:uas-ring-funcs}
\end{align}

Where $\usk \gets \isk$ for each issuer/user and, consequently 
\sring is some (arbitrary) ad hoc set containing the public keys of the
users (issuers) that the signer wants to include in its ring%
\footnote{Note that, implicitly, this conditions that the space of issuer public
  keys is a subset of \AttrSpace. If this is not the case ``by default'', this
  can be achieved via some appropriate mapping (e.g., an FDH or a XOF
  \needcite). That is, instead of having the ring be a set of {\ipk}s, we
  have the ring be a set of $F({\ipk}s)$, where $F$ is an appropriate mapping.
  For simplicity of exposition, we assume that the space of issuer public keys
  is compatible with \AttrSpace.}.
In $\NIZKRel_{\Sign-prv}$
the public keys of the issuers are now part of the witness in the NP relation,
which means that they will not be revealed in the proofs. Also, observe that, if
we let the attributes in a credential be public keys, $\fissue^{\sring}$ means
that a credential will be issued if all the public keys specified as attributes
to be included in the credential are part of the union of the ring and the
public key of the signer. To build $\feval^{\attrs}$, we define $\attrs(\cred)$
to be the function that returns all the attributes encoded in a credential.
Then, if the public key of the signer is included in the ring, $\feval^{\attrs}$
returns all the attributes in the credential (i.e., all the public keys in the
ring); otherwise, it aborts. Finally, let $\finsp^0$ disallows any
de-anonymization.

Then, an \UAS scheme with $\NIZKRel^{\Sign-prv}$ as NP relation for
$\NIZK^{\Sign}$, and $\fissue^{ring}$, $\feval^{\attrs}$, and $\finsp^0$ is
intuitively a ring signature scheme. To see this, observe that any user can
act as an issuer. Thus, the owner of \usk can issue to himself a credential
for any arbitrary ring it desires. While the $\NIZKRel^{\Issue}$ reveals the
issuer's public key, note that this is irrelevant, as the issuer is the user
himself. Then, the newly defined $\NIZKRel^{\Sign-prv}$ does exactly the same
as in our original \UAS scheme, but without revealing the issuer's public key.
$\feval^{\attrs}$ reveals all the attributes in the credential, which are
the public keys of the ring, including the signer's public key. This is actually
what ring signatures do: the signer, who is the owner of the private key
associated to one of the public keys in the ring, advertises which are the
public keys in the ring, and proves knowledge of one of them. If the NIZK
proof verifies, this means that the signer is indeed the owner of such a private
key. Finally, since $\finsp^0$ returns always $0$, then no de-anonymization is
possible.
%
Interestingly, this construction based on \UAS allows adding extra attributes
(beyond the public keys in the ring) to the produced signatures, which may be
useful for real world use cases.

\subsection{Multimodal Private Signatures}
\label{ssec:uas-mps}

To show that Multimodal Private Signatures (MPS) \needcite can be built from
\UAS, we need to give an alternative, simulation-based, definition of the
anonymity property. In the simulation-based approach, we require the adversary
to guess the bit $b$ defining whether it is interacting with the real world,
where it gets signatures by real users, or with a simulation, where all
signatures are simulated and do not contain information about the signer.
This alternative formulation is given in \figref{fig:exp-uas-simanon}, where
$\Osimanon = (\lbrace\HU,\CU\rbrace\GEN,\lbrace\II,\OO\rbrace\GEN,\lbrace\II,
\OO\rbrace\CORR,\OBTAIN,\WREG)$.

\begin{figure}[htp!]

  \centering
  \procedure[linenumbering]{$\ExpSimAnonb(1^\secpar)$}{%
    \pcif b = 0: \\
    \parm \gets \Setup(1^\secpar) \\    
    \pcind b^* \gets \adv^{\Osimanon,\SIGN,\OPEN}(\parm) \\
    \pcelse: \\
    \parm \gets \SIMSETUP(1^\secpar) \\
    \pcind b^* \gets \adv^{\Osimanon,\SIMSIGN,\SIMOPEN}(\parm) \\    
    \pcreturn b^*
  }
  
  \caption{Simulation-based anonymity experiment for \UAS schemes.}
  \label{fig:exp-uas-simanon}
\end{figure}

\begin{definition}{(Simulatable Anonymity of \UAS)}
  \label{def:sim-anonymity-uas}  
  We define the advantage \AdvSimAnon of $\adv$ against \ExpSimAnonb as
  $\AdvSimAnon=|\Pr\lbrack\ExpSimAnono(1^\secpar)=1\rbrack-
  \Pr\lbrack\ExpSimAnonz(1^\secpar)=1\rbrack|$.
  %
  An \UAS scheme satisfies simulatable anonymity if there exists simulators
  \SIMSETUP, \SIMSIGN and \SIMOPEN such that, for any p.p.t. adversary $\adv$,
  \AdvSimAnon is a negligible function of $1^\secpar$.
\end{definition}

Note that, for our generic construction \CUASGen, \SIMSETUP, \SIMSIGN, and
\SIMOPEN are straightforward, as all we need to do is:

\begin{itemize}
\item To build \SIMSETUP, just replace the $\NIZKSetup^{\Sign}$ and
  $\NIZKSetup^{\Open}$ algorithms within \Setup, by their respective
  $\NIZKSimSetup$ algorithms.
\item To build \SIMSIGN from the \SIGN oracle, just simulate the NIZK proof
  contained in the signatures produced by the \Sign algorithm.
\item To build \SIMOPEN from the \OPEN oracle, just simulate the NIZK proof
  showing opening correctness, instead of producing them as in the \Open
  algorithm.
\end{itemize}

Indistinguishability then follows from the zero-knowledge property of each
corresponding NIZK system. Since both scenarios are indistinguishable, and
all signature-related proofs received by the adversary in the simulation do not
depend on any witness (as they are simulated), then it follows that the
adversary cannot gain any information about signers in the real world.

With this alternative definition, we can achieve a generalisation of MPS
basically by, instead of returning the plaintext \yeval value, returning an
encryption of it under the opener's public key; additionally proving 
that the value used to compute the output of \finsp is the plaintext \yeval
value. In more detail, we define an extended $\NIZKRel_{\Sign-enc}$ relation,
and, given any \feval function, a variant $\feval^{enc}$ operating on an
encryption of \yeval. Both are specified in \esref{eq:uas-mps-funcs}.

\begin{align}
  & \NIZKRel_{\Sign-enc} \coloneqq \lbrace (\usk,\scred,\attrs_{\scred},\yinsp,r),
    (\msg,\feval,\yeval,\Ec,\sipk_{\scred},\Eek): \nonumber \\
  & \hspace*{6.00em}\Ec = \EEnc(\Eek,\yinsp;r) \land \nonumber \\
  & \hspace*{6.00em}\forall \cred \in \scred,\SBCMVerify(\ipk_{\cred},\cred,\usk,
  \attrs_{\cred}) = 1) \land \nonumber \\
  & \hspace*{6.00em}\yeval = \EEnc(\opk,\feval(\usk,\scred,\msg)) \land \nonumber \\
  & \hspace*{6.00em}\yinsp = \finsp(\feval(\usk,\scred,\msg),\usk,\scred,\msg)
    \rbrace \nonumber \\
  & \feval^{enc}(\usk,\scred,\msg) \coloneqq \lbrace \yeval
    \gets \feval(\usk,\scred,\msg); \yeval' \gets \EEnc(\opk,\yeval); \pcreturn
    \yeval' \rbrace \label{eq:uas-mps-funcs}
\end{align}

This would also require that the public-key encryption scheme is IND-CCA.

Given this approach, the need to provide an alternative definition for anonymity
is clearer. Namely, in the original \CHALb oracle, we restrict that the
signatures produced by both challenge users output the same \yeval value.
However, if instead of returning the plaintext \yeval value, we return an
encrypted version of it (as we do in the variant just described), this clearly
becomes unachievable.

\jdv{It would seem that the simulation-based anonymity definition is more
  general. Can we prove that? Concretely, if it is also good for selective
  disclosure, we may just adopt it as default.}

\subsection{Delegatable Credentials}
\label{ssec:uas-delcred}

Building delegatable credentials is also straightforward, thanks to the \fissue
function. For instance, take the simplification of level-based delegation, where
the owner of a credential of level $n$ can only issue credentials of level
$n+1$. Without loss of generality, assume that the credential level is encoded
in the first attribute of the credential (after the \usk), which we denote with
$a_1$. In this context, any owner of a credential of level $n$ can define an
issuance function $\fissue^n$, as in \eref{eq:uas-delcred-func}.

\begin{align}
  \fissue^n \coloneqq \lbrace \pcif a_1 = n+1: \pcreturn 1;
  \pcelse \pcreturn 0 \rbrace \label{eq:uas-delcred-func}
\end{align}

\jdv{\subsection{Functional Signatures}}
\jdv{I think that, by making \yeval explicit, \UAS is essentially a
  privacy-preserving extension to Functional Signatures. Still, I read that paper
  quite some time ago. Re-check.}

%%% Local Variables:
%%% mode: latex
%%% TeX-master: "uas"
%%% End:


%%% Local Variables:
%%% mode: latex
%%% TeX-master: "uas"
%%% End:
