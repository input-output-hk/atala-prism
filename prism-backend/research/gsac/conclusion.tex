\section{Conclusion}
\label{sec:conclusion}

\comment{This section is just a placeholder for now...}

Some notes on future work, to expand/refine/remove:

\begin{itemize}
\item Add support for open functions over sets of previously issued signatures.
  This would cover existing schemes like \cite{dl21}, and possibly cool cases
  such as training AI models on private data, in a federated manner, without
  revealing the identity of the node running the model.
\item Currently, in \UAS, we need to restrict challenge users to present
  credentials issued by the same set of issuers. Technically, I guess that this
  could be done by making the issuers' public keys part of the witness (although,
  at what cost?). But then, how would verifiers be able to determine whether
  they trust the issuers of the credentials used to compute the signature? It
  would be as if an employer has to accept a University degree, without knowing
  which university issued it.  
\item Potentially very cool improvement in the modelling: maybe we can leverage
  a decentralized append-only bulletin board (so, a blockchain) to reduce trust
  on the issuers (except identity proofing, we could probably eliminate it...)
  That would be quite innovative, as in GS literature, the issuer has so far
  alwasy been thought to be necessarily honest. (Disclaimer: this was Ezequiel's
  suggestion!)
\item The current syntax of \UAS does not seem to support blind issuance. This
  would however be trivial to achieve by adding a set of attributes to be
  blindly issued in the \Obtain algorithm, and then requiring that they are
  validated in \fissue. In the generic construction, this would be directly
  supported too, as a block of messages to be signed as part of the block
  commitment.
\end{itemize}

%%% Local Variables:
%%% mode: latex
%%% TeX-master: "uas"
%%% End:
