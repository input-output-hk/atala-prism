\section{Conclusion}
\label{sec:conclusion}

In this work, we have described how a direct combination of the group signatures
(GS) and anonymous credentials (AC) research lines results in a scheme that can
bring benefit to real world use cases. Going further, we generalized such simple
combination into Universal Anonymous Signatures, or \UAS. Our model for \UAS
covers straight away a large subset of GS and AC variants, while still giving
system designers the flexibility to decide on the specific tradeoff between
privacy, utility, accountability and computational overhead that fits their
setting best. More concretely, our model for \UAS supports arbitrary issuance
policies, signature evaluation functions, and opening functions, without losing
definitional meaning. We also gave a generic construction from well known
building blocks, and prove that it meets our security model.

Even though the generalization effort we do is significant, there are still
ways to go further.
%
For instance, we do not hide the issuer(s) of the
credential(s) employed to produce a signature (resp., request a new credential).
Thus, we can only ensure privacy among signatures (resp., credential requests)
that involve credentials issued by the same issuer set. While hiding the issuers
of the involved credentials is certainly possible (we show how to do it at
signing time, in order to build ring signatures), we argue that it may be
desirable to leave it as is, at least, initially. The reason being that issuers
are still what conveys trust to a system where credentials contain attributes
attested by these issuers. Indeed, if it is already hard for verifiers to decide
whether or not to trust a non-anonymous issuer, one can imagine that it would be
even harder to decide whether or not to trust an anonymous one. This question
is core to the domain of trust registries, for instance.
%
A second desirable improvement, perhaps clearer from a technical point of view,
is to extend our model to support non-trivial utility and accountability across
multiple signatures from the same signer. More concretely, let our \feval and
\finsp functions (and perhaps also \fissue?) operate on multiple signatures by
the same \usk. That can lead to an extended \UAS scheme which would support
advanced use cases such as rate limiting (or $k$-TAA \cite{asm06}); or might
be even privately training an AI model based on one's authenticated
past activity, and prove that the model has been trained honestly.

%%% Local Variables:
%%% mode: latex
%%% TeX-master: "uas"
%%% End:
