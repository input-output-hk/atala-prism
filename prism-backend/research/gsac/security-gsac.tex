\subsection{Correctness and Security of \GSACGen}
\label{ssec:security-gsac}

\begin{theorem}[Correctness of \GSACGen]
  \label{thm:correctness-gsac}
  If the underlying schemes for \todo{xxx}, our generic construction \GSACGen
  satisfies correctness as defined in \defref{def:correctness-gsac}.
\end{theorem}

\begin{proof}[\thmref{thm:correctness-gsac}]
  \todo{XXX}
\end{proof}

\begin{theorem}[Anonymity of \GSACGen]
  \label{thm:anonymity-gsac}
  If the NIZK system used for $\NIZKRel_{\Sign}$ is zero-knowledge and
  simulation-extractable, and the encryption system is IND-CCA secure,
  our \GSACGen construction satisfies anonymity as defined in
  \defref{def:anonymity-gsac}.
\end{theorem}

\begin{proof}[\thmref{thm:anonymity-gsac}]
  We prove that the probability of an adversary distinguishing signatures by
  challenge users is negligible. For the proof, we restrict to the case of
  allowing only one query to the challenge oracle. The extension to a polynomial
  number of queries given in \cite{bsz05} applies here, with the corresponding
  security loss.

  We begin with game $G_0=\ExpGSACAnonb$. From it, we build $G_1$, where we
  simply replace the \NIZKSetup functions for $\NIZKRel_{\Sign}$ and
  $\NIZKRel_{\Open}$ within the \Setup algorithm. $G_1$ is indistinguishable
  from $G_0$ due to the zero-knowledge property of the NIZK systems.

  From $G_1$, we build $G_2$, where within the challenge signature produced by
  \CHALb, we replace the \Ccom value encrypted within \Sign, with a random
  string of the appropriate length (e.g., the same length of the encrypted
  message in $G_1$). The corresponding proof for $\NIZKRel_{\Sign}$ is thus
  also simulated. The zero-knowledge and simulation-extractability properties
  of the NIZKs for $\NIZKRel_{\Sign}$, and IND-CCA of the encryption
  scheme, ensure that the result is indistinguishable to the adversary.

  Now, observe that in $G_2$, the challenge signature received by the adversary
  is completely independent from the challenge bit $b$. Moreover, as argued,
  $G_2$ is indistinguishable from $G_0$, where the adversary plays the original
  \ExpGSACAnonb game. Thus, our generic construction of \GSAC satisfies
  anonymity, except with negligible probability.
  %
  \qed
\end{proof}

\begin{theorem}[Traceability of \GSACGen]
  \label{thm:trace-gsac}
  If the underlying NIZK schemes for $\NIZKRel_{\Sign}$ %and $\NIZKRel_{\Open}$
  is sound, and the signature scheme for blocks of committed messages is
  existentially unforgeable, then our \GSACGen construction satisfies
  traceability as defined in \defref{def:trace-gsac}, except with negligible
  probability.
\end{theorem}

\todo{\usk belongs to \AttrSpace! I think this can lead to malleability attacks.
  Make them disjoint?}

\begin{proof}[\thmref{thm:trace-gsac}]
  Assuming the NIZK systems are sound, we build an adversary against existential
  unforgeability of the signature scheme for blocks of committed messages.
  %
  Consider the following events:

  \begin{description}
  \item[$O$.] The \Open algorithm returns $\bot$.
  \item[$J$.] The \Judge algorithm returns $0$.
  \item[$D$.] No credential owned by a user in \CU contains \DAttrs.
  \end{description}

  Given these events, the adversary wins if:

  \begin{description}
  \item[$W_1 = O$.] \Open returns $\bot$ (line 5 of \ExpGSACTrace).
  \item[$W_2 = \overline{O} \land J$.] \Open does not fail, but \Judge rejects
    the output of \Open (line 7 of \ExpGSACTrace).
  \item[$W_3 = \overline{O} \land \overline{J} \land D$.] \Open does not fail,
    \Judge accepts the output by \Open, but no corrupt user has a credential
    containing \DAttrs (line 8 of \ExpGSACTrace).
  \end{description}

  Clearly, since the group signature is verified at step 4 of the game, if
  verification succeeds, soundness of $\NIZK_{\Sign}$ implies that $(\Ccom_m,
  \Ec,\DAttrs) \in \NIZKLang_{\Sign}$, except with negligible probability, say
  $2^{-\NIZKsecpar}$. Also, observe that each of the winning events for the
  adversary described above are disjoint. Thus:

  \begin{equation}
    \AdvGSACTrace(\secpar) = \Pr[W_1]+\Pr[W_2]+\Pr[W_3]+2^{-\NIZKsecpar}
  \end{equation}

  $\Pr[W_1]=0$ is directly deduced from the fact that $(\Ccom_m,\Ec,\DAttrs)
  \in \NIZKLang_{\Sign}$ due to soundness of $\NIZKRel_{\Sign}$. Concretely,
  we know that \Open does not abort due to rejection of the signature by
  \Verify (as this is checked in line 4 of \ExpGSACTrace too) and, thus,
  $\Ec = \EEnc(\opk,\Ccom)$ which, by correctness of the encryption algorithm,
  implies that $\Ccom = \EDec(\osk,\Ec)$, where $\upk=\Ccom=\CCommit(\usk;0)$.
  Completeness of $\NIZK_{\Open}$ thus implies that the \Open algorithm can
  compute the opening proof \oproof, and return $(\upk,\oproof)$.
  
  $\Pr[W_2] = 0$ is similarly deduced from soundness of $\NIZKRel_{\Sign}$ and
  completeness of $\NIZKRel_{\Open}$, since the opening proof is computed
  honestly in line 6 of \ExpGSACTrace.

  Finally, assume that there is an adversary $\adv$ against \ExpNonTrace that
  wins by event $W_3$. We build an adversary against unforgeability of the
  signature scheme on blocks of committed messages.
  
  \qed
\end{proof}

\begin{theorem}[Non-frameability of \GSACGen]
  \label{thm:frame-gsac}
  If the underlying schemes for \todo{xxx}, then our \GSACGen construction
  satisfies non-frameability as defined in \defref{def:frame-gsac}, except with
  negligible probability.
\end{theorem}

\begin{proof}[\thmref{thm:frame-gsac}]
  \qed
\end{proof}

%%% Local Variables:
%%% mode: latex
%%% TeX-master: "gsac"
%%% End:
