\section{Introduction}
\label{sec:introduction}

Identity management is one of the hardest problems in the digital domain%
\footnote{\url{https://www.maddyness.com/uk/2021/11/04/digital-identity-is-one-of-the-21st-centurys-biggest-challenges/}. Last access, August 30th, 2022.}.
It requires to address needs that are already complex on their own, such as:

\begin{itemize}
\item Establishing a connection between a digital persona and a physical one.
\item Interoperability across solutions.
\item Security and privacy, without sacrificing functionality.
\end{itemize}

As usual, the third item makes things even more complicated and, still, is
essential. Without proper handling of one's secrets, impersonation is trivial.
Without robust privacy protection mechanisms, high user adoption is unlikely,
and regulatory compliance, impossible. Yet, adding security makes systems less
usable, and adding privacy typically comes at the cost of using functionality.

From the previous, is not surprising that (privacy-preserving and secure)
digital identity has been a fertile ground for research. Indeed, at least for
the last four decades, it has been a recurring topic in the main cryptography
and security academic venues. Also, roughly during the last decade, it has
received increased attention in the industry\footnote{See, for instance, the
  Idemix and PrivacyABC projects, the W3C DID and Verifiable Credentials
  specification efforts, or the related ``anoncreds'' working group, etc.}

In fact, if one tries to get up to date with the latest advances in the domain,
a non-negligible added complexity is to navigate through the vast amount of
previous work. Quickly, seemingly incompatible systems arise, that offer a
different privacy-vs-utility tradeoff: from private systems that do not lose
utility at the cost of introducing a fully trusted third party who can revoke
anyone's privacy; to fully private systems that do not allow privacy revocation,
but lose all utility; passing through many things in between. Needing to make a
hard decission for a concrete privacy-vs-utility tradeoff is one of the big
drawbacks of existing systems.

We introduce here Universal Anonymous Signatures -- UAS, for short. UAS
is a flexible framework for privacy-preserving identities. Cryptographically,
it generalizes much work done durng the last 40 years. In UAS, identities are
sets of attested attributes, which have proven to be a powerful way to represent
physical-world identities. In addition, UAS enables that arbitrary claims are
proven based on these attributes, both at the time that a signature is verified
and, also \uline{afterwards} -- if needed and the request is granted by an
authority chosen by the signer. The claims to be proven are chosen by the
verifier (possibly jointly with the signer), but the signer is fully aware of
the information that will or may be revealed, if any. Indeed, the option to
reveal extra customizable information after the signature has been produced
enables very relevant use cases and, as we will see, makes a big difference in
addressing the privacy-vs-utility tradeoff. Finally, UAS is compatible with
industry efforts that focus on interoperability, such as W3C's Verifiable
Credentials or Anoncreds.

%%% Local Variables:
%%% mode: latex
%%% TeX-master: "uas-onepager"
%%% End:
