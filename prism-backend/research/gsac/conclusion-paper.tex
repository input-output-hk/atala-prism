\section{Conclusion}
\label{sec:conclusion}

We present a general model for privacy-preserving signatures and authentication,
and provide a generic construction, secure in our model, that allows
different privacy-vs-utility tradeoffs to be reached.
%
Then, we prove that well-known schemes are special cases of our generic
construction. The implied schemes include group signatures, anonymous
credentials (with and without revocation), ring signatures, or the recently
proposed multimodal private signatures.

Flexibility in our model is achieved via functional placeholders that
modulate the behaviour at credential issuance time, the utility
information revealed along with the produced signatures, and the utility
information extractable from already produced signatures. These functions are
computed by the users, who prove correct computation. We include features of
anonymous credential schemes, like attributes and multiple
issuers, which we extend with openers (like in group signatures), enabling
multiple openers to coexist. This further increases flexibility, by removing
the tight issuer-opener coupling in previous works.

For future work, an obvious line is to come up with (and implement) concrete
instantiations of our generic construction in order to evaluate efficiency in
real-world use cases. Alternative generic constructions are also of interest --
for instance, based on structure-preserving signatures.
%
Finally, in this work we focus only on utility over \emph{one} signature. But
extending our model and constructions to span utility over \emph{multiple}
signatures promises to be interesting, both regarding its modelling, the
potential use cases, and the generalization capabilities.


%%% Local Variables:
%%% mode: latex
%%% TeX-master: "uas-paper"
%%% End:
