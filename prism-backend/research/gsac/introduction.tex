\section{Introduction}
\label{sec:introduction}

In this work, we combine two closely related schemes -- group signatures (GS)
and anonymous credentials (AC) -- which, for some reason, have followed a
separate path up to now, beyond individual constructions from both fields
drawing inspiration (and techniques) from one another. This combination is in
some sense a natural evolution, which we do believe may have relevant impact in
practical settings. On the one hand, anonymous credentials enable a richer
functionality in regards to what one can prove about his own identity, in a
privacy-preserving way. Academic proof of this is the fact that AC systems have
naturally evolved to enabling arbitrary claims that only reveal the minimally
possible 1-bit of information \needcite. Also, AC systems have been proposed to
go even beyond what one can prove about his identity, covering also orthogonal
functionality such as delegation \needcite, or aggregation of multiple
credentials into one \needcite. Even outside academia, prominent
technical workgroups \needcite, startups \needcite, and big techs \needcite are
promoting and adopting solutions which are directly inspired by AC systems. On
the other hand, GS 
schemes have inspection functionality (e.g., opening and linking of signatures)
at their core, which is reflected formally through non-frameability properties.
This has originated a large body of research focusing on the construction and
formalisation of variants of these notions \needcite. Consequently, group
signatures tend to be proposed for scenarios that require some sort of
de-anonymization \needcite. Indeed, this is not possible in ACs, which at most
support revocation of issued credentials, which essentially can compare to
linking, but not to opening-related notions.

Our goal is to get the best of both worlds: leverage the richer functionality
of ACs in what respects to proving, in a privacy-preserving manner, useful
claims on one's identity; and, at the same time support, through inspection
capabilities, use cases that may require some sort of de-anonymiztion. We
could target the ideal case of allowing to create proofs of arbitrary predicates
on one's identity (as some AC systems do), as well as provide flexible or several
notions of non-frameability. However, this being -- as far as we know -- the
first work that explores this intersection, we opt for keeping things simple
for the sake of facilitating technical precission and readability, and fostering
discussion. Therefore, we focus on the case of selective attribute disclosure
for ACs, and conventional opening for GSs, building a GSAC scheme that supports
both.

%%% Local Variables:
%%% mode: latex
%%% TeX-master: "sok-privsig"
%%% End:
