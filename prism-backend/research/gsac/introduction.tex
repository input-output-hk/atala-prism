\section{Introduction}
\label{sec:introduction}

% What is the context for this research? Introduce previous works on this topic,
% in one paragraph. %%% Privacy-preserving identities, with accountability (APPI)

% What are the main approaches of previous works to achieve APPI?
% How are they lacking?

%% 2 pages max for the previous

\iffalse
\commentwho{Jesus}{The following will most probably change completely if we
  are to submit this for publication (it least, it probably needs some deep
  reorganisation). However, this should capture tha main ideas.}

Two major research lines within cryptography that work in the space of
privacy-preserving authentication and identfication are group
signatures (GS) and anonymous credentials (AC). In fact, despite having followed
separate paths, the overall approaches to build them are quite similar in a
general sense.
%
In short, GSs allow a group of users, accepted into the group by an issuer, to
create signatures on behalf of the group, without revealing their identity. Only
an opener can (sometimes partially) deanonymize these signatures. In ACs, users
obtain credentials, typically containing multiple attributes, from trusted
issuers. Subsequently, users can prove ownership of these credentials, usually
revealing only subsets (or arbitrary claims) of these attritubes, while keeping
the rest hidden. Both GSs and ACs are used for authentication.

\paragraph{Accountability in GSs and identification in ACs.} %

\todo{Take inspiration from the SoK paper on accountability to argue that
  accountability is needed to prevent misbehaviour and support utility in
  a privacy-preserving manner.}
Typically, GSs make special focus in accountability (via opening), while ACs
place more attention in dealing with identifying information of the credential
owners.
%
Indeed, the opening functionality has been a core component of GSs. Even for
GS variants that do not have opening, there is always some mechanism that
allows to somehow ``correlate'' a signature with its signer -- even when this is
controlled by the signers themselves, as in \cite{dl21} \needcite. Still, even
when group signatures can be opened in the conventional sense, the notion of
\emph{identity} that is obtained from this deanonymization is too lax. Most
frequently, opening just returns an integer (e.g., $i$), which can
be associated to a user who joined the system (the $i$-th). Who this ($i$-th)
user is, or what identifies him/her, is left as out of scope.
%
On the other hand, the capability to encode authenticated identifying
information via attributes within a credential, and then prove claims about
those attributes without revealing more than strictly needed, is probably the
most attractive feature of ACs. Through these attributes (and claims over
them), ACs actually allow building systems where the participating entities
know \emph{who} they are interacting with. However, if a credential owner
misbehaves, typically not much can be done beyond (in some schemes) blacklisting
the credential. Even when we do not want to somehow penalize a credential owner,
but rather be able to apply some utility (alas, partially deanonymizing)
function \emph{after the authentication has taken place}, not having some sort
of ``accountability'' mechanism (like GSs' open) in ACs makes it impossible.

It is reasonable that accountability and identification have followed separate
paths. After all, they are challenging features on their own. Even more, when
combined with privacy, as is the case for GSs and ACs. However, real world use
cases claim for both \emph{at the same time} and \emph{in a flexible manner}.
\fi

\subsection{Our Contributions}
\label{ssec:contributions}

In this work, we focus on the generalization of privacy-preserving cryptographic
systems for authentication and identification. Our approach towards this
generalization comes in two steps: first, 


On the one hand, the security models of GS and AC hold important similarities.
Security in group signatures is typically captured with game-based definitions,
where the main properties are anonymity, traceability and non-frameability.
Anonymity ensures privacy, essentially requiring that group signatures do not
leak information about their signer. Traceability and non-frameability are
unforgeability-related properties. Traceability requires that any valid group
signature must have been created by a user who joined the group;
non-frameability requires that no adversary can create a signature that somehow
``blames'' an honest user. Of course, variations in the functionality of group
signatures frequently require specialized properties, but these three are always
the main ones. In anonymous credentials, security is also usually captured with
game-based definitions (although UC models also exist). But, for anonymous
credentials, the main properties (again, up to variations) are simply anonymity
and unforgeability. Anonymity captures a similar privacy notion than group
signatures. Unforgeability of anonymous credentials also maps more clearly to
traceability in group signatures (non-frameability assumes a corrupt issuer,
which must be honest in traceability and unforgeability), and requires that
no adversary can succeed in proving ownership of a credential it does not own.
In both cases -- anonymity and unforgeability in anonymous credentials -- the
focus is on preventing leakage (anonymity) or forging (unforgeability) of the
attributes in a credential. Since group signatures do not (usually) have
attributes, this is a core difference in the modelling.

The first thing that raises attention here is the fact that, despite following
very similar approaches to build them, there is no equivalent property for
non-frameability in anonymous credentials. The most probable reason is that
anonymous credentials do not include the concept of ``opening'' which, in
group signatures, allows to extract the identity of the signer and is essential
to the non-frameability definition\footnote{Exceptions exist \cite{dl21}, but
  the definition is much easier with opening.}. Something similar (more limited
than opening, though) is sometimes considered by means of revocation of
anonymous credentials -- revoking credentials is essentially equivalent to
being able to link signatures by the same signer, which is considered in some
variants of group signatures \needcite. However, while revocation may be enough
if we aim to use credentials ``just'' for authentication, if we target more
advanced use cases, we may need to learn something else from the credential (or
its owner), beyond whether or not it has been revoked. Then, without an
equivalent to non-frameability, we would not be able to make any security claim
in the presence of dishonest issuers. On the other hand, it is
also noteworthy that the literature of group signatures has barely explored
the inclusion of attributes. With the exception of a few works on Attribute-Based
Group Signatures (ABGS) \needcite, members of a group in group signatures just
get a credential attesting for their membership to the group. Moreover, the
existence of attributes in anonymous credentials would seem to have been
instrumental in the surge of real-world use cases in the past few years. Indeed,
initiatives such as W3C's Verifiable Credentials (VCs) \needcite, or large
projects such as Idemix \needcite, which has in turn influenced VCs, are heavily
based on concepts from the anonymous credential literature, even to the point
that Idemix and its ``blockchain-based successor'', Indy \needcite, are now a
type of verifiable credentials \needcite.

Following the previous reasoning, we identify two aspects in which we can
improve group signatures and anonymous credentials. The first is by coming up
with some sort of primitive that combines both lines. That is, a scheme which
supports both (a variant of) opening and non-frameability as well as
attribute-based credentials. This seemingly natural evolution would thus
enable rich use cases in which accountable (by means of opening) and expressive
(by means of the attributes) authentication is necessary. While this is in
line with the previously mentioned ABGS works, we argue that our approach
is more general and flexible. Still, we anticipate that this may not be enough
to reach a scheme that can adapt to the majority of situations that may arise in
the real world. The reason is that the issuance and opening capabilities in the
previous schemes are generally either too rigid or have lacked a formal analysis
so far. For instance, in group signatures, the issuance protocol has barely been
the focus of study. Essentially, this interactive protocol requires the
prospective member to prove, in zero-knowledge, knowledge of a secret key
that will be attested in the membership credential. On the other hand, multiple
variants of the issuance protocol exist in anonymous credentials: protocols that
assume that the issuer knows all the attributes to be included in the credential,
protocols that allow blind issuance of attributes, protocols that let the user
leverage attributes in previous credentials, protocols that include ad hoc
rules at issuance time to enable extra functionality, such as delegation
\todo{cross-check that all this, especially the latter, is correct; and add
  cites}, etc. The problem is that a single change in this aspect, requires to
completely rethink the model. Something similar happens with the inspection
capabilities in group signatures. This frequently takes the shape of opening,
but many variants also exist, message-depdendent opening, linking, or variants
of linking, such as restricted linking only by other members of the group,
user controlled linking, or sequential linking \needcite. Again, any small
change in the inspection capabilities requires a full redesign of the model.
This rigidness in inspection is specially crucial, as inspection is the key
to the very nature of the privacy notion in group signatures. Taking this into
account, it is understandable that many have tried to find some sort of
equilibrium between full anonymity (no inspection at all), and full
accountability (by means of conventional open). Thus, we aim at finding a
generic way to support multiple variants of inspection, from the common model.
Finally, actual evaluation of group signatures is the last component that is
too rigid. Namely, in group signatures, when verifying a signature, either it
has been rightly produced by a group member using his secret key and membership
credential, or not. Luckily for us, anonymous credentials have done a great job
in supporting high expressiveness into what a presentation can tell beyond just
proving knowledge of a (group) key. Thus, by integreating anonymous credentials
and group signatures in a sort of combined primitive, we get \emph{for free}
the benefits of this high expressiveness that previous work on anonymous
credentials have reached in regards to what one can prove given a credential.

%%% Local Variables:
%%% mode: latex
%%% TeX-master: "uas"
%%% End:
