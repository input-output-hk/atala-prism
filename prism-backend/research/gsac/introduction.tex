\section{Introduction}
\label{sec:introduction}

% What is the context for this research? Introduce previous works on this topic,
% in one paragraph. (Privacy-preserving identities, with accountability (APPI))
Successful identity representation, management, and usage are, on its own, among
the toughest problems in the Internet and digital communications at large
\needcite. Just as an illustration, think
of usernames and passwords, which end up being unmanageable as soon as we need
to start remembering a few \needcite; or traditional vertical PKIs, like X.509,
which are too rigid, and frequently require too much trust \needcite; or
horizontal PKIs, such as PGP, that are also too rigid, and do not scale well
\needcite. If, in addition, we require privacy, the scenario becomes even more
challenging. Technologies like OpenID and OAuth \needcite make identity more
manageable (from the point of view of service providers) through Single Sign
On solutions. But this comes at the cost of introducing centralization points
that gain way too much knowledge beyond what is actually necessary. Holding
users accountable, or extracting utility out of their actions, is trivial in
this cases -- although that comes at the cost of barely having privacy.

% What are the main approaches of previous works to achieve APPI? How are they
% lacking? If possible, use titled paragraphs to emphasize the main points.

From a cryptographic point of view, and combining both privacy and
identity/authentication requirements, group signatures (GS) and anonymous
credentials (AC) are surely among the most well known advanced schemes that
aim at addressing subsets of the previous problems. However, they typically fall
short of enabling utility or accountability without sacrificing privacy.

\paragraph{Limitations of Group Signatures.} %
For instance, in a vanilla GS scheme, a group of users can create a signature on
behalf of the group, without revealing the specific user who created it. At any
later point in time, an authority can extract from such group signature the
``identity'' of it signer (process called \emph{opening}). We find here two main
drawbacks:

\begin{enumerate}
\item This usually results in a sort of all-or-nothing approach. 
\item The concept of ``identity'' is barely defined.
\end{enumerate}

Concerning (1), either the signer is anonymous (before opening); or fully
deanonymized (after opening). Some variants on the traditional opening do not
fully deanonymize, but incorporate some other strict tradeoff. Like giving the
users full control \cite{dl21}, only allowing linking \needcite, or some
arbitrary extension such as allowing deanonymization only to the eyes of other
group members \needcite, or sequential linkability \cite{dl21}. While these
variants may be very suitable for concrete settings, it is not hard to see that
adapting them to other scenarios will be hard -- if possible at all.

Concerning (2), it is even surprising that, during the almost 30 years of
research in group signatures, the notion of ``identity'' of a group member
mostly reduces to an integer representing the order in which a user joined
the group -- or, at the most, some long-term public key. Thus, even with
notorious efforts to generalize accountability in group signatures,
re-identifying the signer from an index still forces to assume an external
ecosystem that endows that index with actual real world identyfing information.
Needless to say, this introduces further points of centralization, and
trust assumptions.

\paragraph{Limitations of Anonymous Credentials.} %
In a vanilla AC scheme, an (set of) issuer(s) grant credentials to users,
attesting for attributes that these users own. For intance, their age,
nationality, blood type, etc. Users can later prove arbitrary claims over
these attributes -- the most simple being selective disclosure; i.e., just
plainly revealing a subset of the attributes in a credential. Even from basic
selective disclosure, realistic applications can be built. The rationale being
that these attributes actually establish direct links with real identifying
information. However, accountability is barely considered, being blacklisting
of credentials almost the only explored variant.

While this may initially seem ok, closer thought quickly shows limitations. For
instance, think of a user who authenticates with an anonymous credential to get
in a night club. If the user proves being over the age limit, s/he will be let
in. However, if some time later, s/he is diagnosed positive in some contagious
disease, blacklisting does not serve any purpose. Incentivation mechanisms also
fit very well: a scenario in which a user authenticates anonymously to perform
some action that may receive some reward later will also not be implementable
with a credential scheme that only provides blacklisting. The problem is not
only practical, though. Without having suitable models to cater for these needs,
existing schemes just cannot be proven secure easily if they were to be used
to approximate solutions to the previous problems.

Given the previous, there is one first logical question: \emph{Can we come up
  with a scheme that combines the utility of AC with the accountability of
  GS?} But even then, we may run into a too rigid scheme. Thus, the follow-up
question would be: \emph{Can we design this combined scheme, such that it covers
a wide range of real world scenarios?}

%% 2 pages max for the previous; end with introduction (1-2 sentences) to what
%% would be ideal.

% Contributions first a small paragraph with an overview. Then one titled
% paragraph per contribution. Linking when possible with previous limitations.
\subsection{Our Contributions}
\label{ssec:contributions}

In this work, we make our attempt on addressing both questions.

%%% Local Variables:
%%% mode: latex
%%% TeX-master: "uas"
%%% End:
