\section{Introduction}
\label{sec:introduction}

% What is the context for this research? Introduce previous works on this topic,
% in one paragraph. (Privacy-preserving identities, with accountability (APPI))
Successful identity representation, management, and usage are, on its own, among
the toughest problems in the Internet and digital communications at large%
\footnote{``Digital identity is one of the 21st century's biggest challenges'':
  \href{https://www.maddyness.com/uk/2021/11/04/digital-identity-is-one-of-the-21st-centurys-biggest-challenges/}{https://www.maddyness.com/uk/2021/11/04/digital-identity-is-one-of-the-21st-centurys-biggest-challenges/}. Last access, May 12th, 2022.}.
Just as an illustration, think
of usernames and passwords, which end up being unmanageable as soon as we need
to start remembering a few%
\footnote{``The problem of passwords and how to deal with it'': \url{https://www.computerweekly.com/feature/The-problem-of-passwords-and-how-to-deal-with-it}. Last access,
May 12th, 2022.}; or traditional vertical PKIs, like X.509 \cite{rfc5280},
which are too rigid, and frequently require too much trust; or
horizontal PKIs, such as PGP, that are also too rigid, and are hard to manage
and scale\footnote{``What's the matter with PGP?'': \url{https://blog.cryptographyengineering.com/2014/08/13/whats-matter-with-pgp/}. Last access, May 12th, 2022}.
If, in addition, we require privacy, the scenario becomes even more
challenging. Technologies like OpenID\footnote{\url{https://openid.net/}. Last
  access, May 12th, 2022.} and OAuth\footnote{\url{https://oauth.net/}. Last
  access, May 12th, 2022.} make identity more
manageable (from the point of view of service providers) through Single Sign
On solutions. But this comes at the cost of introducing centralization points
that gain way too much knowledge beyond what is actually necessary. Holding
users accountable, or extracting utility out of their actions, is trivial in
this cases -- although that comes at the cost of barely having privacy.

% What are the main approaches of previous works to achieve APPI? How are they
% lacking? If possible, use titled paragraphs to emphasize the main points.

From a cryptographic point of view, and combining both privacy and
identity/authentication requirements, group signatures (GS) and anonymous
credentials (AC) are surely among the most well known advanced schemes that
aim at addressing subsets of the previous problems. However, they typically fall
short of enabling utility or accountability without sacrificing privacy.

\paragraph{Limitations of Group Signatures.} %
For instance, in a vanilla GS scheme, a group of users can create a signature on
behalf of the group, without revealing the specific user who created it. At any
later point in time, an authority can extract from such group signature the
``identity'' of it signer (process called \emph{opening}). We find here two main
drawbacks:

\begin{enumerate}
\item This usually results in a sort of all-or-nothing approach. 
\item The concept of ``identity'' is barely defined.
\end{enumerate}

Concerning (1), either the signer is anonymous (before opening); or fully
deanonymized (after opening). Some variants on the traditional opening do not
fully deanonymize, but incorporate some other strict tradeoff. Like giving the
users full control \cite{dl21}, only allowing de-anonymization based on the
signed message \cite{ehk+19}, or some
arbitrary extension such as allowing deanonymization only to the eyes of other
group members \cite{manu06}, or sequential linkability \cite{dl21}. While these
variants may be very suitable for concrete settings, it is not hard to see that
adapting them to other scenarios will be hard -- if possible at all.

Concerning (2), it is even surprising that, during the almost 30 years of
research in group signatures, the notion of ``identity'' of a group member
mostly reduces to an integer representing the order in which a user joined
the group -- or, at the most, some long-term public key. Thus, even with
notorious efforts to generalize accountability in group signatures,
re-identifying the signer from an index still forces to assume an external
ecosystem that endows that index with actual real world identyfing information.
Needless to say, this introduces further points of centralization, and
trust assumptions.

\paragraph{Limitations of Anonymous Credentials.} %
In a vanilla AC scheme, an (set of) issuer(s) grant credentials to users,
attesting for attributes that these users own. For intance, their age,
nationality, blood type, etc. Users can later prove arbitrary claims over
these attributes -- the most simple being selective disclosure; i.e., just
plainly revealing a subset of the attributes in a credential. Even from basic
selective disclosure, realistic applications can be built. The rationale being
that these attributes actually establish direct links with real identifying
information. However, accountability is barely considered, being blacklisting
of credentials almost the only explored variant.

While this may initially seem ok, closer thought quickly shows limitations. For
instance, think of a user who authenticates with an anonymous credential to get
in a night club. If the user proves being over the age limit, s/he will be let
in. However, if some time later, s/he is diagnosed positive in some contagious
disease, blacklisting does not serve any purpose. Incentivation mechanisms also
fit very well: a scenario in which a user authenticates anonymously to perform
some action that may receive some reward later will also not be implementable
with a credential scheme that only provides blacklisting. The problem is not
only practical, though. Without having suitable models to cater for these needs,
existing schemes just cannot be proven secure easily if they were to be used
to approximate solutions to the previous problems.

Given the previous, there is one first logical question: \emph{Can we come up
  with a scheme that combines the utility of AC with the accountability of
  GS?} But even then, we may run into a too rigid scheme. Thus, the follow-up
question would be: \emph{Can we design this combined scheme, such that it covers
a wide range of real world scenarios?}

%% 2 pages max for the previous; end with introduction (1-2 sentences) to what
%% would be ideal.

% Contributions first a small paragraph with an overview. Then one titled
% paragraph per contribution. Linking when possible with previous limitations.
\subsection{Our Contributions}
\label{ssec:contributions}

In this work, we make our attempt on addressing both questions. We start with a
direct combination of GS and AC. While this does not pose many definitional nor
technical challenges, it is a good exercise to showcase benefits obtained from
even a simple combination. This directly answers in the affirmative our first
question above. However, a direct combination does not address at all the
flexibility that we aim at -- simply because it inherits the limitations of
the vanilla GS and AC approaches we build on. Thus, we make an extra step with
the goal to add such flexibility, spanning three diretions, allowing:

\begin{itemize}
\item Arbitrary issuance policies.
\item Arbitrary signature evaluation functions.
\item Arbitrary opening functions.
\end{itemize}

As we discuss next, this directly gives place to a scheme that can be seen as a
strict generalization of many existing GS and AC schemes. To achieve it, we
encounter some definitional challenges, although the result is a simple-enough
model that still follows main practices of its GS and AC ancestors.
Interestingly, despite the challenges for the modelling, we show a generic
construction from well known building blocks.

\paragraph{\GSAC: Direct combination of GS and AC.} %
As stated, on the one hand GS lacks mechanisms to add useful identifying
information associated to signers, but has well grounded mechanisms for
accountability. On the other hand, AC is in the exact opposite situation. Thus,
taking advantage of their frequently shared approaches, we merge them in an
attempt to reach a scheme that is good in both aspects.

In fact, taking this unifying path is a low hanging fruit if one looks at recent
works. For instance, BBS+ signatures \cite{asm06,cdl16b} have been used recently
to build both group signatures \cite{gl19,dl21,fgl21}, but are also one of the
main candidates to incorporate anonymity into verifiable credentials%
\footnote{``W3C Credential Exchange using BBS+ Signatures'',
  \url{https://github.com/hyperledger/aries-rfcs/blob/main/features/0646-bbs-credentials/README.md}. Last access May 11th, 2020.}.%
The interest of the verifiable credentials community in BBS+ is not accidental,
though, as BBS+ follow the CL signatures line of work \cite{cl02}. CL signatures
allow creating a single signature over a set of messages, which can be directly
used to bundle a set of attributes (messages) into the same signature
(credential). Indeed, this was the trigger for some of the first, and possibly
the most fruitful branch of, anonymous credentials systems (see
\cite{cl01,cl02,cl04}, among others). Other very closely related signature
scheme, PS16 \cite{ps16} can
also be used to build group signatures, see \cite[Appendix A.1]{ps15} or
\cite{cdl+20}; as well as anonymous credentials \cite[Appendix A.2]{ps15} or
\cite{sms+19}.

While GS and AC schemes share many internal aspects, they do differ in some,
even beyond the obvious differences in functionality. In those cases, our
approach is to adopt the one that seems more general.
%
For instance,
anonymous credentials typically employ interactive credential presentations,
meaning that a credential owner and a verifier run an interactive
protocol by means of which the verifier gets convinced that the credential owner
indeed owns a credential meeting some requirement defined in terms of the
attributes it attests. On the other hand, in the GS domain, signing and
verification are non-interactive processes. We follow the non-interactive
approach, and then show how to trivially convert it into an interactive
protocol.
%
Another difference lies in the fact that, in group signatures, each member gets
only one membership credential, while in anonymous credentials, users can get
arbitrarily many (different) credentials. This is natural, as in group
signatures the only goal is to prove membership into a group and, thus, one
credential is enough. But, in anonymous credentials, the same user can own
multiple credentials (even from the same issuer), each attesting a different
set of attributes. On this occasion, our logical decission is to lean towards
anonymous credentials, as we aim at supporting arbitrarily many attributes per
credential.

An exception to this rule is that, in our modelling of \GSAC, we restrict to
only one issuer and one credential per signature. This is not the most general
choice, though, as in AC schemes it is frequent to assume a scenario with
multiple issuers, and let users leverage multiple credentials in the same
presentation. This is merely for definitional simplicity, and we ``fix'' it
in \UAS.

Additionally, within the AC domain, some works allow proving arbitrary claims
over the attributes in a credential \cite{bcc+09,cklm14,dmm+18,fhs19}; while
others simply allow to
selectively disclose subsets of these attributes. For \GSAC, we do the latter.
Despite not being the most general option (although it is more general than the
GS approach, as it does not even have attributes), it is enough to illustrate
the advantages of unifying both worlds. Again, in \UAS we generalize this
aspect.

Finally, modelling-wise we also have two choices: follow a UC-based definition
\cite{cane20}, or a game-based one \cite{shoup04}. Independently on the pros and
cons of each, in GS the standard is the game-based approach
\cite{bmw03,bsz05,bcc+16,kty04,ky05}, although close relatives to GS, like DAA,
prefer the UC approach \cite{cdl16,cdl16b,ccd+17,cdl17}. On the AC line of work,
while some UC models exist \cite{}, game-based definitions also seem prevalent
\cite{bcc+09,cks10,cl11,cklm12,cmz14,dmm+18,fhs19}.
Thus, the logical choice (at least, for a first step) seems to be game-based
definitions. Concretely, we take strong inspiration from \cite{bsz05} for the
GS branch, and from \cite{fhs19} for the AC one.

\subsubsection{\UAS: Generalizing \GSAC.} %
\GSAC is the first step towards achieving identity and authentication systems
with both utility and accountability, yet it presents some limitations. Some we
have already mentioned: it models a system with only one issuer, only allows
using one credential to produce a signature, and only supports selective
disclosure. We overview other limitations, that we address also with \UAS.

\paragraph{Generalizing issuance protocols.} %
In \GSAC, it is also not possible to leverage previously obtained credentials in
order to obtain new ones -- from the same issuer, or from another. Note that, if
we want to be really flexible, addressing this does not just mean letting the
user specify a set of credentials in the interactive protocol for obtaining a
new one. Concretely: What do we want to allow the user to prove with those extra
credentials? This is essentially the same case as in interactive credential
presentations. Thus, it seems reasonable to allow the same flexibility: namely,
proving arbitrary claims over the attributes in the credentials involved in the
request, without revealing them. We make only one exception, though: we do not
include support to blindly carry attributes from one credential used in the
request, to the new credential. This is indeed possible in some schemes
\cite{bcl04}. However, we argue that, by including support to merge multiple
credentials into the same signature, this is no longer necessary. And indeed
it is even realistic, as it does not make much sense to expect that an issuer
attests an attribute that he cannot access in the clear (even despite it being
attested by some other issuer). In any case, adding support to prove arbitrary
claims at issuance time does introduce additional modelling needs. Namely, we
need to give a generic predicate definition, say \fissue. Then, our model needs
to make sure that, during the interactive issuance protocol, satisfaction of
this predicate is ensured. This is not straightforward, though. One option is,
as \cite{ckl+15} does for their issuance tokens, interpret that, if no
illegitimately obtained credential is used to produce a signature, then this
does no harm; so we only need to check at signature verification time that all
involved credentials were legitimately obtained. However, this turns into a
convoluted definition, which requires to mix claims of actions performed during
credential issuance with actions performed during signing (credential
presentation in \cite{ckl+15}). Instead, we opt for creating two separate
unforgeability-related properties: one capturing security at issuance time, and
another capturing security at signature creation time. Naturally, both require
honest issuers, as we detail later on.

\paragraph{Generalizing signature generation.} %
Another generalization point is on the utility revealed at signing time. As
stated, in \GSAC we restrict to selective disclosure. This is already a
generalization with respect to GS, where this is not considered, possibly with
the only exception of \cite{khad07a,khad07b,emo09,aa14}. These works on
Attribute-Based Group Signatures (ABGS) are however too restricive:
\cite{khad07a,khad07b,aa14} only consider fixed rules to apply over an
attribute set in order to be able to create signatures (i.e., if a different
ruleset is to be applied, a new group is needed). \cite{emo09} allows some
alterations to previous rules, but instead require updates of the group public
key, which is probably unacceptable in most situations. Instead of following the
ABGS approach, we adopt the AC one, and simply require that the signer has to
prove some arbitrary claim over the attributes in the credentials s/he uses to
create a signature. This is thus orthogonal to group/issuer public key
generation, which is what forces ABGS to update them every time a change in the
rules takes place. Our approach to allow revealing arbitrary utility information
at signing time is to let signers/verifiers decide on a \emph{signature
  evaluation function}, \feval, which operates on the user secret key, the
user credentials, and the signed message. This \feval function thus defines what
verifiers (or third parties) will learn straight away from a valid signature.
Note though that signing does not only reduce to simply computing (and proving)
\feval's output. In addition, users have to prove that all the employed
credentials are bound to the same user key, and that the opening information
(more on this next) is also correct. This means that it is perfectly valid to
define \feval functions that reveal absolutely nothing (e.g., always output
$0$), and yet have a meaningful signing operation. Modelling-wise, this forces
us to restrict that, in the anonymity property, the produced challenge
signatures must produce the same \feval output, which may seem too strong,
although it seems hard to avoid. The only alternative we find is actually a
useful generalization, which is to formulate anonymity with a simulation-based
approach. While still compatible with cases in which the output of \feval has
the same in both challenge signatures, it additionally allows useful
constructions in which the output of \feval is encrypted, e.g., under the
opener's key. In this way, only the opener will learn it, when/if needed -- it
removes utility at signing time, but gives more flexibility at opening time,
without sacrificing privacy.

\paragraph{Generalizing accountability.} %
Finally, the last generalization point is on accountability. In \GSAC, we simply
added conventional opening functionality of GS. Recent works have taken a next
step towards generalizing this. For instance, in BiAS \cite{lnpy21}, the user
learns on a per signature basis whether its signature will be deanonymizable or
not, depending on whether a given policy (over the message, user identity, and
some NP witness) is satisfied. This is even generalized further in MPS \needcite%
\footnote{\textcolor{orange}{The MPS work and our \UAS generalization have been
    simultaneously designed. Coincidentally, an early MPS draft came to our
    knowledge as part of a peer-review process while we were working on \UAS.
    Consequently, even though \UAS had already essentially been specified by
    then, MPS has influenced our work, and an acknowledgement in that regard is
    in order.}},
where, somewhat like in \UAS, it is also possible to output any arbitrary
function of the user identity. In \UAS, we define a function \finsp to govern
what identifying information is revealed upon signature opening. There is one
crucial difference, though. Both BiAS and MPS rely on a vaguely defined concept
of ``identity''. That is, it
is not clear how to interpret the fact that the output of a function depends
on someone's identity, when this identity boils down to a member index or
long-term public key. On more practical terms, it seems hard to avoid having
to rely on yet extra third parties that vouch for extra identifying information
associated to those member indexes or public keys; or to use the member
indexes/public keys as labels to a somewhat publicly verifiable table with
information about that user (which seems to go against privacy). Note that, by
explicitly including \emph{meaningful} identifying information in the shape of
attested attributes into the membership credentials (which \UAS inherits from
its AC ancestor), producing an output that depends on the identity of the
signer does have ``real-world'' meaning. We also note that our signature
evaluation function has an equivalent in MPS, which also receives the signer's
identity as a parameter. Therefore, the same rationale applies.
%
Yet another generalization on the accountability side is that, by adhering to
the multiple issuer setting, the concept of group (very present in GS) gets
blurred in \UAS. Somehow naturally, we also allow multiple openers and, what is
more important in practical terms, there is no tight relation between issuer
and opener. That is, a user who obtained a credential from issuer $I_A$, can use
it to produce a signature which can be opened by opener $O_1$, and later use
that same credential to produce another signature openable by opener $O_2$.
Furthermore, while we define a syntax where only one opener is considered, the
extension to multiple openers per signature is trivial (with the corresponding
unavoidable loss in privacy).

On the definitional side, incorporating this flexible opening policies leads to
considering alternative definitions. This was already observed in prior GS works
that do not rely on traditional opening \cite{gl19,dl21,fgl21}. These works
follow the line of DAA, where a helper Identify function is added that formally
allows to test whether a user (key) is indeed the signer behind a signature
(necessary for traceability/unforgeability), or if it is not (necessary for
non-frameability). This is however not enough in our case since, as we rely on
further pieces of information to produce the utility and accountability outputs
(concretely, the message and the identity of the signer; not just its signing
key). Thus, we need to explicitly resort to extraction helper functions, as
\cite{lnpy21}. Looking at our generic construction, this seems a natural choice,
and easy to achieve via zero-knowledge proof of knowledge systems. However, it
raises an interesting theoretical question of whether it is possible to
build, or even model, systems like \UAS without relying on extraction.

\paragraph{Universality of \UAS.} %
As stated, a main goal of \UAS is to reach a general enough model that covers a
large enough portion of the functionality modelled by previous GS and AC works,
and even goes beyond (e.g., through the \fissue, \feval and \finsp functions).
To showcase the universality of \UAS, we describe how, given a generic
construction of \UAS, the \fissue, \feval, and \finsp functions can be used to
come up with generic constructions of very well known schemes in the GS and AC
domains. Directly, \UAS is a generalization of \GSAC, which we prove to be a
strict generalization of vanilla GS and vanilla AC. Consequently, vanilla GS and
AC are a strict subset of \UAS. We also show how to build MPS schemes \needcite
which is on itself a generalization of BiAS \cite{lnpy21}; as well as group
signatures with message-dependent opening \cite{ehk+19}, or delegatable
anonymous credentials \cite{bcc+09,cklm14}.
%
Interestingly, ``pure'' ring signatures \cite{rst06} can also be built from
\UAS, where with ``pure'' we mean that \UAS-based ring signatures do not require
an actual issuer -- i.e., users can act as issuers. This may even be suprising,
given that we build on
group signatures, where having no issuer of group credentials is just not
possible\footnote{Nobatly, mimicking ring signatures without issuers is
  trivially not possible in MPS, as its model does not support multiple
  issuers. This seems crucial as, in essence, a ring signature is a group
  signature where every user can become an issuer.}. Moreover, the
unforgeability notion, that requires trust on the issuer in the setting of group
signatures, is still applicable here, as long as the verifier trusts all
the public keys involved in the \UAS-based ring signature (just the same
as in conventional ring signatures). \jdv{Although, on this regard, it is
  interesting to explore how attributes come into the picture.}

\subsubsection{Is the extra overhead reasonable?} %
Finally, a logical question to ask, of the utmost pratical relevance and
especially given the generalization points we include, is whether all this
overhead is worthy, or even acceptable, in the real world. Obviously, without
concrete instantiations all we can do is abstractly argue about this. In this
regard, we make an important observation, though: the \UAS model we propose is
meaningful even when defining \fissue, \feval and \finsp to be the trivial
issuance, signature evaluation, and opening functions. More concretely, one
perfectly can: (1) set \fissue to return always $1$, implying that no extra
constrain is placed on the credentials used to request a new one; (2) define
\feval to be a constant function, meaning that no signature reveals any
identifying information on its signer; and (3), let \finsp always return the
user's public key, as in conventional group signatures; or also a constant
function, meaning that no revocation is possible. Even in those cases, we reach
a scheme that may be suitable for certain scenarios, as it would basically be
a vanilla GS or AC scheme, or GSAC, with similar computational costs.
This means that a system designer that builds on an \UAS scheme can apply
whatever tradeoff between privacy, utility, accountability, and extra costs
is most suitable to his setting, without the need to model everything from
scratch if the need to tweak something arises.


%%% Local Variables:
%%% mode: latex
%%% TeX-master: "uas"
%%% End:
