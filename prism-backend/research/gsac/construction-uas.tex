\subsection{\CUASGen: A Generic \UAS Construction}
\label{ssec:generic-construction-uas}

In this section, we give a generic construction of an \UAS scheme, based on
generic building blocks. We use three different NP relations in our generic
construction. Namely:

\begin{description}
\item[$\NIZKRel_{\Issue}$:] Set by issuers, governs via \fissue the rules for
  issuing new credentials. It is defined as $\NIZKRel_{\Issue} = \lbrace
  (\usk,\scred,\attrs_{\scred},r), (\Ccom,\attrs,\sipk_{\scred}): \Ccom =
  \CCommit(usk;r) \land \fissue(\usk,\scred,\attrs) = 1 \land \forall \cred \in
  \scred, \SBCMVerify(\ipk_{\cred},\cred,\usk,\attrs_{\cred}) = 1 \rbrace$,
  where for readability we write $\attrs_{\scred}$ as abbreviation for $\lbrace
  \attrs_{\cred} \rbrace_{\cred \in \scred}$, and similarly for $\sipk_{\scred}$.
  Basically, this relation requires that, in order to issue a credential with
  attribtes \attr, and bound to \usk, the requester has to commit to \usk and
  prove knowledge of it, as well as prove that all additional credentials needed
  for the issuing to be granted, meet some defined policy \fissue.
\item[$\NIZKRel_{\Sign}$:] Can be set by anyone but, for simplicity, assume
  verifiers define them. Governs via \feval and \finsp what utility
  information is directly revealed by signatures, or extracted by openers.
  It is defined as $\NIZKRel_{\Sign} = \lbrace (\usk,\scred,
  \attrs_{\scred},\yeval^1,\yinsp,r,r'),(\msg,\feval,\yeval^0,\ceval,\cinsp,
  \sipk_{\scred},\Eek,\widetilde{\Eek}): \ceval = \EEnc(\widetilde{\Eek},\yeval^1;r)
  \land \cinsp= \EEnc (\Eek,\yinsp;r') \land (\yeval^0,\yeval^1) = \feval(\usk,
  \scred,\msg) \land \yinsp = \finsp((\yeval^0,\yeval^1),\usk,\scred,\msg) \land
  \forall \cred \in \scred, \SBCMVerify(\ipk_{\cred},\cred,\usk,\attrs_{\cred})
  = 1) \rbrace$, with $\attrs_{\scred}$ and $\sipk_{\scred}$ as in
  $\NIZKRel_{\Issue}$. In a
  nutshell, this relation ensures that signatures will produce the correct
  signature evaluation and opening values, and that any credential used to
  create the signature is bound to the same \usk key.
\item[$\NIZKRel_{\Open}$:] Ensures that the utility information extracted by
  the \Open algorithm is correct. It is defined as $\NIZKRel_{\Open} =
  \lbrace (\osk),(\Ec,\yinsp): \yinsp = \EDec(\osk,\Ec) \rbrace$.
\end{description}

\paragraph{$\parm \gets \Setup(\secpar,\AttrSpace)$.} %
The setup process essentially consists on generating the public parameters
for all the building blocks. In detail, it parses \secpar as $(\Csecpar,
\NIZKsecpar,\SBCMsecpar,\Esecpar)$. Then, run $\Cparm \gets
\CSetup(\Csecpar)$, $\SBCMparm \gets  \SBCMSetup(\SBCMsecpar)$, $\Sparm \gets
\SSetup(\Ssecpar)$, $\Eparm \gets \ESetup(\Esecpar)$, $\NIZKcrs_{\Issue} \gets
\NIZKSetup^{\NIZKRel_{\Issue}}(\NIZKsecpar)$, $\NIZKcrs_{\Sign} \gets
\NIZKSetup^{\NIZKRel_{\Sign}}(\NIZKsecpar)$, and $\NIZKcrs_{\Open} \gets
\NIZKSetup^{\NIZKRel_{\Open}}(\NIZKsecpar)$. Return $(\Cparm,\SBCMparm,
\Sparm,\Eparm,\NIZKcrs_{\Issue},\NIZKcrs_{\Sign},\NIZKcrs_{\Open},
\AttrSpace)$. Note that many of these parameter setup algorithms can
actually be run by separate entities (respecting some commonalities like,
e.g., same pairing-friendly curves). For instance, each opener could
run its own \ESetup algorithm, or each verifier its
$\NIZKSetup^{\NIZKRel_{\Sign}}$, to advertise the signing evaluation functions
it accepts. For simplicity of exposition, we bundle them together here.

\paragraph{$(\ipk,\isk) \gets \IKeyGen(\parm,\fissue)$.} %
To generate its key pair, each issuer first parses \parm as $(\cdot,\SBCMparm,
\Sparm,\cdot,\cdot,\cdot,\cdot,\cdot)$. Then, runs $(\Svk,\Ssk) \gets
\SKeyGen(\Sparm)$, $(\SBCMvk,\SBCMsk) \gets \SBCMKeyGen(\SBCMparm)$,
$\sig_{\fissue} \gets \SSign(\Ssk,\fissue)$, $\ipk \gets (\Svk,\fissue,
\sig_{\fissue})$, $\isk \gets \Ssk$ and return $(\ipk,\isk)$.

\paragraph{$(\opk,\osk) \gets \OKeyGen(\parm,\finsp)$.} %
To generate its key pair, each opener first parses \parm as $(\cdot,\cdot,
\cdot,\Eparm,\cdot,\cdot,\cdot,\cdot)$. Then, runs $(\Svk,\Ssk) \gets \SKeyGen
(\Sparm)$, $(\Eek,\Edk) \gets \EKeyGen(\Eparm)$, $\sig_{\finsp} \gets \SSign
(\Ssk,\finsp)$, $\opk \gets (\Svk,\Eek,\finsp,\sig_{\finsp})$, and $\osk \gets
(\Ssk,\Edk)$.

\paragraph{$\usk \gets \UKeyGen(\parm)$.} %
Each user, prior to requesting credentials, generates his secret key by parsing
\parm as $(\cdot,\cdot,\cdot,\cdot,\cdot,\AttrSpace)$, and picking randomly
$\usk \getr \AttrSpace$.

\paragraph{$\langle \cred/\bot,\utrans/\bot \rangle \gets
  \langle\Obtain(\usk,\ipk,\scred,\attrs),\Issue(\isk,\sipk,\attrs)\rangle$.} %
The protocol is run between an issuer with key pair $(\ipk,\isk)$, and a user
with secret key \usk and credentials \scred, where each $\cred \in \scred$ is
issued by an issuer with
public key $\ipk_{\cred}$ (which we assume that the user can easily retrieve,
e.g., from secure storage, given \cred), and attests attributes
$\attrs_{\cred}$. The user requests a signature on a commitment to the user key,
as well as on the attributes in \attrs. In addition, the user proves that the
issuance function \fissue established by the issuer is satisfied by the
credentials in $\scred$
and its user secret key. For this, we use relation $\NIZKRel_{\Issue}$, as
defined above. The interactive protocol for a user to obtain a credential from
an issuer of the system is then simply an execution of the interactive signing
protocol of an \SBCM scheme, where the user runs $\SBCMCom(\ipk,\usk,\attrs)$,
and the issuer runs $\SBCMSign(\isk,\attrs)$; in both cases, using
$\NIZKRel_{\Issue}$ as \NIZK relation. The credential \cred produced by the user
is the result of the interactive signing protocol, and the \utrans entry for
the issuer is its transcript which is a $(\Ccom,\attrs,\sipk,\cred,\pi)$ tuple.

\iffalse
\begin{itemize}
\item \uline{User}: Commit to the user secret key with $\Ccom \gets
  \CCommit(\usk)$. Compute proof $\NIZKproof \gets
  \NIZKProve^{\NIZKRel_{\Issue}}(\NIZKcrs_{\Issue},(\usk,\scred,\attrs_{\scred}),
  (\Ccom,\attrs,\sipk_{\scred}))$. Send $(\Ccom,\NIZKproof)$ to Issuer.
\item \uline{Issuer}: Verify \NIZKproof with $\NIZKVerify^{\NIZKRel_{\Issue}}
  (\NIZKcrs_{\Issue},\NIZKproof,(\Ccom,\attrs,\sipk))$, and abort if it fails. Then,
  compute the credential by running $\cred \gets \SBCMSign(\SBCMsk,\Ccom,
  \attrs)$. Send \cred to User. Output $\utrans \gets (\Ccom,\attrs,\sipk,
  \cred,\NIZKproof)$.
\item \uline{User}: Verify the credential with $\SBCMVerify(\SBCMvk,\cred,
  \attrs \cup \lbrace \usk \rbrace)$. Reject if verification fails.
  Otherwise, return \cred.
\end{itemize}
\fi

\paragraph{$\Sig \gets \Sign(\usk,\opk,\scred,\msg,\feval)$.} %
In the signing algorithm, we make use of relation $\NIZKRel_{\Sign}$.
% 
From this, in order to produce a valid signature, the user first evaluates
$(\yeval^0,\yeval^1) \gets \feval (\usk,\scred,\msg)$, and decides whether or
not to continue with the signing process -- this may depend, e.g., on the
opening policy of \opk, as the output of \feval may influence whether the user
will be de-anonymizable or not, depending on the \finsp function in \opk.
%
The user also computes a random encryption key pair with $(\widetilde{\Eek},
\widetilde{\Edk}) \gets \EKeyGen(\Eparm)$. Then, the user parses \opk as $(\Svk,\Eek,
\finsp,\sig_{\finsp})$ and checks that $\Verify(\Svk,\sig_{\finsp},\finsp) = 1$
(note that this step may be cached), to compute $\yinsp \gets \finsp((\yeval^0,
\yeval^1),\usk,\scred,\msg)$, and separately encrypts both $\yeval^1$ and
\yinsp by running $\ceval \gets \EEnc(\widetilde{\Eek},\yeval^1;r)$ and $\cinsp \gets
\EEnc(\Eek,\yinsp; r')$ for some fresh randomness $r,r'$. Finally, the user
computes $\NIZKproof \gets \NIZKProve^{\NIZKRel_{\Sign}}(\NIZKcrs_{\Sign},(\usk,
\scred, \attrs_{\scred},\yeval^1,\yinsp,r,r'),(\msg,\feval,\yeval^0,\ceval,\cinsp,
\sipk_{\scred},\Eek,\widetilde{\Eek}))$ and outputs $(\sig = (\NIZKproof,\ceval,
\cinsp),\yeval^0)$.

\paragraph{$1/0 \gets \Verify(\opk,\sipk,\Sig,\msg,\feval)$.} %
The ``cryptographic'' side of the verification essentially consists on checking
the NIZK proof. That is, parse \Sig as $(\sig = (\NIZKproof,\cinsp),\yeval,
\ceval)$ and check whether $\NIZKVerify(\NIZKcrs,\NIZKproof,(\msg,\feval,\yeval,
\ceval,\cinsp,\sipk,\opk)) = 1$. In addition, the verifier may further check
whether \yeval meets its needs.

\paragraph{$(\yinsp,\NIZKproof)/\bot \gets
  \Open(\osk,\sipk,\Sig,\msg,\feval)$.} %
Here we leverage relation $\NIZKRel_{\Open}$.
%
To open a signature, the opener first verifies the signature by running $\Verify
(\opk,\sipk, \Sig,\msg,\feval)$. If verification succeeds, it parses
\Sig as $(\sig=(\NIZKproof,\cinsp),\yeval,\ceval)$, decrypts \Ec by running $\yinsp
\gets \EDec(\osk,\cinsp)$, and computes $\NIZKproof_{\Open} \gets
\NIZKProve^{\NIZKRel_{\Open}}(\NIZKcrs_{\Open},\osk,(\cinsp,\yinsp))$. It
returns $(\yinsp,\NIZKproof_{\Open})$.

\paragraph{$1/0 \gets \Judge(\opk,\yinsp,\NIZKproof,\Sig,\msg)$.} %
To assess the validity of an opening proof, first check the signature
by running $\Verify(\opk,\sipk,\Sig,\msg,\feval)$. If the check succeeds,
parse \Sig as $((\cdot,\cinsp),\cdot,\cdot)$ and verify \NIZKproof with
$\NIZKVerify(\NIZKcrs_{\Open},\NIZKproof,(\cinsp,\yinsp))$. Accept it the NIZK
verification passes, and reject otherwise.

%\subsection{Correctness and Security of \CUASGen}
\label{ssec:security-uas}

First, we define the \Identify, \ExtractIssue and \ExtractSign functions that
are needed for some of the properties to be meaningful, in
\figref{fig:helper-funcs}.

\begin{figure}[ht!]
  \begin{minipage}[t]{\textwidth}
    \procedure{$\ExtractIssue(\parm,\trans)$}{%
      \textrm{Parse \parm as $(\cdot,\cdot,\cdot,\cdot,\NIZKcrs_{\Issue},\cdot,
        \cdot,\cdot)$; $\NIZKcrs_{\Issue}$ as $(\NIZKcrs,\NIZKtrap)$; and
        \trans as $(\Ccom,\attrs,\sipk,\cred,\NIZKproof)$} \\
      \pcif \NIZKVerify(\NIZKcrs,\NIZKproof,(\Ccom,\attrs,\sipk)): 
      \pcreturn \bot \\
      (\usk,\scred,\attrs_{\scred}) \gets \NIZKExtract(\NIZKcrs,\NIZKtrap,
      (\Ccom,\attrs,\sipk),\NIZKproof) \\
      \pcreturn (\usk,\scred,\attrs_{\scred}) \\
    }
    
    \procedure{$\ExtractSign(\parm,\oid,\siid,\sig,\yeval,\msg,\feval)$}{%
      \textrm{Parse \parm as $(\cdot,\cdot,\cdot,\cdot,\cdot,\NIZKcrs_{\Sign},
        \cdot,\cdot)$; $\NIZKcrs_{\Sign}$ as $(\NIZKcrs,\NIZKtrap)$; and
        \sig as $(\NIZKproof,\Ec)$} \\
      \textrm{Parse $\PUBOK[\oid]$ as $(\opk,\cdot)$ and let $\sipk \gets
        \PUBIK[\siid]$} \\
      \pcif \NIZKVerify(\NIZKcrs,\NIZKproof,(\msg,\feval,\yeval,\Ec,
      \sipk,\opk)): \pcreturn \bot \\
      (\usk,\scred,\attrs_{\scred},\yinsp,r) \gets \NIZKExtract(\NIZKcrs,\NIZKtrap,
      (\msg,\feval,\yeval,\Ec,\sipk,\opk),\NIZKproof) \\
      \pcreturn (\usk,\scred,\attrs_{\scred},\yinsp) \\
    }
    
    \procedure{$\Identify(\usk,\attrs_{\cred},\cred)$}{%
      \pcreturn \SBCMVerify(\ipk_{\cred},\cred,\attrs_{\cred} \cup
      \lbrace \usk \rbrace) \\
    }    
  \end{minipage}
  \label{fig:helper-funcs}
  \caption{Definition of helper functions \Identify, \ExtractIssue and
    \ExtractSign, for \CUASGen.}
\end{figure}

\begin{theorem}[Correctness of \CUASGen]
  \label{thm:correctness-uas}
  If the underlying schemes for vector commitments, encryption, digital
  signatures, signatures on blocks of committed messages, and NIZKs are
  correct, our generic construction \CUASGen satisfies correctness as
  defined in \defref{def:correctness-uas}.
\end{theorem}

\begin{proof}[\thmref{thm:correctness-uas}]
  \todo{XXX}
\end{proof}

\begin{theorem}[Anonymity of \CUASGen]
  \label{thm:anonymity-uas}
  If the underlying encryption scheme is \todo{IND-CCA}, the vector commitment
  scheme is \todo{binding}, the scheme for signatures on blocks of committed
  messages is \todo{XXX}, and the NIZKs used for $\NIZKRel_{\Issue},
  \NIZKRel_{\Sign}$, and $\NIZKRel_{\Inspect}$ are \todo{zero-knowledge} and
  \todo{simulation-sound?}, our \CUASGen construction satisfies anonymity as
  defined in \defref{def:anonymity-uas}.
\end{theorem}

\begin{proof}[\thmref{thm:anonymity-uas}]
\end{proof}

\begin{theorem}[Issuance unforgeability of \CUASGen]
  \label{thm:issue-forge-uas}
  If the underlying scheme for signatures on blocks of committed messages is
  existentially unforgeable, and the NIZK used for $\NIZKRel_{\Issue}$ is
  simulation extractable and sound, then our \CUASGen construction satisfies
  issuance unforgeability as defined in \defref{def:issue-forge-uas}, except
  with negligible probability. \todo{EUF of \SBCM?}
\end{theorem}

\todo{\usk belongs to \AttrSpace! I think this can lead to malleability attacks.
  Make them disjoint?}

\begin{proof}[\thmref{thm:issue-forge-uas}]
  We show that the probability that \fissue outputs $0$ is negligible, as well
  as the probability that the extracted \usk is not the one that was used to
  request some of the credentials employed to obtain the credential specified by
  the adversary.
  %
  For this purpose, we define two games, $G_0=\ExpForgeIssue$, and $G_1$, which
  is exactly the same, but where, within the \Setup algorithm, we replace
  $\NIZKSetup^{\Issue}$ with $\NIZKSimSetup^{\Issue}$. As per the definition of
  \NIZK in \appref{sapp:nizk}, both games are indistinguishable.

  Now, observe that the adversary is required to output a credential
  identifier for which associated entries in \trans and \CRED exist; moreover,
  if such a credential was produced by an issuer, we must have access to those
  entries, as issuers are assumed to be honest.
  %
  Then, given that $\NIZKRel_{\Issue}$ is knowledge extractable, in game $G_1$
  we can apply the \NIZKExtract function, which produces a tuple $(\usk,\scred,
  \attrs_{\scred})$ from $\utrans = (\Ccom,\attrs,\sipk,\cred,\NIZKproof)$.
  %
  Since \NIZKproof is accepted by \ExtractIssue, from the soundness of \NIZK and
  existential unforgeability of \SBCM, we know that all $\cred \in \scred$ are
  valid signatures over \usk, and their respective $\attrs_{\cred}$. Thus,
  \Identify returns $1$ for all $(\usk,\attrs_{\cred},\cred)$ tuples.
  Moreover, all the credentials in \scred given to \fissue belong to the same
  user, who is the owner of \usk.
  %
  Finally, since issuers are honest, we know that $\ATTR[\cid] = \attrs$ and,
  consequently, $\fissue(\usk,\scred,\ATTR[\cid]) = \fissue(\usk,\scred,\attrs)
  = 1$, due to the soundness of \NIZK.
  %
  \qed
\end{proof}

\begin{theorem}[Signing unforgeability of \CUASGen]
  \label{thm:sign-forge-uas}
  If the underlying NIZK scheme for $\NIZKRel_{\Sign}$ is sound and simulation
  extractable, the NIZK scheme for $\NIZKRel_{\Inspect}$ is sound and simulation
  extractable, and \SBCM is existentially unforgeable, then our \CUASGen
  construction satisfies signing unforgeability as defined in
  \defref{def:sign-forge-uas}, except with negligible probability.
\end{theorem}

\begin{proof}[\thmref{thm:sign-forge-uas}]
  As for \thmref{thm:issue-forge-uas}, we define two games, $G_0=\ExpForgeSign$,
  and $G_1$, which is exactly the same, but where, within the \Setup algorithm,
  we replace $\NIZKSetup^{\Sign}$ with $\NIZKSimSetup^{\Sign}$. As per the
  definition of \NIZK in \appref{sapp:nizk}, both games are indistinguishable.

  From $G_1$, and in order to define the winning conditions for the adversary
  in the signing unforgeability game, consider the following events:

  \begin{description}
  \item[$V$.] Where $V = \Verify(\opk,\sipk,\sig,\yeval,\msg,\feval) = 1$.
  \item[$J$.] Where $J = \Judge(\opk,\sipk,\yinsp,\iproof,\sig,\yeval,\msg,
    \feval) = 0$.
  \item[$L$.] Where $L = (\msg,\feval,\yeval,\Ec,\sipk,\opk) \in
    \NIZKLang^{\Sign}$.    
  \item[$I$.] Where $I = \exists \cred \in \scred~\st~\Identify(\usk,
    \attrs_{\cred},\msg) = 0$.
  \end{description}

  $\adv$ wins if $V \land (J \lor I) = (\overline{L} \land V \land (J \lor I))
  \lor (L \land V \land (J \lor I))$.
  %
  $V$ implies that $(\msg,\feval,\yeval,\Ec,\sipk,\opk) \in \NIZKRel^{\Sign}$.
  Thus, after soundness of $\NIZK^{\Sign}$, the probability of $\overline{L}
  \land V \land (J \lor I)$ is negligible in the security parameter.
  %
  For $(L \land V \land (J \lor I))$ to be satisfied, there are three cases:
  \begin{enumerate}
  \item $L \land V \land J \land \overline{I}$. The game returns 1 in step 6.
  \item $L \land V \land J \land I$.  The game returns 1 in step 6.
  \item $L \land V \land \overline{J} \land I$. The game returns 1 in step 10. 
  \end{enumerate}

  In case 1, $L \land V$ implies that $(\msg,\feval,\yeval,\Ec,\sipk,\opk) \in
  \NIZKRel^{\Sign}$. More concretely, soundness of $NIZK^{\Sign}$ implies that
  $\Ec = \EEnc(\opk,\yinsp;r)$, for $\yinsp$ and $r$ known to the signer. Since
  $(\yinsp,\iproof)$ is generated honestly by the challenger from
  $(\opk,\sipk,\sig = (\NIZKproof_{\Sign},\Ec),\yeval,\msg)$, correctness of
  public key encryption implies that $\EDec(\osk,\Ec) = \yinsp$. Consequently,
  the probability of $L \land V \land J \land \overline{I}$ = 0, as \Judge
  checks precisely that \Ec is a correct encryption of \yinsp under \opk.

  The analysis for case 2 is the same as for case 1.

  For case 3, $(\msg,\feval,\yeval,\Ec,\sipk,\opk) \in \NIZKRel^{\Sign}$,
  $\Judge(\opk,\sipk,\yinsp,\iproof,\sig,\yeval,\msg,\feval)
  = 1$, but there exists some credential \cred for which $\Identify(\usk,
  \attrs_{\cred},\msg)=0$. Note that \usk, $\attrs_{\cred}$ and \cred (for all
  $\cred \in \scred$) are output by \ExtractSign. Thus, after the
  simulation-extractability property and soudness of $\NIZK^{\Sign}$, $(\msg,
  \feval,\yeval,\Ec,\sipk,\opk) \in \NIZKRel^{\Sign}$, which more concretely
  means that $\SBCMVerify(\ipk_{\cred},\cred,\attrs_{\cred} \cup \lbrace \usk
  \rbrace) = 1 = \Identify(\usk,\attrs_{\cred},\cred)$, for all $\cred \in
  \scred$. The probability of case 3 is therefore $0$.
  %
  Moreover, since \SBCMVerify returns $1$ for all $\cred \in \scred$, it must
  be that all of them were obtained via queries to the \ISSUE or \OBTISS
  oracles. Otherwise, if there exists some \cred that was not obtained via
  a call to these oracles, the pair $(\lbrace \usk \rbrace \cup \attrs_{\cred},
  \cred)$ constitutes an existential forgery of \SBCM.

  Cases 1, 2 and 3 above account for winning conditions at steps 6 and 10.
  % 
  Additionally, $\adv$ wins at step 8 if $\feval(\usk,\scred,\msg) \neq \yeval$,
  where \yeval is the value output by the adversary in step 2. However, since
  $\Verify(\opk,\sipk,\sig,\yeval,\msg,\feval) = 1$, soundness of $NIZK^{\Sign}$
  implies that this has negligible probability.
  %
  Similarly, $\adv$ wins at step 9 if (1) $\finsp(\yeval,\usk,\scred,\msg) \neq
  \yinsp$, where \yinsp is the value output by \Inspect at line 5; or if (2)
  $\yinsp \neq \yinsp'$, where $\yinsp'$ is the value extracted by \ExtractSign
  at step 7. For (1), soundness of $\NIZK^{\Sign}$ ensures that \yinsp is the
  correct evaluation of \finsp, whereas soundness of $\NIZK^{\Inspect}$ and
  correctness of the public key encryption ensure that this is also the value
  output by \Inspect. Thus, the probability of $\adv$ winning the game because
  of (1) is negligible. Finally, for (2), simulation-extractability of
  $\NIZK^{\Inspect}$ and correctness of the public key encryption ensure that
  the $\yinsp'$ value extracted by \NIZKExtract matches the value produced by
  \Inspect.
  %
  \qed
\end{proof}

\begin{theorem}[Non-frameability of \CUASGen]
  \label{thm:frame-uas}
  If the underlying NIZK schemes are zero-knowledge and the scheme used for
  $\NIZK^{\Sign}$ is simulation-extractable, and the commitment scheme is
  hiding, then our \CUASGen construction satisfies non-frameability as defined
  in \defref{def:frame-uas}, except with negligible probability.
\end{theorem}

\begin{proof}[\thmref{thm:frame-uas}]
  We prove that, given an adversary $\adv$ against non-frameability of \CUASGen,
  we can build an adversary \advB that breaks the hiding property of the
  underlying commitment scheme, with non-negligible probability.

  We start from $G_0=\ExpNonframe$. $\adv$ makes queries to the oracles in
  \Oframe.

  For $G_1$, within \Setup, we replace the \Setup algorithms for the three
  NIZKs (\Issue, \Sign and \Inspect) with their corresponding \SimSetup
  variants. Consequently, the corresponding queries to \Prove are also
  simulated via the simulator. By the zero-knowledge property of the NIZK
  systems, $G_1$ is indistinguishable from $G_0$.
  
  We build adversary \advB against hiding of commitments from $G_1$ against
  non-frameability. In the hiding game (see \figref{fig:com-games}), \advB first
  picks two messages $\msg_0$ and $\msg_1$, and then receives a commitment \Ccom
  of $\msg_b$. Let \advB pick both $\msg_0$ and $\msg_1$ from \AttrSpace. Then,
  \advB initializes $G_1$ for $\adv$ against non-frameability, and randomly
  picks a number $u \getr [1,q]$, where $q$ can be as large as \advB wants, but
  will be the maximum number of honest users to let $\adv$ add to the game.
  Then, when $\adv$ asks to create the $u$-th user, \advB ignores the call to
  \UKeyGen. For every call that $\adv$ makes to the \OBTAIN oracle associated to
  the $u$-th user, \advB uses the commitment \Ccom received as challenge in its
  game against the hiding property of commmitments, and uses it as commitment to
  the $u$-th user's \usk. \advB then simulates all the NIZK proofs associated to
  the $u$-th user, in calls to \OBTAIN, \SIGN and \INSPECT. Again, due to the
  zero-knowledge property of the associated NIZK systems, the outputs of the
  moddified oracles are indistinguishable from the original outputs (as \Ccom
  is a valid commitment of a user secret key). Eventually, $\adv$ outputs a
  $(\sig,\yeval,\msg,\feval,\yinsp,\iproof)$ tuple that is accepted by \Verify
  and \Judge. Due to simulation extractability of $NIZK^{\Sign}$, \ExtractSign
  must be able to produce a $(\usk,\scred,\attrs_{\scred},\yinsp')$ tuple. Since
  $\adv$ wins the
  non-frameability game with non-neglibible probability, with probability $1/q$,
  the \usk value belongs to the $u$-th honest user, so it must be equal to
  either $\msg_0$ or $\msg_1$; \advB responds accordingly to its challenge in
  the game for the hiding property of the commmitment scheme. By assumption,
  \advB wins with non-negligible probability.
  %
  \qed
\end{proof}


%%% Local Variables:
%%% mode: latex
%%% TeX-master: "uas"
%%% End:


%%% Local Variables:
%%% mode: latex
%%% TeX-master: "uas"
%%% End:
