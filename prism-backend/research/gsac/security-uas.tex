\subsection{Correctness and Security of \CUASGen}
\label{ssec:security-uas}

First, we define the \ExtractIssue, \ExtractSign, \IdentifyCred and \IdentifyUK
functions that are needed for some of the properties to be meaningful, in
\figref{fig:helper-funcs}.

\begin{figure}[ht!]
  \begin{minipage}[t]{\textwidth}
    \procedure{$\ExtractIssue(\parm,\utrans)$}{%
      \textrm{Parse \parm as $(\cdot,\cdot,\cdot,\cdot,\NIZKcrs_{\Issue},\cdot,
        \cdot,\cdot)$; $\NIZKcrs_{\Issue}$ as $(\NIZKcrs,\NIZKtrap)$; and
        \utrans as $(\Ccom,\sipk,\cred,\NIZKproof)$} \\
      \pcif \NIZKVerify(\NIZKcrs,\NIZKproof,(\Ccom,\attrs,\sipk)): 
      \pcreturn \bot \\
      (\usk,\scred,\attrs_{\scred}) \gets \NIZKExtract(\NIZKcrs,\NIZKtrap,
      (\Ccom,\attrs,\sipk),\NIZKproof) \\
      \pcreturn (\usk,\attrs,\scred,\attrs_{\scred}) \\
    }
    
    \procedure{$\ExtractSign(\parm,\oid,\siid,\sig,\yeval,\msg,\feval)$}{%
      \textrm{Parse \parm as $(\cdot,\cdot,\cdot,\cdot,\cdot,\NIZKcrs_{\Sign},
        \cdot,\cdot)$; $\NIZKcrs_{\Sign}$ as $(\NIZKcrs,\NIZKtrap)$; and
        \sig as $(\NIZKproof,\Ec)$} \\
      \textrm{Parse $\PUBOK[\oid]$ as $(\opk,\cdot)$ and let $\sipk \gets
        \PUBIK[\siid]$} \\
      \pcif \NIZKVerify(\NIZKcrs,\NIZKproof,(\msg,\feval,\yeval,\Ec,
      \sipk,\opk)): \pcreturn \bot \\
      (\usk,\scred,\attrs_{\scred},\yinsp,r) \gets \NIZKExtract(\NIZKcrs,
      \NIZKtrap, (\msg,\feval,\yeval^0,\ceval,\cinsp,\sipk,\opk,\widetilde{\Eek}),
      \NIZKproof) \\
      \pcreturn (\usk,\scred,\attrs_{\scred},\yeval^1,\yinsp) \\
    }
    
    \procedure{$\IdentifyCred(\usk,\attrs_{\cred},\cred)$}{%
      \pcreturn \SBCMVerify(\ipk_{\cred},\cred,\usk,\attrs_{\cred}) \\
    }

    \procedure{$\IdentifyUK(\uid,\usk)$}{%
      \pcif \uid \in \HU: \pcreturn \usk = \UK[\uid] \\
      \pcfor \cid~\suchthat~\CRED[\cid] = (\uid,\cdot,\cdot,\cdot,\cdot,\cdot) \\
      \pcind (\usk',\cdot,\cdot,\cdot) \gets \ExtractIssue(\parm,\trans[\cid]) \\
      \pcind \pcif \usk = \usk': \pcreturn 1 \\
      \pcreturn 0
    }
  \end{minipage}
  \label{fig:helper-funcs}
  \caption{Definition of helper functions \ExtractIssue, \ExtractSign,
    \IdentifyCred, and \IdentifyUK, for \CUASGen.}
\end{figure}

\begin{theorem}[Correctness of \CUASGen]
  \label{thm:correctness-uas}
  If the underlying schemes for commitments, public-key encryption and \SBCM,
  are correct,
  as well as the NIZKs for $\NIZKRel_{\Issue}$, $\NIZKRel_{\Sign}$, and
  $\NIZKRel_{\Open}$, our generic construction \CUASGen satisfies correctness as
  defined in \defref{def:correctness-uas}.
\end{theorem}

\begin{proof}[\thmref{thm:correctness-uas}; Correctness of \CUASGen]
  By correctness of \SBCM and the NIZK for $\NIZKRel_{\Sign}$, the signature
  produced at line 5 of \ExpCorrect is accepted at line 6 by \Verify.
  Moreover, all the credentials employed to honestly produce the signature,
  identified with \scid, meet their respective issuance policies due to
  correctness of the NIZK for $\NIZKRel_{\Issue}$, so no $\fissue^\cid$ check
  returns $0$ at line 9. Similarly, as $\feval \in \famfeval$ is checked at
  line 3, and due to correctness of the NIZK for $\NIZKRel_{\Sign}$, the
  output of \feval matches $\yeval^0$ at line $11$, which must have been
  computed over $\usk=\UK[\uid]$, as in line $10$, due to correctness of the
  commitment scheme. Finally, correctness of the NIZKs for $\NIZKRel_{\Sign}$
  and $\NIZKRel_{\Open}$, and correctness of the encryption scheme, ensure that
  \Judge accepts the proof produced by \Open, and \yinsp is the correct value
  for the chosen $\finsp^{\oid}$.
\end{proof}

\begin{theorem}[Signature anonymity of \CUASGen]
  \label{thm:sign-anonymity-uas}
  If the NIZK system used for $\NIZKRel_{\Sign}$ is zero-knowledge, and the
  public-key encryption scheme is IND-CCA secure, our \CUASGen construction
  satisfies signature anonymity as defined in \defref{def:sign-anonymity-uas}.
\end{theorem}

\begin{proof}[\thmref{thm:sign-anonymity-uas}; Signature anonymity of \CUASGen]
  \iffalse
  First, we define \SimSetup, \SIMOBTAIN, \SIMSIGN and \SIMOPEN as in
  \figref{fig:anon-sim}.

  \begin{figure}[ht!]
    \begin{minipage}[t]{\textwidth}
      \procedure{$\SimSetup(\secpar,\AttrSpace)$}{%
        \textrm{Parse}~\secpar~\textrm{as}~(\cdot,\secpar_{\NIZK},\cdot,\cdot) \\
        (\NIZKcrs_{\Issue},\NIZKtrap_{\Issue})
        \gets \NIZKSimSetup^{\NIZKRel_{\Issue}}(\secpar_{\NIZK}) \\
        (\NIZKcrs_{\Sign},\NIZKtrap_{\Sign})
        \gets \NIZKSimSetup^{\NIZKRel_{\Sign}}(\secpar_{\NIZK}) \\
        (\NIZKcrs_{\Open},\NIZKtrap_{\Open})
        \gets \NIZKSimSetup^{\NIZKRel_{\Open}}(\secpar_{\NIZK}) \\
        \textrm{Compute}~\parm \gets (\Cparm,\SBCMparm,\Sparm,\Eparm)~
        \textrm{as in}~\Setup \\
        \pcreturn ((\parm,\NIZKcrs_{\Issue},
        \NIZKcrs_{\Sign},\NIZKcrs_{\Open},\AttrSpace),
        (\NIZKtrap_{\Issue},\NIZKtrap_{\Sign},\NIZKtrap_{\Open})) \\
      }

      \begin{minipage}[t]{\textwidth}
        \begin{minipage}[t]{0.55\textwidth}
          \procedure{$\SIMOBTAIN(\cid,\uid,\iid,\attrs,\scid)$}{%
            \pcif \uid \notin \HU: \pcreturn \bot \\
            \pcif \iid \notin \CI: \pcreturn \bot \\
            \pcif \CRED[\cid] \neq \bot: \pcreturn \bot \\
            \sipk_{\scred} \gets \PUBIK[\scid] \\
            \langle \cred, \cdot \rangle \gets
            \langle SimObt(\attrs,\PUBIK[\iid],\sipk_{\scred}),\adv \rangle \\
            \CRED[\cid] \gets (\uid, \cred, \iid, \attrs, \scid, \siid) \\
            \pcreturn \top \\      
          }
        \end{minipage}
        \begin{minipage}[t]{0.45\textwidth}
          \procedure{$SimObt(\attrs,\ipk,\sipk_{\scred})$}{%
            \textrm{Runs $\langle \SBCMCom^{\NIZKRel_{\Issue}},\adv \rangle$} \\
            \pcind \textrm{using $\NIZKSim^{\NIZKRel_{\Issue}}
              (\NIZKcrs_{\Issue},\NIZKtrap_{\Issue},\cdot)$} \\
            \pcind \textrm{instead of $\NIZKProve^{\NIZKRel_{\Issue}}
              (\NIZKcrs_{\Issue},\cdot,\cdot)$}
          }
        \end{minipage}
      \end{minipage}

      \begin{minipage}[t]{\textwidth}
        \begin{minipage}[t]{0.55\textwidth}
          \procedure{$\SIMSIGN(\oid,\uid,\scid,\msg,\feval)$}{%
            \pcif \uid \notin \HU: \pcreturn \bot \\
            \pcif \feval \notin \famfeval: \pcreturn \bot \\
            \Sig \gets SimSign(\PUBOK[\oid],\msg,\feval) \\
            \SIG[\uid] \gets \SIG[\uid] \cup
            \lbrace (\oid,\scid,\Sig,\msg,\feval) \rbrace \\
            \pcreturn \Sig \\
          }
        \end{minipage}
        \begin{minipage}[t]{0.45\textwidth}
          \procedure{$SimSign(\opk,\msg,\feval)$}{%
            \textrm{Like}~\Sign()~\textrm{using} \\
            \pcind \NIZKSim^{\NIZKRel_{\Sign}}(\NIZKcrs_{\Sign},
            \NIZKtrap_{\Sign},\cdot) \\
            \pcind \textrm{instead of}~\NIZKProve^{\NIZKRel_{\Sign}}
            (\NIZKcrs_{\Sign},\cdot,\cdot)        
          }
        \end{minipage}
      \end{minipage}

      \begin{minipage}[t]{\textwidth}
        \begin{minipage}[t]{0.55\textwidth}
          \procedure{$\SIMOPEN(\oid,\Sig,\msg)$}{%
            \textrm{Let}~\uid~\textrm{be s.t.}~(\oid,\scid,\Sig,\msg,\feval)
            \in \SIG[\uid] \\
            % \textrm{Parse}~\Sig~\textrm{as}~(\sig,\yeval,\ceval) \\
            (\yinsp,\iproof) \gets
            SimOpen(\PRVOK[\oid],\PUBIK[\scid],\Sig,\msg,\feval) \\
            \pcif \CSIG[\Sig] \neq \bot: \\
            \pcind \textrm{Parse $\CSIG[\Sig]$ as $(\oid,\cuid_b,\scid_b,\msg,
              \feval$} \\
            \hspace*{83pt}\cuid_{1-b},\cSig_{1-b},\scid_{1-b}) \\
            \pcind (\yinsp',\iproof') \gets
            \Open(\PRVOK[\oid],\IK[\siid],\\
            \hspace*{107pt} \cSig_{1-b},\msg,\feval) \\
            \pcind \pcif \yinsp' \neq \yinsp: \pcreturn \bot \\
            \pcreturn (\yinsp,\iproof)          
          }
        \end{minipage}
        \begin{minipage}[t]{0.45\textwidth}
          \procedure{$SimOpen(\opk,\msg,\feval)$}{%
          }
        \end{minipage}
      \end{minipage}    
    \end{minipage}
    \label{fig:sim-anon}
    \caption{Definition of \SimSetup, \SIMOBTAIN, \SIMSIGN, and \SIMOPEN for
      anonymity in \CUASGen.}
  \end{figure}
  \fi

  In this proof, we restrict to the case in which the adversary can only make
  one query to the challenge oracle. Note however that the generalization to
  polynomially many queries given in \cite{bsz05} applies here too (with the
  corresponding security loss). Thus, proving security for one query to the
  challenge oracle is enough.

  We start from $G_0=\ExpSigAnonb$, and define game $G_1$ to be exactly the same
  as $G_0$, except that, within the $\Setup$ algorithm, we replace
  $\NIZKSetup^{\Sign}$ with $\NIZKSimSetup^{\Sign}$. By zero-knowledgeness,
  $G_1$ is indistinguishable from $G_0$.
  
  From $G_1$, we consider $G^0_1$, which we define to be $G_1$, for $b=0$
  (i.e., \ExpSigAnonz, using $\NIZKSimSetup^{\Sign}$). The challenge sent to the
  adversary is $(\csig_0,\yeval) \gets \Sign(\PRVUK[\cuid_0],\PUBOK[\oid],
  \CRED[\scid_0],\msg,\feval)$, where $\csig_0 = (\pi_0,\Ec_{\yinsp})$, with
  $\pi_0 = \NIZKProve^{\NIZKRel_{\Sign}}(\NIZKcrs_{\Sign},(\msg,\feval,\yeval,
  \ceval,\cinsp,\PUBIK[\scid_0],\widetilde{\Eek},\PUBOK[\oid]),(\PRVUK[\cuid_0],
  \CRED[\scid_0],\attrs_{\scid_0},\yeval^1,\yinsp,r,r'))$, $\ceval = \EEnc
  (\widetilde{\Eek},\yeval^1;r)$, and $\cinsp = \EEnc(\PUBOK[\oid],\yinsp;r')$.
  % 
  Further, we build $G_2^0$ from $G_1^0$ by simulating the proof $\pi_0$. That
  is, in $G_2^0$, $\csig_0 = (\pi_0^s,\ceval,\cinsp)$, where $\pi^s_0 =
  \NIZKSim^{\NIZKRel_{\Sign}}(\NIZKcrs_{\Sign},\NIZKtrap,(\msg,\feval,\yeval,
  \ceval,\cinsp,\PUBIK[\scid_0],\PUBOK[\oid]))$. By zero-knowledgeness
  of $\NIZK^{\Sign}$, $G_2^0$ is indistinguishable from $G_1^0$.

  Similarly, we consider $G_1^1$ and $G_2^1$. That is, $G_1^1$ is $G_1$
  for $b=1$, where the challenge
  sent to the adversary is $(\csig_1,\yeval) \gets \Sign(\PRVUK[\cuid_1],
  \PUBOK[\oid],\CRED[\scid_1],\msg,\feval)$, where $\csig_1 = (\pi_1,\ceval,
  \cinsp)$, with $\pi_1 = \NIZKProve^{\NIZKRel_{\Sign}}(\NIZKcrs_{\Sign},
  (\msg,\feval,\yeval,\ceval,\cinsp,\PUBIK[\scid_1],\widetilde{\Eek},
  \PUBOK[\oid]),(\PRVUK[\cuid_1],\CRED[\scid_1],\attrs_{\scid_1},\yeval^1,
  \yinsp,r,r'))$, $\ceval = \EEnc(\widetilde{\Eek},\yeval^1;r)$, and
  $\cinsp = \EEnc(\PUBOK[\oid],\yinsp;r')$. As before, $G_2^1$ is built from
  $G_1^1$, simulating $\pi_1$. That is, in $G_2^1$, $\csig_1 = (\pi_1^s,\ceval,
  \cinsp)$, where $\pi^s_1 = \NIZKSim^{\NIZKRel_{\Sign}}(\NIZKcrs_{\Sign},
  \NIZKtrap,(\msg,\feval,\yeval,\ceval,\cinsp,\PUBIK[\scid_1],\widetilde{\Eek},
  \PUBOK[\oid]))$. Again, by zero-knowledge of $\NIZK^{\Sign}$, $G_2^1$ is
  indistinguishable from $G_1^1$. Note also that $G_2^1$ and $G_2^0$ are
  indistinguishable, due to the IND-CCA property of the encryption scheme
  (so, the \ceval values in $\pi^s_0$ and $\pi^s_1$ are indistinguishable),
  in the  challenge oracle used in the anonymity game, we restrict to
  $\PUBIK[\scid_0] = \PUBIK[\scid_1]$, and the respective \cinsp values encrypt
  the same \yinsp value.

  Finally, consider the definition of $\AdvSigAnon=|\Pr\lbrack
  \ExpSigAnono(1^\secpar)=1\rbrack-\Pr\lbrack\ExpSigAnonz(1^\secpar)=1\rbrack|$. As
  argued, $G_1$ is indistinguishable from $\ExpSigAnonb$, thus
  $\AdvSigAnon \approx |\Pr\lbrack G_1^1(1^\secpar)=1\rbrack-\Pr\lbrack
  G_1^0(1^\secpar)=1\rbrack| \approx
  |\Pr\lbrack G_2^1(1^\secpar)=1\rbrack-\Pr\lbrack
  G_2^0(1^\secpar)=1\rbrack|$. Since $G_2^1=G_2^0$, it follows that
  \AdvSigAnon is negligible.
  % 
  \qed
\end{proof}

\begin{theorem}[Issuance unforgeability of \CUASGen]
  \label{thm:issue-forge-uas}
  If the underlying NIZK used for $\NIZKRel_{\Issue}$ is zero-knowledge,
  simulation extractable and sound, then our \CUASGen construction satisfies
  issuance unforgeability as defined in \defref{def:issue-forge-uas}.
\end{theorem}

\begin{proof}[\thmref{thm:issue-forge-uas}; Issuance unforgeability of \CUASGen]
  We show that the probability that \fissue outputs $0$ is negligible, as well
  as the probability that the extracted \usk is not the one that was used to
  request some of the credentials employed to obtain the credential specified by
  the adversary.
  %
  For this purpose, we define two games, $G_0=\ExpForgeIssue$, and $G_1$, which
  is exactly the same, but where, within the \Setup algorithm, we replace
  $\NIZKSetup^{\Issue}$ with $\NIZKSimSetup^{\Issue}$. Due to zero-knowledgeness
  of $\NIZK^{\NIZKRel_{\Issue}}$, both games are indistinguishable.

  Now, observe that the adversary is required to output a credential
  identifier for which associated entries in \trans and \CRED exist; moreover,
  if such a credential was produced by an issuer, we must have access to those
  entries, as issuers are assumed to be honest.
  %
  Then, given that $\NIZKRel_{\Issue}$ is knowledge extractable (which is implied
  by simulation-extractability), in game $G_1$
  we can apply the \NIZKExtract function, which produces a tuple $(\usk,\scred,
  \attrs_{\scred})$ from $\utrans = (\Ccom,\attrs,\sipk,\cred,\NIZKproof)$.
  %
  Since \NIZKproof is accepted by \ExtractIssue, from the soundness of \NIZK, we
  know that all $\cred \in \scred$ are valid signatures over \usk, and their
  respective $\attrs_{\cred}$. Thus, \IdentifyCred returns $1$ for all $(\usk,
  \attrs_{\cred},\cred)$ tuples. That is, all the credentials in \scred given
  to \fissue belong to the same user, who is the owner of \usk.
  %
  Finally, since issuers are honest, we know that $\ATTR[\cid] = \attrs$ and,
  consequently, $\fissue(\usk,\scred,\ATTR[\cid]) = \fissue(\usk,\scred,\attrs)
  = 1$, due to the soundness of \NIZK.
  %
  \qed
\end{proof}

\begin{theorem}[Signing unforgeability of \CUASGen]
  \label{thm:sign-forge-uas}
  If the underlying NIZK scheme for $\NIZKRel_{\Sign}$ is simulation
  extractable,the NIZK scheme for $\NIZKRel_{\Open}$ is complete, the public-key
  encryption scheme is correct, and \SBCM is correct and one-more unforgeable,
  then our \CUASGen construction satisfies signing unforgeability as defined in
  \defref{def:sign-forge-uas}, except with negligible probability.
\end{theorem}

\begin{proof}[\thmref{thm:sign-forge-uas}; Sign unforgeability of \CUASGen]
  As in \thmref{thm:issue-forge-uas}, we define two games, $G_0=\ExpForgeSign$,
  and $G_1$, which is exactly the same but where, within the \Setup algorithm,
  we replace $\NIZKSetup^{\Sign}$ with $\NIZKSimSetup^{\Sign}$. After
  zero-knowledgeness of $\NIZK^{\Sign}$, both games are indistinguishable.
  %
  Next, we show that an adversary winning $G_1$ can be used to break the
  one-more unforgeability property of \SBCM.
 
  If the verification at line 4 holds, then $(\msg,\feval,\yeval,\ceval,
  \cinsp,\sipk,\opk,\widetilde{\Eek}) \in \NIZKLang^{\Sign}$. Then:
  %

  \paragraph{(a) \Judge accepts $(\yinsp,\iproof)$.} %
  Simulation extractability of $\NIZK^{\Sign}$ thus ensures that $\cinsp =
  \EEnc(\Eek,\yinsp)$, and since the $(\yinsp,\iproof)$ is generated honestly at
  line 5, then \yinsp is the correct decryption of \cinsp, and  correctness of
  $\NIZK^{\Open}$ ensures that \Judge outputs $1$ at line 6.

  \paragraph{(b) $(\yeval^0,\yeval^1)$ are the correct signature evaluation
    pair.} Also, due to simulation extractability of $\NIZK^{\Sign}$:

  \begin{itemize}
  \item $\yeval^0 = \yeval$, where $(\yeval^0,\cdot) = \feval(\usk,\scred,
    \msg)$, and $\yeval$ is as output by \adv~at step 2.    
  \item $\tyeval^1 = \yeval^1$, where $(\cdot,\yeval^1) = \feval(\usk,\scred,
    \msg)$, and $\tyeval^1$ is as extracted by \ExtractSign at line 7.
  \end{itemize}

  Thus, the probability of \adv~winning at line 9 is $0$.

  \paragraph{(c) The output of \finsp matches the output of \Open.} %
  After (a), the \yinsp value output by \Open is the correct decryption of
  \cinsp. After (b), the $(\yeval^0=\yeval,\yeval^1)$ values output by
  \ExtractSign match the evaluation of \feval. Thus, simulation extractability
  of $\NIZK^{\Sign}$ ensures that $\finsp((\yeval^0,\yeval^1),\usk,\scred,\msg)
  = \yinsp$ and, also, that \yinsp matches the $\yinsp'$ value extracted by
  \ExtractSign. Thus, the probability of \adv~winning at line 10 is $0$.

  \paragraph{(d) All {\cred}s are bound to the same \usk.} %
  $\NIZKRel^{\Sign}$ includes a condition that $\forall \cred \in \scred,
  \SBCMVerify(\ipk_{\cred},\cred,\usk,\attrs_{\cred}) = 1$. Thus, simulation
  extractability of $\NIZK^{\Sign}$ and correctness of \SBCM, ensure that all
  credentials involved in the signature contain \usk as their user key (first)
  attribute. Consequently, \IdentifyCred returns $1$ for all the involved
  credentials, and the probability of \adv~winning at line 11 is $0$.

  \paragraph{(e) \usk must belong to a known user.} %
  The only remaining option for $\adv$ to win is via winning condition at line
  12, meaning that \IdentifyUK fails to find an honest or corrupt users with
  a \usk matching the one used to request the credentials used to produce the
  signature output by \adv. However, at line 12, we already know that all
  credentials are valid signatures by the issuer and that, also, all are bound
  to the same user key. If this key is not associated to any known user, this
  means that there is no matching $\langle\Obtain,\Issue\rangle$ transcript for
  a credential over \usk --i.e., the honest issuer did not issue any credential
  to \usk. But, as \sig must have been produced with at least one credential
  (extracted at step 7), then this credential is a forgery, breaking security
  against one-more forgery of the \SBCM scheme.
  %
  \qed
\end{proof}

\begin{theorem}[Non-frameability of \CUASGen]
  \label{thm:frame-uas}
  If the underlying scheme for $\NIZK^{\Issue}$ is zero-knowledge, the schemes
  for $\NIZK^{\Sign}$ are zero-knowledge and simulation-extractable, and the
  commitment scheme is hiding, then our \CUASGen construction satisfies
  non-frameability as defined in \defref{def:frame-uas}, except with negligible
  probability.
\end{theorem}

\begin{proof}[\thmref{thm:frame-uas}; Non-frameability of \CUASGen]
  We prove that, given an adversary $\adv$ against non-frameability of
  \CUASGen, we can build an adversary \advB that breaks the hiding property of
  the underlying commitment scheme, with non-negligible probability.

  We start from $G_0=\ExpNonframe$. $\adv$ makes queries to the oracles in
  \Oframe. For $G_1$, within \Setup, we replace the \Setup algorithms for the
  three NIZKs (\Issue, \Sign and \Open) with their corresponding \SimSetup
  variants. Consequently, the corresponding queries to \Prove are also
  simulated via the simulator. By the zero-knowledge property of the NIZK
  systems, $G_1$ is indistinguishable from $G_0$. In the sequel, we build on
  top of $G_1$, depending on whether or not a matching \Sig is found in \SIG,
  in line 7.

  \paragraph{(a) A matching \Sig exists in \SIG.} %
  If we can find a matching \Sig in \SIG, for some \uid, then \adv~does not win
  at line 7 of the game. Given that the signature is accepted by \Verify at line
  3, and after simulation extractability of $\NIZK^{\Sign}$, $\yeval =
  \yeval^0$, and $\tyeval^1 = \yeval^1$. Similarly, since the $(\yinsp,\iproof)$
  pair output by \adv~is accepted by \Judge, simulation extractability of
  $\NIZK^{\Sign}$ and $\NIZK^{\Open}$ implies that both checks at line 9 pass.
  Hence, the probability that the adversary wins at either lines 8 or 9 is $0$,
  after simulation extractability of $\NIZK^{\Sign}$ and $\NIZK^{\Open}$.

  \paragraph{(b) No matching \Sig exists in \SIG.} %
  This means that \adv~creates an \UAS forgery. We leverage such an adversary
  against $G_1$ to break the hiding property of the underlying commitment
  scheme (see \appref{sapp:commitments}). For each new honest user \uid creation
  request by \adv~(via \HUGEN),
  \advB leverages its own $COM$ oracle to embed its challenges in the commitment
  scheme into credentials of the \UAS non-frameability game. More precisely, in
  calls to \HUGEN by \adv, \advB does not call \UKeyGen and instead picks two
  random messages, $\msg_0^{\uid},\msg_1^{\uid}$, both from \AttrSpace. Whenever
  a query to \OBTAIN is made by \adv~on behalf of \uid, \advB makes a call to
  its $COM$ oracle, to get a commitment \Ccom for $\usk^{\uid}=\msg_b^{\uid}$
  (for \advB's challenge bit $b$). Then,  \advB uses \Ccom in its \SBCMCom part
  of the underlying \SBCM interactive signing protocol, and simulates the
  corresponding $\NIZK^{\Issue}$ proof. Due to the zero-knowledge property, the
  produced output is indistinguishable to that of the real execution. Similarly,
  \advB can simulate the calls to the \SIGN and \OPEN oracles due to
  simulation extractability of $\NIZK^{\Sign}$ and $\NIZK^{\Open}$.
  %
  Eventually, $\adv$ outputs a $(\oid,\siid,\Sig,\msg,\feval,\yinsp,\iproof)$
  tuple that is accepted by \Verify and \Judge. Due to simulation extractability
  of $NIZK^{\Sign}$, \ExtractSign must be able to produce a $(\usk,\scred,
  \attrs_{\scred}, \tyeval^1,\yinsp')$ tuple. As, by assumption (line 6),  there
  exists an \uid matching that \usk, all \advB has to do is find the
  \uid for which \usk was used as either $\msg_0$ or $\msg_1$ in its hiding
  game. Thus, \advB wins its hiding game with the same probability than \adv~
  has to win the non-frameability game.
  %
  \qed
\end{proof}


%%% Local Variables:
%%% mode: latex
%%% TeX-master: "uas"
%%% End:
