\subsection{Correctness and Security of \CUASGen}
\label{ssec:security-uas}

First, we define the \Identify, \ExtractIssue and \ExtractSign functions that
are needed for some of the properties to be meaningful, in
\figref{fig:helper-funcs}.

\begin{figure}[ht!]
  \begin{minipage}[t]{\textwidth}
    \procedure{$\ExtractIssue(\parm,\trans)$}{%
      \textrm{Parse \parm as $(\cdot,\cdot,\cdot,\cdot,\NIZKcrs_{\Issue},\cdot,
        \cdot,\cdot)$; $\NIZKcrs_{\Issue}$ as $(\NIZKcrs,\NIZKtrap)$; and
        \trans as $(\Ccom,\attrs,\sipk,\cred,\NIZKproof)$} \\
      \pcif \NIZKVerify(\NIZKcrs,\NIZKproof,(\Ccom,\attrs,\sipk)): 
      \pcreturn \bot \\
      (\usk,\scred,\attrs_{\scred}) \gets \NIZKExtract(\NIZKcrs,\NIZKtrap,
      (\Ccom,\attrs,\sipk),\NIZKproof) \\
      \pcreturn (\usk,\scred,\attrs_{\scred}) \\
    }
    
    \procedure{$\ExtractSign(\parm,\oid,\siid,\sig,\yeval,\msg,\feval)$}{%
      \textrm{Parse \parm as $(\cdot,\cdot,\cdot,\cdot,\cdot,\NIZKcrs_{\Sign},
        \cdot,\cdot)$; $\NIZKcrs_{\Sign}$ as $(\NIZKcrs,\NIZKtrap)$; and
        \sig as $(\NIZKproof,\Ec)$} \\
      \textrm{Parse $\PUBOK[\oid]$ as $(\opk,\cdot)$ and let $\sipk \gets
        \PUBIK[\siid]$} \\
      \pcif \NIZKVerify(\NIZKcrs,\NIZKproof,(\msg,\feval,\yeval,\Ec,
      \sipk,\opk)): \pcreturn \bot \\
      (\usk,\scred,\attrs_{\scred},\yinsp,r) \gets \NIZKExtract(\NIZKcrs,\NIZKtrap,
      (\msg,\feval,\yeval,\Ec,\sipk,\opk),\NIZKproof) \\
      \pcreturn (\usk,\scred,\attrs_{\scred},\yinsp) \\
    }
    
    \procedure{$\Identify(\usk,\attrs_{\cred},\cred)$}{%
      \pcreturn \SBCMVerify(\ipk_{\cred},\cred,\attrs_{\cred} \cup
      \lbrace \usk \rbrace) \\
    }    
  \end{minipage}
  \label{fig:helper-funcs}
  \caption{Definition of helper functions \Identify, \ExtractIssue and
    \ExtractSign, for \CUASGen.}
\end{figure}

\begin{theorem}[Correctness of \CUASGen]
  \label{thm:correctness-uas}
  If the underlying schemes for vector commitments, encryption, digital
  signatures, signatures on blocks of committed messages, and NIZKs are
  correct, our generic construction \CUASGen satisfies correctness as
  defined in \defref{def:correctness-uas}.
\end{theorem}

\begin{proof}[\thmref{thm:correctness-uas}]
  \todo{XXX}
\end{proof}

\begin{theorem}[Anonymity of \CUASGen]
  \label{thm:anonymity-uas}
  If the underlying encryption scheme is \todo{IND-CCA}, the vector commitment
  scheme is \todo{binding}, the scheme for signatures on blocks of committed
  messages is \todo{XXX}, and the NIZKs used for $\NIZKRel_{\Issue},
  \NIZKRel_{\Sign}$, and $\NIZKRel_{\Inspect}$ are \todo{zero-knowledge} and
  \todo{simulation-sound?}, our \CUASGen construction satisfies anonymity as
  defined in \defref{def:anonymity-uas}.
\end{theorem}

\begin{proof}[\thmref{thm:anonymity-uas}]
\end{proof}

\begin{theorem}[Issuance unforgeability of \CUASGen]
  \label{thm:issue-forge-uas}
  If the underlying scheme for signatures on blocks of committed messages is
  existentially unforgeable, and the NIZK used for $\NIZKRel_{\Issue}$ is
  simulation extractable and sound, then our \CUASGen construction satisfies
  issuance unforgeability as defined in \defref{def:issue-forge-uas}, except
  with negligible probability. \todo{EUF of \SBCM?}
\end{theorem}

\todo{\usk belongs to \AttrSpace! I think this can lead to malleability attacks.
  Make them disjoint?}

\begin{proof}[\thmref{thm:issue-forge-uas}]
  We show that the probability that \fissue outputs $0$ is negligible, as well
  as the probability that the extracted \usk is not the one that was used to
  request some of the credentials employed to obtain the credential specified by
  the adversary.
  %
  For this purpose, we define two games, $G_0=\ExpForgeIssue$, and $G_1$, which
  is exactly the same, but where, within the \Setup algorithm, we replace
  $\NIZKSetup^{\Issue}$ with $\NIZKSimSetup^{\Issue}$. As per the definition of
  \NIZK in \appref{sapp:nizk}, both games are indistinguishable.

  Now, observe that the adversary is required to output a credential
  identifier for which associated entries in \trans and \CRED exist; moreover,
  if such a credential was produced by an issuer, we must have access to those
  entries, as issuers are assumed to be honest.
  %
  Then, given that $\NIZKRel_{\Issue}$ is knowledge extractable, in game $G_1$
  we can apply the \NIZKExtract function, which produces a tuple $(\usk,\scred,
  \attrs_{\scred})$ from $\utrans = (\Ccom,\attrs,\sipk,\cred,\NIZKproof)$.
  %
  Since \NIZKproof is accepted by \ExtractIssue, from the soundness of \NIZK and
  existential unforgeability of \SBCM, we know that all $\cred \in \scred$ are
  valid signatures over \usk, and their respective $\attrs_{\cred}$. Thus,
  \Identify returns $1$ for all $(\usk,\attrs_{\cred},\cred)$ tuples.
  Moreover, all the credentials in \scred given to \fissue belong to the same
  user, who is the owner of \usk.
  %
  Finally, since issuers are honest, we know that $\ATTR[\cid] = \attrs$ and,
  consequently, $\fissue(\usk,\scred,\ATTR[\cid]) = \fissue(\usk,\scred,\attrs)
  = 1$, due to the soundness of \NIZK.
  %
  \qed
\end{proof}

\begin{theorem}[Signing unforgeability of \CUASGen]
  \label{thm:sign-forge-uas}
  If the underlying NIZK scheme for $\NIZKRel_{\Sign}$ is sound and simulation
  extractable, the NIZK scheme for $\NIZKRel_{\Inspect}$ is sound and simulation
  extractable, and \SBCM is existentially unforgeable, then our \CUASGen
  construction satisfies signing unforgeability as defined in
  \defref{def:sign-forge-uas}, except with negligible probability.
\end{theorem}

\begin{proof}[\thmref{thm:sign-forge-uas}]
  As for \thmref{thm:issue-forge-uas}, we define two games, $G_0=\ExpForgeSign$,
  and $G_1$, which is exactly the same, but where, within the \Setup algorithm,
  we replace $\NIZKSetup^{\Sign}$ with $\NIZKSimSetup^{\Sign}$. As per the
  definition of \NIZK in \appref{sapp:nizk}, both games are indistinguishable.

  From $G_1$, and in order to define the winning conditions for the adversary
  in the signing unforgeability game, consider the following events:

  \begin{description}
  \item[$V$.] Where $V = \Verify(\opk,\sipk,\sig,\yeval,\msg,\feval) = 1$.
  \item[$J$.] Where $J = \Judge(\opk,\sipk,\yinsp,\iproof,\sig,\yeval,\msg,
    \feval) = 0$.
  \item[$L$.] Where $L = (\msg,\feval,\yeval,\Ec,\sipk,\opk) \in
    \NIZKLang^{\Sign}$.    
  \item[$I$.] Where $I = \exists \cred \in \scred~\st~\Identify(\usk,
    \attrs_{\cred},\msg) = 0$.
  \end{description}

  $\adv$ wins if $V \land (J \lor I) = (\overline{L} \land V \land (J \lor I))
  \lor (L \land V \land (J \lor I))$.
  %
  $V$ implies that $(\msg,\feval,\yeval,\Ec,\sipk,\opk) \in \NIZKRel^{\Sign}$.
  Thus, after soundness of $\NIZK^{\Sign}$, the probability of $\overline{L}
  \land V \land (J \lor I)$ is negligible in the security parameter.
  %
  For $(L \land V \land (J \lor I))$ to be satisfied, there are three cases:
  \begin{enumerate}
  \item $L \land V \land J \land \overline{I}$. The game returns 1 in step 6.
  \item $L \land V \land J \land I$.  The game returns 1 in step 6.
  \item $L \land V \land \overline{J} \land I$. The game returns 1 in step 10. 
  \end{enumerate}

  In case 1, $L \land V$ implies that $(\msg,\feval,\yeval,\Ec,\sipk,\opk) \in
  \NIZKRel^{\Sign}$. More concretely, soundness of $NIZK^{\Sign}$ implies that
  $\Ec = \EEnc(\opk,\yinsp;r)$, for $\yinsp$ and $r$ known to the signer. Since
  $(\yinsp,\iproof)$ is generated honestly by the challenger from
  $(\opk,\sipk,\sig = (\NIZKproof_{\Sign},\Ec),\yeval,\msg)$, correctness of
  public key encryption implies that $\EDec(\osk,\Ec) = \yinsp$. Consequently,
  the probability of $L \land V \land J \land \overline{I}$ = 0, as \Judge
  checks precisely that \Ec is a correct encryption of \yinsp under \opk.

  The analysis for case 2 is the same as for case 1.

  For case 3, $(\msg,\feval,\yeval,\Ec,\sipk,\opk) \in \NIZKRel^{\Sign}$,
  $\Judge(\opk,\sipk,\yinsp,\iproof,\sig,\yeval,\msg,\feval)
  = 1$, but there exists some credential \cred for which $\Identify(\usk,
  \attrs_{\cred},\msg)=0$. Note that \usk, $\attrs_{\cred}$ and \cred (for all
  $\cred \in \scred$) are output by \ExtractSign. Thus, after the
  simulation-extractability property and soudness of $\NIZK^{\Sign}$, $(\msg,
  \feval,\yeval,\Ec,\sipk,\opk) \in \NIZKRel^{\Sign}$, which more concretely
  means that $\SBCMVerify(\ipk_{\cred},\cred,\attrs_{\cred} \cup \lbrace \usk
  \rbrace) = 1 = \Identify(\usk,\attrs_{\cred},\cred)$, for all $\cred \in
  \scred$. The probability of case 3 is therefore $0$.
  %
  Moreover, since \SBCMVerify returns $1$ for all $\cred \in \scred$, it must
  be that all of them were obtained via queries to the \ISSUE or \OBTISS
  oracles. Otherwise, if there exists some \cred that was not obtained via
  a call to these oracles, the pair $(\lbrace \usk \rbrace \cup \attrs_{\cred},
  \cred)$ constitutes an existential forgery of \SBCM.

  Cases 1, 2 and 3 above account for winning conditions at steps 6 and 10.
  % 
  Additionally, $\adv$ wins at step 8 if $\feval(\usk,\scred,\msg) \neq \yeval$,
  where \yeval is the value output by the adversary in step 2. However, since
  $\Verify(\opk,\sipk,\sig,\yeval,\msg,\feval) = 1$, soundness of $NIZK^{\Sign}$
  implies that this has negligible probability.
  %
  Similarly, $\adv$ wins at step 9 if (1) $\finsp(\yeval,\usk,\scred,\msg) \neq
  \yinsp$, where \yinsp is the value output by \Inspect at line 5; or if (2)
  $\yinsp \neq \yinsp'$, where $\yinsp'$ is the value extracted by \ExtractSign
  at step 7. For (1), soundness of $\NIZK^{\Sign}$ ensures that \yinsp is the
  correct evaluation of \finsp, whereas soundness of $\NIZK^{\Inspect}$ and
  correctness of the public key encryption ensure that this is also the value
  output by \Inspect. Thus, the probability of $\adv$ winning the game because
  of (1) is negligible. Finally, for (2), simulation-extractability of
  $\NIZK^{\Inspect}$ and correctness of the public key encryption ensure that
  the $\yinsp'$ value extracted by \NIZKExtract matches the value produced by
  \Inspect.
  %
  \qed
\end{proof}

\begin{theorem}[Non-frameability of \CUASGen]
  \label{thm:frame-uas}
  If the underlying NIZK schemes are zero-knowledge and the scheme used for
  $\NIZK^{\Sign}$ is simulation-extractable, and the commitment scheme is
  hiding, then our \CUASGen construction satisfies non-frameability as defined
  in \defref{def:frame-uas}, except with negligible probability.
\end{theorem}

\begin{proof}[\thmref{thm:frame-uas}]
  We prove that, given an adversary $\adv$ against non-frameability of \CUASGen,
  we can build an adversary \advB that breaks the hiding property of the
  underlying commitment scheme, with non-negligible probability.

  We start from $G_0=\ExpNonframe$. $\adv$ makes queries to the oracles in
  \Oframe.

  For $G_1$, within \Setup, we replace the \Setup algorithms for the three
  NIZKs (\Issue, \Sign and \Inspect) with their corresponding \SimSetup
  variants. Consequently, the corresponding queries to \Prove are also
  simulated via the simulator. By the zero-knowledge property of the NIZK
  systems, $G_1$ is indistinguishable from $G_0$.
  
  We build adversary \advB against hiding of commitments from $G_1$ against
  non-frameability. In the hiding game (see \figref{fig:com-games}), \advB first
  picks two messages $\msg_0$ and $\msg_1$, and then receives a commitment \Ccom
  of $\msg_b$. Let \advB pick both $\msg_0$ and $\msg_1$ from \AttrSpace. Then,
  \advB initializes $G_1$ for $\adv$ against non-frameability, and randomly
  picks a number $u \getr [1,q]$, where $q$ can be as large as \advB wants, but
  will be the maximum number of honest users to let $\adv$ add to the game.
  Then, when $\adv$ asks to create the $u$-th user, \advB ignores the call to
  \UKeyGen. For every call that $\adv$ makes to the \OBTAIN oracle associated to
  the $u$-th user, \advB uses the commitment \Ccom received as challenge in its
  game against the hiding property of commmitments, and uses it as commitment to
  the $u$-th user's \usk. \advB then simulates all the NIZK proofs associated to
  the $u$-th user, in calls to \OBTAIN, \SIGN and \INSPECT. Again, due to the
  zero-knowledge property of the associated NIZK systems, the outputs of the
  moddified oracles are indistinguishable from the original outputs (as \Ccom
  is a valid commitment of a user secret key). Eventually, $\adv$ outputs a
  $(\sig,\yeval,\msg,\feval,\yinsp,\iproof)$ tuple that is accepted by \Verify
  and \Judge. Due to simulation extractability of $NIZK^{\Sign}$, \ExtractSign
  must be able to produce a $(\usk,\scred,\attrs_{\scred},\yinsp')$ tuple. Since
  $\adv$ wins the
  non-frameability game with non-neglibible probability, with probability $1/q$,
  the \usk value belongs to the $u$-th honest user, so it must be equal to
  either $\msg_0$ or $\msg_1$; \advB responds accordingly to its challenge in
  the game for the hiding property of the commmitment scheme. By assumption,
  \advB wins with non-negligible probability.
  %
  \qed
\end{proof}


%%% Local Variables:
%%% mode: latex
%%% TeX-master: "uas"
%%% End:
