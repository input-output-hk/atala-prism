\section{\CUASGen with Interactive Sign and Verify}
\label{app:interactive-uas}

\todo{The following was written for \GSAC. It should apply for \UAS, but check
  and adapt.}
\todo{Markulf's comment on non-transferability and non-deniability of each
  option.}

In group signatures, the signing and verification process is non-interactive.
That is, a group member produces a signature and eventually sends it to the
verifier, who can check it without further interaction. Nevertheless, in
anonymous credentials, this process is typically interactive, and consists of
at least one round-trip. This can be useful, for instance, when some notion of
freshness is necessary.

The approach to turn our non-interactive singing-verification into an
interactive protocol that ensures freshness is evident: prior to signing the
message, we require that the honest verifier sends to the user a number picked
uniformly at random from an appropriate domain, which must be concatenated to
the signed message. If there is no need to actually sign a message, then the
message is set to an empty string, and the user only signs the random number.
%
Syntactically, instead of having non-interactive $\sig \gets \Sign(\gpk,\usk,
\cred,\dattrs,\msg)$ and $1/0 \gets \Verify(\gpk,\sig,\dattrs,\msg)$ algorithms,
we have a $1/0 \gets \langle \Sign(\gpk,\usk,\cred,\dattrs),\Verify(\gpk,
\dattrs) \rangle$ interactive protocol.
%
We sketch next a proof for why does this result in a secure \GSAC scheme with
interactive signing and verification.

\paragraph{Issuance Anonymity.} \todo{XXX}

\paragraph{Signature Anonymity.} To see why the interactive variant is
anonymous, assume an adversary $\adv$ against anonymity in that case. We build
\advB against the non-interactive \ExpGSACSigAnonb from \adv~ as follows. \advB
initializes
everything as in the \ExpGSACSigAnonb game. When $\adv$ initiates a call to its
interactive \SIGN (resp. \CHALb) oracle, \advB picks a random number, and sends
it to the adversary, along with the response of its own (non-interactive) \SIGN
(resp. \CHALb) oracle where, to the message passed as parameter to its oracle,
concatenates the produced random number. After such simulation, \advB just
outputs whatever $\adv$ outputs. Clearly, the simulation is perfect and, if
$\adv$ wins with non-negligible probability in the interactive case, then so
does \advB in the non-interactive case.

\paragraph{Traceability.} The simulation described for the anonymity case (i.e.,
\advB choosing the random numbers, and concatenating them to the message to be
signed) applies here too. Thus, an adversary against traceability in the
interactive case, can be used to build an adversary that breaks traceability in
the non-interactive counterpart.

\paragraph{Non-frameability.} The simulation described for the anonymity case
(i.e., \advB choosing the random numbers, and concatenating them to the message
to be signed) applies here too. Thus, an adversary against non-frameability in
the interactive case, can be used to build an adversary that breaks
non-frameability in the non-interactive counterpart. 

%%% Local Variables:
%%% mode: latex
%%% TeX-master: "uas-paper"
%%% End:
