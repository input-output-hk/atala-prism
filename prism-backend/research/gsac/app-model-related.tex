\section{Models of Related Schemes}
\label{app:related-models}

In the following subsections we reproduce models of related schemes in the
literature. Then, we give concrete \fissue, \feval and \finsp functions that
make \UAS mimic their functionality. Finally, we prove for each case that
\UAS -- restricted with the corresponding functions -- indeed meets their
security models, thus proving that \UAS is a generalization of all of them.

\subsection{Group Signatures}
\label{sapp:related-models-gs}

We adopt the model in \cite{bsz05}, where security of group signatures is
defined according to the games in \figref{fig:model-gs}.

\subsection{Group Signatures with Message Dependent Opening}
\label{sapp:related-models-gsmdo}

\subsection{Ring Signatures}
\label{sapp:related-models-rs}

We adopt the model in \cite{bkm06}, where security of ring signatures is
defined according to the games in \figref{fig:model-rs}. A ring signature
scheme consists of three algorithms $(KeyGen,Sign,Verify)$. Roughly,
$((\pk_1,\sk_1),\dots,(\pk_n,\sk_n)) \gets KeyGen(1^\secpar)$ produces $n$
signing key pairs, $\sig \gets Sign(\sk_i,R,\msg)$ creates a signature for
anonymity set $R \subseteq \lbrace \pk_1, \dots, \pk_t \rbrace$, using secret
key $\sk_i$, associated to some $\pk_i \in R$; and $1/0 \gets Verify(R,\sig,
\msg)$ checks if \sig is a valid signature over \msg for ring $R$.

\begin{figure}[ht!]
  \centering
  \scalebox{0.9}{      
    \begin{minipage}[t]{0.55\textwidth}        
      \procedure[linenumbering]{$\Exp^{anon-b}_{\adv,rs}(1^\secpar)$}{%
        ((\pk_1,\sk_1),\dots,(\pk_n,\sk_n)) \gets KeyGen(1^\secpar) \\
        (i_0,i_1,R,\msg,\st) \gets \adv^{SIGN}(\pk_1,\dots,\pk_n) \\
        \pcif % R \not\subseteq \lbrace [n] \rbrace \lor
        \lbrace \pk_{i_0},\pk_{i_1} \rbrace \notin R \lor %\\
        i_0 = i_1: \pcreturn \bot \\
        \sig \gets Sign(\sk_{i_b},R,\msg)) \\
        b^* \gets \adv(\sig,\st) \\
        \pcreturn b^*
      }        
    \end{minipage}
  }
  \scalebox{0.9}{      
    \begin{minipage}[t]{.5\textwidth}
      \procedure[linenumbering]{$\Exp^{forge}_{\adv,rs}(1^\secpar)$}{%
        ((\pk_1,\sk_1),\dots,(\pk_n,\sk_n)) \gets KeyGen(1^\secpar) \\
        (R,\sig,\msg) \gets \adv^{SIGN,CORR}(\pk_1,\dots,\pk_n) \\
        \pcif Verify(R,\sig,\msg) = 0: \pcreturn 0 \\
        \pcreturn 1~\pcif R \subseteq [n] \setminus CU = \emptyset~\land \\
        \pcind \adv~\textrm{never queried}~SIGN(\cdot,R,\msg) \\
        \pcreturn 0
      }       
    \end{minipage}      
  }
  \caption{Security games for ring signatures \cite{bkm06}. The SIGN oracle
    accepts a $(i,R,\msg)$ tuple, where $pk_i$ must be part of $R$, and
    returns $\sig \gets Sign(\sk_i,R,\msg)$. The oracle $CORR$ accepts an
    index $i$, and lets $\adv$ learn $\sk_i$. Corrupted user indexes are
    added to list $CU$.}
  \label{fig:model-rs}  
\end{figure}

Intuitively, a ring signature scheme $rs$ is anonymous if, for every p.p.t.
\adv, $|\Pr[\Exp^{anon-1}_{\adv,rs}(1^\secpar)] = 1| -
|\Pr[\Exp^{anon-1}_{\adv,rs}(1^\secpar)] = 0|$ is a negligible function of
$1^\secpar$; and a ring signature scheme $rs$ is unforgeable if, for every
p.p.t. \adv, the probability that $\Exp^{forge}_{\adv,rs}(1^\secpar)$ returns
$1$ is also negligible in $1^\secpar$. For short, we say that a ring signature
scheme meeting both anonymity and unforgeability as specified above, is a secure
ring signature scheme.

\subsubsection{\CUASRing Construction.} %
We prove that, as specified in \secref{ssec:uas-ring}, our \CUASGen construction
for \UAS, with alternative NIZK relationship $\NIZKRel_{\Sign-prv}$ for signing,
issuance function $\fissue^{\sring}$, signature evaluation function
$\feval^{\attrs}$, and inspection function $\finsp^0$, is a secure ring
signature scheme. Thus, in this section, when we write \CUASGen, we refer to
its specific variant built from the previous modifications.

\begin{description}  
\item[$\parm \gets \Setup(\secpar,\AttrSpace)$.] As in \CUASGen, but using
  $\NIZKRel_{\Sign-prv}$ instead of $\NIZKRel_{\Sign}$.  
\item[$(\ipk,\isk) \gets \IKeyGen(\parm,\fissue)$.] As in \CUASGen, using
  $\fissue^{\sring}$.  
\item[$(\opk,\osk) \gets \OKeyGen(\parm,\finsp).$] As in \CUASGen, where every
  opener that wants to be a valid opener in \CUASRing, needs to set \finsp to
  $\finsp^0$.  
\item[$\usk \gets \UKeyGen(\parm).$] Can only be run after \IKeyGen. Let 
  $\usk \gets \isk$.
\item[$\langle \cred,\bot,\utrans,\bot \rangle \gets
  \langle \Obtain(\usk,\scred,\attrs),\Issue(\isk,\sipk,\attrs)\rangle.$]
  The user picks a ring from an (publicly) available list of issuers (users),
  i.e. $\sring \gets \lbrace \ipk_i \rbrace_{i \in [t]}$, for some $t$. Then,
  runs the $\langle \Obtain,\Issue \rangle$ protocol as in \CUASGen, acting both
  as user and as issuer, and using $\fissue^{\sring}$ as issuance function.
  \attrs are the public keys of the users (issuers) to be included in the ring.
\item[$(\sig,\yeval) \gets \Sign(\usk,\opk,\scred,\msg,\feval).$] As in
  \CUASGen, using $\feval^{\attrs}$ as sign evaluation function, using
  as credential for signing the credential used in the desired $\langle \Obtain,
  \Issue \rangle$ protocol and, as \opk, any previously registered \opk (i.e.,
  such that $\finsp=\finsp^0$).
\item[$1/0 \gets \Verify(\opk,\sipk,\sig,\yeval,\msg,\feval).$] As in \CUASGen.
  Note that $\sipk = \sring$, and $\feval=\feval^{\attrs}$.
\item[$\Open$ and $\Judge$.] As in \CUASGen. Note that $\finsp=\finsp^0$.
\end{description}

\todo{I guess we need to show that changing the NIZK for signing does not affect
  security. This leads to ``What are the conditions that need to be checked/cannot
  be altered in the default NIZK sign, that make \CUASGen secure?'' }

\paragraph{Security of \CUASRing.} %
We prove next that our \CUASRing construction is anonymous and unforgeable, if
the underlying \CUASGen construction is secure.

\begin{theorem}[Anonymity of \CUASRing]
  If the base \CUASGen construction is anonymous according to
  \defref{def:anonymity-uas}, then \CUASRing is an anonymous ring signature
  scheme.
\end{theorem}

\begin{proof}
  Assume $\adv$ is an adversary against ring signature anonymity for \CUASRing.
  Then, we build \advB against anonymity of \CUASGen. Concretely, \advB first
  generates $n$ keypairs by running its \IGEN (with $\fissue^{\sring}$) and
  \HUGEN oracles, and runs \OGEN at least once (with $\finsp^0$). Then, invokes
  $\adv$ passing as public keys the public keys of the $n$ generated issuers
  (users in \CUASRing). $\adv$'s $SIGN(i,R,\msg)$ queries are processed as
  follows. If user $i$ has not yet produced a credential for ring $R$, \advB
  makes a call to its \OBTAIN oracle, using $R$ as attribute set. If such a
  credential was already produced, then reuse it. Then, \advB uses the
  corresponding user and credential identifiers as parameters to a query to its
  own \SIGN oracle (specifying $\feval^{\attrs}$ as signature evaluation
  function). Eventually, $\adv$ outputs two user indexes, a ring $R$, and
  message \msg. If the checks at line 4 of \figref{fig:model-rs} pass, \advB
  generates the challenge signature by calling its own \CHALb oracle, and passes
  the result to $\adv$. Finally, \advB outputs whatever $\adv$ outputs.
  %
  \qed
\end{proof}

\begin{theorem}[Unforgeability of \CUASRing]
  If the base \CUASGen construction is non-frameable according to
  \defref{def:frame-uas}, then \CUASRing is an unforgeable ring signature
  scheme.
\end{theorem}

\begin{proof}
  Assume $\adv$ against unforgeability of \CUASRing wins the game. We build
  \advB winning the non-frameability game of \CUASGen with the same probability.
  %
  \advB prepares the environment as the adversary \advB against anonymity,
  and answers \adv's queries to the $SIGN$ oracle also as in the anonymity
  proof. To answer \adv's queries to its $CORR$ oracle, \advB leverages the
  \ICORR oracle in the non-frameability game for \UAS. Finally, \advB outputs
  whatever $\adv$ outputs.

  Note that, if $\adv$ wins its game, it means that the public keys of the
  issuers in $R$ all belong to uncorrupted users. Since the signature is
  accepted by $Verify$ at line 3 of the ring signature unforgeability game,
  there exists some \uid with an \usk matching the check at line 6 in the
  non-frameability game. Moreover, since $\adv$ never queried its $SIGN$
  oracle, then by construction there cannot exist a matching entry in the
  $\SIG[\uid]$ table. Thus, \advB wins the non-frameability game whenever
  $\adv$ wins its unforgeability game. Since \CUASGen is non-frameable, then
  this probability must be negligible.
  %
  \qed
\end{proof}

\subsection{Anonymous Credentials}
\label{sapp:related-models-ac}

We adopt the model in \cite{fhs19}, where security of anonymous credentials
(with selective disclosure) is defined according to the games in
\figref{fig:model-ac}.

\subsection{Delegatable Anonymous Credentials}
\label{sapp:related-models-dac}

\subsection{Multimodal Private Signatures}
\label{sapp:related-models-mps}


%%% Local Variables:
%%% mode: latex
%%% TeX-master: "uas"
%%% End:
