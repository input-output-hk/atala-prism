\section{Relating \UAS with Other Schemes, and Variations}
\label{sec:transformations}

In order to justify \UAS's universality, in this section we describe how can
we leverage our \UAS scheme to build other known schemes, only by specifying
different issuance, signature evaluation, and opening functions; as well as,
occasionally, some minor variations in some NIZK relations. Table
\tabref{tab:uas-alt-funcs} summarizes the different functions we use in our
subsequent variations of \UAS. \jdv{The relationships seem quite
  straightforward; thus, we only informally sketch how one could build other
  schemes (or something very similar to them) from \UAS, and leave the formal
  analysis for futher work.}

\begin{table}[ht!]
  \begin{tabular}{c | c | c | c | c | c}
    \bf Target scheme & \bf $\NIZKRel_{\Sign}$ variant & \bf \fissue variant & \bf \feval variant & \bf \finsp variant & \bf  Defined in \\
    \hline
    GS & None & $\fissue^1$ & $\feval^0$ & $\finsp^{\upk}$ & Implied by \GSAC \\
    AC\footnote{We show how to build ACs with selective disclosure, but ACs for
    arbitrary claims are direct too.} & None & $\fissue^1$ & $\feval^{\dattrs}$ & $\finsp^0$ & Implied by \GSAC \\
    \GSAC & None & $\fissue^1$ & $\feval^{\dattrs}$ & $\finsp^{\upk}$
    & \secref{ssec:uas-gsac} \\
    \bf GS-MDO\footnote{We achieve something similar to GS-MDO, but not the
    exact same model as in \cite{ehk+19}} & None & $\fissue^1$ & $\feval^0$ & $\finsp^{\smsg}$
    & \secref{ssec:uas-gsmdo} \\
    \bf Ring sigs. & $\NIZKRel_{\Sign-prv}$ & $\fissue^{\sring}$ & $\feval^{\attrs}$ & $\finsp^0$ & \secref{ssec:uas-ring} \\
    \bf MPS\footnote{Building MPS from \UAS also requires a simple definitional change. See
    \secref{ssec:uas-mps}.} & $\NIZKRel_{\Sign-enc}$ & Any & $\feval^{enc}$ & Any & \secref{ssec:uas-mps} \\
    \bf Delegatable creds & None & $\fissue^n$ & $\feval^n$ & Any & \secref{ssec:uas-delcred} \\
  \end{tabular}
  \caption{How to build related schemes from \UAS.}
  \label{tab:uas-alt-funcs}
\end{table}

\subsection{\GSAC from \UAS}
\label{ssec:uas-gsac}

Take the generic construction for \UAS in \secref{ssec:generic-construction-uas}.
First, consider the constant issuance function $\fissue^1$, the signature
evaluation function $\feval^{\dattrs}$, and the opening function $\finsp^{\upk}$,
as defined in \esref{eq:uas-gsac-funcs}

\begin{align}
  & \fissue^1(\usk,\scred,\attrs) \coloneqq 1 \nonumber \\
  & \feval^{\dattrs}(\usk,\cred,\msg) \coloneqq \dattrs(\cred) \nonumber \\
  & \finsp^{\upk}(\yeval,\usk,\scred,\msg) \coloneqq \CCommit(\usk;0)
    \label{eq:uas-gsac-funcs}
\end{align}

Where $\dattrs(\cred)$ is the function returning the attributes indexed by
\dattrs within \cred (which we can assume to be easy to do with proper
encoding). It is straightforward to see that an \UAS scheme with $\fissue^1$
as issuance function, $\feval^{\dattrs}$ as signing evaluation function, and
$\finsp^{\upk}$ as opening function, is an implementation of the functionality
described for \GSAC --which, in addition, lets the signer use more than one
credential per signature; although this can of course be restricted
implementation-wise.
%
Furthermore, the anonymity, sign unforgeability, and non-frameability properties
of \UAS directly imply the anonymity, traceability, and non-frameability
properties of \GSAC.

\paragraph{Vanilla Group Signatures and Anonymous Credentials from \UAS.} As a
corollary to the previous analysis, recall that, in \secref{ssec:variants-gsac}
we proved that \GSAC implies both group signatures and anonymous credentials.
Thus, since \UAS implies \GSAC, \UAS implies also both group signatures and
anonymous credentials.

\subsection{\UAS with Multiple Openers}
\label{ssec:uas-multiopen}

We briefly mention that we have defined \UAS signatures as being ``openable'' by
only one opener. However, since there is no tight relation between issuer and
opener (as opposed to conventional group signatures), it is trivial to
extend to the case with multiple openers per signature. In that case, the
$\NIZKRel_{\Sign}$ relation should be extended to include proofs of correct
encryption of the \yinsp values produced by each chosen opener.
Similarly, the sign unforgeability and non-frameability definitions should be
extended to check correctness of the corresponding extra open values.

\subsection{Group Signatures with Message Dependent Opening}
\label{ssec:uas-gsmdo}

In \cite{khk+19}, one of the first group signature schemes with a certain
flexibility in the open-related functionality was introduced. Briefly, a group
signature scheme with message dependent opening works as a conventional group
signature scheme, with the addition that it is only possible to open signatures
over specific messages (e.g., text with offensive language). For this, the
authors introduce an extra authority, the admitter, who creates a
``message-specific token'' for each message that is deemed unacceptable.
These message-specific tokens are later needed by the opener, in order to open
signatures over unacceptable messages. \UAS can implement this functionality in
a straightforward manner, by leveraging $\fissue^1$ as defined in
\esref{eq:uas-gsac-funcs}, and $\feval^0$ and $\finsp^{\smsg}$, as defined in
\esref{eq:uas-gsmdo-funcs}.

\begin{align}
  & \feval^0(\usk,\scred,\msg) \coloneqq \pcreturn 0 \nonumber \\
  & \finsp^{\smsg}(\yeval,\usk,\cred,\msg) \coloneqq \lbrace \pcif \msg \in \smsg:
    \pcreturn \CCommit(\usk;0); \pcelse \pcreturn \bot \rbrace
    \label{eq:uas-gsmdo-funcs}
\end{align}

\jdv{In \cite{khk+19}, the authors prove that GS-MDO implies identity-based
  encryption. \UAS implying GS-MDO would thus imply IBE, but we do not use
  IBE in our constructions. Is there some connection between some of the main
  building blocks we use, and IBE?}

\subsection{Ring Signatures}
\label{ssec:uas-ring}
\todo{This subsection needs much refinement.}

In a nutshell, ring signatures \cite{rst06} are like group signatures where
rather than having an issuer accepting users into a group, any user can create
an ad hoc group composed by himself, and any arbitrary set of other users that
have advertised their public keys in some publicly accessible way. Also, in
ring signatures no de-anonymization is possible\footnote{At least, in vanilla
  ring signatures. Some variants allow some sort of linkability; for instance
  \cite{lww04}.}, although the most distinguishing property is probably the lack
of issuer, which has led some relevant systems like Monero%
\footnote{\url{https://www.getmonero.org/resources/moneropedia/ringsignatures.html}.
  Last access on May 8th, 2022.} to opt for ring signatures rather than, e.g.,
group signatures.

\todo{Explain the following better...}
To reach (vanilla) ring signatures from \UAS, we have to slightly alter our
generic construction. This is due to the fact that the NP relation
$\NIZKRel_{\Sign}$ defined in \secref{ssec:generic-construction-uas} reveals the
public key of the issuer of each credential used to produce the signatures.
This is a problem as, intuitively, the signer in a ring
signature is a sort of issuer for the ad hoc group. To hide the issuer, we
replace the NP relation in \secref{ssec:generic-construction-uas} with
$\NIZKRel_{\Sign-prv}$. We also use issuance function $\fissue^{\sring}$,
signature evaluation function $\feval^{\attrs}$, and opening function
$\finsp^0$. All of them are specified in \esref{eq:uas-ring-funcs}.

\begin{align}
  & \NIZKRel_{\Sign-prv} \coloneqq \lbrace (\usk,\cred,\yinsp,r,
    \ipk),(\msg,\feval,\yeval,\Ec,\Eek,\sring): \nonumber \\
  & \hspace*{6.405em}\Ec = \EEnc(\Eek,\yinsp;r) \land \nonumber \\
  & \hspace*{6.40em}\exists \ipk \in \sring~\suchthat~\SBCMVerify(\ipk,\cred,\usk,
    \sring) = 1) \land \nonumber \\
  & \hspace*{6.40em}\yeval = \feval(\usk,\cred,\msg) \land
    \yinsp = \finsp(\yeval,\usk,\cred,\msg)
     \rbrace \nonumber \\
  & \fissue^{\sring}(\usk,\scred,\attrs) \coloneqq \lbrace \pcif \attrs
    \subseteq \sring \cup \lbrace \CCommit(\usk;0) \rbrace: \pcreturn 1; \pcelse
    \pcreturn 0 \rbrace \nonumber \\
  & \feval^{\attrs}(\usk,\cred,\msg) \coloneqq \lbrace \pcif \CCommit(\usk;0) \in
    \attrs: \pcreturn \attrs(\cred); \pcelse \pcreturn \bot \rbrace \nonumber \\
  & \finsp^0(\yeval,\usk,\cred,\msg) \coloneqq 0 \label{eq:uas-ring-funcs}
\end{align}

Where $\usk \gets \isk$ for each issuer/user and, consequently 
\sring is some (arbitrary) ad hoc set containing the public keys of the
users (issuers) that the signer wants to include in its ring%
\footnote{Note that, implicitly, this conditions that the space of issuer public
  keys is a subset of \AttrSpace. If this is not the case ``by default'', this
  can be achieved via some appropriate mapping (e.g., an FDH or a XOF
  \needcite). That is, instead of having the ring be a set of {\ipk}s, we
  have the ring be a set of $F({\ipk}s)$, where $F$ is an appropriate mapping.
  For simplicity of exposition, we assume that the space of issuer public keys
  is compatible with \AttrSpace.}.
In $\NIZKRel_{\Sign-prv}$
the public keys of the issuers are now part of the witness in the NP relation,
which means that they will not be revealed in the proofs. Also, observe that, if
we let the attributes in a credential be public keys, $\fissue^{\sring}$ means
that a credential will be issued if all the public keys specified as attributes
to be included in the credential are part of the union of the ring and the
public key of the signer. To build $\feval^{\attrs}$, we define $\attrs(\cred)$
to be the function that returns all the attributes encoded in a credential.
Then, if the public key of the signer is included in the ring, $\feval^{\attrs}$
returns all the attributes in the credential (i.e., all the public keys in the
ring); otherwise, it aborts. Finally, let $\finsp^0$ disallows any
de-anonymization.

Then, an \UAS scheme with $\NIZKRel^{\Sign-prv}$ as NP relation for
$\NIZK^{\Sign}$, and $\fissue^{ring}$, $\feval^{\attrs}$, and $\finsp^0$ is
intuitively a ring signature scheme. To see this, observe that any user can
act as an issuer. Thus, the owner of \usk can issue to himself a credential
for any arbitrary ring it desires. While the $\NIZKRel^{\Issue}$ reveals the
issuer's public key, note that this is irrelevant, as the issuer is the user
himself. Then, the newly defined $\NIZKRel^{\Sign-prv}$ does exactly the same
as in our original \UAS scheme, but without revealing the issuer's public key.
$\feval^{\attrs}$ reveals all the attributes in the credential, which are
the public keys of the ring, including the signer's public key. This is actually
what ring signatures do: the signer, who is the owner of the private key
associated to one of the public keys in the ring, advertises which are the
public keys in the ring, and proves knowledge of one of them. If the NIZK
proof verifies, this means that the signer is indeed the owner of such a private
key. Finally, since $\finsp^0$ returns always $0$, then no de-anonymization is
possible.
%
Interestingly, this construction based on \UAS allows adding extra attributes
(beyond the public keys in the ring) to the produced signatures, which may be
useful for real world use cases.

\subsection{Multimodal Private Signatures}
\label{ssec:uas-mps}

To show that Multimodal Private Signatures (MPS) \needcite can be built from
\UAS, we need to give an alternative, simulation-based, definition of the
anonymity property. In the simulation-based approach, we require the adversary
to guess the bit $b$ defining whether it is interacting with the real world,
where it gets signatures by real users, or with a simulation, where all
signatures are simulated and do not contain information about the signer.
This alternative formulation is given in \figref{fig:exp-uas-simanon}, where
$\Osimanon = (\lbrace\HU,\CU\rbrace\GEN,\lbrace\II,\OO\rbrace\GEN,\lbrace\II,
\OO\rbrace\CORR,\OBTAIN,\WREG)$.

\begin{figure}[htp!]

  \centering
  \procedure[linenumbering]{$\ExpSimAnonb(1^\secpar)$}{%
    \pcif b = 0: \\
    \parm \gets \Setup(1^\secpar) \\    
    \pcind b^* \gets \adv^{\Osimanon,\SIGN,\OPEN}(\parm) \\
    \pcelse: \\
    \parm \gets \SIMSETUP(1^\secpar) \\
    \pcind b^* \gets \adv^{\Osimanon,\SIMSIGN,\SIMOPEN}(\parm) \\    
    \pcreturn b^*
  }
  
  \caption{Simulation-based anonymity experiment for \UAS schemes.}
  \label{fig:exp-uas-simanon}
\end{figure}

\begin{definition}{(Simulatable Anonymity of \UAS)}
  \label{def:sim-anonymity-uas}  
  We define the advantage \AdvSimAnon of $\adv$ against \ExpSimAnonb as
  $\AdvSimAnon=|\Pr\lbrack\ExpSimAnono(1^\secpar)=1\rbrack-
  \Pr\lbrack\ExpSimAnonz(1^\secpar)=1\rbrack|$.
  %
  An \UAS scheme satisfies simulatable anonymity if there exists simulators
  \SIMSETUP, \SIMSIGN and \SIMOPEN such that, for any p.p.t. adversary $\adv$,
  \AdvSimAnon is a negligible function of $1^\secpar$.
\end{definition}

Note that, for our generic construction \CUASGen, \SIMSETUP, \SIMSIGN, and
\SIMOPEN are straightforward, as all we need to do is:

\begin{itemize}
\item To build \SIMSETUP, just replace the $\NIZKSetup^{\Sign}$ and
  $\NIZKSetup^{\Open}$ algorithms within \Setup, by their respective
  $\NIZKSimSetup$ algorithms.
\item To build \SIMSIGN from the \SIGN oracle, just simulate the NIZK proof
  contained in the signatures produced by the \Sign algorithm.
\item To build \SIMOPEN from the \OPEN oracle, just simulate the NIZK proof
  showing opening correctness, instead of producing them as in the \Open
  algorithm.
\end{itemize}

Indistinguishability then follows from the zero-knowledge property of each
corresponding NIZK system. Since both scenarios are indistinguishable, and
all signature-related proofs received by the adversary in the simulation do not
depend on any witness (as they are simulated), then it follows that the
adversary cannot gain any information about signers in the real world.

With this alternative definition, we can achieve a generalisation of MPS
basically by, instead of returning the plaintext \yeval value, returning an
encryption of it under the opener's public key; additionally proving 
that the value used to compute the output of \finsp is the plaintext \yeval
value. In more detail, we define an extended $\NIZKRel_{\Sign-enc}$ relation,
and, given any \feval function, a variant $\feval^{enc}$ operating on an
encryption of \yeval. Both are specified in \esref{eq:uas-mps-funcs}.

\begin{align}
  & \NIZKRel_{\Sign-enc} \coloneqq \lbrace (\usk,\scred,\attrs_{\scred},\yinsp,r),
    (\msg,\feval,\yeval,\Ec,\sipk_{\scred},\Eek): \nonumber \\
  & \hspace*{6.00em}\Ec = \EEnc(\Eek,\yinsp;r) \land \nonumber \\
  & \hspace*{6.00em}\forall \cred \in \scred,\SBCMVerify(\ipk_{\cred},\cred,\usk,
  \attrs_{\cred}) = 1) \land \nonumber \\
  & \hspace*{6.00em}\yeval = \EEnc(\opk,\feval(\usk,\scred,\msg)) \land \nonumber \\
  & \hspace*{6.00em}\yinsp = \finsp(\feval(\usk,\scred,\msg),\usk,\scred,\msg)
    \rbrace \nonumber \\
  & \feval^{enc}(\usk,\scred,\msg) \coloneqq \lbrace \yeval
    \gets \feval(\usk,\scred,\msg); \yeval' \gets \EEnc(\opk,\yeval); \pcreturn
    \yeval' \rbrace \label{eq:uas-mps-funcs}
\end{align}

This would also require that the public-key encryption scheme is IND-CCA.

Given this approach, the need to provide an alternative definition for anonymity
is clearer. Namely, in the original \CHALb oracle, we restrict that the
signatures produced by both challenge users output the same \yeval value.
However, if instead of returning the plaintext \yeval value, we return an
encrypted version of it (as we do in the variant just described), this clearly
becomes unachievable.

\jdv{It would seem that the simulation-based anonymity definition is more
  general. Can we prove that? Concretely, if it is also good for selective
  disclosure, we may just adopt it as default.}

\subsection{Delegatable Credentials}
\label{ssec:uas-delcred}

Building delegatable credentials is also straightforward, thanks to the \fissue
and \feval functions. For instance, take the simplification of level-based
delegation, where the owner of a credential of level $n$ can only issue
credentials of level $n+1$. Without loss of generality, assume that the
credential level is encoded in the first attribute of the credential (after the
\usk), which we denote with $a_1$. In this context, any owner of a credential of
level $n$ can define an issuance function $\fissue^n$, as in
\eref{eq:uas-delcred-func}. Similarly, we may just have the signing evaluation
function reveal the level of each credential used for signing, along with any
other desired attribute, as in $\feval^{\dattrs,n}$, in
\eref{eq:uas-delcred-func}. Note that it is straightforward to create more
complex functions in both cases which, for instance, also take into account
roles, or hiding the level but ensuring that it meets whatever desired rules.

\begin{align}
  & \fissue^n(\usk,\scred,\attrs) \coloneqq \lbrace \pcif a_1 = n+1: \pcreturn 1;
  \pcelse \pcreturn 0 \rbrace \nonumber \\
  & \feval^{\dattrs,n}(\usk,\scred,\msg) \coloneqq \lbrace \pcreturn
    \dattrs(\scred),a_1(\scred) \rbrace  
  \label{eq:uas-delcred-func}
\end{align}

\jdv{\subsection{Functional Signatures}}
\jdv{I think that, by making \yeval explicit, \UAS is essentially a
  privacy-preserving extension to Functional Signatures. Still, I read that paper
  quite some time ago. Re-check.}

%%% Local Variables:
%%% mode: latex
%%% TeX-master: "uas"
%%% End:
