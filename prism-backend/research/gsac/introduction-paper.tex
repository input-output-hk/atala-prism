\section{Introduction}
\label{sec:introduction}

% [4 pages max]

%% Brief description of main precedents (simple words)
\paragraph{Anonymous authentication and signatures.} %
Privacy-preserving digital signature and authentication systems have been a
prolific field of study in the last few decades. Anonymous credentials
\cite{chau85} and group signatures \cite{ch91} are among the most relevant ones.
While there are many variants, the high-level idea is common: after obtaining a
credential, users can authenticate anonymously and indistinguishably from other
users with similar credentials. 

%% Usefulness of main precedents / motivation
Their usefulness stems from the combination, within a single primitive, of
seemingly contradictory concepts: anonymity, authentication, and utility; where
utility can appear in the shape of information revealed at signature/authentication
time, or after (in the latter case, typically called accountability).

\paragraph{Utility at authentication-time and after authentication-time.} %
Anonymous credentials (AC) focus on utility at authentication time. For
instance, encoding multiple attributes within a credential and allowing to
selectively disclose a subset of them, or even arbitrary predicates.
The real world applications of such systems are direct,
e.g., to prove ``Over 18''-type of claims. Indeed, the scientific research
behind anonymous credentials is making its own way to become a foundational
block of the technology some call ``anoncreds''%
\footnote{\url{https://hyperledger-indy.readthedocs.io/projects/sdk/en/latest/docs/design/002-anoncreds/README.html}. Last access, June 18th, 2022.},
the privacy-preserving cousin of W3C's Verifiable Credentials%
\footnote{\url{https://www.w3.org/TR/vc-data-model/}. Last access, June 18th,
  2022.}.

Group signatures (GS) enhance utility after authentication, by allowing partial
reidentification like linking of signatures, or even full deanonymization.
While actual applications of group signatures seem hard to find in the real
world, Direct Anonymous Attestation (DAA) \cite{bcc04,bl07,cdl16b} has been deployed to
millions of computing devices into what's called TPM (Trusted Privacy Module)%
\footnote{\url{https://trustedcomputinggroup.org/resource/trusted-platform-module-tpm-summary/}. Last access, June 18th, 2022.}.
Indeed, DAA is a very close relative of group signatures, where the signer is
divided into two components (trusted and partially trusted), and signing
and verification are turned into an interactive protocol.

%% Previous works summary w/ refs directly related contribs of UAS
%% (or, tech approaches of main precedents)
\paragraph{Evolution of AC and GS.} %
As the previous shows, AC and GS systems have followed separate paths, although
there are some exceptions. Functionality-wise, Attribute-Based Group Signatures
(ABGS) introduce attributes into the credentials issued to group members, who
can later use them to prove pre-fixed claims \cite{emo09,aa14}. Hidden-identity
signatures also add attributes into the credentials, with the aim of preventing
the opener (the trusted party who can deanonymize signers) from having to rely
on the issuer in order to perform the opening \cite{ks07}. Some AC schemes have
also tried to incorporate some sort of utility after authentication, like AC
schemes with revocation \cite{cks10}, or with ``watchlists'' \needcite.

This separation does not hold in the approaches to build them, though. The
probably most fruitful way to design and instantiate AC systems
is with CL-type signatures \cite{cl02}. But the fact that they support efficient
zero-knowledge protocols for proving knowledge of a signature, makes them very
suitable to build GS schemes -- even without exploiting the possibility to
encode many attributes. Indeed, BBS+ signatures \cite{bbs04,cdl16b} have been
used to design and instantiate GS schemes \cite{gl19,dl21}, as well as  PS16
signatures \cite{ps16,cdl+20}. Both are CL-like signature schemes, and thus
can be (and have been) used to design AC schemes \needcite.

Also, both AC and GS branches have experienced praiseworthy efforts towards
generalization. In the AC domain, malleable signatures are used as core building
block in \cite{cklm14}, to allow building both conventional AC and Delegatable
AC systems, depending on the transformation supported by the malleable
signatures. In GS, very recently, BiAS \cite{lnpy21} and MPS \needcite
flexibilize the tradeoff between anonymity and accountability.

%% Main drawbacks of precedents
\paragraph{Limitations of GS and AC.} %
However, even the generalization efforts that have independently taken place in
GS and AC, are quickly seen to have limited applicability in the real world.
On the GS side, even though MPS achieves a nice balance on when a signature is
deanonymizable and what information can be extracted from it, upon closer look,
this information is somehow meaningless. To understand it, consider how does the
generic construction work. Roughly, the signer encrypts the output of an
arbitrary function of its ``identity'', and proves correctness of this
computation. Yet, the concept of ``identity'' is not defined at all (it is a
value within some undefined domain). Thus, it seems hard to associate any real
world meaning to a function of something that is not clearly defined. In any
case, proving membership to a group may be enough in some use cases, but
definitely not as a general tool -- that is, GS  offer minimal utility at
signing time.
%
On the other hand, ACs have a well defined concept of identity (the attributes
encoded within a credential), yet their utility-after-authentication
capabilities are 
minimal -- something on which their GS counterpart has done decades of research.
A somehow indirect proof of this is that the AC line of work does not even
consider what security we can get in the presence of malicious issuers, which
is precisely captured in the non-frameability properties of GS.

\subsection{Our Contribution} 
%% Our results; TLDR (simple words)

\paragraph{General model for anonymous signatures.} %
Acknowledging that the strengths of AC are the weaknesses of GS, and vice versa,
we come up with a unified model. In a nutshell, we give a sytnax and model for
anonymous signatures -- which can be turned into an interactive authentication
system -- with attribute-enabled credentials, and generalized utility
both at authentication/signature-time, and after. We also generalize the
credential issuance process, subsuming an even larger set of previously
known schemes. Given its generalization power, we refer to schemes meeting this
model as Universal Anonymous Signatures.

\paragraph{Generic construction, and connections with previous schemes.} %
We give a generic construction based on well-known building blocks, and prove
its security in the proposed model. To prove its universality, we show how
restrictions of our generic construction directly imply the direct ancestors of
UAS: group signatures and anonymous credentials. This includes not only their
vanilla versions, but also relevant variants, such as MPS \needcite, (a modified
version of) group signatures with message-dependent opening \cite{seh+12}, or
anonymous credentials with revocation \cite{cks10}. We also show how minimal
variations in our generic construction -- and leveraging the generalization of
the issuance protocol -- allow us to build ring signatures \cite{rst06} or
delegatable anonymous credentials \cite{bcc+09}. The most interesting part being
that, while we prove implications with other previously known schemes, our
generalization approach allows to merge features of them: e.g., one can easily
model and instantiate GSs with with no revocation at all, ACs with full
de-anonymization, or anything in between.

%% Our results; Tech nuances
\paragraph{Modelling challenges.} %
Combining features of GS and AC into a unified model unavoidably brings
complexity. For instance, we allow proving arbitrary claims over previously
obtained credentials at issuance time, in order to inform the issuance
decission. Thus, it is necessary to
include unforgeability assurances already at that point -- something not
considered in previous works. Also, in order to meet the general practice in AC,
the model for \UAS includes multiple issuers, requiring careful bookkeeping to
prevent trivial wins. Interestingly, this generalization also paves the way to
include multiple openers, each with its own opening functionality -- also
requiring careful bookkeeping. While powerful from a practical point of view,
this introduces similar limitations privacy-wise as in functional encryption
schemes \needcite -- e.g., privacy cannot be ensured if some chosen opening
function produces different outputs for different users. However, on this
occasion, it is the signer (perhaps jointly with the verifier) who decides the
opening policy. Thus, the user is in full control of how deanonymizable its
signatures will be. This flexibility on the opening values also leads to the
need of using extraction-based techniques for the modelling, since we cannot
expect that opening outputs the real identity of the signer. This comes as no
surprise, though, as it was already observed in GS schemes with non-conventional
opening \cite{dl21}, in DAA schemes \cite{cdl16,cdl16b} or in AC schemes that
need to ensure correctness of information generated before authentication, like
delegatable credentials \cite{bcc+09}, that prove validity of the delegation
chain. In fact, simulation-extractability seems to be most useful here,
something that was also observed in \cite{cl06} for producing signatures out of
knowledge of witnesses for concrete NP statements.

A crucial aspect to cater for all possible variants that we aim to subsume, is
the incorporation of ``placeholders'' enabling customization. We introduce abstract
functions whose computation needs to be proven correct by the user. Depending on
the choice of these functions, one can instantiate a different related scheme,
which is anyway secure according to our \UAS model. And
besides them, we have to enable tweaking what utility information is (or can be)
revealed at signing-time and after, in a robust manner -- a scheme that allows
proving some claim at signing time, but a somehow contradictory one afterwards,
is of little use. Hence, the statements proving correctness of computation by
the signer need to include cross-references. This is also very useful, as one
can modulate the information to be revealed after signing-time, with the
information revealed at signing-time, allowing policies such as ``deanonymize
only if 18+ years old''.

%% Our techniques
%%% E.g., describe how we compatibilize both with AC sel disc., ring sigs, and MPS
\paragraph{Our approach.} %
The high-level idea for our generic construction is simple. Credentials are
signatures over blocks of messages (some signed in committed form), where each
message can be a different attribute, and a user who wants to simultaneously
leverage multiple credentials needs to bind all of them to the same key. We add
three generic functions: \fissue, leveraged by users to
prove claims over attributes in other credentials, in order to get a new one;
\feval, proving correctness of the utility revealed at signing time; and \finsp,
proving correctness of the utility extractable after signing. The latter is, in
addition, fed and modulated with the output of \feval. The output of \feval is
actually double: one part is shared in the clear along with the signature, and
the other is encrypted, so that it can be used to tweak -- in a privacy-enhanced
manner -- the behaviour of \finsp. The output of \finsp is encrypted with an
opener's key. The pattern of the statements proven in each function is
summarised in \figref{fig:proof-blocks-uas}.

\begin{figure}[ht!]
  \centering
  \scalebox{0.9}{
    \begin{tikzpicture}

  \pgfdeclarelayer{bg}    % declare background layer
  \pgfsetlayers{bg,main}  % set the order of the layers (main is the standard layer)

  \node (GSAC) at (3.00,3.50) { \GSAC };
  \node (UAS) at (9.00,3.50) { \UAS };

  \node (issue) at (0.00,2.50) { \bf Issue };
  \node (sign) at (0.00,1.50) { \bf Sign };  
  \node (open) at (0.00,0.50) { \bf Open };

  \node (issue-gsac) at (3.00,2.50) { $K(\usk)$ };
  \node[align=center] (sign-gsac) at (3.00,1.50)
  { $K(\usk) \land K(\cred) ~\land$ \\
    $\lbrace \usk,\attrs \rbrace \in \cred \land E(\upk)~\land$ \\
    $E(\attrs) \land \dattrs \subseteq \attrs$ };
  \node (open-gsac) at (3.00,0.50) { $D(\upk) \land D(\attrs)$ };

  \node (issue-uas) at (9.00,2.50)
  { $K(\usk) \land K(\scred) \land \usk \in \scred \land \fissue(\cdot) = 1$ };
  \node[align=center] (sign-uas) at (9.00,1.50)
  { $K(\usk) \land K(\scred) \land \usk \in \scred \land E(\yinsp)~\land$ \\
    $\yeval = \feval(\cdot) \land \yinsp = \finsp(\yeval,\cdot)$
  };
  \node (open-uas) at (9.00,0.50) { $D(\yinsp)$ };

  \draw[draw=black,thick] (10.80,2.29) rectangle ++(1.43,0.44);
  \draw[draw=black,dashed,thick] (6.80,1.09) rectangle ++(1.75,0.44);
  \draw[draw=black,dotted,thick] (8.85,1.09) rectangle ++(2.30,0.44);
    
\end{tikzpicture}

%%% Local Variables:
%%% mode: latex
%%% TeX-master: t
%%% End:

  }
  \caption{Summary of statements proved per operation in \UAS.
    $K(x)$ means proving knowledge of $x$; $E(y)$ means proving correct
    encryption of $y$; $D(z)$ means proving correct decryption of (an encryption
    of) $z$. Solid boxes span configurable statements that control issuance;
    dashed boxes span configurable statements that control utility at signing
    time; dotted boxes span configurable statements that control utility after
    signing time (accountability). Unboxed statements are fixed, and are thus
    common to all \UAS restrictions. $\usk \coloneqq$ user secret key; $\scred
    \coloneqq$ set of credentials.}
  \label{fig:proof-blocks-uas}
\end{figure}

From this generic construction, \CUASGen, one can already build known (and more
functionality-limited) schemes from what we call \emph{restrictions} of
\CUASGen simply by giving concrete definitions of \fissue, \feval, and \finsp.
Yet, the model remains the same, and we prove that if such restrictions of
\CUASGen are secure in the \UAS model, they lead to secure constructions for
the related primitives (GS variants, AC variants, etc). To define our generic
construction, we also specify concrete NP relations for the underlying NIZK
proofs for issuing, signing, and opening. Again with (unavoidable) minor variations
of these NP relations -- concretely to the one used for signing -- we show how
our \CUASGen variant can securely build ring signatures. That is, we prove that,
by simply tweaking a function, and treating all users as issuers, group
signatures an ring signatures are in essence the same thing.

We emphasize that, while -- already before \UAS -- it may have not seemed hard
to build either of these primitives from another, the core of our contribution
is that, with \UAS, one just needs to give concrete definitions of three
functions. Security of the final primitive then automatically follows from the
fact that it is a secure instantiation of an \UAS scheme. We believe that this
is a powerful notion, that may finally allow research lines (AC and GS) that
have followed different paths to benefit from the findings of each other.
\figref{fig:relations} graphically summarizes the relations we prove between
\UAS and other schemes -- although, of course, many more are probably
achievable.

\begin{figure}[ht!]
  \centering
  \scalebox{0.9}{
    \begin{tikzpicture}

  \node (uas) at (-0.25,2.25) { \bf \UAS };
  
  \node (gs) at (6.00,4.50) { GS \cite{bsz05}
    (\secref{ssec:related-models-gs})};
  \node (gsmdo) at (7.35,3.75) { GS-MDO (based on \cite{seh+12};
    \appref{sapp:related-models-gsmdo}) };
  \node (mps) at (6.20,3.00) { MPS \cite{ngsy22}
    (\secref{ssec:related-models-mps}) };
  \node (ac) at (6.00,2.25) { AC \cite{fhs19}
    (\secref{ssec:related-models-ac}) };
  \node (rac) at (7.35,1.50) { RAC (based on \cite{fhs19,cks10};
    \secref{ssec:related-models-rac}) };
  \node (dac) at (7.15,0.75) { DAC (based on \cite{fhs19,bcc+09};
    \appref{sapp:related-models-dac}) };
  \node (rs) at (6.00,0.00) { RS \cite{bkm06}
    (\secref{ssec:related-models-rs}) };

  \draw[->] (uas) -- (0.50,2.25) -- (0.50,4.50) -- node[midway,above]
  {$(\fissue^{single,L},\feval^0,\finsp^{\upk})$-\CUASGen} (gs);
  \draw[->] (uas) -- (0.50,2.25) -- (0.50,3.75) -- node[midway,above]
  {$(\fissue^{single,L},\feval^0,\finsp^{\smsg})$-\CUASGen} (gsmdo);
  \draw[->] (uas) -- (0.50,2.25) -- (0.50,3.00) -- node[midway,above]
  {$(\fissue^{single,L},\feval^0,\finsp^{\ast})$-\CUASGen} (mps);
  \draw[->] (uas) -- (0.50,2.25) -- node[midway,above]
  {$(\fissue^1,\feval^{\dattrs},\finsp^0)$-$\CUASGen^{int}$} (ac);
  \draw[->] (uas) -- (0.50,2.25) -- (0.50,1.50) -- node[midway,above]
  {$(\fissue^1,\feval^{\dattrs,F,V,L},\finsp^{F})$-$\CUASGen^{int}$} (rac);
  \draw[->] (uas) -- (0.50,2.25) -- (0.50,0.75) -- node[midway,above]
  {$(\fissue^{l,\dattrs},\feval^{l,\dattrs},\finsp^0)$-$\CUASGen^{int}$} (dac);
  \draw[->] (uas) -- (0.50,2.25) -- (0.50,0.00) -- node[midway,above]
  {$(\fissue^{\sring},\feval^{\sring},\finsp^0)$-$\CUASGen^{hide-iss}$} (rs);
    
\end{tikzpicture}

%%% Local Variables:
%%% mode: latex
%%% TeX-master: t
%%% End:

  }
  \caption{Summary of schemes that can be instantiated as restrictions of our
    generic construction, \CUASGen, and proved secure under our \UAS model.
    $\CUASGen^{\ast}$ denotes that minor changes to the basic construction
    beyond the concrete $(\fissue,\feval,\finsp)$-restrictions are needed.
    Between parenthesis we show the adopted reference model, and the section
    offering full detail.
  }
  \label{fig:relations}
\end{figure}

We give some preliminaries in \secref{sec:preliminaries}. Then proceed with our
\UAS syntax, model, and construction, in \secref{sec:formal-uas}, and show
relationships with other primitives in \secref{sec:relationships}.. We conclude
in \secref{sec:conclusion}. Deferred proofs and additional information appear
in the appendices.

%%% Local Variables:
%%% mode: latex
%%% TeX-master: "uas-paper"
%%% End:
