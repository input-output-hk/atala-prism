\section{Introduction}
\label{sec:introduction}

% [4 pages max]

%% Brief description of main precedents (simple words)
\paragraph{Anonymous authentication and signatures.} %
Privacy-preserving digital signature and authentication systems have been a
prolific field in the last few decades. Anonymous credentials (AC)
\cite{chau85} and group signatures (GS) \cite{ch91} are among the most relevant
schemes. For both, the high-level idea is common: after obtaining a credential,
users can authenticate anonymously and indistinguishably from other users with
similar credentials. 
%
%% Usefulness of main precedents / motivation
Their usefulness stems from the combination, within a single primitive, of
seemingly contradictory concepts: anonymity, authentication, and utility; where
utility can appear in the shape of information revealed at signature/authentication
time, or after (in the latter case, typically called accountability).

\paragraph{Utility at authentication time and after authentication time.} %
AC schemes focus on utility at authentication time. For instance, encoding
multiple attributes within a credential and allowing to selectively disclose a
subset of them. The real world applications of such systems are direct, e.g.,
to prove ``Over 18''-type of claims. Indeed, the scientific research behind
anonymous credentials is making its own way to become a foundational block of
the technology some call ``anoncreds''%
\footnote{\url{https://hyperledger-indy.readthedocs.io/projects/sdk/en/latest/docs/design/002-anoncreds/README.html}. Last access, June 18th, 2022.},
the privacy-preserving cousin of W3C's Verifiable Credentials%
\footnote{\url{https://www.w3.org/TR/vc-data-model/}. Last access, June 18th,
  2022.}.

GS schemes enhance utility after authentication, by allowing partial
re-identification like linking of signatures, or even full deanonymization.
While actual applications of group signatures seem hard to find in the real
world, Direct Anonymous Attestation (DAA) \cite{bcc04,bl07,cdl16b} has been
deployed to millions of computing devices into what's called TPM (Trusted
Privacy Module)%
\footnote{\url{https://trustedcomputinggroup.org/resource/trusted-platform-module-tpm-summary/}. Last access, June 18th, 2022.}.
Indeed, DAA is a very close relative of group signatures, where the signer is
divided into two components (trusted and partially trusted), and signing
and verification are interactive.

%% Previous works summary w/ refs directly related contribs of UAS
%% (or, tech approaches of main precedents)
\paragraph{Evolution of AC and GS.} %
As the previous shows, AC and GS systems have followed separate paths, although
there are some exceptions. Functionality-wise, Attribute-Based Group Signatures
introduce attributes into the credentials issued to group members, who can later
use them to prove pre-fixed claims \cite{emo09,aa14}. Hidden identity-based
signatures also add attributes into the credentials, with the aim of preventing
the opener (the trusted party who can deanonymize signers) from having to rely
on the issuer in order to return meaningful identifying information when opening
a signature \cite{ks07}. Some AC schemes have also tried to incorporate some
sort of utility after authentication, like AC schemes with revocation
\cite{cks10}, or with ``watchlists'' \needcite.

This separation does not hold in the approaches to build them, though. The
probably most fruitful way to instantiate AC is with CL-type signatures
\cite{cl02}. But the fact that they support efficient
zero-knowledge protocols for proving knowledge of a signature, makes them very
suitable to build GS schemes -- even without exploiting the possibility to
encode many attributes. Indeed, BBS+ signatures \cite{bbs04,cdl16b} have been
used to design and instantiate GS schemes \cite{gl19,dl21}, as is the case for
PS16 signatures \cite{ps16,cdl+20} -- both, CL-type signatures.

Also, both AC and GS branches have experienced remarkable efforts towards
generalisation. In the AC domain, malleable signatures are used as core building
block in \cite{cklm14}, to allow building both conventional AC and Delegatable
AC systems, depending on the transformation supported by the malleable
signatures. In GS, very recently, BiAS \cite{lnpy21} and MPS \cite{ngsy22}
flexibilize the trade-off between anonymity and accountability.

%% Main drawbacks of precedents
\paragraph{Limitations of GS and AC.} %
However, even the generalisation efforts that have independently taken place in
GS and AC, are quickly seen to have limited applicability in the real world. On
the GS side, even though MPS achieves a powerful balance on when a signature is
deanonymizable and what information can be extracted from it, upon closer look,
this information is somehow meaningless. Namely, the notion of identity is not
well defined and, thus, it is not possible to claim whether the value output by
their generalised open function makes sense or even if it is legitimate. Also,
just being able to prove membership at signing time is typically not enough, and
one needs to prove more general claims. That is precisely what AC schemes do,
while also providing a well defined notion of identity via credential
attributes. Precisely, it is this defined concept of identity what enables AC
schemes to state when some information revealed at authentication time is a
forgery or not. On the other hand, the notion of accountability  is barely
present in AC schemes. For instance, non-frameability is not even considered
-- and it is non-frameability what ensures when cannot a user be held
accountable.
%
In any case, perhaps the most impractical limitation of all is that too many
models exist in both fields. Although similar, each model focuses on slightly
different utility variants, making them hardly comparable. And, of course, AC
focus on utility at authentication (signing) time, while GS focus on
utility after signing (authentication) time. To the best of our knowledge, no
model exists that combines both, in a general enough manner. Probably, the
most ambitious work in this regard is \cite{ckl+15}, although it focuses only on
anonymous credentials. In practice, needing to resort to too specific models
forces system designers to decide which feature (utility at or after signing
time, etc) they do not want to lose -- or, which one they will lose.

\subsection{Our Contribution} 
%% Our results; TLDR (simple words)

\paragraph{General model for anonymous signatures.} %
Acknowledging that the strengths of AC are the weaknesses of GS, and vice versa,
and the inconvenience in practice of having a miriad of similar-but-different
models, we come up with a unified one. In a nutshell, we give a syntax and model
for anonymous signatures -- which can be turned into an interactive
authentication system -- with attribute-enabled credentials, and generalised
utility both at signature time, and after. Concretely, we propose a scheme
through which it is possible to model revealing arbitrary claims at signing time
(as in AC), enable conditional release of utility information after signing time
(as in MPS) and, more importantly, come up with the combination of the previous
that is more suitable for each scenario. We also generalise the credential
issuance process, subsuming an even larger set of previously known schemes.
Given its generalisation power, we refer to schemes meeting this model as
Universal Anonymous Signatures.

\paragraph{Generic construction, and connections with previous schemes.} %
We give a main generic construction based on well-known building blocks, and
prove its security. To prove its universality, we show how restrictions of our
generic construction directly imply ancestors of UAS: group
signatures and anonymous credentials. This includes not only their vanilla
versions, but also relevant variants, such as MPS \cite{ngsy22}, (a modified
version of) group signatures with message-dependent opening \cite{seh+12}, or
anonymous credentials with revocation \cite{cks10} or delegation \cite{bcc+09}.
Strictly speaking, for AC schemes, we build on a simple transform of our main
construction and model that turns non-interactive signatures into interactive
authentication protocols with protection against replay attacks\footnote{This is
  the well-known folklore approach that has the verifier send a fresh nonce to
  be signed by the alleged owner of the private key related to a known
  public key.}. We further show how to
build ring signatures (RS) in the model given by \cite{bkm06}, from an
alternative \UAS construction that improves issuer privacy.
%
The most interesting part being that, while we prove implications with other
previously known schemes, our generalisation approach allows to merge features
of them: e.g., one can easily model and instantiate GSs with with no revocation
at all, ACs with full de-anonymization, RS with linkability, or anything in
between.

%% Our results; Tech nuances
\paragraph{Modelling challenges.} %
Combining and extending features of GS and AC into a unified model unavoidably
brings complexity. For instance, we allow proving arbitrary claims over
previously obtained credentials at issuance time, in order to inform the
issuance decision. Thus, it is necessary to model unforgeability guarantees
already at that point -- something not considered in previous works as far as
we know. Also, in
order to meet the general practice in AC, the model for \UAS includes multiple
issuers. Interestingly, combining this generalisation with the GS world also
paves the way to include multiple openers, each with its own opening
functionality. While powerful from a practical point of view, this introduces
similar (unavoidable?) limitations privacy-wise as in functional encryption
schemes \cite{bsw11} -- e.g., revealing too granular information at signing or
opening time makes privacy go away. However, on this occasion, it is the signer
(perhaps jointly with the verifier) who decides the signing and
opening policy. Thus, the user is in full control of how deanonymizable its
signatures will be, and knows it at signature generation time. This flexibility
on the opening values also leads to the
need of using extraction-based techniques for the modelling, since we cannot
expect that opening outputs the real identity of the signer. This comes as no
surprise, though, as it was already observed in GS schemes with non-conventional
opening \cite{dl21}, in DAA schemes \cite{cdl16,cdl16b} or in AC schemes that
need to ensure correctness of information generated before authentication, like
delegatable credentials \cite{bcc+09}, that prove validity of the delegation
chain. In fact, simulation-extractability seems to be most useful here,
something that was also observed in \cite{cl06} for producing signatures out of
knowledge of witnesses for concrete NP statements.

A crucial aspect to cater for all possible variants that we aim to subsume, is
the incorporation of ``functional placeholders'' enabling customisation. We
introduce abstract functions whose computation needs to be proven correct by the
user. Depending on the choice of these functions, one can instantiate a
different related scheme, which is anyway secure according to our \UAS model.
We also allow to modulate the information to be revealed after signing time,
with the information revealed at signing time, allowing policies such as
``deanonymize only if 18+ years old''.

\paragraph{Sample real-world use cases.} %
By enabling customization at issuance, signing and opening times, as well as
multiple uncoupled issuers and openers, the
possibilities are large. As an example, consider a social network that requires
under-age users to get permission from one parent. This can be defined via a
delegatable credential scheme, checking that the delegation chain of the user
requesting to join is valid via a policy at issuance time. Later, in order to
make some post in the system advertising something to be sold in a local
second-hand page in the social network, the user may need to reveal, at signing
time, being a citizen of the local area (e.g. city). This can be done via an
appropriate policy at signing time. If, after the product is sold, it proves
to be defective, the social network may directly output the email of the user
if s/he is over-age, or the email of the adult in charge if s/he is under-age.
And that could be achieved via a policy for utility at opening time. Finally,
if, in the same social network, some other
issuer-user-verifier situation comes up that requires a different set of utility
choices, just the appropriate functions would need to be changed, without the
need to re-model from scratch (or at all).

%% Our techniques
%%% E.g., describe how we compatibilize both with AC sel disc., ring sigs, and MPS
\paragraph{Our construction approach.} %
The high-level idea for our generic construction \CUASGen is simple. Credentials
are signatures over blocks of messages (some signed in committed form), where
each message can be a different attribute, and a user who wants to
simultaneously leverage multiple credentials needs to bind all of them to the
same key. We add three generic functions: \fissue, leveraged by users at
credential issuance time to prove claims over attributes in other credentials,
in order to get a new one; \feval, proving correctness of the utility revealed
at signing time; and \finsp, proving correctness of the utility information
extractable after signing. The latter is, in addition, fed and modulated with
the output of \feval. The pattern of the statements proven in each function is
summarised in \figref{fig:proof-blocks-uas}.

\begin{figure}[ht!]
  \centering
  \scalebox{0.9}{
    \begin{tikzpicture}

  \pgfdeclarelayer{bg}    % declare background layer
  \pgfsetlayers{bg,main}  % set the order of the layers (main is the standard layer)

  \node (GSAC) at (3.00,3.50) { \GSAC };
  \node (UAS) at (9.00,3.50) { \UAS };

  \node (issue) at (0.00,2.50) { \bf Issue };
  \node (sign) at (0.00,1.50) { \bf Sign };  
  \node (open) at (0.00,0.50) { \bf Open };

  \node (issue-gsac) at (3.00,2.50) { $K(\usk)$ };
  \node[align=center] (sign-gsac) at (3.00,1.50)
  { $K(\usk) \land K(\cred) ~\land$ \\
    $\lbrace \usk,\attrs \rbrace \in \cred \land E(\upk)~\land$ \\
    $E(\attrs) \land \dattrs \subseteq \attrs$ };
  \node (open-gsac) at (3.00,0.50) { $D(\upk) \land D(\attrs)$ };

  \node (issue-uas) at (9.00,2.50)
  { $K(\usk) \land K(\scred) \land \usk \in \scred \land \fissue(\cdot) = 1$ };
  \node[align=center] (sign-uas) at (9.00,1.50)
  { $K(\usk) \land K(\scred) \land \usk \in \scred \land E(\yinsp)~\land$ \\
    $\yeval = \feval(\cdot) \land \yinsp = \finsp(\yeval,\cdot)$
  };
  \node (open-uas) at (9.00,0.50) { $D(\yinsp)$ };

  \draw[draw=black,thick] (10.80,2.29) rectangle ++(1.43,0.44);
  \draw[draw=black,dashed,thick] (6.80,1.09) rectangle ++(1.75,0.44);
  \draw[draw=black,dotted,thick] (8.85,1.09) rectangle ++(2.30,0.44);
    
\end{tikzpicture}

%%% Local Variables:
%%% mode: latex
%%% TeX-master: t
%%% End:

  }
  \caption{Summary of statements proved per operation in \UAS.
    $K(x)$ means proving knowledge of $x$; $E(y)$ means proving correct
    encryption of $y$; $D(z)$ means proving correct decryption of (an encryption
    of) $z$. $\usk \coloneqq$ user secret key; $\scred \coloneqq$ set of
    credentials.}
  \label{fig:proof-blocks-uas}
\end{figure}

From \CUASGen, one can already build known (and more functionality-limited)
schemes from what we call \emph{restrictions} of \CUASGen simply by giving
concrete definitions of \fissue, \feval, and \finsp. Yet, the model remains the
same, and we prove that if such restrictions of \CUASGen are secure in the \UAS
model, they lead to secure constructions for the related primitives (GS
variants, AC variants, etc). To define our generic construction, we also specify
concrete NP relations for the underlying NIZK proofs for issuing, signing, and
opening. These NP relations must not be considered as immutable, though --
informally, it would seem that they must simply follow the pattern in
\figref{fig:proof-blocks-uas}. Actually, as mentioned, we show how by making a
minor variation into the relation used for signing, a \CUASGen restriction can
securely build ring signatures.

We emphasize that, while -- already before \UAS -- it may have not seemed hard
to build either of these primitives from another, the core of our contribution
is that, with \UAS, one essentially just needs to give concrete definitions of
three functions. Security of the final primitive then basically follows from
the fact that it is a secure instantiation of an \UAS scheme. We believe that
this is a powerful notion, that may finally allow research lines (AC and GS)
that have followed different paths to easily benefit from the findings of each
other. \figref{fig:relations} graphically summarises the relations we prove
between \UAS and other schemes -- although, of course, many more are probably
achievable.

\begin{figure}[ht!]
  \centering
  \scalebox{0.9}{
    \begin{tikzpicture}

  \node (uas) at (-0.25,2.25) { \bf \UAS };
  
  \node (gs) at (6.00,4.50) { GS \cite{bsz05}
    (\secref{ssec:related-models-gs})};
  \node (gsmdo) at (7.35,3.75) { GS-MDO (based on \cite{seh+12};
    \appref{sapp:related-models-gsmdo}) };
  \node (mps) at (6.20,3.00) { MPS \cite{ngsy22}
    (\secref{ssec:related-models-mps}) };
  \node (ac) at (6.00,2.25) { AC \cite{fhs19}
    (\secref{ssec:related-models-ac}) };
  \node (rac) at (7.35,1.50) { RAC (based on \cite{fhs19,cks10};
    \secref{ssec:related-models-rac}) };
  \node (dac) at (7.15,0.75) { DAC (based on \cite{fhs19,bcc+09};
    \appref{sapp:related-models-dac}) };
  \node (rs) at (6.00,0.00) { RS \cite{bkm06}
    (\secref{ssec:related-models-rs}) };

  \draw[->] (uas) -- (0.50,2.25) -- (0.50,4.50) -- node[midway,above]
  {$(\fissue^{single,L},\feval^0,\finsp^{\upk})$-\CUASGen} (gs);
  \draw[->] (uas) -- (0.50,2.25) -- (0.50,3.75) -- node[midway,above]
  {$(\fissue^{single,L},\feval^0,\finsp^{\smsg})$-\CUASGen} (gsmdo);
  \draw[->] (uas) -- (0.50,2.25) -- (0.50,3.00) -- node[midway,above]
  {$(\fissue^{single,L},\feval^0,\finsp^{\ast})$-\CUASGen} (mps);
  \draw[->] (uas) -- (0.50,2.25) -- node[midway,above]
  {$(\fissue^1,\feval^{\dattrs},\finsp^0)$-$\CUASGen^{int}$} (ac);
  \draw[->] (uas) -- (0.50,2.25) -- (0.50,1.50) -- node[midway,above]
  {$(\fissue^1,\feval^{\dattrs,F,V,L},\finsp^{F})$-$\CUASGen^{int}$} (rac);
  \draw[->] (uas) -- (0.50,2.25) -- (0.50,0.75) -- node[midway,above]
  {$(\fissue^{l,\dattrs},\feval^{l,\dattrs},\finsp^0)$-$\CUASGen^{int}$} (dac);
  \draw[->] (uas) -- (0.50,2.25) -- (0.50,0.00) -- node[midway,above]
  {$(\fissue^{\sring},\feval^{\sring},\finsp^0)$-$\CUASGen^{hide-iss}$} (rs);
    
\end{tikzpicture}

%%% Local Variables:
%%% mode: latex
%%% TeX-master: t
%%% End:

  }
  \caption{Summary of schemes that can be instantiated as restrictions of our
    generic constructions and proved secure under our \UAS model. Between
    parenthesis we show the adopted reference model, and the section offering
    full detail.}
  \label{fig:relations}
\end{figure}

\paragraph{Outline.} %
We give some preliminaries in \secref{sec:preliminaries}. Then proceed with our
\UAS syntax, model, and construction, in \secref{sec:formal-uas}, and show
relationships with other primitives in \secref{sec:relationships}. We conclude
in \secref{sec:conclusion}. Deferred proofs and additional information appear
in the appendices.

%%% Local Variables:
%%% mode: latex
%%% TeX-master: "uas-paper"
%%% End:
