\section{Introduction}
\label{sec:introduction}

% [4 pages max]

%% Brief description of main precedents (simple words)
Privacy-preserving digital signature and authentication systems have been a
prolific field of study in the last few decades. Anonymous credentials
\cite{chau85} and group signatures \cite{ch91} are among the most relevant ones.
\todo{Other closely related schemes?}
While there are many variants, the high-level idea is common: after obtaining a
credential, users can authenticate anonymously and indistinguishably from other
users with similar credentials. 

%% Usefulness of main precedents / motivation
The usefulness of these systems stems from the combination, within a single
primitive, of seemingly contradictory concepts: anonymity, authentication, and
utility.

Anonymous credentials focus on utility at authentication time. For
instance, encoding multiple attributes within a credential and allowing to
selectively disclose a subset of them (or arbitrary predicates) at
authentication time. The real world applications of such systems is direct,
e.g., to prove ``Over 18''-type of claims. Indeed, anonymous credentials are
making their own way to become a foundational block of ``anoncreds''%
\footnote{\url{https://hyperledger-indy.readthedocs.io/projects/sdk/en/latest/docs/design/002-anoncreds/README.html}. Last access, June 18th, 2022.},
the privacy-preserving cousin of W3C's Verifiable Credentials%
\footnote{\url{https://www.w3.org/TR/vc-data-model/}. Last access, June 18th,
  2022.}.

Group signatures enhance utility after authentication, by allowing partial
reidentification, like linking of signatures, or even full deanonymization.
While direct applications of group signatures seem hard to find, Direct
Anonymous Attestation (DAA) \cite{bcc04,bl07,cdl16b} has been deployed to
millions of computing devices into what's called TPM (Trusted Privacy Module)%
\footnote{\url{https://trustedcomputinggroup.org/resource/trusted-platform-module-tpm-summary/}. Last access, June 18th, 2022.}.
Indeed, DAA is a very close relative of group signatures, where the signer is
divided into two components (trusted and partially trusted), and treats signing
and verification as an interactive protocol.

%% Previous works summary w/ refs directly related contribs of UAS
%% (or, tech approaches of main precedents)


%% Main drawbacks of precedents

%% Our results; TLDR (simple words)

%% Our results; Tech nuances

%% Our techniques

%%% E.g., describe how we compatibilize both with AC sel disc., ring sigs, and MPS



%%% Local Variables:
%%% mode: latex
%%% TeX-master: "uas-paper"
%%% End:
