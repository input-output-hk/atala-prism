\section{Introduction}
\label{sec:introduction}

% [4 pages max]

%% Brief description of main precedents (simple words)
\paragraph{Anonymous authentication and signatures.} %
Privacy-preserving signature and authentication systems have been a prolific
field in the last decades. Anonymous credentials (AC) \cite{chau85} and group
signatures (GS) \cite{ch91} are among the most relevant schemes. In both, the
high-level idea is common: users obtain credentials, which can then use to
authenticate anonymously and indistinguishably from other users with similar
credentials. 
%
%% Usefulness of main precedents / motivation
Their usefulness stems from the combination of seemingly contradictory concepts:
anonymity, authentication, and utility; where utility can appear in the shape of
information revealed at signature/authentication time, or after (in the latter
case, typically called accountability).

\paragraph{Utility at, and after, authentication time.} %
AC schemes target utility at authentication time. For instance, allowing
users to selectively disclose subsets of the attributes in their credentials.
Their real world applications are direct, e.g., to prove ``Over 18''-type of
claims. Indeed, the scientific research behind anonymous credentials is becoming
a foundational block of the technology some call ``anoncreds''%
\footnote{\url{https://hyperledger-indy.readthedocs.io/projects/sdk/en/latest/docs/design/002-anoncreds/README.html}. Last access, June 18th, 2022.},
the privacy-preserving cousin of W3C's Verifiable Credentials%
\footnote{\url{https://www.w3.org/TR/vc-data-model/}. Last access, June 18th,
  2022.}.

GS schemes enhance utility after authentication, allowing partial
re-identification like linking of signatures, or even full deanonymization.
While actual applications of group signatures seem hard to find in the real
world, Direct Anonymous Attestation (DAA) \cite{bcc04,bl07,cdl16b} has been
deployed to millions of computing devices into what's called TPM (Trusted
Privacy Module)%
\footnote{\url{https://trustedcomputinggroup.org/resource/trusted-platform-module-tpm-summary/}. Last access, June 18th, 2022.}.
Indeed, DAA is a very close relative of group signatures, where the signer is
divided into two components (trusted and partially trusted), and signing
and verification are interactive.

%% Previous works summary w/ refs directly related contribs of UAS
%% (or, tech approaches of main precedents)
\paragraph{Evolution of AC and GS.} %
AC and GS systems have followed separate paths, with some exceptions.
Functionality-wise, Attribute-Based Group Signatures add attributes to the
credentials issued to group members, who can later use them to prove pre-fixed
claims \cite{emo09,aa14}. Hidden identity-based signatures also add attributes
to the credentials, with the aim of preventing the opener (the trusted party who
can deanonymize signers) from having to rely on the issuer in order to return
meaningful identifying information when opening a signature \cite{ks07}. Some AC
schemes also incorporate some sort of utility after authentication, like
AC schemes with revocation \cite{cks10}.

This separation does not hold in the approaches to build them, though. The
probably most fruitful way to instantiate AC is with CL-type signatures
\cite{cl02}. But the fact that CL signatures support efficient ZK protocols for
proving knowledge of a signature makes them also very suitable to build GS
schemes. Indeed, BBS+ signatures \cite{bbs04,cdl16b} have been used to design
and instantiate GS schemes \cite{gl19,dl21}, as is the case for PS16 signatures
\cite{ps16,cdl+20} -- both, CL-type signatures.

AC and GS have experienced remarkable efforts towards generalisation. In
the AC domain, malleable signatures are used as core building block in
\cite{cklm14}, to allow building both conventional AC and Delegatable AC
systems, depending on the transformation supported by the malleable signatures.
Also, \cite{ckl+15} gives a very generic approach to build AC schemes. In GS,
very recently, BiAS \cite{lnpy21} and MPS \cite{ngsy22} flexibilize the
trade-off between anonymity and accountability.

%% Main drawbacks of precedents
\paragraph{Limitations of GS and AC.} %
However, even the generalisation efforts that have independently taken place in
GS and AC have limited applicability in the real world. On the GS side, even
though MPS achieves a powerful balance on when a signature is
deanonymizable and what information can be extracted from it, upon closer look,
this information is somehow meaningless. Namely, the notion of identity is not
well defined and, thus, it is not possible to assess whether the value output by
their generalised open function makes sense or even if it is legitimate. Also,
just being able to prove membership at signing time is typically not enough. In
other words, in GS, utility at authentication time is very limited -- but this
is precisely the main strength in AC schemes, which also have a well defined
notion of identity via attributes. On the other hand, the notion of
accountability  is barely present in AC schemes. For instance, non-frameability
is not even considered, and it is non-frameability what prevents honest users
from being held accountable of dishonest actions they did not take part in.
%
In any case, perhaps the most impractical limitation of all is that too many
models exist in both fields. Although similar, each model focuses on slightly
different utility variants, making them hardly comparable. To the best of our
knowledge, no model exists that combines both fields, in a general enough
manner. In practice, needing to resort to too specific models forces system
designers to decide which feature they do not want to lose -- or, which one they
will lose.

\subsection{Our Contribution} 
%% Our results; TLDR (simple words)

\paragraph{General model for anonymous signatures.} %
We give a syntax and model for anonymous signatures with attribute-enabled
credentials and, inspired by GS and AC works, generalised utility.

We observe that in all GS and AC models, concrete
claims related to user knowledge are proven either at issuance, signature
generation, and opening times. Typically, at issuance time, the user proves
knowledge of a secret key; at signing time, she proves that her secret key (and,
potentially, more attributes) has been attested by some issuer; finally, at
opening time, the opener proves that some fixed user-related information is
contained in the de-anonymized signature.

We note that instead of fixing in the model what concrete claim has to
be made at each point, we can introduce \emph{functional placeholders} to be
specified at the construction level. Of course, these functions need
to depend on the user key and/or user credentials, attested attributes and
signed message. But, in the modelling, we should only be concerned about their
consistency -- i.e., that whatever is proved, relates to the correct user
credential, signed message, etc.
%
It is through these functional placeholders that we can later tweak what
information is required and revealed, and when. Given its generalisation
capabilities, we refer to schemes meeting this model as Universal Anonymous
Signatures.

But the flexibility obtained from combining models from the AC and GS
domains into one also brings complexity. For instance, in order to meet
the general practice in AC, our UAS model includes multiple issuers. Along with
the combination of GS features, this paves the way to including multiple
openers, each with its own opening functionality. While powerful from a
practical point of view, this introduces similar (unavoidable?) limitations
privacy-wise as in functional encryption schemes \cite{bsw11} -- e.g., revealing
too granular information at signing or opening time reduces privacy. However, on
this occasion, signers control ``how deanonymizable'' their signatures will
be, and know it at signature generation time. This flexibility on the opening
values also leads to the need of using extraction-based techniques for the
modelling, since we cannot expect that opening outputs the real identity of the
signer. This is no surprise, though, as it was already observed in GS schemes
with non-conventional opening \cite{dl21}, in DAA schemes \cite{cdl16,cdl16b}.
In fact, simulation-extractability seems to be most useful here, something that
was also observed in \cite{cl06} for producing signatures out of knowledge of
witnesses for concrete NP statements.

\paragraph{Generic construction.} %
We give \CUASGen, a generic construction of UAS based on well-known
building blocks, and prove its security. Despite its functional flexibility, our
construction is simple. Like many previous related schemes, credentials are
signatures over blocks of messages -- each message encoding a different
attribute. Users simultaneously leveraging multiple credentials need to prove
that they are bound to the same key. Utility information to be revealed at
signing time is output in plaintext, and utility information extractable by
openers is encrypted under their public key -- in both cases, users prove
correct computation.

The three functional placeholders mentioned above are \fissue, \feval, and
\finsp. \fissue is the function that users are required to evaluate at issuance
time, to prove knowledge of a user key and, potentially, claims over attributes
in previously obtained credentials. \feval is computed at signing time by the
user, and controls and proves correctness of the information learned by
verifiers. Finally, \finsp, although  computed also at signing time by the
users, proves correctness of the utility information extractable after signing.
The latter is, in addition, fed and modulated with the output of \feval. With
these functions, we allow achieving configurable utility-vs-privacy tradeoffs
throughout the whole ``lifecycle'' (from issuance to opening). They also take
a central role in defining the NP statements that need to be proven by different
algorithms/protocols, summarised in \figref{fig:proof-blocks-uas}.

\begin{figure}[ht!]
  \centering
  \scalebox{0.85}{
    \begin{tikzpicture}

  \pgfdeclarelayer{bg}    % declare background layer
  \pgfsetlayers{bg,main}  % set the order of the layers (main is the standard layer)

  \node (issue) at (0.00,2.50) { \bf Issue };
  \node (sign) at (0.00,1.50) { \bf Sign };  
  \node (open) at (0.00,0.50) { \bf Open };

  \node (issue-uas) at (4.00,2.50)
  { $K(\usk) \land K(\scred) \land \usk \in \scred \land \fissue(\cdot) = 1$ };
  \node[align=center] (sign-uas) at (4.00,1.50)
  { $K(\usk) \land K(\scred) \land \usk \in \scred \land E(\yinsp)~\land$ \\
    $\Yeval = \feval(\cdot) \land \yinsp = \finsp(\Yeval,\cdot)$
  };
  \node (open-uas) at (4.00,0.50) { $D(\yinsp)$ };

  % \draw[draw=black,thick] (5.65,2.29) rectangle ++(1.43,0.44);
  % \draw[draw=black,dashed,thick] (1.80,1.09) rectangle ++(1.75,0.44);
  % \draw[draw=black,dotted,thick] (3.85,1.09) rectangle ++(2.30,0.44);
    
\end{tikzpicture}

%%% Local Variables:
%%% mode: latex
%%% TeX-master: t
%%% End:

  }
  \caption{Summary of statements proved per operation in \UAS.
    $K(x)$ means proving knowledge of $x$; $E(y)$ means proving correct
    encryption of $y$; $D(z)$ means proving correct decryption of (an encryption
    of) $z$. $\usk \coloneqq$ user secret key; $\scred \coloneqq$ set of
    credentials.}
  \label{fig:proof-blocks-uas}
\end{figure}

\paragraph{Universality.}
To formally showcase that UAS can indeed claim to be universal, we prove that
restrictions of our generic construction imply the GS and AC ancestors of UAS.
This includes not only their vanilla versions, but also relevant variants,
such as MPS \cite{ngsy22}, (a modified version of) group signatures with
message-dependent opening \cite{seh+12}, or anonymous credentials with
revocation \cite{cks10} or delegation \cite{bcc+09}.
%
Strictly speaking, for AC schemes, we build on a simple transform of our main
construction and model that turns non-interactive signatures into interactive
authentication protocols with protection against replays\footnote{This is
  the well-known folklore approach that has the verifier send a fresh nonce to
  be signed by the alleged owner of the private key related to a known
  public key.}. We further build ring signatures (RS) in the model given by
\cite{bkm06}, from an alternative UAS construction that improves issuer privacy.
%
The most interesting part being that, while we prove implications with
previously known schemes, our approach allows to merge their features: e.g.,
one can easily instantiate GSs with with no revocation at all, ACs
with full de-anonymization, RS with linkability, etc.

While already before \UAS it may have not seemed hard to build either of these
primitives from another, with \UAS one essentially just needs to give concrete
definitions of \fissue, \feval and \finsp in what we call a $(\fissue,\feval,
\finsp)$-\CUASGen restriction. Security then basically follows from
the fact that \CUASGen is a secure \UAS scheme. We believe that this is a
powerful notion, that may finally allow research lines that have
followed different paths to easily benefit from the findings of each other.
\figref{fig:relations} graphically summarises the relations we prove between
\UAS and other schemes -- although, of course, many more are probably
achievable.

\begin{figure}[ht!]
  \centering
  \scalebox{0.85}{
    \begin{tikzpicture}

  \node (uas) at (0.00,2.25) { \bf \UAS };
  
  \node (gs) at (6.00,4.50) { GS \cite{bsz05} (\secref{xxx})};
  \node (gsmdo) at (7.35,3.75) { GS-MDO (based on \cite{seh+12}; \secref{xxx}) };
  \node (mps) at (6.20,3.00) { MPS \needcite (\secref{xxx}) };
  \node (ac) at (6.00,2.25) { AC \cite{fhs19} (\secref{xxx}) };
  \node (rac) at (7.15,1.50) { RAC (based on \cite{fhs19,cks10}; \secref{xxx}) };
  \node (dac) at (7.15,0.75) { DAC (based on \cite{fhs19,bcc+09}; \secref{xxx}) };
  \node (rs) at (6.00,0.00) { RS \cite{bkm06} (\secref{xxx}) };

  \draw[->] (uas) -- (1.00,2.25) -- (1.00,4.50) -- node[midway,above]
  {$(\fissue^1,\feval^0,\finsp^{\upk})$-\CUASGen} (gs);
  \draw[->] (uas) -- (1.00,2.25) -- (1.00,3.75) -- node[midway,above]
  {$(\fissue^1,\feval^0,\finsp^{\smsg})$-\CUASGen} (gsmdo);
  \draw[->] (uas) -- (1.00,2.25) -- (1.00,3.00) -- node[midway,above]
  {$(\fissue^1,\feval^0,\finsp^{\ast})$-\CUASGen} (mps);
  \draw[->] (uas) -- (1.00,2.25) -- node[midway,above]
  {$(\fissue^1,\feval^{\dattrs},\finsp^0)$-\CUASGen} (ac);
  \draw[->] (uas) -- (1.00,2.25) -- (1.00,1.50) -- node[midway,above]
  {$(\fissue^1,\feval^{\dattrs,F,V,L},\finsp^{F})$-\CUASGen} (rac);
  \draw[->] (uas) -- (1.00,2.25) -- (1.00,0.75) -- node[midway,above]
  {$(\fissue^{n,\dattrs},\feval^{n,\dattrs},\finsp^0)$-$\CUASGen^{\ast}$} (dac);
  \draw[->] (uas) -- (1.00,2.25) -- (1.00,0.00) -- node[midway,above]
  {$(\fissue^{\sring},\feval^{\attrs},\finsp^0)$-$\CUASGen^{\ast}$} (rs);
    
\end{tikzpicture}

%%% Local Variables:
%%% mode: latex
%%% TeX-master: t
%%% End:

  }
  \caption{Summary of schemes that can be instantiated as restrictions of our
    generic constructions and proved secure under our \UAS model. Between
    parenthesis we show the adopted reference model, and the section offering
    full detail. \CUASGenInt and \CUASGenHideIss are variations of \CUASGen,
    described in \secref{sec:formal-uas}.}
  \label{fig:relations}
\end{figure}



% \paragraph{Modelling challenges.} %
% Combining and extending features of GS and AC into a unified model unavoidably
% brings complexity. For instance, we allow proving arbitrary claims over
% previously obtained credentials at issuance time, in order to inform the
% issuance decision. Thus, it is necessary to model unforgeability guarantees
% already at that point -- something not considered in previous works as far as
% we know. Also, in
% order to meet the general practice in AC, the model for \UAS includes multiple
% issuers. Interestingly, combining this generalisation with the GS world also
% paves the way to include multiple openers, each with its own opening
% functionality. While powerful from a practical point of view, this introduces
% similar (unavoidable?) limitations privacy-wise as in functional encryption
% schemes \cite{bsw11} -- e.g., revealing too granular information at signing or
% opening time makes privacy go away. However, on this occasion, it is the signer
% (perhaps jointly with the verifier) who decides the signing and
% opening policy. Thus, the user is in full control of how deanonymizable its
% signatures will be, and knows it at signature generation time. This flexibility
% on the opening values also leads to the
% need of using extraction-based techniques for the modelling, since we cannot
% expect that opening outputs the real identity of the signer. This comes as no
% surprise, though, as it was already observed in GS schemes with non-conventional
% opening \cite{dl21}, in DAA schemes \cite{cdl16,cdl16b} or in AC schemes that
% need to ensure correctness of information generated before authentication, like
% delegatable credentials \cite{bcc+09}, that prove validity of the delegation
% chain. In fact, simulation-extractability seems to be most useful here,
% something that was also observed in \cite{cl06} for producing signatures out of
% knowledge of witnesses for concrete NP statements.

% A crucial aspect to cater for all possible variants that we aim to subsume, is
% the incorporation of ``functional placeholders'' enabling customisation. We
% introduce abstract functions whose computation needs to be proven correct by the
% user. Depending on the choice of these functions, one can instantiate a
% different related scheme, which is anyway secure according to our \UAS model.
% We also allow to modulate the information to be revealed after signing time,
% with the information revealed at signing time, allowing policies such as
% ``deanonymize only if 18+ years old''.

\paragraph{Sample real-world use cases.} %
By enabling customization at issuance, signing and opening times, as well as
multiple uncoupled issuers and openers, the possibilities are large and
especially useful in contexts where different parties may have different
privacy-vs-utility tradeoffs. As an example, consider a social network that
requires under-age users to get permission from one parent. This can be defined
via a delegatable credential scheme, checking that the delegation chain of the
user requesting to join is valid via a policy at issuance time. Later, in order
to make some post in the system advertising something to be sold in a local
second-hand page in the social network, the user may need to reveal, at signing
time, being a citizen of the local area (e.g. city). This can be done via an
appropriate policy at signing time. If, after the product is sold, it proves
to be defective, the social network may directly output the email of the user
if s/he is over-age, or the email of the adult in charge if s/he is under-age.
And that could be achieved via a policy for utility at opening time. Finally,
if, in the same social network, some other
issuer-user-verifier situation comes up that requires a different set of utility
choices, just the appropriate functions would need to be changed, without the
need to re-model from scratch (or at all).

% %% Our techniques
% %%% E.g., describe how we compatibilize both with AC sel disc., ring sigs, and MPS
% \paragraph{Our construction approach.} %
% The high-level idea for our generic construction \CUASGen is simple. Credentials
% are signatures over blocks of messages (some signed in committed form), where
% each message can be a different attribute, and a user who wants to
% simultaneously leverage multiple credentials needs to bind all of them to the
% same key. We add three generic functions: \fissue, leveraged by users at
% credential issuance time to prove claims over attributes in other credentials,
% in order to get a new one; \feval, proving correctness of the utility revealed
% at signing time; and \finsp, proving correctness of the utility information
% extractable after signing. The latter is, in addition, fed and modulated with
% the output of \feval. The pattern of the statements proven in each function is
% summarised in \figref{fig:proof-blocks-uas}.

% \begin{figure}[ht!]
%   \centering
%   \scalebox{0.9}{
%     \begin{tikzpicture}

  \pgfdeclarelayer{bg}    % declare background layer
  \pgfsetlayers{bg,main}  % set the order of the layers (main is the standard layer)

  \node (issue) at (0.00,2.50) { \bf Issue };
  \node (sign) at (0.00,1.50) { \bf Sign };  
  \node (open) at (0.00,0.50) { \bf Open };

  \node (issue-uas) at (4.00,2.50)
  { $K(\usk) \land K(\scred) \land \usk \in \scred \land \fissue(\cdot) = 1$ };
  \node[align=center] (sign-uas) at (4.00,1.50)
  { $K(\usk) \land K(\scred) \land \usk \in \scred \land E(\yinsp)~\land$ \\
    $\Yeval = \feval(\cdot) \land \yinsp = \finsp(\Yeval,\cdot)$
  };
  \node (open-uas) at (4.00,0.50) { $D(\yinsp)$ };

  % \draw[draw=black,thick] (5.65,2.29) rectangle ++(1.43,0.44);
  % \draw[draw=black,dashed,thick] (1.80,1.09) rectangle ++(1.75,0.44);
  % \draw[draw=black,dotted,thick] (3.85,1.09) rectangle ++(2.30,0.44);
    
\end{tikzpicture}

%%% Local Variables:
%%% mode: latex
%%% TeX-master: t
%%% End:

%   }
%   \caption{Summary of statements proved per operation in \UAS.
%     $K(x)$ means proving knowledge of $x$; $E(y)$ means proving correct
%     encryption of $y$; $D(z)$ means proving correct decryption of (an encryption
%     of) $z$. $\usk \coloneqq$ user secret key; $\scred \coloneqq$ set of
%     credentials.}
%   \label{fig:proof-blocks-uas}
% \end{figure}

% From \CUASGen, one can already build known (and more functionality-limited)
% schemes from what we call \emph{restrictions} of \CUASGen simply by giving
% concrete definitions of \fissue, \feval, and \finsp. Yet, the model remains the
% same, and we prove that if such restrictions of \CUASGen are secure in the \UAS
% model, they lead to secure constructions for the related primitives (GS
% variants, AC variants, etc). To define our generic construction, we also specify
% concrete NP relations for the underlying NIZK proofs for issuing, signing, and
% opening. These NP relations must not be considered as immutable, though --
% informally, it would seem that they must simply follow the pattern in
% \figref{fig:proof-blocks-uas}. Actually, as mentioned, we show how by making a
% minor variation into the relation used for signing, a \CUASGen restriction can
% securely build ring signatures.


\paragraph{Outline.} %
We give some preliminaries in \secref{sec:preliminaries}. Then proceed with our
\UAS syntax, model, and construction, in \secref{sec:formal-uas}, and show
relationships with other primitives in \secref{sec:relationships}. We conclude
in \secref{sec:conclusion}. Deferred proofs and additional information appear
in the appendices.

%%% Local Variables:
%%% mode: latex
%%% TeX-master: "uas-paper"
%%% End:
