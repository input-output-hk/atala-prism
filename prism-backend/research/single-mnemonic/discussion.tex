\section{Discussion}
\label{sec:discussion}

From a theoretical point of view, it seems very reasonable to derive all
keys, even for different purposes (payments, staking, or identities) from
the same mnemonic. \secref{ssec:unification}, listed a series of topics
that should be taken into account prior to unifying Cardano and Atala
wallets under the same mnemonic, to avoid security pitfalls. In addition to
those,  there are also some aspects that may need careful consideration, as they
may have impact in the overall security and privacy properties of the wallet
(and related systems).

\paragraph{Malleability considerations when combining multiple
  attributes into one address.} %
Are the keys for the different purposes going to be used \emph{only} in a
standalone manner? Or, rather, it can be expected that they are somehow
encoded and distributed jointly? The latter seems to be the case for PoS
wallets, which are expected to be used in the Cardano ecosystem. As pointed
out by \cite{kkl20}, naive encoding of keys with different purposes, into
a single address, may enable certain (malleability) attacks.

\paragraph{Concrete differences in prior security analysis.} %
The security analysis in \cite{def+21} (which is one of the main references
used in this research spike) is on BIP32, not BIP44. While BIP44
is actually a subset of BIP32, it further specifies it in concrete ways
that may alter the result of the analysis. For instance, \cite{def+21}
assumes that non-hardened nodes are leaves of the tree (which, alternatively,
can be seen as roots in new trees). However, in BIP44 the leaves are
non-hardened nodes, and intermediate nodes are mostly hardened. Yet another
possible source of discrepancy is that the analysis assumes that non-hardened
nodes are maintained in a hot/cold wallet setting and, thus, the private keys
cannot be compromised. This assumption may not hold for all use cases, and
that would change completely the security reasoning. If we want to determine
the concrete security level we want/need in some specific use cases that differ
from the analysis of \cite{def+21} as mentioned above, a new model might be
necessary.

\paragraph{Careful consideration of extra security properties.} %
The main related work analysed so far\footnote{BIP32, BIP44, \cite{def+21}
  and, partially, \cite{kkl20}.} study the security of either BIP32
wallets for payments, or PoS wallets. The introduction of identity-related
(frequently associated to actual people) data may require additional properties,
only mentioned in these works (or not considered, as in the BIPs). For instance,
some sort of forward security will be very desirable. While, for the case of
payments or staking, the impact of a compromise (either of a leaf node --
i.e., a key -- or a complete subtree) can be mitigated by prompt detection
and transfer of funds to an uncompromised address, this may be harder for
keys representing in some way real world identities. Most probably, the
latter will frequently be long-lived keys, that also can, in turn, be used
to issue further identities (e.g., as in the case of identity providers).
Thus, the consequences of a compromise may be more cumbersome to address
(re-issuing credentials, distributing revocation lists, etc.) and,
consequently, it seems desirable to look for constructions that give some
sort of guarantee about the security of keys produced at time $t' < t$, if a
compromise happens at time $t$.

Also, it cannot be discarded that further desirable privacy or security
properties arise, as we get more familiar with the targetted use cases.

\paragraph{Are you aware of further challenges that may be worth
  looking into?} Please, reach out to us:

\begin{itemize}
\item Jesus Diaz Vico (Atala Semantics team).
\item Christos KK Loverdos (Atala Technical Director).
\item Ezequiel Postan (Atala Semantics team).
\item Tony Rose (Atala Head of Product).
\end{itemize}

  
\subsection{Further related work}

There seems to be a growing body of related research, which is specially
growing lately (probably, due to the rise of cryptocurrencies). The following
just includes some of the main references for self-bookkeeping (besides
\cite{kkl20} and \cite{def+21}, already mentioned in earlier sections). Note
that they, in turn, include references to further related work.

\begin{itemize}
\item ``Arcula: A Secure Hierarchical Deterministic Wallet for Multi-asset
  Blockchains'', \cite{lfa20}.
\item ``Simple, Efficient and Strongly KI-Secure Hierarchical Key Assignment
  Schemes'', \cite{fpp13}.
\item ``A Formal Treatment of Deterministic Wallets'', \cite{dfl19}.
\end{itemize}

%%% Local Variables:
%%% mode: latex
%%% TeX-master: "single-mnemonic"
%%% End:
