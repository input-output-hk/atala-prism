\section{Discussion}
\label{sec:discussion}

From a theoretical point of view, it seems very reasonable to derive all
keys, even for different purposes (payments, staking, or identities) from
the same mnemonic. However, there are some aspects that may need careful
consideration, as they may have impact in the overall security and privacy
properties of the wallet (and related systems).

\paragraph{Domain separation when different cryptosystems are
  necessary.} %
As mentioned, Cardano uses EdDSA (i.e., it is based on Ed25519 curve),
while Atala uses ECDSA with Secp256k1. It is \emph{not} a good idea%
\footnote{I have not been able to find a concrete paper that describes
  some concrete related attack or gives some impossibility result for
  proving security under this circumstance but, at the least, it seems to
  be folklore knowledge. See Lindell's answer at
  \url{https://crypto.stackexchange.com/a/54666/52362} for instance.
  \cite{dlp12+,thorm21} seem good references to study this topic further, if
  needed.} to use the same key for different cryptosystems, nor for different
curves, even though ``structurally'' it may be possible (e.g., in EdDSA and
ECDSA, private keys are 32-byte random numbers). However, this is easily
avoidable by ensuring domain separation in the key derivation functions that
are applied on the common seed. This is an important concern on its own; yet,
it should be addressed natively through the considerations in the next
paragraph.

\paragraph{Ensure a correct hierarchical key derivation strategy.} %
We need to take into account what specific usage we expect from the wallets
and, from there, define hierarchical derivation rules that ensure a correct
separation. For instance, related to the previous paragraph, we should make
sure that keys that are aimed to be used in different cryptosystems, are
derived in tree branches that ensure domain separation (just using a
different \texttt{purpose} branch, or \texttt{coin type} branch, would probably
be enough). Similarly, one needs to consider if specific subtrees are expected
to be used in an standalone manner, even probably by different (but related)
entities. For instance, different faculties of a university may need to maintain
their own subrees for the sake of efficiency (i.e., each one independently
deriving keys as needed). But then, compromises to one of them should not
affect the other. Again, this can be easily achieved using hardened
derivation in BIP44 slang (e.g., at the \texttt{account} level). From this
point of view, a formalization along the lines of \cite{def+21} would be
desirable, but the one given there is probably too strict (as, for instance,
it assumes that hardened nodes are leaves of the tree, and secret key
compromise of non-hardened nodes, is not possible.)

\paragraph{Malleability considerations when combining multiple
  attributes into one address.} %
Are the keys for the different purposes going to be used \emph{only} in a
standalone manner? Or, rather, it can be expected that they are somehow
encoded and distributed jointly? The latter seems to be the case for PoS
wallets, which are expected to be used in the Cardano ecosystem. As pointed
out by \cite{kkl20}, naive encoding of keys with different purposes, into
a single address, may enable certain (malleability) attacks.

\paragraph{Careful consideration of extra security properties.} %
The main related work analysed so far\footnote{BIP32, BIP44, \cite{def+21}
  and, partially, \cite{kkl20}.} analyses the security of either BIP44
wallets, or PoS wallets. The introduction of identity-related (frequently
associated to actual people) data may require additional properties, only
mentioned in these works (or not considered, as in the BIPs). For instance,
some sort of forward security will be very desirable. While, for the case of
payments or staking, the impact of a compromise (either of a leaf node --
i.e., a key -- or a complete subtree) can be mitigated by prompt detection
and transfer of funds to an uncompromised address, this may be harder for
keys representing in some way real world identities. Most probably, the
latter will frequently be long-lived keys, that also can, in turn, be used
to issue further identities (e.g., as in the case of identity providers).
Thus, the consequences of a compromise may be more cumbersome to address
(re-issuing credentials, distributing revocation lists, etc.) and,
consequently, it seems desirable to look for constructions that give some
sort of guarantee about the security of keys produced at time $t' < t$, if a
compromise happens at time $t$.

Also, it cannot be discarded that further desirable privacy or security
properties arise, as we get more familiar with the targetted use cases.
  
\subsection{Further related work}

There seems to be a large body of related research, which is specially
growing lately (probably, due to the rise of cryptocurrencies). Here,
I just include some of the main references I've come across for
self-bookkeeping (besides \cite{kkl20} and \cite{def+21}, already
mentioned in earlier sections). Note that they, in turn, include references
to further related work.

\begin{itemize}
\item ``Arcula: A Secure Hierarchical Deterministic Wallet for Multi-asset
  Blockchains'', \cite{lfa20}.
\item ``Simple, Efficient and Strongly KI-Secure Hierarchical Key Assignment
  Schemes'', \cite{fpp13}.
\item ``A Formal Treatment of Deterministic Wallets'', \cite{dfl19}.
\end{itemize}

%%% Local Variables:
%%% mode: latex
%%% TeX-master: "single-mnemonic"
%%% End:
