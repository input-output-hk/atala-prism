\section{Introduction}
\label{sec:introduction}

Hierarchical Deterministic Wallets (HD wallets, for short) in the cryptocurrency
domain derive from the BIP32 specification%
\footnote{\url{https://github.com/bitcoin/bips/blob/master/bip-0032.mediawiki}.
  Last access, December 13th, 2021.}
The motivation behind creating this type of wallets is clear if one thinks about
how typical cryptocurrency systems work: each address is associated with a key
pair and, in order to achieve high security and privacy levels, it is best to
limit the reusage of addresses (thus, key pairs). Therefore, a too high number
of needed key pairs is quickly reached. If each key is generated independently
at random, management becomes too complex -- especially, if several devices
are expected to be synchronised. As a solution to this challenge, a hierachical
tree-based data structure was proposed, in which each parent node can be used to
derive multiple child nodes, deterministically. This approach seems also useful
to foster some sort of separation -- i.e., create sub-trees that are
(computationally) independent from one another, in the sense that a compromise
in one does not lead to a compromise in the other. Typically, the root node
of the tree is derived from a seed encoded as a mnemonic. BIP39%
\footnote{\url{https://github.com/bitcoin/bips/blob/master/bip-0039.mediawiki}.
  Last access, December 13th, 2021.} is the main specification for this latter
purpose.

\subsection{Preliminaries}

In the sequel, the processes that compose an (BIP44-compliant) HD wallet are
described, including high-level cryptographic algorithms. For the sake of
readability of non-cryptography savvy audience, we give informal descriptions of
the cryptographic concepts and schemes mentioned next.

\paragraph{Signature schemes.} A signature scheme uses an asymmetric key pair,
composed of a public key and a private key (also frequently referred to as
verification key and signing key, respectively). Intuitively, the owner of the
signing key can produce a digital signature over arbitrary messages, which can
be verified by anyone with the verification key, which is frequently somehow
made publicly accessible. Secure signature schemes ensure that only the owner
of the signing key can produce signatures verifiable with the corresponding
verifying key. In the HD wallets studied next, typical signing algorithms are
ECDSA and EdDSA.

\paragraph{HMACs and HKDFs.} A Hash-based Message Authentication Code scheme is
a concrete approach to build MAC functions, using cryptographic hashes as main
building block. A MAC function is a kind of symmetric equivalent to digital
signature -- i.e., it relies on a shared secret rather than on an asymmetric key
pair. HMACs have been proved to be secure pseudo random number
generators \cite{bck96}, which is a useful building block to build secure HKDFs
\cite{kraw10}, or Hash-based Key Derivation Functions. Specifically, an HKDF, as
defined in \cite{kraw10}, can be applied to produce secure cryptographic keys
from low entropy sources, by applying a two-phase process of randomness
extraction and expansion (the extract-expand approach). In summary, given a
(possibly low-entropy) bitstring, an HKDF produces an arbitrarily long pseudo
random bitstring that is fit for cryptographic purposes.

%%% Local Variables:
%%% mode: latex
%%% TeX-master: "single-mnemonic"
%%% End:
