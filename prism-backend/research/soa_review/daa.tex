\section{Direct Anonymous Attestation}
\label{sec:daa}

% \begin{enumerate}
% \item Main intuition.
% \item Evolution of main security models.
% \item Variations concerning functionality.
% \item Evolution in efficiency.
% \item Main issues.
% \item Approx. number of related works in main crypto+security venues: IACR,
%   Esorics, CCS, AsiaCCS, SP, EuroSP, Usenix Sec and Privacy, PETs...
% \end{enumerate}

\paragraph{Main intuition.}
%
Direct Anonymous Attestation \cite{bcc04} was initially proposed to address a
need in the Trusted Computing Group
\footnote{\url{https://trustedcomputinggroup.org/}. Last access on October 29th,
  2020.}. Take a computing platform equipped with a trusted hardware element
(referred to as TPM, from Trusted Platform Module, in the TCG jargon). A TPM
is trusted to securely store secrets and perform some cryptographic operations,
but is otherwise not able to communicate directly with the outside world. To
do that, it has to leverage other components of the larger computing platform,
which are not fully trusted. Then, the challenge is to come up with a way for
third parties (verifiers) to check that they are indeed communicating with a
computing platform that is equipped with a valid TPM. Moreover, in order to
preserve privacy, a requirement is not to reveal which specific TPM the
verifier is communicating with. Initially, several proposals were made
\needcite\footnote{\cite{bcc04} references works by Boneh et al, and Brickel,
  but no academic reference seems to exist.}, but it was \cite{bcc04} the one
that was finally adopted.

\subsection{Approaches, Models, and Variations}
\label{ssec:daaapproach}

\note{For now, just write down notes on the DAA papers. Then, unify.}
\note{Remember: talk about the properties that security models aim to achieve
  in DAA.}

DAA has many similarities with group signatures, but also important differences.
While in group signatures the user joining the group is an ``indivisible''
entity, in DAA this role is divided among the TPM and the larger platform
containing the TPM (usually referred to as the Host). This undoubtedly
complicates the modelling, as different trust assumptions have to be made for
each of them. Also, security in DAA is modelled in the UC paradigm \todo{why?}
\todo{(not true. Some works use game-based definitions, as \cite{bfg+13} -- see
  also the discussion and references therein)},
whereas group signatures are modelled with game-based definitions. From a
functionality perspective, DAA does not allow opening and, also, uses pseudonyms
that can be leveraged to provide linkability%
\footnote{We note that, recently, group signature schemes have been proposed
  that do not implement opening either \cite{dl21}, and also leverage pseudonyms
  to introduce some kind of linkability \cite{gl19,kss19,dl21} -- and, indeed,
  these schemes draw techniques from DAA.}.

Besides these differences, the typical approach is similar: a new TPM runs an
interactive $\langle Join,Issue \rangle$ protocol with an issuer whereby the
TPM receives a membership credential. Then, to produce attestations, the TPM
creates SPKs of a valid credential, over a message including information about
the status of the Host. However, since there is no opening, DAA employs the
\emph{Sign-Randomize-Prove} strategy from group signatures, which produces more
efficient attestations (in group signatures, at the cost of more inefficient
opening). In addition, the pseudonyms added in DAA are built from \emph{scopes}
(also called \emph{basenames}) and cryptographic hash functions \todo{more
  insight on this}.

In \cite{bl07}, an extension of the DAA approach in \cite{bcc04} is made so
that it is possible to revoke TPMs not only based on leaked private keys (which
is a restricted approach), but also based on signatures and on requests by the
issuer. To do so, the membership proving protocol (roughly, signing) is extended
so that the prover demonstrates that its membership credential is different to
the ones who have been included in revocation lists by a revocation authority.
This, however, incurs in extra costs of the signing and verification protocols,
which grow linearly with the number of revoked users. During signing, \cite{bl07}
also remove the basename concept, which in \cite{bcc04} is used to derive
linkable pseudonyms when needed. Instead, they employ random bases. Finally, in
their model, \cite{bl07} do not differentiate between TPM and host. While the
authors give some insight into how to separate them, extending the formalization
(and proofs) has been shown to be problematic \todo{Bernhard work and later
  criticism?}

\cite{cms08} points out a flaw in the reduction of \cite{bcc04} and fixes it.
Interesting definition of DAA (maybe useful for giving a high-level overview).
It also provides a security proof (simulation/UC-based model) for the
scheme in \cite{cms08b}. In \cite{cms08b}, the authors present a pairing-based
DAA system using asymmetric pairings, achieving improved efficiency.
However, as pointed out in \cite{bfg+13} linking is still not correctly modelled,
being possible to link different authentications from a same TPM \todo{no detail
  is given. Try to see how/why.}
\note{\cite{cms08b} contains some nice (albeit now old) cost comparison.}

\cite{bcl09} proposes an alternative model using games, rather than the
simulation paradigm. They show that security in the UC model by \cite{bcc04}
implies security in their game-based model, but the converse is left as an
open question. However, as pointed out in \cite{bfg+13}, the model in
\cite{bcc04} has a flaw concerning linkability, which is thus inherited by
\cite{bcl09}. \todo{Cross-check this with subsequent Camensich-Lehmann works.}
Construction-wise, they propose an instantiation based on bilinear pairings,
which greatly improves costs.

In \cite{bfg+13}, flaws are pointed out in previous schemes (notably in the
one adopted by the TCG, \cite{bcc04}) that affect unforgeability and linkability
of the produced signatures. They argue that proving security in the UC model is
hard and error-prone in for DAA. Thus, they to game-based approach. In this
context, they propose a black box approach for building DAA systems from
\emph{weakly blind signatures} and \emph{Linkable Indistinguishable Tag} systems.
Moreover, they first capture security for scenarios in which TPM and host are
the same entity; then, they provide an approach to generalise to the case
in which they are different entities (with the different trust assumptions).
However, in their model, they make some odd requirements in the way they
capture identity \todo{delve more into this, and reference the other paper
  that points this out.}

\subsection{Related Primitives}

\begin{itemize}
\item Anonymous Credentials. \cite{bcc04} builds on Camenisch-Lysyanskaya
  signatures.
\item \cite{cms08b} mentions two schemes that are based on two different
  variants of CL signatures (one uses pairings, one does not).
\end{itemize}

\subsection{Applications}
\label{ssec:daaapplication}

\subsubsection{Theory}
\label{sssec:daatheoryapp}

\subsubsection{Real World Deployments}
\label{sssec:daarwdeploy}

\subsection{TODO}

\cite{cdl16,cdl16b,ccd+17,cdl17,cu15}.

Pairing-based DAA: see references in page 2 of \cite{bfg+11}.

%%% Local Variables:
%%% mode: latex
%%% TeX-master: "sok-privsig"
%%% End:
