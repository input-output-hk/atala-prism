\documentclass{llncs}%[10pt]{llncs}
\pagestyle{plain}

\usepackage[n,advantage,operators,sets,adversary,landau,probability,notions,
logic,ff,mm,primitives,events,complexity,asymptotics,keys]{cryptocode}
\usepackage{amssymb}
\usepackage{xspace}
\usepackage[normalem]{ulem}
\usepackage{hyperref}


\title{State of the Art Review on Privacy-Preserving Signatures, Credentials, and Authentication}
\author{}

% Custom shorthands for this paper
\definecolor{darkgreen}{rgb}{0.1,0.7,0.1}
\newcommand{\todo}[1]{{\colorbox{red}{\bf TODO:}\textcolor{red}{#1}}}
\newcommand{\doubt}[1]{{\colorbox{blue}
    {\textcolor{white}{\bf DOUBT:}}\textcolor{blue}{#1}}}
\newcommand{\response}[1]{{\colorbox{orange}{\bf RESPONSE:}
                                             \textcolor{orange}{#1}}}
\newcommand{\modified}[1]{{\textcolor{orange}{#1}}}
\newcommand{\commentwho}[2]{{\colorbox{darkgreen}{{\bf #1:}}
    \textcolor{darkgreen}{#2}}}
\newcommand{\comment}[1]{{\textcolor{orange}{\# #1}}}
%\newcommand{\note}[1]{{\colorbox{gray}{\bf NOTE:}\textcolor{gray} {#1}}}
\newcommand{\needcite}{[\colorbox{red}{?}]}
\newcommand{\figref}[1]{Fig. \ref{#1}}
\newcommand{\tabref}[1]{Table \ref{#1}}
\newcommand{\lstref}[1]{Listing \ref{#1}}
\newcommand{\secref}[1]{Section \ref{#1}}
\newcommand{\appref}[1]{Appendix \ref{#1}}
\newcommand{\defref}[1]{Definition \ref{#1}}

%\renewcommand{\qedsymbol}{$\blacksquare$}
\newcommand{\br}[1]{\ensuremath{\lbrack #1 \rbrack}}
\newcommand{\gen}[1]{\ensuremath{#1}}

\mathchardef\mhyphen="2D

\def\parm{\ensuremath{param}\xspace}
\def\Exp{\ensuremath{\mathsf{Exp}}\xspace}
\def\Adv{\ensuremath{\mathsf{Adv}}\xspace}
\def\status{\ensuremath{\mathsf{st}}\xspace}

% Symbols for GSAC
\def\GSAC{\ensuremath{\mathsf{GSAC}}\xspace}
\def\HUGEN{\ensuremath{\mathsf{HUGEN}}\xspace}
\def\CUGEN{\ensuremath{\mathsf{CUGEN}}\xspace}
\def\OBTAIN{\ensuremath{\mathsf{OBTAIN}}\xspace}
\def\OBTISS{\ensuremath{\mathsf{OBTISS}}\xspace}
\def\ISSUE{\ensuremath{\mathsf{ISSUE}}\xspace}
\def\SIGN{\ensuremath{\mathsf{SIGN}}\xspace}
\def\OPEN{\ensuremath{\mathsf{OPEN}}\xspace}
\def\CHALb{\ensuremath{\mathsf{CHAL}_b}\xspace}
\def\ATTR{\ensuremath{\mathsf{ATTR}}\xspace}
\def\OWNR{\ensuremath{\mathsf{OWNR}}\xspace}
\def\CRED{\ensuremath{\mathsf{CRED}}\xspace}
\def\USK{\ensuremath{\mathsf{\MakeUppercase{\usk}}}\xspace}
\def\PUBUSK{\ensuremath{\mathsf{P}\USK}\xspace}
\def\PRVUSK{\ensuremath{\mathsf{S}\USK}\xspace}
\def\HU{\ensuremath{\mathsf{HU}}\xspace}
\def\CU{\ensuremath{\mathsf{CU}}\xspace}
\def\SIG{\ensuremath{\mathsf{SIG}}\xspace}
\def\CSIG{\ensuremath{\mathsf{CSIG}}\xspace}
\def\CCRED{\ensuremath{\mathsf{CC}}\xspace}
\def\secpar{\ensuremath{\kappa}\xspace}
\def\param{\ensuremath{\param}\xspace}
\def\isk{\ensuremath{isk}\xspace}
\def\ipk{\ensuremath{ipk}\xspace}
\def\osk{\ensuremath{osk}\xspace}
\def\opk{\ensuremath{opk}\xspace}
\def\gpk{\ensuremath{gpk}\xspace}
\def\usk{\ensuremath{usk}\xspace}
\def\upk{\ensuremath{upk}\xspace}
\def\Attrs{\ensuremath{A}\xspace}
\def\DAttrs{\ensuremath{D}\xspace}
\def\cred{\ensuremath{cred}\xspace}
\def\utrans{\ensuremath{reg}\xspace}
\def\trans{\ensuremath{\MakeUppercase{reg}}\xspace}  
\def\msg{\ensuremath{m}\xspace}
\def\sig{\ensuremath{\sigma}\xspace}
\def\oproof{\ensuremath{\pi}\xspace}
\def\Setup{\ensuremath{Setup}\xspace}
\def\IKeyGen{\ensuremath{IKG}\xspace}
\def\OKeyGen{\ensuremath{OKG}\xspace}
\def\UKeyGen{\ensuremath{UKG}\xspace}
\def\Obtain{\ensuremath{Obtain}\xspace}
\def\Issue{\ensuremath{Issue}\xspace}
\def\Sign{\ensuremath{Sign}\xspace}
\def\Verify{\ensuremath{Verify}\xspace}
\def\Open{\ensuremath{Open}\xspace}
\def\Judge{\ensuremath{Judge}\xspace}
\def\ExpCorrect{\ensuremath{\Exp_{\GSAC,\adv}^{corr}}\xspace}
\def\ExpAnonb{\ensuremath{\Exp_{\GSAC,\adv}^{anon-b}}\xspace}
\def\ExpAnonz{\ensuremath{\Exp_{\GSAC,\adv}^{anon-0}}\xspace}
\def\ExpAnono{\ensuremath{\Exp_{\GSAC,\adv}^{anon-1}}\xspace}
\def\AdvAnon{\ensuremath{\Adv_{\GSAC,\adv}^{anon}}\xspace}
\def\ExpTrace{\ensuremath{\Exp_{\GSAC,\adv}^{trace}}\xspace}
\def\AdvTrace{\ensuremath{\Adv_{\GSAC,\adv}^{trace}}\xspace}
\def\ExpNonframe{\ensuremath{\Exp_{\GSAC,\adv}^{frame}}\xspace}
\def\AdvNonframe{\ensuremath{\Adv_{\GSAC,\adv}^{frame}}\xspace}
\def\choose{\ensuremath{\mathsf{choose}}\xspace}
\def\guess{\ensuremath{\mathsf{guess}}\xspace}
\def\uid{\ensuremath{\mathsf{uid}}\xspace}
\def\cuid{\ensuremath{\mathsf{uid}^*}\xspace}
\def\cid{\ensuremath{\mathsf{cid}}\xspace}
\def\ccid{\ensuremath{\mathsf{cid}^*}\xspace}
\def\csig{\ensuremath{\mathsf{\sigma}^*}\xspace}


%%% Local Variables: 
%%% mode: pdflatex
%%% TeX-master: "gsac.tex"
%%% End:


\begin{document}
\maketitle


\begin{abstract}
	\textbf{DISCLAIMER}: Please, keep in mind that, as of now, this document
	basically contains a research log in the shape of personal notes. Do not take 
	anything written here as ground truth.
\end{abstract}

\section{Introduction}
\label{sec:introduction}

Privacy is a human right \needcite. It is also one of the most threatened ones
as a consequence of digitalization and hyperconnectivity \needcite. Maybe one
of the reasons behind this is that, on many occasions, it conflicts with another
human right: security \needcite. The most representative occasion in which this
happens is when we, as individuals%
\footnote{This, of course, also happens when authenticating machines -- as they
  carry to some extent information of some individual(s).}%
, want to gain access to some (possibly
digital) good or service. In that context, one has to prove having the legitimate
rights to access that good or service. However, in the process of doing so, the
entity verifying that we have the rights, typically learns something about us,
like our real name. This has been traditionally fine in physical systems, as the
impact was mostly local. However, with the advent of digitalization and
hyperconnectivity, this usually means that something related to our identity is
being logged in some server, who knows where and with what security controls,
and probably vulnerable to a number of criminal activities \needcite.

Of course, the cryptographic community has not been stranger to the need of
providing satisfactory solutions that enable both privacy and security. We
refer to all works trying to address to this need as \emph{privacy-preserving
  authentication} mechanisms. For decades, a huge amount of research, with many
variations, has been produced in the field. Sadly, still with limited impact in
the real world. Within this context, the goal of this review is manyfold: from a
practical perspective, serve
as a hub to security and privacy professionals that may find themselves in need
to balance privacy and security in some real world application; from a more
academic one, provide a (somehow detailed) overview of the vast amount of
literature in the field, which is already overwhelming enough to make it hard
to find the most appropriate variant of some property we may find ourselves
in need to provide; and finally, attempt to throw some insight into why such a
vast academic work has so far failed to gain more real world adoption.

In order to achieve this, %
we go over the main cryptographic primitives used to build privacy-preserving
authentication mechanisms. For all of them, we first provide a high-level
intuition of the approach to protect users' privacy. Then, we overview the
most prominent security models since their inception, and complement this
with some relevant variations concerning the provided functionality. We also
summarise the main results from an efficiency standpoint, and the main
challenges found so far both in theory and practice. To gain some initial
insight into the academic impact of each primitive, we provide (approximate)
numbers on publications in the main cryptography, security and privacy venues.
Finally, with a more practical mindset, we overview the applications that each
primitive has ``in theory'', and the extent to which they have been deployed
in the real world.

\note{Argue/mention that focus is on primitives that can be used for signing.
  This therefore excludes protocols that can only be used for anonymous
  authentication, but not anonymous singing per se (e.g., PAKE, PASTA, PESTO...
  and probably also Nymble, PEREA, BLAC...)}

\note{Create a graph of relationships between basic primitives and their
  applications. E.g., much of
  the work in SRP group signatures and DAA seems to be derived from randomizable
  signatures (which, afaik, are also a foundation for anonymous credentials).}

%%% Local Variables:
%%% mode: latex
%%% TeX-master: "sok-privsig"
%%% End:

\section{Group Signatures}
\label{sec:gs}

\paragraph{Main intuition.}
%
Group signatures were first proposed by Chaum and van Heyst in 1991 \cite{ch91}
as a privacy-respectful extension to conventional digital signatures.
The intuition is simple: in order to avoid revealing the identity of the signer,
we make her signatures look indistinguishable to the signatures of all other
members of a group. Then, any entity with access to the group public key can
verify that a group signature comes from a member of the group, but cannot
tell which specific one actually created it.
%
Beyond this, group signature schemes have one main feature: a (set of) group
manager(s) control who can become a member of the group. Also, most frequently,
a (set of) group manager(s) decide when to expell a member from the group, or
when to re-identify the creator of a specific signature.
%
The initial idea put forward in \cite{ch91} had, as is to be expected, a number
of limitations. In the following years, the seminal work was extended, both
formally, functionally, and efficiency-wise.

\subsection{Approaches, Models, and Variations}
\label{ssec:gsapproach}

\subsubsection{Main approaches.}
%
There are two main ways that are recurrently used to instantiate group signature
schemes: \emph{Sign-Encrypt-Prove} and \emph{Sign-Randomize-Prove}. In both
variants, when new members of the group join, they receive a credential (i.e.,
a signature by the issuer) derived from a secret they choose. Then, when
creating a group signature, they prove control a valid credential and the
associated secret. The difference lies on how this credential is kept hidden
from the verifier.
%
In the sign-encrypt-prove variant, proposed in \cite{bmw03}, signers
encrypt their credential and then
create a NIZK \needcite proof of knowledge of such a valid credential
over their secret (plus the message they want to sign). This typically results
in schemes with more efficient openings, as the opening authority just has
to decrypt the encrypted credential. However, group signatures require more
space.
%
In the sign-randomize-prove variant, proposed in \needcite \todo{Get Shorty?},
credentials are actually randomizable signatures by the issuer. Signers then
randomize the signature (credential), and create a NIZK proof of knowledge that
this credential is issued for a secret they know (again, plus the message
they want to sign). This usually results in schemes with less efficient openings,
as the opening authority needs to iterate over the registered credentials until
finding the one from which the randomized credential was derived. However, the
produced group signatures are also shorter.

When a new user wants to become a group member, a common habit is also to have
her prove ownership of a conventional digital identity. This identity is then
added to the de-identifying information leveraged by the opening authority, in
order to be able to associate (anonymous) group signatures to a real-world
identity. These conventional digital identities can be any type of PKI-based
(e.g. X.509 or equivalent) certificate. \todo{This is important for
  non-frameability! Check \cite[Section 3.2]{cdl+20} and re-state this.}
\todo{This is indeed necessary for non-repudiation (mentioned in
  \cite[Section 4]{lpy12}).}
\todo{Typically used in schemes with verifiable openings. See
  \cite[Section 1.3.3]{bsi12}.}

\subsubsection{Main security models.}
%
Security for group signatures is given in the form of game-based definitions.
The first formal security model appeared in \cite{bmw03}. It captured security
for groups with a prefixed set of members, added by the group manager upon the
creation of the group. This static requirement was too restrictive to be
realistic, and was quickly extended in \cite{bsz05} and \cite{kty04,ky06}, to
model that new users can join the group after it has been created. The models
in \cite{bsz05,kty04,ky06} are usually referred to as dynamic or partially
dynamic. \cite{bcc+16} generalizes the (partially) dynamic setting into a fully
dynamic model, which considers extra desirable security properties when users
can be revoked (i.e., removed from the group).

Up to some subtleties, the main security properties in most works that appeared
after \cite{bmw03,bsz05,kty04,ky06} are as follows:

\begin{itemize}
\item {\bf Anonymity}. This is the core property concerning privacy. It
  ensures that given a signature produced by one out of two honest (challenge)
  users picked at random from the set of honest group members, it should be
  unfeasible to determine which one actually created it. 
  
\item {\bf Traceability}. This property (covered by the misidentification attack
  in \cite{kty04,ky06}, and spaning also non-frameability in \cite{bmw03})
  captures that it must be possible (by the proper authority) to unambiguously
  recover the identity of the signer of any valid group signature.

\item {\bf Non-frameability}. This property (considered in the framing attacks
  in \cite{kty04,ky06}) requires that no adversary should be able to create
  valid signatures that can be used to incriminate honest users by having them
  open to an honest user who did not create them.
 
\end{itemize}

\cite{bcc+16} also includes a trace soundness property, ensuring that signatures
can be associated (via opening proofs) to only one user \todo{why does this not
  appear in the other models? Maybe because they don't support verifiable
  openings? Check \cite{kss19} and, if so, move this to the variations in
  functionality.}

\subsubsection{Variations in functionality.}
%
Note that, while the security properties just described roughly have the same
purpose across models, there are still differences making some models stronger
or more general than others in some respects. This includes subtleties like who
generates the keys of the authorities (the environment vs the adversary), how
many queries can be made to the challenge oracles and whether or not the
signatures are fixed, etc. However, while important, there are usually generic
techniques that allow to migrate from one variant to another \todo{\needcite,
  ellaborate more, or rewrite}. Now we are interested in revisiting some of
the variations that have impact into the functionality that schemes
following each model can provide, or variations that are specific to group
signatures schemes and do not appear in other cryptosystems.

\paragraph{Separating issuer and opener.}
In some schemes \cite{bmw03,kty04} one single authority concentrates the
power to add new members to the group, and open signatures. This endows such authority with great
power -- and trust -- and is therefore one of the main concerns for privacy
advocates: not only this authority can prevent users from joining (in the case
of dynamic groups), but it can also reidentify group members. In order to reduce
this trust, and since it is a clear distinction between both functionalities
(issuing and opening), many schemes separate both roles into an issuer and an
opener. While both authorities need to be trusted for their respective tasks,
one can now make independent assumptions and even ensure certain properties
if only one of the two authorities is compromised or malicious. For instance,
non-frameability can be achieved even when both authorities are malicious;
traceability can be achieved with malicious openers; and anonymity can be
kept as long as the opener is honest. This is the case in \cite{bsz05} and
\cite{ky06} (which, although starts from a single authority, also considers
and formalises the case of separate authorities).

\paragraph{Protecting against corrupt authorities.}
Even when the issuance and opening functionalities are separated, as mentioned,
each entity can unilaterally break crucial properties in a group signature scheme.
While placing such kind of trust may be reasonable in some use cases, it may not
be so for others. Thus, a natural extension is to reduce this trust as much as
possible. In this direction, there have been works that have applied threshold
and similar techniques \needcite to require that a set of openers (as opposed
to a single opener) need to cooperate in order to open any group signature
\cite{bcly08} \needcite.
Recently, the role of the issuer was also subject to these techniques
\cite{cdl+20}, and it is now known how to produce group signatures such that no
single entity can prevent users from joining nor reidentify them.

\paragraph{Forward security.}
Forward security aims at restricting the impact of security compromises (e.g.,
key leakage) to the future: i.e., if an attacker gains control of a user's
private key, only signatures produced after the compromise should be considered
invalid. In schemes that do not ensure forward secrecy, such a compromise also
renders invalid signatures produced before the compromise. Note that, for
conventional signatures, only the compromised user is (directly) affected by
such event. However,  in group signatures, all group members are directly
affected, as the signer's identity is not revealed by the signature. Thus,
in order to learn whether or not a past signature was issued by a compromised
user, it is necessary to resort to the opener -- therefore, reidentifying
\emph{all} users in the system.
%
Forward security has been introduced in group signatures in works like
\cite{song01,ly12}.
%
Interestingly (but logically), the approach to incorporate this property
holds similarities with the functionality to implement fully dynamic group
signatures: for forward security, users are usually required to apply an
update procedure to their private keys \cite{ly12}; in fully dynamic group
signatures \todo{fully dynamic gs have not been introduced yet!}, it is the
group manager (or the opener) who has to update some
key material \cite{bcc+16}. In both cases, time has to be discretized in epochs,
and conditions are introduced into the security properties that prevent an
adversary being able to deanonymize (produce) a valid signature generated
(using) a key in an epoch in which such key was not supposed to be active
(either because the user updated it -- in a forward secure scheme -- or the
group manager revoked it -- in a fully dynamic scheme).

\paragraph{\todo{Revocation mechanisms}.}
Verifier Local Revocation \cite{bs04}, accumulators approach.

\paragraph{Opening controls.}
Even when separating the roles of the authorities or distributing them, the
fact that some entity (or set of entities) can open group signatures is still
a delicate matter that creates friction. A minimal control is to enable the
opener to prove that the opening process has been performed honestly. This is
precisely what is done by schemes that support \emph{verifiable opening}
\needcite. To further aleviate this issue, some schemes
introduce mechanisms to even prevent opening as long as some condition is
met. For instance, in \cite{seh+12,ehk+19}, the condition is determined by
the message (e.g., only signatures over offending messages can be opened).
To implement this, the authors introduce an extra authority, the
\emph{admitter}, such that the opener needs to get some information from the
admitter in order to be able to open signatures over some specific message.
In \cite{sbm10}, the authors propose \emph{contractual} anonymity: before a user
is able to anonymously access some service, both user and service sign a
contract defining the conditions upon which anonymity can be revoked. This
contract can only be enforced by a trusted entity, the \emph{accountability
  service} who runs a trusted execution environment (such as today's Intel
SGX \needcite) ensuring that the conditions are checked in a trustworthy manner.

\paragraph{Alternatives to opening.}
%
Even with separated authorities, with distributed openers, or with more generic
opening controls such as message-dependent opening, opening may not even be
the type of reduction of privacy that is necessary. In this vein, alternative
schemes have been proposed that do not include any kind of
opening at all.
%
In the schemes proposed in \cite{kty04,cpy06,bcly08}, there is an extra
authority that can produce \emph{tracing} trapdoors on a per-user basis.
Once the tracing trapdoor for a specific user is given to some entity, this
entity can link (or trace) all group signatures produced by that user.
These trapdoors are also an effective mechanism for implementing verifier-local
revocation.
%
In \cite{gl19}, the opener is replaced by a \emph{converter} who, rather than
opening any given group signature, can link sets of signatures. More
specifically, when queried with a set of signatures, the converter transforms
them so that every signature coming from the same user gets assigned a same
randomized pseudonym. Furthermore, to prevent the converter from gradually
relinking all the signatures, these conversions are non-transitive: i.e.,
signatures produced by user $A$ will receive a different randomized pseudonym in
different queries.
%
In \cite{dl21}, there is directly no trusted entity that can open or link group
signatures. Instead, every user can create proofs of several signatures having
been issued by herself. In this case, the authors also propose a scheme whereby
a user can prove sequentiality of a set of signatures: i.e., that all signatures
were produced by herself, and no signature has been omitted or inserted into the
sequence, which is also given in the order in which the original signatures were
produced.

\subsubsection{Efficiency.}

\begin{itemize}
\item Size of (group) keys.
\item Size of signatures.
\item Verification time (batching, including revocation checks...)
\end{itemize}

\subsubsection{Main challenges and interesting lines of work.}

Efficiency is possibly good for many real-world cases. Still: utility
and unavailability of implementations.

\note{Concurrent joins -- avoid extraction \cite{ky05,klap20}.}

\subsection{Related Primitives}

\subsubsection{Ring Signatures}
\label{sssec:rs}

\cite{rst06}

\cite{bcc+15} uses ring signatures to build a fully dynamic group signature
scheme.

\subsection{Applications}
\label{ssec:gsapplication}

See \cite[Section 1.2.3]{bsi12}: it seems mostly theoretical applications.

\subsubsection{Theory}
\label{sssec:gstheoryapp}

\begin{itemize}
\item GS-MDO imply IBE \cite{ehk+19}.
\item GS are implied by trapdoor permutations \cite{bmw03}.
\item GS imply one-way functions \cite{romp90}.
\item GS imply public key encryption \cite{aw04}.
\item Commitments from GS (ref. 25 in \cite{bfg+11}).
\end{itemize}

\subsubsection{Real World Deployments}
\label{sssec:gsrwdeploy}

\cite{ehk+19} mentions anonymous auctions, anonymous bulletin boards and
anomaly detection systems.

Maybe make some comments on how to deploy gs schemes, or what aspects may be
relevant. For instnace, some schemes/models (e.g. \cite{ly12}) assume a
trusted party who generates the keys for the authorities. While this has
a big impact in security/trust model, in the real world it would probably
be circumventable with audited processes for most use cases. Also, the
issue with concurrent issuance and the approach we took for implementing
\cite{gl19} in ICT4Cart, or the limitation to logarithmic numbers of users, etc.

%%% Local Variables:
%%% mode: latex
%%% TeX-master: "sok-privsig"
%%% End:

\section{Attribute-Based Signatures}
\label{sec:abs}

\begin{enumerate}
\item Main intuition.
\item Evolution of main security models.
\item Variations concerning functionality.
\item Evolution in efficiency.
\item Main issues.
\item Approx. number of related works in main crypto+security venues: IACR,
  Esorics, CCS, AsiaCCS, SP, EuroSP, Usenix Sec and Privacy, PETs...
\end{enumerate}

\subsection{Related Primitives}

\subsection{Applications}
\label{ssec:absapplication}

\subsubsection{Theory}
\label{sssec:abstheoryapp}

\subsubsection{Real World Deployments}
\label{sssec:absrwdeploy}

%%% Local Variables:
%%% mode: latex
%%% TeX-master: "sok-privsig"
%%% End:

\section{Anonymous Credentials}
\label{sec:ac}

\paragraph{Main intuition.}
%
Anonymous Credentials (ACs) were first proposed by Chaum \cite{chau85}, and
further studied, formally treated, and optimized, in a series of papers by
Camenisch and Lysyanskaya \cite{cl01,cl02,cl04} and, independently, by Brands
\cite{bran00}\footnote{Camenisch and Lysyanskaya line of work is well known for
  IBM's Idemix system, and Brands's work for Microsoft's U-Prove. More detail
  appears in the sequel.}. In an AC system, roughly speaking, users receive
credentials from an issuing organization, which they can then use to
authenticate against some verifier. The main property is that this
authentication process can be done in a privacy-preserving way. This typically
means that the verifier cannot link different authentication processes by that
same user. Also, usually, credentials are composed by multiple attributes, such
that the verifier only learns the attributes that the user is willing to share
(or a function of them).

\subsection{Approaches, Models, and Variations}
\label{ssec:acapproach}

\subsubsection{Main approaches.}
%
The more fruitful line of work within the AC domain is the one initiated by
Camenisch and Lysyanskaya \cite{cl01,cl02,cl04} (henceforth, CL credentials or,
when referring to the underlying signature schemes defined in \cite{cl02,cl04},
CL signatures). The core idea is simple: in order to build a (basic) AC
system, all that is needed are a commitment scheme, and a signature scheme
with efficient protocols for signing committed values, and proving knowledge
of a signature. From this, a basic AC system can be derived essentially as
follows:

\begin{enumerate}
\item User $A$, with secret value $s_A$, commits to it into $c_A$.
\item User $A$ interacts with issuer $I$ to get a signature on $s_A$. To
  avoid leaking the secret to $I$, $A$ uses $c_A$, proving knowledge of the
  associated secret. The produced signature $\sigma_A$, is $A$'s anonymous
  credential.
\item When $A$ needs to convince a verifier $B$ that she owns a valid credential
  issued by $I$, she just has to create a zero-knowledge proof of knowledge
  of $\sigma_A$.
\end{enumerate}

Usually, the commitment to the user's secret key is known as a pseudonym, or
in short form, \emph{nym}. These nyms have proven to be a useful component to
ensure security and privacy properties -- and are also leveraged in many group
signature and DAA works. Not all anonymous credentials systems use pseudonyms,
though.

A large part of the academic works in the AC domain follow this approach, which
is then extended to support enhanced functionality. For instance, in order to
let a user with two different credentials prove being the owner of both of
them, \cite{cl01,cl02,cl04} require an extra protocol for proving, in
zero-knowledge, equality of two committed and signed values.
This allows user $A$, with credential $\sigma_A^I$ obtained from issuer $I$ and
credential $\sigma_A^J$ obtained from issuer $J$, to prove $J$ that she has a
valid credential from $I$, by issuing a zero-knowledge proof of equality for
both credentials. Furthermore, if each credential is associated with a different
(set of) attribute(s), this also leads to ``naive'' constructions of
multi-attribute AC systems.

For multi-attribute ACs, though, much more efficient variants are possible. To
begin with, CL signatures allow signing blocks of messages (and commitments),
which can be seen as attributes signed by the issuer. BBS+ signatures
\cite{asm06,cdl16b}, and PS16 signatures \cite{ps16} also support similar
functionality, and have been leveraged to build multi-attribute AC systems
too (e.g., \cite{cks10} gives a construction with BBS+ credentials, and
\cite{sms+19,halp20} are based on PS16). These family of signatures on blocks
of messages have yet another very useful property: signatures are easily
randomizable, which can be used (along with the previously mentioned
zero-knowledge proofs) to make unlinkable showings of the same credential.
AC systems achieving the latter property of unlinkable showings are said
to offer \emph{multi-show} credentials.

Most of these systems also have in common that, being based on building blocks
leveraging discrete logarithm structures, they are easily extended to prove
arbitrary statements on the credentials attributes -- such as the usual ``Over
18'' claim. However, while it is certainly doable, the approach does not scale
for complex statements involving multiple attributes \todo{some concrete
  figures?}

Restricting the expressiveness, and focusing on letting the credential holder
reveal only a subset of the attributes in a credential, more efficient schemes
have been proposed, leveraging algebraic tricks such as structure-preserving
signatures on equivalence classes (SPS-EQ), along with vector commitment schemes
\cite{cdhk15}, or set commitment schemes \cite{fhs19}. The former uses Lagrange
interpolation to allow revealing only a subset of the committed messages;
while \cite{fhs19} employs set commitment schemes, mapping the attributes to
be committed to roots of a polynomial, so that opening a specific subset of the
commmitted elements just requires revealing a subset of the roots of the
polynomial.
%
Also improving efficiency, but restricting to the scenario in which the issuer
is the same entity as the verifier (as in ticketing systems for public transport),
\cite{cmz14} replaces signatures with algebraic MACs.

Within this line of credentials from randomizable signatures, we can also find
systems that can achieve high expressiveness, like the (delegatable) credentials
building on signatures with controlled malleability \cite{bcc+09,cklm14},
where \cite{cklm14} is a generalization of \cite{bcc+09}. Here, at a high level,
credentials are signatures, built from non-interactive zero-knowledge proofs of
knowledge, on commitments to the user key, that support \emph{controlled
  malleability}. This malleability means
that both the proof and the statements can be randomized and transformed, but
only for allowed transformatoins. The specific transformation considered for
delegation requires that a credential with depth $L$, can only produce a credential
with level $L+1$. Credential showings are also represented through transformations,
through which the holder proves knowledge, and equality, of the secret in the
``source'' and ``target'' nyms. Although the credentials in these works do not
include multiple attributes (some hint on how to support public attributes is
given in \cite{bcc+09}), it is certainly possible to do so through a proper
transformation -- although, probably, the result won't be very efficient if
done naively \todo{specific figures?} The notion of transformations is interesting,
though, as the authors prove that it is similar (or even more generic) than those
used to achieve expressiveness in related digital singature schemes (such as
policy-based signatures \cite{bf14} and delegatable functional signatures \cite{bms16}).
Thus, it is reasonable to expect that very expressive AC systems could be built
with this approach.

All AC systems mentioned up to this point follow, up to ``small'' variations,
the same approach: randomizable signatures on sets of (committed) messages, with
(somehow) efficient proofs. A completely different approach is described in
\cite{dmm+18}, based on predicate encryption. In predicate encryption, given
a master key pair, decryption keys associated with specific predicates can be
derived. Messages, and attribute sets, are encrypted with the master public key,
such that the resulting ciphertext can only be decrypted with a decryption key
associated with a predicate that is satisfied by the attributes in the
ciphertext. Assume an attribute space that can be represented by binary strings
in a set $\Sigma$. On one hand, policies defined by verifiers in \cite{dmm+18}
are encoded
as binary circuits through strings in $\Sigma$ too\footnote{\todo{This seems
    like an abuse of predicate encryption, as encryption receives a predicate,
    rather than an attribute set.}}, which can thus be associated to
ciphertexts through the predicate encryption process. On the other hand,
credentials received by users are decryption keys on a generic policy $f$, with
a hardcoded set of attributes (the attributes that the user is granted), that
can evaluate circuits. Then, intuitively, in order for a verifier to check
whether a user owns an attribute-based credential meeting certain policy $g$, it
creates a ciphertext, associating as attribute an encoding a circuit for $g$
(that belongs to $\Sigma$). If the user can decrypt this ciphertext, the
verifier accepts, as this means that the decryption key in the user's credential
has a hardcoded set of attributes that makes $g$ evaluate to true. This only
supports public attributes
(i.e., known to the issuer), though. A negative part of this approach is that
credential showings can only be interactive, whereas in the randomizable proofs
approach, they can be made non-interactive too.

\subsubsection{Main Security Models.}
%
All AC systems include \emph{anonymity} and \emph{unforgeability} as main
security properties. Up to minor variations depending on extra functionality,
anonymity typically captures that, concerning a user's credential, nothing
should be learnt by third parties that the user does not explicitly reveal.
E.g., for ACs that support multiple attributes and selectively disclosing
subsets of them, anonymity captures that the information leaked during a showing
of such credential is limited to the subset of attributes voluntarily revealed
by the user. In multi-show credentials, anonymity also captures that two
different showings must be unlinkable.
As mentioned, further variations in the functionality affect anonymity --
e.g., in systems supporting delegation, the delegation chain may be required
to be kept private.
% 
Concerning unforgeability, it basically captures that no adversary can make
a verifier accept a credential that has not been honestly produced by an
issuer of the system. Again, there are variations to this. For instance, when
credentials are composed of attributes that can further be selectively
disclosed. In this case, the adversary should not be able to prove ownership
of a set of attributes not contained within the set of attributes that have
been obtained by the adversary or leaked to it.

Special mention is worth to the model variation in \cite{hs21}, which divides
the user into two entities: core and helper, thus taking anonymous credentials
closer to DAA. This is reflected into the model by adding a property named
\emph{dependability}, which requires that no adversary should be capable to
show an honestly generated credential withouth compromising the core entity.

On the approach to formalize AC security notions, game-based definitions tend to
be more frequent \cite{bcc+09,cks10,cl11,cklm14,cmz14,ckl+15,dmm+18,fhs19,hs21},
also including definitions in the simulation paradigm. The UC framework
has also been used \cite{cdhk15}.

\subsubsection{Variations in functionality.}
%
\paragraph{Attributes, selective disclosure, and expressiveness.} %
Multi-attribute and multi-show credentials are the most frequent
\cite{cks10,cl11,cdhk15,ckl+15,fhs19,sms+19,halp20,hs21}, although one-show
credentials have also been studied \cite{bran00}.
Similarly, selective disclosure has also been more studied
\cite{cks10,cl11,cdhk15,sms+19,halp20,hs21}, when compared to 
more expressive AC systems \cite{bcc+09,cklm14,dmm+18}, while \cite{fhs19}
supports selective disclosure and restricted policies (specifically,
conjunctions), for the sake of efficiency. Probably, the higher attention to
selective disclosure so far is due to the fact that the building blocks enabling
efficient constructions with high expressiveness (such as SNARKs and functional
encryption) are recent in the cryptographic literature.

\paragraph{Security and privacy against issuers.} %
Blind issuance, i.e., letting users include in their credentials attributes that
are kept hidden from issuers, is also frequent (and certainly useful to, e.g.,
include secret keys; or support private attributes in general)
\cite{cklm14,ckl+15}, although it is also frequent to assume (even implicitly)
that the issuer \emph{needs} to learn the attributes it signs, and thus private
attributes are not considered, e.g. as in \cite{dmm+18,fhs19,sms+19}.
%
Threshold issuance is also interesting for decentralized scenarios, as it
protects against misbehaving issuers, but it was only considered recently in
\cite{sms+19} and, consequently, in \cite{halp20} (which actually builds on
\cite{sms+19}).

\paragraph{Delegation.} %
Delegation has gained attention in the last decade \cite{bcc+09,cklm14},
including delegated verification \cite{dmm+18}, as well as supporting more
expressive proofs \cite{cklm14,dmm+18}. While the latter is indeed possible in
the early works (one can use general zero-knowledge proofs on discrete
logarithms given a CL-, BBS- or PS16- based credential) it is worth noting that
the costs are probably too high. An interesting learning, that can be extracted
from \cite{cklm14}, is that we can see delegation as a sort of policy that can
be checked with expressive policies.

\paragraph{Revocation.} %
Perhaps surprisingly, revocation is not a common functionality, although it is
certainly considered by several works, e.g. \cite{cl01,cks10,ckl+15}, and for
some it is just sketched \cite{hs21}.

\paragraph{Credential aggregation.} %
And, finally, an interesting variation that only one work seems to target,
is that of efficient aggregation of credentials created by different issuers
\cite{cl11}. As in the topic of expressiveness, the conventional approach
can easily be extended to support this: e.g., if the owner of credential A and
credential B wants to use both credentials in the same showing, she just has to
prove that they share the same secret key (although this is only possible in
schemes that embed a user secret within the credential). Still, even if
possible, this process is inefficient. \cite{cl11} leverages aggregatable
signatures, which target precisely that setting.

\subsubsection{Efficiency.}
%
\comment{This section is just a placeholder for now...}

\subsubsection{Main challenges and interesting lines of work.}
%
\comment{This section is just a placeholder for now...}

I find interesting that, given that ACs and group signatures are
essentially the same thing, there is no work (or I have not come across it)
trying to connect both. Specifically, the following questions:

\begin{itemize}
\item Anonymity properties in ACs and GSs are roughly similar. However, concerning
  unforgeability-related properties:
  \begin{itemize}
  \item The notion of traceability in GSs seems incompatible in AC systems that
    do not include a user secret key in the credentials. Even for AC systems that
    do include it, unforgeability in AC is different than traceability, as there
    is no guarantee that a credential can be traced to a ``join'' process. Or
    there is?
  \item The notion of non-frameability of GSs does not seem to have an equivalent
    in ACs. I find this kind of surprising from a practical point of view, as an
    adversary in AC systems can very well aim at impersonating an honest user,
    which may then carry consequences (e.g., legal actions) for the honest user.
    Probably, this is related to ACs not having opening, although it has been
    shown in the GS literature that schemes without opening, but with linking
    can still have a notion of non-frameability \cite{dl21} (and, certainly,
    blacklisting/revocation allows linking, and this has already been done in
    the AC domain.)
  \item Maybe I'm wrong in the two previous points, and we can draw a connection
    between (some basic notion of) unforgeability in ACs, and the usual notions
    of traceability and non-frameability in GSs. If I recall correctly, the
    \cite{bmw03} paper merged non-frameability and traceability into
    unforgeability, and this was later divided again in \cite{bsz05}. Check the
    reasoning again...
  \end{itemize}  
\item I find very surprising that essentially all the body of work in ACs is
  based on what seems to be the sign-randomize-proof approach of GSs, yet there
  is nothing (or I have not seen it) similar to the encrypt-then-prove approach.
  Can we do it? This may be due to the fact that revocation has not been
  considered as a main requirement in ACs (which, IMHO, is a drawback if we want
  systems applicable in the real world).
  \begin{itemize}
  \item Can we come up with a black-box construction of ACs from GSs, or vice
    versa? It should not be hard, and this may open a completely new breed of
    ACs (i.e., those based on the encrypt-then-prove approach).
  \end{itemize}
\item Also, forward security does not seem to have been explored in the AC
  domain (again, I may just overlooked works on it). But it has been studied in
  GSs. Can we leverage that?
\end{itemize}

\subsection{Related Primitives and Systems}
\label{ssec:acrelated}

\comment{This section is just a placeholder for now...}

\begin{itemize}
\item Anonymous Authentication systems in the line of Nymble, BLACR, PEREA...
\end{itemize}

\subsection{Real World Deployments}
\label{ssec:acrealworld}

\comment{This section is just a placeholder for now...}

\begin{itemize}
\item Idemix.
\item Hyperledger Indy.
\item U-Prove.
\end{itemize}

%%% Local Variables:
%%% mode: latex
%%% TeX-master: "sok-privsig"
%%% End:

\section{Direct Anonymous Attestation}
\label{sec:daa}

{\bf \em DISCLAIMER: This section is in a very immature state, read at your
  own risk!}

% \begin{enumerate}
% \item Main intuition.
% \item Evolution of main security models.
% \item Variations concerning functionality.
% \item Evolution in efficiency.
% \item Main issues.
% \item Approx. number of related works in main crypto+security venues: IACR,
%   Esorics, CCS, AsiaCCS, SP, EuroSP, Usenix Sec and Privacy, PETs...
% \end{enumerate}

\paragraph{Main intuition.}
%
Direct Anonymous Attestation \cite{bcc04} was initially proposed to address a
need in the Trusted Computing Group
\footnote{\url{https://trustedcomputinggroup.org/}. Last access on October 29th,
  2020.}. Take a computing platform equipped with a trusted hardware element
(referred to as TPM, from Trusted Platform Module, in the TCG jargon). A TPM
is trusted to securely store secrets and perform some cryptographic operations,
but is otherwise not able to communicate directly with the outside world. To
do that, it has to leverage other components of the larger computing platform,
which are not fully trusted. Then, the challenge is to come up with a way for
third parties (verifiers) to check that they are indeed communicating with a
computing platform that is equipped with a valid TPM. Moreover, in order to
preserve privacy, a requirement is not to reveal which specific TPM the
verifier is communicating with. Initially, several proposals were made
\needcite\footnote{\cite{bcc04} references works by Boneh et al, and Brickel,
  but no academic reference seems to exist.}, but it was \cite{bcc04} the one
that was finally adopted.

\subsection{Approaches, Models, and Variations}
\label{ssec:daaapproach}

\note{For now, just notes on the DAA papers. Then, unify.}

DAA has many similarities with group signatures, but also important differences.
While in group signatures the user joining the group is an ``indivisible''
entity, in DAA this role is divided among the TPM and the larger platform
containing the TPM (usually referred to as the Host). This undoubtedly
complicates the modelling, as different trust assumptions have to be made for
each of them. Also, security in DAA is modelled in the UC paradigm \todo{why?}
\todo{(not true. Some works use game-based definitions, as \cite{bfg+13} -- see
  also the discussion and references therein)},
whereas group signatures are modelled with game-based definitions. From a
functionality perspective, DAA does not allow opening and, also, uses pseudonyms
that can be leveraged to provide linkability%
\footnote{We note that, recently, group signature schemes have been proposed
  that do not implement opening either \cite{dl21}, and also leverage pseudonyms
  to introduce some kind of linkability \cite{gl19,kss19,dl21} -- and, indeed,
  these schemes draw techniques from DAA.}.

Besides these differences, the typical approach is similar: a new TPM runs an
interactive $\langle Join,Issue \rangle$ protocol with an issuer whereby the
TPM receives a membership credential. Then, to produce attestations, the TPM
creates SPKs of a valid credential, over a message including information about
the status of the Host. However, since there is no opening, DAA employs the
\emph{Sign-Randomize-Prove} strategy from group signatures, which produces more
efficient attestations (in group signatures, at the cost of more inefficient
opening). In addition, the pseudonyms added in DAA are built from \emph{scopes}
(also called \emph{basenames}) and cryptographic hash functions \todo{more
  insight on this}.

In \cite{bl07}, an extension of the DAA approach in \cite{bcc04} is made so
that it is possible to revoke TPMs not only based on leaked private keys (which
is a restricted approach), but also based on signatures and on requests by the
issuer. To do so, the membership proving protocol (roughly, signing) is extended
so that the prover demonstrates that its membership credential is different to
the ones who have been included in revocation lists by a revocation authority.
This, however, incurs in extra costs of the signing and verification protocols,
which grow linearly with the number of revoked users. During signing, \cite{bl07}
also remove the basename concept, which in \cite{bcc04} is used to derive
linkable pseudonyms when needed. Instead, they employ random bases. Finally, in
their model, \cite{bl07} do not differentiate between TPM and host. While the
authors give some insight into how to separate them, extending the formalization
(and proofs) has been shown to be problematic \todo{Bernhard work and later
  criticism?}

\cite{cms08} points out a flaw in the reduction of \cite{bcc04} and fixes it.
Interesting definition of DAA (maybe useful for giving a high-level overview).
It also provides a security proof (simulation/UC-based model) for the
scheme in \cite{cms08b}. In \cite{cms08b}, the authors present a pairing-based
DAA system using asymmetric pairings, achieving improved efficiency.
However, as pointed out in \cite{bfg+13} linking is still not correctly modelled,
being possible to link different authentications from a same TPM \todo{no detail
  is given. Try to see how/why.}
\note{\cite{cms08b} contains some nice (albeit now old) cost comparison.}

\cite{bcl09} proposes an alternative model using games, rather than the
simulation paradigm. They show that security in the UC model by \cite{bcc04}
implies security in their game-based model, but the converse is left as an
open question. However, as pointed out in \cite{bfg+13}, the model in
\cite{bcc04} has a flaw concerning linkability, which is thus inherited by
\cite{bcl09}. \todo{Cross-check this with subsequent Camensich-Lehmann works.}
Construction-wise, they propose an instantiation based on bilinear pairings,
which greatly improves costs.

In \cite{bfg+13}, flaws are pointed out in previous schemes (notably in the
one adopted by the TCG, \cite{bcc04}) that affect unforgeability and linkability
of the produced signatures. They argue that proving security in the UC model is
hard and error-prone in for DAA. Thus, they to game-based approach. In this
context, they propose a black box approach for building DAA systems from
\emph{weakly blind signatures} and \emph{Linkable Indistinguishable Tag} systems.
Moreover, they first capture security for scenarios in which TPM and host are
the same entity; then, they provide an approach to generalise to the case
in which they are different entities (with the different trust assumptions).
However, in their model, they make some odd requirements in the way they
capture identity \todo{delve more into this, and reference the other paper
  that points this out.}

\subsection{Related Primitives}

\begin{itemize}
\item Anonymous Credentials. \cite{bcc04} builds on Camenisch-Lysyanskaya
  signatures.
\item \cite{cms08b} mentions two schemes that are based on two different
  variants of CL signatures (one uses pairings, one does not).
\end{itemize}

\subsection{Applications}
\label{ssec:daaapplication}

\subsubsection{Theory}
\label{sssec:daatheoryapp}

\subsubsection{Real World Deployments}
\label{sssec:daarwdeploy}

\subsection{TODO}

\cite{cdl16,cdl16b,ccd+17,cdl17,cu15}.

Pairing-based DAA: see references in page 2 of \cite{bfg+11}.

%%% Local Variables:
%%% mode: latex
%%% TeX-master: "sok-privsig"
%%% End:

\section{Other Related Works}
\label{sec:other}

\comment{This section is just a placeholder for now...}

%\doubt{Functional Signatures? \cite{bgi14}}
\doubt{PAKE? Is privacy an inherent concern in PAKE?}
%\doubt{W3C DIDs and VCs?}
%$\doubt{Blind Signatures?}
%\doubt{k-times anonymity?}
\doubt{PEREA, BLAC, Nymble, etc.?}

%%% Local Variables:
%%% mode: latex
%%% TeX-master: "sok-privsig"
%%% End:

\section{Discussion}
\label{sec:discussion}

From a theoretical point of view, it seems very reasonable to derive all
keys, even for different purposes (payments, staking, or identities) from
the same mnemonic. However, there are some aspects that may need careful
consideration, as they may have impact in the overall security and privacy
properties of the wallet (and related systems).

\paragraph{Domain separation when different cryptosystems are
  necessary.} %
As mentioned, Cardano uses EdDSA (i.e., it is based on Ed25519 curve),
while Atala uses ECDSA with Secp256k1. It is \emph{not} a good idea%
\footnote{I have not been able to find a concrete paper that describes
  some concrete related attack or gives some impossibility result for
  proving security under this circumstance but, at the least, it seems to
  be folklore knowledge. See Lindell's answer at
  \url{https://crypto.stackexchange.com/a/54666/52362} for instance.
  \cite{dlp12+,thorm21} seem good references to study this topic further, if
  needed.} to use the same key for different cryptosystems, nor for different
curves, even though ``structurally'' it may be possible (e.g., in EdDSA and
ECDSA, private keys are 32-byte random numbers). However, this is easily
avoidable by ensuring domain separation in the key derivation functions that
are applied on the common seed. This is an important concern on its own; yet,
it should be addressed natively through the considerations in the next
paragraph.

\paragraph{Ensure a correct hierarchical key derivation strategy.} %
We need to take into account what specific usage we expect from the wallets
and, from there, define hierarchical derivation rules that ensure a correct
separation. For instance, related to the previous paragraph, we should make
sure that keys that are aimed to be used in different cryptosystems, are
derived in tree branches that ensure domain separation (just using a
different \texttt{purpose} branch, or \texttt{coin type} branch, would probably
be enough). Similarly, one needs to consider if specific subtrees are expected
to be used in an standalone manner, even probably by different (but related)
entities. For instance, different faculties of a university may need to maintain
their own subrees for the sake of efficiency (i.e., each one independently
deriving keys as needed). But then, compromises to one of them should not
affect the other. Again, this can be easily achieved using hardened
derivation in BIP44 slang (e.g., at the \texttt{account} level). From this
point of view, a formalization along the lines of \cite{def+21} would be
desirable, but the one given there is probably too strict (as, for instance,
it assumes that hardened nodes are leaves of the tree, and secret key
compromise of non-hardened nodes, is not possible.)

\paragraph{Malleability considerations when combining multiple
  attributes into one address.} %
Are the keys for the different purposes going to be used \emph{only} in a
standalone manner? Or, rather, it can be expected that they are somehow
encoded and distributed jointly? The latter seems to be the case for PoS
wallets, which are expected to be used in the Cardano ecosystem. As pointed
out by \cite{kkl20}, naive encoding of keys with different purposes, into
a single address, may enable certain (malleability) attacks.

\paragraph{Concrete differences in prior security analysis} %
The security analysis in \cite{def+21} is on BIP32, not BIP44. While BIP44
is actually a subset of BIP32, it further specifies it in concrete ways
that may alter the result of the analysis. For instance, \cite{def+21}
assumes that non-hardened nodes are leaves of the tree (which, alternatively,
can be seen as roots in new trees). However, in BIP44 the leaves are
non-hardened nodes, and intermediate nodes are mostly hardened. Yet another
possible source of discrepancy is that the analysis assumes that non-hardened
nodes are maintained in a hot/cold wallet setting and, thus, the private keys
cannot be compromised. This assumption may not hold for all use cases, and
that would change completely the security reasoning. If we want to determine
the concrete security level we want/need in some specific use cases that differ
from the analysis of \cite{def+21} as mentioned above, a new model might be
necessary.

\paragraph{Careful consideration of extra security properties.} %
The main related work analysed so far\footnote{BIP32, BIP44, \cite{def+21}
  and, partially, \cite{kkl20}.} study the security of either BIP32
wallets for payments, or PoS wallets. The introduction of identity-related
(frequently associated to actual people) data may require additional properties,
only mentioned in these works (or not considered, as in the BIPs). For instance,
some sort of forward security will be very desirable. While, for the case of
payments or staking, the impact of a compromise (either of a leaf node --
i.e., a key -- or a complete subtree) can be mitigated by prompt detection
and transfer of funds to an uncompromised address, this may be harder for
keys representing in some way real world identities. Most probably, the
latter will frequently be long-lived keys, that also can, in turn, be used
to issue further identities (e.g., as in the case of identity providers).
Thus, the consequences of a compromise may be more cumbersome to address
(re-issuing credentials, distributing revocation lists, etc.) and,
consequently, it seems desirable to look for constructions that give some
sort of guarantee about the security of keys produced at time $t' < t$, if a
compromise happens at time $t$.

Also, it cannot be discarded that further desirable privacy or security
properties arise, as we get more familiar with the targetted use cases.

\paragraph{On ``generic'' interoperability issues.} %
Interoperability can mean many things. Hence, in this discussion, we restrict
to the structure of the (hierarchically and deterministically) produced tree,
and on the mnemonic length, which are two differing aspects in Atala and
Cardano (lightwallet). On the length of the mnemonic, Atala uses 12-word
mnemonics, while Cardano uses 24 words. According to BIP39, 12 words encode
128 bits of entropy, and 24 words encode 256. Taking into account that ECDSA
(in the case of Atala) provides a security level of roughly half of the size
of the key, and that the deterministic approach to derive keys from the seed
(HMAC-SHA512) does not amplify entropy \cite{rfc5869}, it is advisable to move
to 24-word mnemonics. Even more, considering the analysis of \cite{def+21},
that concludes that the construction of HD wallets based on additively
re-randomized ECDSA results in a security loss of 31 bits.

Related to the structure of the tree, it is worth mentioning that the Cardano
light wallet uses the conventional BIP44 hierarchy, composed of a 6-levels tree
(including the master seed), where the first four levels are derived in
hardened mode, and the last one in non-hardened mode; on the other hand, Atala
uses a 4-levels tree (again, including the master seed), all of them derived
in hardened mode. Essentially, the Atala wallet ``prunes'' the last two levels.
While the format of the produced keys is of course the same (they are just
random bitstrings), this discrepancy may lead to issues implementation-wise,
that should be further evaluated.
  
\subsection{Further related work}

There seems to be a large body of related research, which is specially
growing lately (probably, due to the rise of cryptocurrencies). Here,
I just include some of the main references I've come across for
self-bookkeeping (besides \cite{kkl20} and \cite{def+21}, already
mentioned in earlier sections). Note that they, in turn, include references
to further related work.

\begin{itemize}
\item ``Arcula: A Secure Hierarchical Deterministic Wallet for Multi-asset
  Blockchains'', \cite{lfa20}.
\item ``Simple, Efficient and Strongly KI-Secure Hierarchical Key Assignment
  Schemes'', \cite{fpp13}.
\item ``A Formal Treatment of Deterministic Wallets'', \cite{dfl19}.
\end{itemize}

%%% Local Variables:
%%% mode: latex
%%% TeX-master: "single-mnemonic"
%%% End:

\section{Conclusion}
\label{sec:conclusion}

In this work, we have described how a direct combination of the group signatures
(GS) and anonymous credentials (AC) research lines results in a scheme that can
bring benefit to real world use cases. Going further, we generalized such simple
combination into Universal Anonymous Signatures, or \UAS. Our model for \UAS
covers straight away a large subset of GS and AC variants, while still giving
system designers the flexibility to decide on the specific tradeoff between
privacy, utility, accountability and computational overhead that fits their
setting best. More concretely, our model for \UAS supports arbitrary issuance
policies, signature evaluation functions, and opening functions, without losing
definitional meaning. We also gave a generic construction from well known
building blocks, and prove that it meets our security model.

Even though the generalization effort we do is significant, there are still
ways to go further.
%
For instance, we do not hide the issuer(s) of the
credential(s) employed to produce a signature (resp., request a new credential).
Thus, we can only ensure privacy among signatures (resp., credential requests)
that involve credentials issued by the same issuer set. While hiding the issuers
of the involved credentials is certainly possible (we show how to do it at
signing time, in order to build ring signatures), we argue that it may be
desirable to leave it as is, at least, initially. The reason being that issuers
are still what conveys trust to a system where credentials contain attributes
attested by these issuers. Indeed, if it is already hard for verifiers to decide
whether or not to trust a non-anonymous issuer, one can imagine that it would be
even harder to decide whether or not to trust an anonymous one. This question
is core to the domain of trust registries, for instance.
%
A second desirable improvement, perhaps clearer from a technical point of view,
is to extend our model to support non-trivial utility and accountability across
multiple signatures from the same signer. More concretely, let our \feval and
\finsp functions (and perhaps also \fissue?) operate on multiple signatures by
the same \usk. That can lead to an extended \UAS scheme which would support
advanced use cases such as rate limiting (or $k$-TAA \cite{asm06}); or might
be even privately training an AI model based on one's authenticated
past activity, and prove that the model has been trained honestly.

%%% Local Variables:
%%% mode: latex
%%% TeX-master: "uas"
%%% End:


\section{TODOs}

General/pending doubts and tasks:

\begin{itemize}
\item Many constructions of ACs and GSs seem to be built using the same
  approach: an issuer and a user interact so that the latter receives a signature
  from the former; then the user employs that signature to do something else (in
  ACs, prove attributes; in GSs, produce further signatures). Still, the typical
  security models are different. Concretely, GSs have a very well established
  model, but ACs don't. It also ``bothers'' me that none of the AC models I've
  seen cover something similar to non-frameability, nor traceability, but just
  include conventional unforgeability properties. Why is this?
\end{itemize}

\bibliographystyle{splncs04}
\bibliography{soa-review}

\end{document}

%%% Local Variables:
%%% mode: latex
%%% TeX-master: ucl.tex
%%% End:
