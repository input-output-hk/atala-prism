\documentclass{llncs}%[10pt]{llncs}
\pagestyle{plain}

\usepackage[n,advantage,operators,sets,adversary,landau,probability,notions,
logic,ff,mm,primitives,events,complexity,asymptotics,keys]{cryptocode}
\usepackage{amssymb}
\usepackage{xspace}
\usepackage[normalem]{ulem}
\usepackage{hyperref}


\title{State of the Art Review on Privacy-Preserving Signatures, Credentials, and Authentication}
\author{}

% Custom shorthands for this paper
\definecolor{darkgreen}{rgb}{0.1,0.7,0.1}
\newcommand{\todo}[1]{{\colorbox{red}{\bf TODO:}\textcolor{red}{#1}}}
\newcommand{\doubt}[1]{{\colorbox{blue}
    {\textcolor{white}{\bf DOUBT:}}\textcolor{blue}{#1}}}
\newcommand{\response}[1]{{\colorbox{orange}{\bf RESPONSE:}
                                             \textcolor{orange}{#1}}}
\newcommand{\modified}[1]{{\textcolor{orange}{#1}}}
\newcommand{\commentwho}[2]{{\colorbox{darkgreen}{{\bf #1:}}
    \textcolor{darkgreen}{#2}}}
\newcommand{\comment}[1]{{\textcolor{orange}{\# #1}}}
%\newcommand{\note}[1]{{\colorbox{gray}{\bf NOTE:}\textcolor{gray} {#1}}}
\newcommand{\needcite}{[\colorbox{red}{?}]}
\newcommand{\figref}[1]{Fig. \ref{#1}}
\newcommand{\tabref}[1]{Table \ref{#1}}
\newcommand{\lstref}[1]{Listing \ref{#1}}
\newcommand{\secref}[1]{Section \ref{#1}}
\newcommand{\appref}[1]{Appendix \ref{#1}}
\newcommand{\defref}[1]{Definition \ref{#1}}

%\renewcommand{\qedsymbol}{$\blacksquare$}
\newcommand{\br}[1]{\ensuremath{\lbrack #1 \rbrack}}
\newcommand{\gen}[1]{\ensuremath{#1}}

\newcommand{\setind}[2]{\ensuremath{\set{#1}_{#2}}}

\mathchardef\mhyphen="2D
\def\suchthat{\ensuremath{s.t.}\xspace}

% General
\def\secpar{\ensuremath{1^\kappa}\xspace}
\def\Gen{\ensuremath{Gen}\xspace}
\def\Hash{\ensuremath{Hash}\xspace}
\def\tx{\ensuremath{tx}\xspace}
\def\getr{\ensuremath{\stackrel{\$}{\gets}}\xspace}
\def\arg{\ensuremath{arg}\xspace}
\def\msg{\ensuremath{m}\xspace}

% UC
\def\IdealF{\ensuremath{\mathcal{F}}\xspace}
\def\IdealG{\ensuremath{\mathcal{G}}\xspace}
\newcommand{\ucio}[1]{\ensuremath{\text{#1}}}
\newcommand{\uccmd}[1]{\ensuremath{\mathtt{#1}}}
\def\cmd{\ensuremath{cmd}\xspace}
\def\sid{\ensuremath{sid}\xspace}
\def\ssid{\ensuremath{ssid}\xspace}
\def\ucsend{\ucio{Send}\xspace}
\def\ucrecv{\ucio{Recv}\xspace}
\def\Sim{\ensuremath{\mathcal{S}}\xspace}
\def\Env{\ensuremath{\mathcal{E}}\xspace}
\def\Exec{\ensuremath{\text{EXEC}}}

% Signatures
\def\sig{\ensuremath{\sigma}\xspace}
\def\Sig{\ensuremath{\mathsf{S}}\xspace}
\def\IdealFSig{\ensuremath{\IdealF_{SIG}}\xspace}
\def\SIG{\ensuremath{\mathsf{SG}}\xspace}

% Hashes
\def\IdealGRO{\ensuremath{\IdealG_{RO}}\xspace}
\def\hashlist{\ensuremath{HL}\xspace}

% Key Types
\def\mpk{\ensuremath{mpk}\xspace}
\def\msk{\ensuremath{msk}\xspace}
\def\ipk{\ensuremath{ipk}\xspace}
\def\isk{\ensuremath{isk}\xspace}
\def\apk{\ensuremath{apk}\xspace}
\def\ask{\ensuremath{ask}\xspace}
\def\VK{\ensuremath{\mathsf{VK}}\xspace}

% Clock
\def\IdealGclock{\ensuremath{\IdealG_{clock}}\xspace}

% BB
\def\ldgLedger{\ensuremath{L}\xspace}
\def\IdealGledger{\ensuremath{\IdealG_{Ledger}}\xspace}
\def\IdealGdledger{\ensuremath{\IdealGledger^\delta}\xspace}
\def\ldgHonest{\ensuremath{H}\xspace}
\def\ldgMap{\ensuremath{M}\xspace}
\def\ldgState{\ensuremath{\Sigma}\xspace}
\def\ldgUtxo{\ensuremath{U}\xspace}
\def\ldgTx{\ensuremath{tx}\xspace}
\def\ldgBlock{\ensuremath{B}\xspace}
\def\ldgHead{\ensuremath{Head}\xspace}

% VDR
\def\VDR{\ensuremath{\mathsf{VDR}}\xspace}

% DIDs
\def\IdealGPKIDID{\ensuremath{\IdealG^{\P,\delta}_{\mathsf{PKIdvu}}}\xspace}
\def\did{\ensuremath{did}\xspace}
\def\DID{\ensuremath{\mathsf{DID}}\xspace}
\def\DIDCreate{\ensuremath{\DID.Create}\xspace}
\def\DIDRead{\ensuremath{\DID.Read}\xspace}
\def\DIDUpdate{\ensuremath{\DID.Update}\xspace}
\def\DIDReadLdg{\ensuremath{\mathsf{ParseLedger}}\xspace}
\def\didOp{\ensuremath{op}\xspace}
\def\didOP{\ensuremath{\mathsf{OP}}\xspace}
\def\didOWN{\ensuremath{\mathsf{OWN}}\xspace}
\def\typ{\ensuremath{t}\xspace}
\def\lbl{\ensuremath{l}\xspace}
\def\LBL{\ensuremath{L}\xspace}
\def\val{\ensuremath{v}\xspace}
\def\sval{\ensuremath{\mathbf{v}}\xspace}
\def\VAL{\ensuremath{V}\xspace}
\def\P{\ensuremath{\mathtt{P}}\xspace}
% \def\Pc{\ensuremath{\mathtt{P}_{\mathtt{C}}}\xspace}
% \def\Pu{\ensuremath{\mathtt{P}_{\mathtt{U}}}\xspace}
% \def\Pd{\ensuremath{\mathtt{P}_{\mathtt{D}}}\xspace}

% VCs
\def\VC{\ensuremath{\mathsf{VC}}\xspace}
\def\VCCreate{\ensuremath{\VC.Create}\xspace}
\def\VCPublish{\ensuremath{\VC.Publish}\xspace}
\def\VCRevoke{\ensuremath{\VC.Revoke}\xspace}
\def\VCShow{\ensuremath{\VC.Show}\xspace}
\def\VCVerify{\ensuremath{\VC.Verify}\xspace}

% Misc
\def\IdealFCA{\ensuremath{\IdealF_{CA}}\xspace}

% Atala
\def\RealPKIDIDAtala{\ensuremath{\Pi_{DID}^{Atala}}\xspace}
\def\phiDID{\ensuremath{\phi_{DID}}\xspace}
\def\MasterKey{\texttt{M}\xspace}
\def\AuthKey{\texttt{A}\xspace}
\def\CommKey{\texttt{C}\xspace}
\def\IssueKey{\texttt{I}\xspace}
\def\LMKL{\texttt{LMKL}\xspace}

%%% Local Variables: 
%%% mode: pdflatex
%%% TeX-master: "prism-protocol"
%%% End:


\begin{document}
\maketitle


\begin{abstract}
	\textbf{DISCLAIMER}: Please, keep in mind that, as of now, this document
	basically contains a research log in the shape of personal notes. Do not take 
	anything written here as ground truth.
\end{abstract}

\section{Introduction}
\label{sec:introduction}

%%% Local Variables:
%%% mode: latex
%%% TeX-master: "single-mnemonic"
%%% End:

\section{Group Signatures}
\label{sec:gs}

\paragraph{Main intuition.}
%
Group signatures were first proposed by Chaum and van Heyst in 1991 \cite{ch91}
as a privacy-respectful extension to conventional digital signatures.
The intuition is simple: in order to avoid revealing the identity of the signer,
we make her signatures look indistinguishable to the signatures of all other
members of a group. Then, any entity with access to the group public key can
verify that a group signature comes from a member of the group, but cannot
tell which specific one actually created it.
%
Beyond this, group signature schemes have one main feature: a (set of) group
manager(s) control who can become a member of the group. Also, most frequently,
a (set of) group manager(s) decide when to expell a member from the group, or
when to re-identify the creator of a specific signature.
%
The initial idea put forward in \cite{ch91} had, as is to be expected, a number
of limitations. In the following years, the seminal work was extended, both
formally, functionally and efficiency-wise.

\subsection{Approaches, Models, and Variations}
\label{ssec:gsapproach}

\subsubsection{Main approaches.}
%
There are two main ways that are recurrently used to instantiate group signature
schemes: \emph{Sign-Encrypt-Prove} and \emph{Sign-Randomize-Prove}. In both
variants, when new members of the group join, they receive a credential (i.e.,
a signature by the issuer) derived from a secret they choose. Then, when
creating a group signature, they prove control a valid credential and the
associated secret. The difference lies on how this credential is kept hidden
from the verifier.
%
In the sign-encrypt-prove variant, proposed in \cite{bmw03}, signers
encrypt their credential with a \doubt{IND-CCA2? what type?} scheme and then
create a NIZK \needcite proof of knowledge of such a valid credential
over their secret (plus the message they want to sign). This typically results
in schemes with more efficient openings, as the opening authority just has
to decrypt the encrypted credential. However, group signatures require more
space.
%
In the sign-randomize-prove variant, proposed in \needcite \todo{Get Shorty?},
credentials are actually randomizable signatures by the issuer. Signers then
randomize the signature (credential), and create a NIZK proof of knowledge that
this credential is issued for a secret they know (again, plus the message
they want to sign). This usually results in schemes with less efficient openings,
as the opening authority needs to iterate over the registered credentials until
finding the one from which the randomized credential was derived. However, the
produced group signatures are also shorter.

When a new user wants to become a group member, a common habit is also to have
her prove ownership of a conventional digital identity. This identity is then
added to the de-identifying information leveraged by the opening authority, in
order to be able to associate (anonymous) group signatures to a real-world
identity. These conventional digital identities can be any type of PKI-based
(e.g. X.509 or equivalent) certificate. \todo{This is important for
  non-frameability! Check \cite[Section 3.2]{cdl+20} and re-state this.}
\todo{This is indeed necessary for non-repudiation (mentioned in
  \cite[Section 4]{lpy12}).}
\todo{Typically used in schemes with verifiable openings. See
  \cite[Section 1.3.3]{bsi12}.}

\subsubsection{Main security models.}
%
Security for group signatures is given in the form of game-based definitions
\needcite.
The first formal security model appeared in \cite{bmw03}. It captured security
for groups with a prefixed set of members, added by the group manager upon the
creation of the group. This static requirement was too restrictive to be
realistic, and was quickly extended in \cite{bsz05} and \cite{kty04,ky06}, to
model that new users can join the group after it has been created. The models
in \cite{bsz05,kty04,ky06} are usually referred to as dynamic or partially
dynamic. \cite{bcc+16} generalizes the (partially) dynamic setting into a fully
dynamic model, which considers extra desirable security properties when users
can be revoked (i.e., removed from the group).

Up to some subtleties, the main security properties in most works that appeared
after \cite{bmw03,bsz05,kty04,ky06} are as follows:

\begin{itemize}
\item {\bf Anonymity}. This is the core property concerning privacy. It
  ensures that given a signature produced by one out of two honest (challenge)
  users picked at random from the set of honest group members, it should be
  unfeasible to determine which one actually created it. 
  
\item {\bf Traceability}. This property (covered by the misidentification attack
  in \cite{kty04,ky06}, and spaning also non-frameability in \cite{bmw03})
  captures that it must be possible (by the proper authority) to unambiguously
  recover the identity of the signer of any valid group signature.

\item {\bf Non-frameability}. This property (considered in the framing attacks
  in \cite{kty04,ky06}) requires that no adversary should be able to create
  valid signatures that can be used to incriminate honest users by having them
  open to an honest user who did not create them.
 
\end{itemize}

\cite{bcc+16} also includes a trace soundness property, ensuring that signatures
can be associated (via opening proofs) to only one user \todo{why does this not
  appear in the other models? Maybe because they don't support verifiable
  openings? Check \cite{kss19} and, if so, move this to the variations in
  functionality.}

\subsubsection{Variations in functionality.}
%
Note that, while the security properties just described roughly have the same
purpose across models, there are still differences making some models stronger
or more general than others in some respects. This includes subtleties like who
generates the keys of the authorities (the environment vs the adversary), how
many queries can be made to the challenge oracles and whether or not the
signatures are fixed, etc. However, while important, there are usually generic
techniques that allow to migrate from one variant to another \todo{\needcite,
  ellaborate more, or rewrite}. Now we are interested in revisiting some of
the variations that have impact into the functionality that schemes
following each model can provide, or variations that are specific to group
signatures schemes and do not appear in other cryptosystems.

\paragraph{Separating issuer and opener.}
In some schemes \cite{bmw03,kty04} one single authority concentrates the
power to add new members to the group, and open signatures. This endows such authority with great
power -- and trust -- and is therefore one of the main concerns for privacy
advocates: not only this authority can prevent users from joining (in the case
of dynamic groups), but it can also reidentify group members. In order to reduce
this trust, and since it is a clear distinction between both functionalities
(issuing and opening), many schemes separate both roles into an issuer and an
opener. While both authorities need to be trusted for their respective tasks,
one can now make independent assumptions and even ensure certain properties
if only one of the two authorities is compromised or malicious. For instance,
non-frameability can be achieved even when both authorities are malicious;
traceability can be achieved with malicious openers; and anonymity can be
kept as long as the opener is honest. This is the case in \cite{bsz05} and
\cite{ky06} (which, although starts from a single authority, also considers
and formalises the case of separate authorities).

\paragraph{Protecting against corrupt authorities.}
Even when the issuance and opening functionalities are separated, as mentioned,
each entity can unilaterally break crucial properties in a group signature scheme.
While placing such kind of trust may be reasonable in some use cases, it may not
be so for others. Thus, a natural extension is to reduce this trust as much as
possible. In this direction, there have been works that have applied threshold
and similar techniques \needcite to require that a set of openers (as opposed
to a single opener) need to cooperate in order to open any group signature
\cite{bcly08} \needcite.
Recently, the role of the issuer was also subject to these techniques
\cite{cdl+20}, and it is now known how to produce group signatures such that no
single entity can prevent users from joining nor reidentify them.

\paragraph{Forward security.}
Forward security aims at restricting the impact of security compromises (e.g.,
key leakage) to the future: i.e., if an attacker gains control of a user's
private key, only signatures produced after the compromise should be considered
invalid. In schemes that do not ensure forward secrecy, such a compromise also
renders invalid signatures produced before the compromise. Note that, for
conventional signatures, only the compromised users is (directly) affected by
such event. However,  in group signatures, all group members are directly
affected, as the signer's identity is not revealed by the signature. Thus,
in order to learn whether or not a past signature was issued by a compromised
user, it is necessary to resort to the opener -- therefore, reidentifying
\emph{all} users in the system.
%
Forward security has been introduced in group signatures in works like
\cite{song01,ly12} \needcite.
%
Interestingly (but logically), the approach to incorporate this property
holds similarities with the functionality to implement fully dynamic group
signatures: for forward security, users are usually required to apply an
update procedure to their private keys \cite{ly12}; in fully dynamic group
signatures, it is the group manager (or the opener) who has to update some
key material \cite{bcc+16}. In both cases, time has to be discretized in epochs,
and conditions are introduced into the security properties that prevent an
adversary being able to deanonymize (produce) a valid signature generated
(using) a key in an epoch in which such key was not supposed to be active
(either because the user updated it -- in a forward secure scheme -- or the
group manager revoked it -- in a fully dynamic scheme).

\paragraph{Revocation mechanisms.}
Verifier Local Revocation \cite{bs04}, accumulators approach.

\paragraph{Opening controls.}
Even when separating the roles of the authorities or distributing them, the
fact that some entity (or set of entities) can open group signatures is still
a delicate matter that creates friction. A minimal control is to enable the
opener to prove that the opening process has been performed honestly. This is
precisely what is done by schemes that support \emph{verifiable opening}
\needcite. To further aleviate this issue, some schemes
introduce mechanisms to even prevent opening as long as some condition is
met. For instance, in \cite{seh+12,ehk+19}, the condition is determined by
the message (e.g., only signatures over offending messages can be opened).
To implement this, the authors introduce an extra authority, the
\emph{admitter}, such that the opener needs to get some information from the
admitter in order to be able to open signatures over some specific message.
In \cite{sbm10}, the authors propose \emph{contractual} anonymity: before a user
is able to anonymously access some service, both user and service sign a
contract defining the conditions upon which anonymity can be revoked. This
contract can only be enforced by a trusted entity, the \emph{accountability
  service} who runs a trusted execution environment (such as today's Intel
SGX \needcite) ensuring that the conditions are checked in a trustworthy manner.

\paragraph{Alternatives to opening.}
%
Even with separated authorities, with distributed openers, or with more generic
opening controls such as message-dependent opening, opening may not even be
the type of reduction of privacy that is necessary. In this vein, alternative
schemes have been proposed that do not include any kind of
opening at all.
%
In the schemes proposed in \cite{kty04,cpy06,bcly08}, there is an extra
authority that can produce \emph{tracing} trapdoors on a per-user basis.
Once the tracing trapdoor for a specific user is given to some entity, this
entity can link (or trace) all group signatures produced by that user.
These trapdoors are also an effective mechanism for implementing verifier-local
revocation.
%
In \cite{gl19}, the opener is replaced by a \emph{converter} who, rather than
opening any given group signature, can link sets of signatures. More
specifically, when queried with a set of signatures, the converter transforms
them so that every signature coming from the same user gets assigned a same
randomized pseudonym. Furthermore, to prevent the converter from gradually
relinking all the signatures, these conversions are non-transitive: i.e.,
signatures produced by user $A$ will receive a different randomized pseudonym in
different queries.
%
In \cite{dl21}, there is directly no trusted entity that can open or link group
signatures. Instead, every user can create proofs of several signatures having
been issued by herself. In this case, the authors also propose a scheme whereby
a user can prove sequentiality of a set of signatures: i.e., that all signatures
were produced by herself, and no signature has been omitted or inserted into the
sequence, which is also given in the order in which the original signatures were
produced.

\subsubsection{Efficiency.}

\begin{itemize}
\item Size of (group) keys.
\item Size of signatures.
\item Verification time (batching, including revocation checks...)
\item Accumulators.
\end{itemize}

\subsubsection{Main challenges.}

Efficiency is possibly good for many real-world cases. Still: utility
and unavailability of implementations.

\note{Concurrent joins -- avoid extraction \cite{ky05,klap20}.}

\subsection{Related Primitives}

\subsubsection{Ring Signatures}
\label{sssec:rs}

\cite{rst06}

\cite{bcc+15} uses ring signatures to build a fully dynamic group signature
scheme.

\subsection{Applications}
\label{ssec:gsapplication}

See \cite[Section 1.2.3]{bsi12}: it seems mostly theoretical applications.

\subsubsection{Theory}
\label{sssec:gstheoryapp}

GS-MDO imply IBE \cite{ehk+19}.
GS are implied (shouldn't his by imply?) by trapdoor permutations \cite{bmw03}.
GS imply one-way functions \cite{romp90}.
GS imply public key encryption \cite{aw04}.
Commitments from group signatures (group signatures imply commitments?)
(ref. 25 in \cite{bfg+11}).

\subsubsection{Real World Deployments}
\label{sssec:gsrwdeploy}

\cite{ehk+19} mentions anonymous auctions, anonymous bulletin boards and
anomaly detection systems.

Maybe make some comments on how to deploy gs schemes, or what aspects may be
relevant. For instnace, some schemes/models (e.g. \cite{ly12}) assume a
trusted party who generates the keys for the authorities. While this has
a big impact in security/trust model, in the real world it would probably
be circumventable with audited processes for most use cases. Also, the
issue with concurrent issuance and the approach we took for implementing
\cite{gl19} in ICT4Cart, or the limitation to logarithmic numbers of users, etc.

\subsection{TODOs}

\begin{itemize}
\item Fully anonymous dynamic GS from SPS-EQ \cite{ds16}
\item Multi-group group signature schemes. Compare with ACs with multiple
  issuers.
\end{itemize}

\subsubsection{Further Notes}
\note{Integrate this section or delete it.}

\todo{CCA2 type of anonymity is defined in \cite{bbs04}. Schemes with CPA type
  of anonymity are (according to \cite{ehk+19}): \cite{bw06,bw07}.}

\cite{ehk+19} has a nice related work mentioning the assumptions that are
typically used.

Signature highjacking attacks (referenced in \cite{ehk+19}).

If we want to have traceability, the issuer needs to be honest. See what
implications this has in practice (e.g., what architectures are compatible
with this?) This probably makes schemes with separate issuing and opening
authorities more attractive in practice.

Other minor variants: group blind signatures, democratic group signatures,
mediated group signatures. All referenced in \cite[Section 1.3.5]{bsi12}.

Backward linkability property of group signatures with Timed Verifier-Local
Revocation, proposed in \cite{nf05}. Mentioned in \cite[Section 8.1.2]{bsi12}.

Relation of forward security with proactive security \cite{oy91}?

References 1, 24, 32 and, probably, 35 in \cite{bfg+11} seem to be (traceable)
group signatures that use the nym approach (or similar) for linking. Check.

\paragraph{Academic indicators.}
IACR, Esorics, CCS, AsiaCCS, SP, EuroSP, Usenix Sec and Privacy, PETs... Not
sure where to fit this.

Main cryptographic building blocks:

\begin{itemize}
\item BB signatures \cite{bb04}.
  \begin{itemize}
  \item Used in: \cite{ky05}.
  \item Paillier encryption of discrete-logarithms? Used in \cite{ky05} and,
    I think, \cite{gl19}.    
  \end{itemize}
\item Model for reduction: list of group signature schemes based on RO and
  non-RO are given in \cite{bcc+16}.
\end{itemize}

%%% Local Variables:
%%% mode: latex
%%% TeX-master: "sok-privsig"
%%% End:

\section{Attribute-Based Signatures}
\label{sec:abs}

\begin{enumerate}
\item Main intuition.
\item Evolution of main security models.
\item Variations concerning functionality.
\item Evolution in efficiency.
\item Main issues.
\item Approx. number of related works in main crypto+security venues: IACR,
  Esorics, CCS, AsiaCCS, SP, EuroSP, Usenix Sec and Privacy, PETs...
\end{enumerate}

\subsection{Related Primitives}

\subsection{Applications}
\label{ssec:absapplication}

\subsubsection{Theory}
\label{sssec:abstheoryapp}

\subsubsection{Real World Deployments}
\label{sssec:absrwdeploy}

%%% Local Variables:
%%% mode: latex
%%% TeX-master: "sok-privsig"
%%% End:

\section{Anonymous Credentials}
\label{sec:ac}

First proposed in \cite{chau85}.
CL signatures (idemix) \cite{cl01,cl04}.
UC formalisation in \cite{cdhk15}.
Brands works (U-prove).
Coconut paper, referenced in \cite[Section 1]{cdl+20}: anonymous credentials with
threshold opening.

\begin{enumerate}
\item Main intuition.
\item Evolution of main security models.
\item Variations concerning functionality.
\item Evolution in efficiency.
\item Main issues.
\item Approx. number of related works in main crypto+security venues: IACR,
  Esorics, CCS, AsiaCCS, SP, EuroSP, Usenix Sec and Privacy, PETs...
\end{enumerate}

Anonymous Credentials (ACs) were first proposed by Chaum \cite{chau85}, and
later formally treated and optimized in a series of papers by Camenisch and
Lysyanskaya \cite{cl01,cl02,cl04}. Roughly speaking, users receive credentials
from an issuing organization, which can then use to authenticate against some
verifier. The main property is that this authentication process can be done
in a privacy-preserving way. This typically means that the verifier cannot
link different authentication processes by that same user. And, usually, in
credentials composed by multiple attributes, that the verifier only learns
the attributes that the user is willing to share (or a function of them).

\subsection{Approaches, Models, and Variations}

\paragraph{Main approaches.} %

\begin{enumerate}
\item[\cite{cl01}:]
  \begin{itemize}
  \item Functionality: FormNym, GrantCred, VerifyCred, VerifyCredOnNym.
  \item Approach: Users create a pseudonym with each organization. For this,
    they have to register first with the organization (associating a key to
    a user name). Whenever the user wants to get a credential from a organization
    in which he registered, he has to authenticate, and then the organization
    decides whether or not to grant the credential. Users can show their
    credentials to organizations in two ways: (1) just proving that they own a
    credential issued by some organization; or (2), proving that they own a
    credential issued by some organization \emph{and} this credential is over
    the same pseudonym that the user holds with the verifying organization.
    Essentially, pseudonyms are the concatenation of two random numbers (one
    provided by the user, and the other by the organization). They get
    associated an authenticating tag, which is a statistically hiding commitment
    to the user secret (computed using randomness provided by both user and
    organization). Organizations keep a database of the generated pseudonyms.
    To get a credential from an organization where the user registered
    previously, the user proves knowledge of the secret involved in the
    pseudonym. The organization then creates a ``random signature'' (by using a
    the inverse of random exponent, with the help of the factorization of an RSA
    modulus chosen by the organization at setup time) over the pseudonym, and
    gives that to the user. Users can show the credentials (in any of the
    previously stated two ways) by creating non-interactive zero-knowledge
    proofs of knowledge of the appropriate secret. Note that, in case (2),
    this means that the proof is over two pseudonyms, which must share the
    same secret.    
  \end{itemize}
\item[\cite{cl02}:]
  \begin{itemize}
  \item Approach: Te paper aims at improving efficiency of AC systems, as
    well as still ensuring that the user can retrieve their credentials in
    a privacy-preserving manner. For this purpose, the core of the paper
    describes a signature scheme that allows signing blocks of messages,
    and later prove, in zero-knowledge, knowledge of such a signature. To
    enable private retrieval of credentials, the authors describe a protocol
    for their signature scheme, whereby the user specifices a commitment to
    a message (or a block of messages, although this is not actually
    formalized), and get a signature of the message in return. While the
    authors do not strictly define an AC system, the application of their
    signature system and related protocols to the AC domain seems direct.
    Indeed, the same functionality as in \cite{cl01} seems straight
    forward given the cryptosystems in \cite{cl02}.
  \end{itemize}
\item[\cite{cl04}:] Follows a similar approach to \cite{cl02}, although
  changing the security assumptions to be based only on problems related
  to the discrete logarithm, and on bilinear maps. Basically, defines a
  signature scheme suitable for signing multiple messages, a protocol to
  obtain a signature on a committed value, and a protocol to prove knowledge
  of a signature. Concerning commitments, they rely on Pedersen commitment
  scheme, for which protocols are known to prove knowledge and equality of
  committed values. This lets them follow the same approach to \cite{cl02},
  which is in turn the one in \cite{cl01}.
\item[\cite{cks10}:] The system proposed here follows the conventional
  approach to anonymous credentials based on randomizable signatures (with
  the possibility to sign committed values) over vectors of messages.
  It however enables issuers to produce updates on already signed messages.
  This updates can be leveraged to incorporate revocation. Essentially,
  time is divided in epochs, and credentials include an attribute which
  represents the expiration date/epoch of the credential. Users have to
  prove that their credential is valid for the current epoch. Similarly,
  to add new attributes, the issuer leverages the internal structure of
  the signatures over vectors of messages, to basically add new messages
  (attributes). In both cases, it is responsibility of the user to update
  her credential given the latest update by the issuer. An example
  instantiation is given using BBS signatures.
\item[\cite{cmz14}:] This work differs from the rest in that they use
  secret key cryptography rather than public key. Namely, they assume a
  setting in which the issuer will also be the verifier (e.g., a
  transportation organization that issues tickets for its travelers). In
  this setting, it is reasonable to assume that both issuer and verifier
  will have access to the same secret. Therefore, they replace signatures
  with (algebraic) MACs. They do not create pseudonyms for end users (like
  \cite{sms+19}). Other than that, the supported functionality is similar
  to the rest: Setup and KeyGen to initialize the cryptographic parameters,
  and produce the issuer's keys, respectively; an interactive protocol for
  end users to blindly receive credentials (MACs) on their attributes; and
  an interactive protocol for showing and verifying a credential. In case
  that issuance does not need to be blinded, it becomes a non-interactive
  protocol (i.e., the issuer just produces the credential and sends it to
  the user). Otherwise, the user has to encrypt the attributes and prove
  they are correct. Presentation only requires one message from the user
  to the verifier.
\item[\cite{cdhk15}:] The approach to build anonymous credentials is roughly
  the same as in the previous works \cite{cl01,cl02,cl04}, although with
  relevant differences that make it more efficient. Basically, users create
  pseudonyms derived from a common secret. Then, any user can request a
  credential. For this, the user needs to prove knowledge of the pseudonym's
  secret. In return, the issuer produces an ``unlinkable redactable signature''
  on a set of attributes, for that pseudonym. A redactable signature is
  essentially a signature over a set (block) of messages, that enables
  subsequent verification that a subset of those messages was signed. This
  is essentially the same as in \cite{cl02,cl04}. However, to achieve
  unlinkability, \cite{cdhk15} incorporates efficient zero-knowledge proofs
  to prove correctness of the subset of the messages that a user wants to
  ``show'' in a credential showing process. This ensures unlinkability (hence,
  privacy). The internal scheme used for this is a vector commitment scheme
  that uses Lagrange interpolation to allow checking that a subset of the
  committed messages are indeed contained within the commitment. Moreover,
  both the vector commitment scheme, and the vector signing scheme, are
  structure preserving. This means that, since all the produced messages,
  commitments, and signatures, are elements of the group (rather than of the
  field), proving knowledge is much more efficient (e.g., it prevents costly
  extraction required for security proofs).
\item[\cite{ckl+15}:] Follows the conventional approach: a secure commitment
  scheme with proofs of knowledge, and a signature scheme that allows to
  sign vectors of messages, including commitments. Here, as in the seminal
  works by Camenisch and Lysyanskaya \cite{cl01,cl02,cl04}, pseudonyms are
  used as explicit building blocks. \todo{Not sure yet of how they are
    made explicit in the generic construction. Clarify.}
  A ``syntactic'' variation: they introduce
  the concept of tokens both for credential issuance and presentation. To
  request issuance of a credential, users have to produce an issuance token
  from the set of attributes they want to reveal to the issuer, and the set
  of attributes they want to hide (potentially coming from another credential
  already owned by the user). An equivalent description applies for presentation
  of a credential to a verifier. In both cases, the issuer/verifier has to
  check the received token prior to issuing a credential or granting access.
  They also add revocation functionality. For this, a revocation authority
  (not necessarily the issuer) keeps a revocation information list, which is
  basically a list containing no longer valid attributes. In order to prove
  that their credentials are valid w.r.t. some attribute, users have to generate
  a (non) revocation token, which proves that the attribute has not been
  included in the revocation information list.
\item[\cite{dmm+18}:] Follows a completely different approach to (most) other
  works. Instead of using randomizable signatures (over committed values) or
  some equivalent building block, it relies on predicate encryption.
  Essentially, a user with attributes $A$ is given a decryption key for a
  predicate $f_A(g)$ which, given a predicate $g$, returns $g(A)$. When
  a verifier wants to check if a user has a set of attributes that meet
  certain policy $h$, challenges the user to decrypt a ciphertext over
  a random message that has been associated to predicate $h$. Then, only
  if the attribute set $A$ of the user meets policy $h$, the user will be
  able to decrypt the random message (since $f_A(h) = h(A) = 1$). This
  protocol is actually augmented with some commitments, to prevent
  misbehaviour from user and from verifier, but the approach is the same.
  \doubt{There is something weird here, or that I do not understand --
    predicate encryption requires a
    set of attributes for encryption, rather than another predicate. I have
    reached out to the authors to see if they can help me clarify this.}
\item[\cite{fhs19}:] Based on Structure Preserving Signatures from equivalence
  classes (SPS-EQ). The approach resembles that of \cite{ckl+15} in that it
  uses a structure preserving primitive, and signs (commitments of) sets of
  messages. However, they aim to avoid (as they claim) costly zero-knowledge
  proofs. Basically, upon setup, a set commitment scheme is set up. Each user
  generates a keypair. To obtain a credential on a set of attributes $A$, the
  user proves knowledge of the user secret key, which is used as randomness
  for a set commitment over $A$. The issuer (proves knowledge of the secret
  key and) returns a SPS-EQ signature over $A$. To show the credential, the
  user randomizes the commitment set and the signature, reveals the required
  subset $D$ of $A$, and computes a witness that indeed $D \subset A$, as well
  as a proof knowledge of the randomness used to randomize the credential (to
  prevent replays). 
\item[\cite{sms+19}:] The Coconut system follows a similar approach than the previous
  works with respect to the representation and showing protocol for credentials.
  Namely, credentials are randomizable signatures issued blindly. In this case,
  the authors rely on Pointcheval Sanders signatures \cite{ps16} as main
  building block, to which they incorporate techniques from other schemes to
  support threshold issuance.
\end{enumerate}

\paragraph{Main security models.}

\begin{enumerate}
\item[\cite{cl01}:] Security in \cite{cl01} is proven in an ad-hoc way,
  using the ideal-world/real-world paradigm. The authors describe a
  simulator for the operations supported by the system (Setup, FormNym,
  GrantCred, VerifyCred and VerifyCredOnNym), and prove that the output of this
  simulator is indistinguishable from that of the real protocol, under the
  strong RSA and Diffie-Hellman assumptions.  
\item[\cite{cl02}:] No security model is given for the anonymous credential
  application of the signature scheme (reasonably, as it is not the main focus
  of the paper). The signature scheme is proven secure (no specific mention
  to a security model, although I suppose it would be quite straight forward)
  based on the Strong RSA and Diffie-Hellman assumptions.
\item[\cite{cl04}:] Same as in \cite{cl02} (no actual security model for the
  AC system). The custom proofs are based on the DDH and LRSW assumptions.
\item[\cite{cks10}:] Signer privacy and user privacy properties are required
  for the underlying signature scheme (thus, inherited by the credential system,
  although I have not found explicit model). Signer privacy captures that the
  adversary (including corrupt users) doesn't learn any information about the
  signer from signatures and updates of signatures. User privacy captures that
  the adversary (including a corrupt issuer) does not learn any information
  about the blindly signed messages. \doubt{Not sure I get the utility of
    signer privacy. The original property seems to come from \cite{bckl08}.}
\item[\cite{cmz14}:] The security properties proven in this work are:
  \begin{itemize}
  \item \emph{Unforgeability}: The adversary cannot produce a valid proof for a
    statement, if none of the attribute sets for which it has received a
    credential meets that statement.
  \item \emph{Anonymity}: The proofs produced in a presentation do not leak
    more information than the statement being proved.
  \item \emph{Blind issuance}: The issuance protocol is a secure two-party
    protocol for generating credentials on the user's attributes.
  \item \emph{Key-parameter consistency}: no issuer can use different parameters
    to produce credentials of different users such that their anonymity can
    be compromised afterwards. \doubt{This somehow remembers non-frameability in
      group signatures.}
  \end{itemize}
\item[\cite{cdhk15}:] The model (and associated proofs) presented in this paper
  is very comprehensive, and detailed. Specifically, for the main building block
  (Unlinkable Redactable Signatures), the authors give both game-based
  definitions as well as an ideal functionality. This ideal functionality models
  security in a single-issuer setting. However, as it is done in the UC
  framework, it can be composed with itself, so that a scenario with multiple
  issuers, in which users can combine credentials obtained by different users
  into a single showing process (``presentation'') is directly achievable by
  composing multiple instances of the ideal functionality (one per ``proof
  part'').
\item[\cite{ckl+15}:] The main security and privacy properties required are
  pseudonym collision-resistance (two different user secret keys cannot produce
  the same nym for the same scope), unforgeability (the adversary cannot forge
  issuance tokens or presentation tokens), and privacy (no two presentations by
  the same user can be linked) or weak privacy (presentations by the same user
  can be linked, but not traced to an issuance session). Privacy is modelled
  in the simulation paradigm, the difference in the modelling between privacy
  and weak privacy being a filter that removes some information \note{(I find
    this kind of weird; probably I'm wrong. Would be nice to study further)}.
  The unforgeability and privacy properties are essentially inherited from the
  main building block, privacy-enhancing attribute-based signatures. Although
  there, the authors begin by defining a blind issuance experiment (ensuring
  that the issuer does not learn anything more than what it should during
  isusance), and then define two variants for user privacy: presentation privacy
  and untraceability, which basically match the strong and weak notions for
  PABCs (in untraceability, two presentations by the same user can be linked,
  but not trace to a specific credential issuance). [Strong] user privacy has
  to satisfy both blind issuance and presentation privacy; weak user privacy
  has to satisfy both blind issuance and untraceability. I find this model
  somehow interesting, as it is one of the few (maybe along with \cite{cmz14})
  that includes some notion related to tracing back to issuance of credentials,
  as in group signatures). In both cases (PABC and PABS), [strong] privacy
  implies weak privacy.
  Essentially, the security properties required by the PABC scheme, are
  ensured by the building blocks. Namely: the pseudonym system (requiring
  key extractability, collision resistance and pseudonym unlinkability); the
  revocation scheme (requiring soundness and revocation privacy); the
  PABs (requiring unforgeability, and [weak] privacy); and the commitment
  scheme (requiring blinding and hiding, and opening extractability).
\item[\cite{dmm+18}:] The authors give a game-based security model, covering
  properties of unforgeability, anonymity, and policy-hiding. The latter
  means that, for delegated verification, the verifier is oblivious to the
  policy being verified.
\item[\cite{fhs19}:] Gives a game-based security model, with anonymity and
  unforgeability being the main security properties (besides correctness).
  Unforgeability resembles the traceability property of group signatures,
  which are recognised as an influence in their modelling -- more specifically,
  \cite{bmw03} (the model -- and the oracles therein -- definitely resembles
  the BMW model for GSs). The authors claim to be the firsts to give a well defined and
  formal model for ACs, besides \cite{ckl+15} (which gives game-based properties
  in the simulation paradigm) and \cite{cdhk15} (which follows the UC framework).
  However, \cite{fhs19} claims that adopting non-simulation based definitions
  allows them to reach a more efficient construction. They also support malicious
  issuer key generation, as issuers have to prove knowledge of their secret keys
  (which allows extraction in the games).
  An interesting aspect of their model is that all attributes in their
  credentials are learned by the issuer -- i.e., there are no blindly signed
  attributes. This implies that it is not possible to sign secret keys owned
  by the users (as in the conventional CL credentials). Then, for showing a
  credential, all the user has to do is reveal the attributes s/he is required
  to show. This has the disadvantage that, depending on the use case, this may
  be trivial to attack. For instance, assume a COVID-like pandemic situation,
  in which citizens have to prove being vaccinated. An AC system for this use
  could be naively defined as: { Type of vaccine, Date of vaccination }. But
  this is very susceptible to dictionary attacks, and leaves much to the
  engineering/implementation level. E.g., if it is easy for an eavesdroper
  to get hold of a credential (not even the plaintext attributes), s/he can
  just iterate through the small list of possible values, and then take
  advantage of the ``stolen'' credential. This is certainly secure under the
  model defined in \cite{fhs19}, but it is definitely easy for implementations
  following this model to go wrong...
  They also model only interactive show processes. Furthermore, to prevent
  replay attacks, the user has to prove knowledge of some randomness included
  in the credential. However (\textbf{cross-check this!}), since signatures
  are randomizable without knowledge of any secret value, any adversary that
  learns a credential (not necessarily any plaintext attribute associated to
  it), would be capable of authenticating through a dictionary attack in case
  of small attribute ranges.
  Note: this may just be a modelling issue (or not even that, cross-check later
  in case I misunderstood something). In their construction, they include
  a secret key of the user in the credentials, of which the user has to prove
  knowledge during the issue protocol. They do this by ``reusing'' the user
  secret key as the randomness employed for the set commitment scheme, and
  requiring the user to prove knowledge of the corresponding public key.
  However, I am not sure of whether or not this somehow thwarts the previously
  mentioned ``attack'', which affects the show protocol (\todo{i.e., the user does
    not need to prove knowledge of the usk during show, no?} \note{Actually, it
    should not, as it would probably prevent anonymity}) In any case, even if
  the specific construction and instantiation thwarts this attack, I think their
  model allows it.
\item[\cite{sms+19}:] No security model is given nor adopted.
\end{enumerate}

\paragraph{Variations in functionality.}

\begin{enumerate}
\item[\cite{cl01}:] Extensions provided to the main functionality are:
  \begin{itemize}
  \item PKI-based and all-or-nothing non-transferability: Both options
    involve sharing a verifiable encryption of the user secrets. In the
    case of PKI-based non-transferability, the CA receives a verifiable
    encryption, using as key the user's master secret, of some valuable
    piece of information belonging to the user (e.g., the user's secret
    key associated to his public key with that CA). In the all-or-nothing
    case, the user shares with each organization (with which it has a
    pseudonym), a verifiable encryption of his pseudonym secret data, encrypted
    with his master secret. In both cases, if the user shares his master
    secret with someone else, then he automatically gives access to
    other relevant pieces of information (as the verifiable encryptions are
    made public by the corresponding entity).    
  \item One-show credentials: The validating tags over pseudonyms are extended
    to include an extra value in their commitment. Then, users have to reveal
    that value (or, rather, a value deterministically derived from that) in the
    zero-knowledge proofs they create at show time. Since the value is always
    the same, verifiers can check whether they have seen it before, or not.
  \item Revocation: Both local (i.e., within an organization) and global (i.e.,
    referring to an identity of the user in an external CA) is possible by
    extending the pseudonyms of the user, and the credential showing protocols,
    with values that allow subsequent decryption by a trusted party of their
    pseudonym or external identity, given a transcript of the credential
    showing.    
  \item Attributes: The paper also succinctly describes how multiple attributes
    per credential could be included. This would be done by dividing the inverval
    within the exponent used to compute the pseudonym in sub-intervals, and then
    proving that the exponent lies within a sub-interval.
  \end{itemize}
\item[\cite{cmz14}:] The paper mentions ``Credential translation'', which
  basically means being able to prove that two attributes in different
  credentials are actually the same. This is not formalized or detailed,
  though.
\item[\cite{cdhk15}:] The same as in the previous works (\cite{cl01,cl02,cl04}).
\item[\cite{dmm+18}:] The scheme supports delegated verification. Basically,
  to implement delegated verification, the delegatee receives an encryption of
  the policy it has to verify, plus a signature of the encryption. If the
  signature is valid, it re-randomizes the ciphertext for the policy, and
  sends it to the user, who follows the same protocol as in the normal case.
  The paper also mentions how to add more functionality, like delegation of
  credentials and revocation, but they are not modelled. Also interestingly,
  they claim somewhat informally (in Section 2.3) that no scheme satisfying
  anonymity in the presence of an untrustworthy issuer can support delegation
  of verification with policy hiding.
\item[\cite{sms+19}:] It is essentially the same as previous works, although it
  supports threshold issuance. Interestingly, \cite{sms+19} does not make use
  of pseudonyms. I.e., credentials are just signed (blocks of) attributes. This
  probably lets them achieve better efficiency. \doubt{This makes the scheme
    to be practically like a group signature scheme (without revocation...)}
\end{enumerate}

\paragraph{Efficiency.}

\cite{sms+19} seems to be the most efficient one (note: relies on the
Fiat-Shamir heuristic). Credentials are only composed by two G1 elements,
independently on the number of attributes. \todo{Cross-check this after
  review.}

\cite{cdhk15} and \cite{fhs19} achieve credential showings of size independent
of the number of attributes.

\paragraph{Main challenges.}

\subsection{Related Primitives}

Most schemes \cite{cl01,cl02,cl04,cdhk15,sms+19} are based on randomizable
signatures, with the possibility of signing committed values, that also support
efficient proofs of knowledge.

The most recurrent signature schemes are Camenisch-Lysyanskaya \cite{cl02,cl04},
BBS/BBS+ \cite{asm06,cdl16b}, and Pointcheval Sanders \cite{ps16}, with the
exception of \cite{cdhk15} that leverages their own signature scheme, called
Unlinkable Randomizable Signatures.

\subsubsection{Pseudonym Systems}
\label{sssec:pseudonyms}

\doubt{Not sure if this fits here.}

\subsection{Applications}
\label{ssec:acapplication}

\subsubsection{Theory}
\label{sssec:actheoryapp}

Main approaches to build ACs (from ToPS paper I reviewed):

\begin{itemize}
\item Re-randomizable signatures on commitments (also as in DAA):
  \cite{cl02,cl04,lmpy16,ps16}.
\item Equivalence class signatures: \cite{fhs19,hs14}.
\item Redactable signatures \cite{cdhk15,sand20} and malleable signatures
  \cite{ckl14}.
\item Predicate encryption: \cite{dmm+18}.
\end{itemize}

PS: Tokenization.

\subsubsection{Real World Deployments}
\label{sssec:acrwdeploy}

PS: Tokenization.
U-Prove.
Idemix.
Signal's AC \needcite

\subsection{TODO}

\begin{itemize}
\item \cite{bcc+09,cklm14}: Delegation (\cite{cklm14} relies on \cite{cklm12}).
\item \cite{aks12}: ACs with revocation.
\item \cite{asm06}: This seems to be the origin of the BBS+ branch of ACs (also,
  probably \cite{aks12} is in this family).
\item \cite{cgm16} An AC system that builds ACs from existing (conventional,
  such as ECDSA) signatures.
\item \cite{cl11}: ACs from aggregate signatures.
\end{itemize}

[ToPS paper I reviewed] refers to a strategy to do threshold issuing in \cite{bbh06}.

Concepts in ACs: multishow, one-show (Brands et al); interactive vs non-interactive
showings (differentiation made in \cite[p.6]{fhs19} but, why would one want interactive,
in the real world?)

Which schemes support comibning credentials from multiple issuers? I think this
is directly obtained in the schemes following the CL approach; but others?

A recurrent claim in schemes avoiding using zero-knowledge proofs of knowledge
of some attributes being included in a credential is that these proofs are costly
(like, linear in the number of attributes). Yet, they resort to other primitives
that may also be costly (e.g., predicate encryption \cite{dmm+18}) or somehow move
the cost to some other place (e.g., \cite{fhs19} uses a vector commitment scheme
that requires a public key with size that grows with the number of attributes
that \emph{can} be committed). It would be nice to have a detailed and complete
(e.g., not skipping the data that makes one scheme look bad, like the size of
the public keys) comparison. And, ideally, with concrete instantiations (not only
theoretical analysis)?


%%% Local Variables:
%%% mode: latex
%%% TeX-master: "sok-privsig"
%%% End:

\section{Direct Anonymous Attestation}
\label{sec:daa}

\comment{\bf \em DISCLAIMER: This whole section is in a very immature state,
  read at your own risk!}

% \begin{enumerate}
% \item Main intuition.
% \item Evolution of main security models.
% \item Variations concerning functionality.
% \item Evolution in efficiency.
% \item Main issues.
% \item Approx. number of related works in main crypto+security venues: IACR,
%   Esorics, CCS, AsiaCCS, SP, EuroSP, Usenix Sec and Privacy, PETs...
% \end{enumerate}

\paragraph{Main intuition.}
%
Direct Anonymous Attestation \cite{bcc04} was initially proposed to address a
need in the Trusted Computing Group
\footnote{\url{https://trustedcomputinggroup.org/}. Last access on October 29th,
  2020.}. Take a computing platform equipped with a trusted hardware element
(referred to as TPM, from Trusted Platform Module, in the TCG jargon). A TPM
is trusted to securely store secrets and perform some cryptographic operations,
but is otherwise not able to communicate directly with the outside world. To
do that, it has to leverage other components of the larger computing platform,
which are not fully trusted. Then, the challenge is to come up with a way for
third parties (verifiers) to check that they are indeed communicating with a
computing platform that is equipped with a valid TPM. Moreover, in order to
preserve privacy, a requirement is not to reveal which specific TPM the
verifier is communicating with. Initially, several proposals were made
\needcite\footnote{\cite{bcc04} references works by Boneh et al, and Brickel,
  but no academic reference seems to exist.}, but it was \cite{bcc04} the one
that was finally adopted.

\subsection{Approaches, Models, and Variations}
\label{ssec:daaapproach}

\note{For now, just notes on the DAA papers. Then, unify.}

DAA has many similarities with group signatures, but also important differences.
While in group signatures the user joining the group is an ``indivisible''
entity, in DAA this role is divided among the TPM and the larger platform
containing the TPM (usually referred to as the Host). This undoubtedly
complicates the modelling, as different trust assumptions have to be made for
each of them. Also, security in DAA is modelled in the UC paradigm \todo{why?}
\todo{(not true. Some works use game-based definitions, as \cite{bfg+13} -- see
  also the discussion and references therein)},
whereas group signatures are modelled with game-based definitions. From a
functionality perspective, DAA does not allow opening and, also, uses pseudonyms
that can be leveraged to provide linkability%
\footnote{We note that, recently, group signature schemes have been proposed
  that do not implement opening either \cite{dl21}, and also leverage pseudonyms
  to introduce some kind of linkability \cite{gl19,kss19,dl21} -- and, indeed,
  these schemes draw techniques from DAA.}.

Besides these differences, the typical approach is similar: a new TPM runs an
interactive $\langle Join,Issue \rangle$ protocol with an issuer whereby the
TPM receives a membership credential. Then, to produce attestations, the TPM
creates SPKs of a valid credential, over a message including information about
the status of the Host. However, since there is no opening, DAA employs the
\emph{Sign-Randomize-Prove} strategy from group signatures, which produces more
efficient attestations (in group signatures, at the cost of more inefficient
opening). In addition, the pseudonyms added in DAA are built from \emph{scopes}
(also called \emph{basenames}) and cryptographic hash functions \todo{more
  insight on this}.

In \cite{bl07}, an extension of the DAA approach in \cite{bcc04} is made so
that it is possible to revoke TPMs not only based on leaked private keys (which
is a restricted approach), but also based on signatures and on requests by the
issuer. To do so, the membership proving protocol (roughly, signing) is extended
so that the prover demonstrates that its membership credential is different to
the ones who have been included in revocation lists by a revocation authority.
This, however, incurs in extra costs of the signing and verification protocols,
which grow linearly with the number of revoked users. During signing, \cite{bl07}
also remove the basename concept, which in \cite{bcc04} is used to derive
linkable pseudonyms when needed. Instead, they employ random bases. Finally, in
their model, \cite{bl07} do not differentiate between TPM and host. While the
authors give some insight into how to separate them, extending the formalization
(and proofs) has been shown to be problematic \todo{Bernhard work and later
  criticism?}

\cite{cms08} points out a flaw in the reduction of \cite{bcc04} and fixes it.
Interesting definition of DAA (maybe useful for giving a high-level overview).
It also provides a security proof (simulation/UC-based model) for the
scheme in \cite{cms08b}. In \cite{cms08b}, the authors present a pairing-based
DAA system using asymmetric pairings, achieving improved efficiency.
However, as pointed out in \cite{bfg+13} linking is still not correctly modelled,
being possible to link different authentications from a same TPM \todo{no detail
  is given. Try to see how/why.}
\note{\cite{cms08b} contains some nice (albeit now old) cost comparison.}

\cite{bcl09} proposes an alternative model using games, rather than the
simulation paradigm. They show that security in the UC model by \cite{bcc04}
implies security in their game-based model, but the converse is left as an
open question. However, as pointed out in \cite{bfg+13}, the model in
\cite{bcc04} has a flaw concerning linkability, which is thus inherited by
\cite{bcl09}. \todo{Cross-check this with subsequent Camensich-Lehmann works.}
Construction-wise, they propose an instantiation based on bilinear pairings,
which greatly improves costs.

In \cite{bfg+13}, flaws are pointed out in previous schemes (notably in the
one adopted by the TCG, \cite{bcc04}) that affect unforgeability and linkability
of the produced signatures. They argue that proving security in the UC model is
hard and error-prone in for DAA. Thus, they to game-based approach. In this
context, they propose a black box approach for building DAA systems from
\emph{weakly blind signatures} and \emph{Linkable Indistinguishable Tag} systems.
Moreover, they first capture security for scenarios in which TPM and host are
the same entity; then, they provide an approach to generalise to the case
in which they are different entities (with the different trust assumptions).
However, in their model, they make some odd requirements in the way they
capture identity \todo{delve more into this, and reference the other paper
  that points this out.}

\subsection{Related Primitives}

\begin{itemize}
\item Anonymous Credentials. \cite{bcc04} builds on Camenisch-Lysyanskaya
  signatures.
\item \cite{cms08b} mentions two schemes that are based on two different
  variants of CL signatures (one uses pairings, one does not).
\end{itemize}

\subsection{Applications}
\label{ssec:daaapplication}

\subsubsection{Theory}
\label{sssec:daatheoryapp}

\subsubsection{Real World Deployments}
\label{sssec:daarwdeploy}

\subsection{TODO}

\cite{cdl16,cdl16b,ccd+17,cdl17,cu15}.

Pairing-based DAA: see references in page 2 of \cite{bfg+11}.

%%% Local Variables:
%%% mode: latex
%%% TeX-master: "sok-privsig"
%%% End:

\section{Other Related Works}
\label{sec:other}

\comment{This section is just a placeholder for now...}

%\doubt{Functional Signatures? \cite{bgi14}}
\doubt{PAKE? Is privacy an inherent concern in PAKE?}
%\doubt{W3C DIDs and VCs?}
%$\doubt{Blind Signatures?}
%\doubt{k-times anonymity?}
\doubt{PEREA, BLAC, Nymble, etc.?}

%%% Local Variables:
%%% mode: latex
%%% TeX-master: "sok-privsig"
%%% End:

\section{Discussion}
\label{sec:discussion}

From a theoretical point of view, it seems very reasonable to derive all
keys, even for different purposes (payments, staking, or identities) from
the same mnemonic. However, there are some aspects that may need careful
consideration, as they may have impact in the overall security and privacy
properties of the wallet (and related systems).

\paragraph{Domain separation when different cryptosystems are
  necessary.} %
As mentioned, Cardano uses EdDSA (i.e., it is based on Ed25519 curve),
while Atala uses ECDSA with Secp256k1. It is \emph{not} a good idea%
\footnote{I have not been able to find a concrete paper that describes
  some concrete related attack or gives some impossibility result for
  proving security under this circumstance but, at the least, it seems to
  be folklore knowledge. See Lindell's answer at
  \url{https://crypto.stackexchange.com/a/54666/52362} for instance.
  \cite{dlp12+,thorm21} seem good references to study this topic further, if
  needed.} to use the same key for different cryptosystems, nor for different
curves, even though ``structurally'' it may be possible (e.g., in EdDSA and
ECDSA, private keys are 32-byte random numbers). However, this is easily
avoidable by ensuring domain separation in the key derivation functions that
are applied on the common seed. Since this would be handled by the 

\paragraph{Ensure a correct hierarchical key derivation strategy.} %
We need to take into account what specific usage we expect from the wallets
and, from there, define hierarchical derivation rules that ensure a correct
separation. For instance, related to the previous paragraph, we should make
sure that keys that are aimed to be used in different cryptosystems, are
derived in tree branches that ensure domain separation (just using a
different \texttt{purpose} branch, or \texttt{coin type} branch, would probably
be enough). Similarly, one needs to consider if specific subtrees are expected
to be used in an standalone manner, even probably by different (but related)
entities. For instance, different faculties of a university may need to maintain
their own subrees for the sake of efficiency (i.e., each one independently
deriving keys as needed). But then, compromises to one of them should not
affect the other. Again, this can be easily achieved using hardened
derivation in BIP44 slang (e.g., at the \texttt{account} level). From this
point of view, a formalization along the lines of \cite{def+21} would be
desirable, but the one given there is probably too strict (as, for instance,
it assumes that hardened nodes are leaves of the tree, and secret key
compromise of non-hardened nodes, is not possible.)

\paragraph{Malleability considerations when combining multiple
  attributes into one address.} %
Are the keys for the different purposes going to be used \emph{only} in a
standalone manner? Or, rather, it can be expected that they are somehow
encoded and distributed jointly? The latter seems to be the case for PoS
wallets, which are expected to be used in the Cardano ecosystem. As pointed
out by \cite{kkl20}, naive encoding of keys with different purposes, into
a single address, may enable certain (malleability) attacks.

\paragraph{Careful consideration of extra security properties.} %
The main related work analysed so far\footnote{BIP32, BIP44, \cite{def+21}
  and, partially, \cite{kkl20}.} analyses the security of either BIP44
wallets, or PoS wallets. The introduction of identity-related (frequently
associated to actual people) data may require additional properties, only
mentioned in these works (or not considered, as in the BIPs). For instance,
some sort of forward security will be very desirable. While, for the case of
payments or staking, the impact of a compromise (either of a leaf node --
i.e., a key -- or a complete subtree) can be mitigated by prompt detection
and transfer of funds to an uncompromised address, this may be harder for
keys representing in some way real world identities. Most probably, the
latter will frequently be long-lived keys, that also can, in turn, be used
to issue further identities (e.g., as in the case of identity providers).
Thus, the consequences of a compromise may be more cumbersome to address
(re-issuing credentials, distributing revocation lists, etc.) and,
consequently, it seems desirable to look for constructions that give some
sort of guarantee about the security of keys produced at time $t' < t$, if a
compromise happens at time $t$.

Also, it cannot be discarded that further desirable privacy or security
properties arise, as we get more familiar with the targetted use cases.
  
\subsection{Further related work}

There seems to be a large body of related research, which is specially
growing lately (probably, due to the rise of cryptocurrencies). Here,
I just include some of the main references I've come across for
self-bookkeeping (besides \cite{kkl20} and \cite{def+21}, already
mentioned in earlier sections). Note that they, in turn, include references
to further related work.

\begin{itemize}
\item ``Arcula: A Secure Hierarchical Deterministic Wallet for Multi-asset
  Blockchains'', \cite{lfa20}.
\item ``Simple, Efficient and Strongly KI-Secure Hierarchical Key Assignment
  Schemes'', \cite{fpp13}.
\item ``A Formal Treatment of Deterministic Wallets'', \cite{dfl19}.
\end{itemize}

%%% Local Variables:
%%% mode: latex
%%% TeX-master: "single-mnemonic"
%%% End:

\section{Conclusion}
\label{sec:conclusion}

\comment{This section is just a placeholder for now...}

%%% Local Variables:
%%% mode: latex
%%% TeX-master: "uas"
%%% End:


\section{TODOs}

General/pending doubts and tasks:

\begin{itemize}
\item Many constructions of ACs and GSs seem to be built using the same
  approach: an issuer and a user interact so that the latter receives a signature
  from the former; then the user employs that signature to do something else (in
  ACs, prove attributes; in GSs, produce further signatures). Still, the typical
  security models are different. Concretely, GSs have a very well established
  model, but ACs don't. It also ``bothers'' me that none of the AC models I've
  seen cover something similar to non-frameability, nor traceability, but just
  include conventional unforgeability properties. Why is this?
  \begin{itemize}
  \item Concerning non-frameability, this probably is because the de-anonymizing
    functionality that ACs include (if any) is revocation, which is essentially
    a blacklisting. On the other hand, GSs usually allow opening, which extracts
    the actual identity. Opening is ``stronger'' than blacklisting, in the sense
    that it can be used to build blacklisting, but it is also more threatening
    to privacy and may have further implications. Among these, unequivocable
    framing of users. This, in general, does not seem possible with conventional
    blacklisting (or, at least, not in an unequivocable way).
  \item Concerning traceability.
  \end{itemize}
\end{itemize}

\bibliographystyle{splncs04}
\bibliography{soa-review}

\end{document}

%%% Local Variables:
%%% mode: latex
%%% TeX-master: ucl.tex
%%% End:
