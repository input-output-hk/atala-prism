\section{Log on Review of Group Signatures schemes}
\label{app:ac}

\subsection{TODOs}

\begin{itemize}
\item ABGS line of work: \cite{khad07a,khad07b}...
\item Fully anonymous dynamic GS from SPS-EQ \cite{ds16}
\item Multi-group group signature schemes. Compare with ACs with multiple
  issuers.
\end{itemize}

\subsubsection{Further Notes}
\note{Integrate this section or delete it.}

\todo{CCA2 type of anonymity is defined in \cite{bbs04}. Schemes with CPA type
  of anonymity are (according to \cite{ehk+19}): \cite{bw06,bw07}.}

\cite{ehk+19} has a nice related work mentioning the assumptions that are
typically used.

Signature highjacking attacks (referenced in \cite{ehk+19}).

If we want to have traceability, the issuer needs to be honest. See what
implications this has in practice (e.g., what architectures are compatible
with this?) This probably makes schemes with separate issuing and opening
authorities more attractive in practice.

Other minor variants: group blind signatures, democratic group signatures,
mediated group signatures. All referenced in \cite[Section 1.3.5]{bsi12}.

Backward linkability property of group signatures with Timed Verifier-Local
Revocation, proposed in \cite{nf05}. Mentioned in \cite[Section 8.1.2]{bsi12}.

Relation of forward security with proactive security \cite{oy91}?

References 1, 24, 32 and, probably, 35 in \cite{bfg+11} seem to be (traceable)
group signatures that use the nym approach (or similar) for linking. Check.

\paragraph{Academic indicators.}
IACR, Esorics, CCS, AsiaCCS, SP, EuroSP, Usenix Sec and Privacy, PETs... Not
sure where to fit this.

Main cryptographic building blocks:

\begin{itemize}
\item BB signatures \cite{bb04}.
  \begin{itemize}
  \item Used in: \cite{ky05}.
  \item Paillier encryption of discrete-logarithms? Used in \cite{ky05} and,
    I think, \cite{gl19}.    
  \end{itemize}
\item Model for reduction: list of group signature schemes based on RO and
  non-RO are given in \cite{bcc+16}.
\end{itemize}
