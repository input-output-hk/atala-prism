\section{Anonymous Credentials}
\label{sec:ac}

First proposed in \cite{chau85}.
CL signatures (idemix) \cite{cl01,cl04}.
UC formalisation in \cite{cdhk15}.
Brands works (U-prove).
Coconut paper, referenced in \cite[Section 1]{cdl+20}: anonymous credentials with
threshold opening.

\begin{enumerate}
\item Main intuition.
\item Evolution of main security models.
\item Variations concerning functionality.
\item Evolution in efficiency.
\item Main issues.
\item Approx. number of related works in main crypto+security venues: IACR,
  Esorics, CCS, AsiaCCS, SP, EuroSP, Usenix Sec and Privacy, PETs...
\end{enumerate}

Anonymous Credentials (ACs) were first proposed by Chaum \cite{chau85}, and
later formally treated and optimized in a series of papers by Camenisch and
Lysyanskaya \cite{cl01,cl02,cl04}. Roughly speaking, users receive credentials
from an issuing organization, which can then use to authenticate against some
verifier. The main property is that this authentication process can be done
in a privacy-preserving way. This typically means that the verifier cannot
link different authentication processes by that same user. And, usually, in
credentials composed by multiple attributes, that the verifier only learns
the attributes that the user is willing to share (or a function of them).

\subsection{Approaches, Models, and Variations}

\paragraph{Main approaches.} %

\begin{enumerate}
\item[\cite{cl01}:]
  \begin{itemize}
  \item Functionality: FormNym, GrantCred, VerifyCred, VerifyCredOnNym.
  \item Approach: Users create a pseudonym with each organization. For this,
    they have to register first with the organization (associating a key to
    a user name). Whenever the user wants to get a credential from a organization
    in which he registered, he has to authenticate, and then the organization
    decides whether or not to grant the credential. Users can show their
    credentials to organizations in two ways: (1) just proving that they own a
    credential issued by some organization; or (2), proving that they own a
    credential issued by some organization \emph{and} this credential is over
    the same pseudonym that the user holds with the verifying organization.
    Essentially, pseudonyms are the concatenation of two random numbers (one
    provided by the user, and the other by the organization). They get
    associated an authenticating tag, which is a statistically hiding commitment
    to the user secret (computed using randomness provided by both user and
    organization). Organizations keep a database of the generated pseudonyms.
    To get a credential from an organization where the user registered
    previously, the user proves knowledge of the secret involved in the
    pseudonym. The organization then creates a ``random signature'' (by using a
    the inverse of random exponent, with the help of the factorization of an RSA
    modulus chosen by the organization at setup time) over the pseudonym, and
    gives that to the user. Users can show the credentials (in any of the
    previously stated two ways) by creating non-interactive zero-knowledge
    proofs of knowledge of the appropriate secret. Note that, in case (2),
    this means that the proof is over two pseudonyms, which must share the
    same secret.    
  \end{itemize}
\item[\cite{cl02}:]
  \begin{itemize}
  \item Approach: Te paper aims at improving efficiency of AC systems, as
    well as still ensuring that the user can retrieve their credentials in
    a privacy-preserving manner. For this purpose, the core of the paper
    describes a signature scheme that allows signing blocks of messages,
    and later prove, in zero-knowledge, knowledge of such a signature. To
    enable private retrieval of credentials, the authors describe a protocol
    for their signature scheme, whereby the user specifices a commitment to
    a message (or a block of messages, although this is not actually
    formalized), and get a signature of the message in return. While the
    authors do not strictly define an AC system, the application of their
    signature system and related protocols to the AC domain seems direct.
    Indeed, the same functionality as in \cite{cl01} seems straight
    forward given the cryptosystems in \cite{cl02}.
  \end{itemize}
\item[\cite{cl04}:] Follows a similar approach to \cite{cl02}, although
  changing the security assumptions to be based only on problems related
  to the discrete logarithm, and on bilinear maps. Basically, defines a
  signature scheme suitable for signing multiple messages, a protocol to
  obtain a signature on a committed value, and a protocol to prove knowledge
  of a signature. Concerning commitments, they rely on Pedersen commitment
  scheme, for which protocols are known to prove knowledge and equality of
  committed values. This lets them follow the same approach to \cite{cl02},
  which is in turn the one in \cite{cl01}.
\item[\cite{bcc+09}:] This is mostly a theoretical work, which is relevant
  since it is (one of) the first to propose a delegatable anonymous
  credentials systems. The building blocks are randomizable proofs of
  knowledge that randomize both the proof and the statement (which the authors
  build on top of Groth-Sahai proofs), and message authenticators (a variant
  of signatures that require a special notion of unforgeability). Each user
  can create pseudonyms, which are commitments to his secret key, and use
  these commitments to prove ownership of and delegate credentials. These
  pseudonyms are essentially as in the CL line of work (user with pseudonym
  $N^A_U$ w.r.t. organization $A$ and $^B_U$ w.r.t. orgainzation $B$, can
  prove to $B$ (resp. $A$) that $N^A_U$ and $N^B_U$ are related to the
  same secret key). Delegations here follow a sequential order: i.e., if
  a credential has level $L$, then its owner can further delegate this
  credential, and the result would be a credential with level $L+1$. To
  delegate, essentially, the delegator (with pseudonym $N_I$) and delegatee
  (with pseudonym $N_D$) perform an interactive protocol
  through which the delegatee receives a randomizable NIZKP that lets him
  prove knowledge of an authenticator over $N_D$, issued by $N_I$. Since
  both the proof and the statement can be randomized, verifiers do not
  learn the chain of delegation.  Although this is informally mentioned,
  ``attributes'' can be added to the delegation process too (besides to the
  credentials themselves). This is suggested by adding the attributes -- which
  would be visible to the verifier -- to the application of a hash function
  used to derive the authenticator). 
\item[\cite{cks10}:] The system proposed here follows the conventional
  approach to anonymous credentials based on randomizable signatures (with
  the possibility to sign committed values) over vectors of messages.
  It however enables issuers to produce updates on already signed messages.
  This updates can be leveraged to incorporate revocation. Essentially,
  time is divided in epochs, and credentials include an attribute which
  represents the expiration date/epoch of the credential. Users have to
  prove that their credential is valid for the current epoch. Similarly,
  to add new attributes, the issuer leverages the internal structure of
  the signatures over vectors of messages, to basically add new messages
  (attributes). In both cases, it is responsibility of the user to update
  her credential given the latest update by the issuer. An example
  instantiation is given using BBS signatures.
\item[\cite{cmz14}:] This work differs from the rest in that they use
  secret key cryptography rather than public key. Namely, they assume a
  setting in which the issuer will also be the verifier (e.g., a
  transportation organization that issues tickets for its travelers). In
  this setting, it is reasonable to assume that both issuer and verifier
  will have access to the same secret. Therefore, they replace signatures
  with (algebraic) MACs. They do not create pseudonyms for end users (like
  \cite{sms+19}). Other than that, the supported functionality is similar
  to the rest: Setup and KeyGen to initialize the cryptographic parameters,
  and produce the issuer's keys, respectively; an interactive protocol for
  end users to blindly receive credentials (MACs) on their attributes; and
  an interactive protocol for showing and verifying a credential. In case
  that issuance does not need to be blinded, it becomes a non-interactive
  protocol (i.e., the issuer just produces the credential and sends it to
  the user). Otherwise, the user has to encrypt the attributes and prove
  they are correct. Presentation only requires one message from the user
  to the verifier.
\item[\cite{cl11}:] The approach differs from the conventional ones. In this
  case, the focus is on settings where users may want to combine multiple
  credentials from different issuers. Thus, interest is on reducing the size
  of the resulting credentials, and the efficiency of associated processes.
  The approach is based on aggregate signatures. As usual, each credential
  may contain several attributes. Each attribute is independently signed
  by the issuer. Then, users can combine an arbitrary number of signed
  attributes (by the same or different issuers) into a single aggregate
  signature. Verification only needs the aggregated signature and the set
  of public keys of all the involved issuers. To prevent different users
  from aggregating their credentials, the authors resort to \emph{indexed}
  aggregate signatures: an index is added to each signature, such that only
  signatures that have the same index can be aggregated. In their instantiation,
  the signature is a BGLS signature \cite{bgls03}, and the index is a BB
  signature \cite{bb04}. The actual instantiation seems impractical (too many
  linear costs, see the table in the paper).
\item[\cite{cklm14}:] This work is, in essence, a generalization of \cite{bcc+09}.
  The main novel building block are controlled-malleability NIZKs. Basically,
  these are NIZKs that can be re-randomized, but only for a pre-defined set of
  (NP) transformations. From this cm-NIZKs, malleable signatures are built.
  From malleable signatures and commitments, delegatable credentials are built
  as follows. Credentials are basically a tuple $(nym, l, flag)$, where $nym$
  is a commitment to the user's private key, $l$ is the length of the credential
  in the delegation chain, and $flag$ is either $cred$ or $proof$. Credentials
  are accompanied with a malleable signature (which, for level 1 credentials,
  it is a signature by the issuer). To create ``proofs'' or to delegate, the
  credential owner has to apply one of the two allowed transformations. For
  ``proofs'', basically the transformation that has to be applied to the original
  proof (malleable signature) converts it to a proof showing that the prover
  knows the secret in the ``source'' pseudonym and ``target'' pseudonym, and
  both are the same. In the case of delegations, flag is set to ``cred'', and
  the transformation converts the cm-NIZK proof into one that lets the prover
  prove knowledge of the secret key in the ``source'' pseudonym, and allows
  only to produce a new credential with structure $(nym', l+1, cred)$ -- i.e.,
  from a credential of level $l$, only a credential of level $l+1$ can be
  derived. On the delegation topic, the type of delegation this work ``supports'',
  at least by default, is the simple one of hierarchical delegation.
\item[\cite{cdhk15}:] The approach to build anonymous credentials is roughly
  the same as in the previous works \cite{cl01,cl02,cl04}, although with
  relevant differences that make it more efficient. Basically, users create
  pseudonyms derived from a common secret. Then, any user can request a
  credential. For this, the user needs to prove knowledge of the pseudonym's
  secret. In return, the issuer produces an ``unlinkable redactable signature''
  on a set of attributes, for that pseudonym. A redactable signature is
  essentially a signature over a set (block) of messages, that enables
  subsequent verification that a subset of those messages was signed. This
  is essentially the same as in \cite{cl02,cl04}. However, to achieve
  unlinkability, \cite{cdhk15} incorporates efficient zero-knowledge proofs
  to prove correctness of the subset of the messages that a user wants to
  ``show'' in a credential showing process. This ensures unlinkability (hence,
  privacy). The internal scheme used for this is a vector commitment scheme
  that uses Lagrange interpolation to allow checking that a subset of the
  committed messages are indeed contained within the commitment. Moreover,
  both the vector commitment scheme, and the vector signing scheme, are
  structure preserving. This means that, since all the produced messages,
  commitments, and signatures, are elements of the group (rather than of the
  field), proving knowledge is much more efficient (e.g., it prevents costly
  extraction required for security proofs).
\item[\cite{ckl+15}:] Follows the conventional approach: a secure commitment
  scheme with proofs of knowledge, and a signature scheme that allows to
  sign vectors of messages, including commitments. Here, as in the seminal
  works by Camenisch and Lysyanskaya \cite{cl01,cl02,cl04}, pseudonyms are
  used as explicit building blocks. \todo{Not sure yet of how they are
    made explicit in the generic construction. Clarify.}
  A ``syntactic'' variation: they introduce
  the concept of tokens both for credential issuance and presentation. To
  request issuance of a credential, users have to produce an issuance token
  from the set of attributes they want to reveal to the issuer, and the set
  of attributes they want to hide (potentially coming from another credential
  already owned by the user). An equivalent description applies for presentation
  of a credential to a verifier. In both cases, the issuer/verifier has to
  check the received token prior to issuing a credential or granting access.
  They also add revocation functionality. For this, a revocation authority
  (not necessarily the issuer) keeps a revocation information list, which is
  basically a list containing no longer valid attributes. In order to prove
  that their credentials are valid w.r.t. some attribute, users have to generate
  a (non) revocation token, which proves that the attribute has not been
  included in the revocation information list.
\item[\cite{dmm+18}:] Follows a completely different approach to (most) other
  works. Instead of using randomizable signatures (over committed values) or
  some equivalent building block, it relies on predicate encryption.
  Essentially, a user with attributes $A$ is given a decryption key for a
  predicate $f_A(g)$ which, given a predicate $g$, returns $g(A)$. When
  a verifier wants to check if a user has a set of attributes that meet
  certain policy $h$, challenges the user to decrypt a ciphertext over
  a random message that has been associated to predicate $h$. Then, only
  if the attribute set $A$ of the user meets policy $h$, the user will be
  able to decrypt the random message (since $f_A(h) = h(A) = 1$). This
  protocol is actually augmented with some commitments, to prevent
  misbehaviour from user and from verifier, but the approach is the same.
  \doubt{There is something weird here, or that I do not understand --
    predicate encryption requires a
    set of attributes for encryption, rather than another predicate. I have
    reached out to the authors to see if they can help me clarify this.}
\item[\cite{fhs19}:] Based on Structure Preserving Signatures from equivalence
  classes (SPS-EQ). The approach resembles that of \cite{ckl+15} in that it
  uses a structure preserving primitive, and signs (commitments of) sets of
  messages. However, they aim to avoid (as they claim) costly zero-knowledge
  proofs. Basically, upon setup, a set commitment scheme is set up. Each user
  generates a keypair. To obtain a credential on a set of attributes $A$, the
  user proves knowledge of the user secret key, which is used as randomness
  for a set commitment over $A$. The issuer (proves knowledge of the secret
  key and) returns a SPS-EQ signature over $A$. To show the credential, the
  user randomizes the commitment set and the signature, reveals the required
  subset $D$ of $A$, and computes a witness that indeed $D \subset A$, as well
  as a proof knowledge of the randomness used to randomize the credential (to
  prevent replays). 
\item[\cite{sms+19}:] The Coconut system follows a similar approach than the previous
  works with respect to the representation and showing protocol for credentials.
  Namely, credentials are randomizable signatures issued blindly. In this case,
  the authors rely on Pointcheval Sanders signatures \cite{ps16} as main
  building block, to which they incorporate techniques from other schemes to
  support threshold issuance.
\item[\cite{halp20}:] Builds on Coconut credentials and a blockchain -- in theory,
  any blockchain with smart-contract capability, although it seems they are building
  their own blockchain. This paper is in the frontier of academic and industry
  work. On the cryptographic side, the credentials they build are essentially an
  extension of Coconut credentials (although the extension is not described, nor
  formalized). Users can combine multiple Coconut credentials into one Nym
  Credential. In order to control Sybil attacks, Nym Credentials are obtained
  in exchange of Nym Tokens (how users get hold of these, is not described).
  Service Providers receive Nym Tokens in exchange of offering their services
  through the network.
\item[\cite{hs21}:] This work merges AC approaches based on SPS-EQ and aggregate
  signatures, and also introduces the separation between a trusted but restricted
  device, and a less trusted but powerful device typical in DAA systems. The
  former is referred to as \emph{core}, and the latter as \emph{helper}, hence
  the Core-Helper AC (CHAC) acronym. In addition to the previous basic building
  blocks, the system uses a Signature with Flexible Public Keys (SFPK) scheme.
  Roughly, these are signature schemes where the public keys can be randomized,
  but in a manner that no adversary can decide whether two public keys are
  related or not (i.e., if one is a randomization of the other). Users identities
  are SFPK keypairs, where the public part gets signed by the issuer, using an
  AAEQ scheme (Aggregatable Attribute-based Signature Scheme on Equivalence Classes).
  Roughly, an AAEQ scheme is a combination of attribute-based signatures on
  equivalence classes (which the authors combine with Tag-based Signatures on
  Equivalence Classes), and aggregate signatures. In the proposed AAEQ scheme,
  a tag is derived from an attribute an a nonce, which is then ``signed'' (note:
  this tag resembles much the concept of nyms based on scopes). Since signatures
  (and tags) are group elements sharing the same discrete log (the signer's key),
  they can be aggregated easily via multiplication. Hence, for each attribute that
  the issuer wants to grant to the user, it creates an independent signature where
  the tag depends on the attribute to grant. Then, the user can just aggregate
  all of them into one credential. In addition to this, the authors split the
  operations (which are in the bilinear setting), in a way that requires the
  core to perform only a few operations in $G_1$. The result seems to be quite
  efficient. However, it only supports selective disclosure (no advanced
  privacy-preserving policies), and even revocation is not considered from scratch,
  although the authors sketch an approach to integrate it -- which probably
  would require a more formal analysis and would incur in (possibly bad?)
  added costs.
\end{enumerate}

\paragraph{Main security models.}

\begin{enumerate}
\item[\cite{cl01}:] Security in \cite{cl01} is proven in an ad-hoc way,
  using the ideal-world/real-world paradigm. The authors describe a
  simulator for the operations supported by the system (Setup, FormNym,
  GrantCred, VerifyCred and VerifyCredOnNym), and prove that the output of this
  simulator is indistinguishable from that of the real protocol, under the
  strong RSA and Diffie-Hellman assumptions.  
\item[\cite{cl02}:] No security model is given for the anonymous credential
  application of the signature scheme (reasonably, as it is not the main focus
  of the paper). The signature scheme is proven secure (no specific mention
  to a security model, although I suppose it would be quite straight forward)
  based on the Strong RSA and Diffie-Hellman assumptions.
\item[\cite{cl04}:] Same as in \cite{cl02} (no actual security model for the
  AC system). The custom proofs are based on the DDH and LRSW assumptions.
\item[\cite{bcc+09}:] The article follows a game-based definition approach,
  with two security properties (besides correctness): anonymity and
  unforgeability, which are the usual, but extended to the scenario with
  delegation. Basically, the anonymity (in the ideal/real world paradigm)
  property requires that the adversary should not distinguish real credentials
  from simulated ones. Moreover, the adversary should not learn the delegation
  chain of an honest credential (even if it was issued by himself). For
  unforgeability, the requirement is that the adversary should not create
  a delegation chain that includes a delegation by an honest user, if the
  user did not actually perform that delegation.
\item[\cite{cks10}:] Signer privacy and user privacy properties are required
  for the underlying signature scheme (thus, inherited by the credential system,
  although I have not found explicit model). Signer privacy captures that the
  adversary (including corrupt users) doesn't learn any information about the
  signer from signatures and updates of signatures. User privacy captures that
  the adversary (including a corrupt issuer) does not learn any information
  about the blindly signed messages. \doubt{Not sure I get the utility of
    signer privacy. The original property seems to come from \cite{bckl08}.}
\item[\cite{cl11}:] The security properties in this work are unforgeability
  and anonymity, for which the authors give game-based definitions.
\item[\cite{cklm14}:] Defines security of the DAC scheme using simulation-based
  notions of anonymity and unforgeability.
\item[\cite{cmz14}:] The security properties proven in this work are:
  \begin{itemize}
  \item \emph{Unforgeability}: The adversary cannot produce a valid proof for a
    statement, if none of the attribute sets for which it has received a
    credential meets that statement.
  \item \emph{Anonymity}: The proofs produced in a presentation do not leak
    more information than the statement being proved.
  \item \emph{Blind issuance}: The issuance protocol is a secure two-party
    protocol for generating credentials on the user's attributes.
  \item \emph{Key-parameter consistency}: no issuer can use different parameters
    to produce credentials of different users such that their anonymity can
    be compromised afterwards. \doubt{This somehow remembers non-frameability in
      group signatures.}
  \end{itemize}
\item[\cite{cdhk15}:] The model (and associated proofs) presented in this paper
  is very comprehensive, and detailed. Specifically, for the main building block
  (Unlinkable Redactable Signatures), the authors give both game-based
  definitions as well as an ideal functionality. This ideal functionality models
  security in a single-issuer setting. However, as it is done in the UC
  framework, it can be composed with itself, so that a scenario with multiple
  issuers, in which users can combine credentials obtained by different users
  into a single showing process (``presentation'') is directly achievable by
  composing multiple instances of the ideal functionality (one per ``proof
  part'').
\item[\cite{ckl+15}:] The main security and privacy properties required are
  pseudonym collision-resistance (two different user secret keys cannot produce
  the same nym for the same scope), unforgeability (the adversary cannot forge
  issuance tokens or presentation tokens), and privacy (no two presentations by
  the same user can be linked) or weak privacy (presentations by the same user
  can be linked, but not traced to an issuance session). Privacy is modelled
  in the simulation paradigm, the difference in the modelling between privacy
  and weak privacy being a filter that removes some information \note{(I find
    this kind of weird; probably I'm wrong. Would be nice to study further)}.
  The unforgeability and privacy properties are essentially inherited from the
  main building block, privacy-enhancing attribute-based signatures. Although
  there, the authors begin by defining a blind issuance experiment (ensuring
  that the issuer does not learn anything more than what it should during
  isusance), and then define two variants for user privacy: presentation privacy
  and untraceability, which basically match the strong and weak notions for
  PABCs (in untraceability, two presentations by the same user can be linked,
  but not trace to a specific credential issuance). [Strong] user privacy has
  to satisfy both blind issuance and presentation privacy; weak user privacy
  has to satisfy both blind issuance and untraceability. I find this model
  somehow interesting, as it is one of the few (maybe along with \cite{cmz14})
  that includes some notion related to tracing back to issuance of credentials,
  as in group signatures). In both cases (PABC and PABS), [strong] privacy
  implies weak privacy.
  Essentially, the security properties required by the PABC scheme, are
  ensured by the building blocks. Namely: the pseudonym system (requiring
  key extractability, collision resistance and pseudonym unlinkability); the
  revocation scheme (requiring soundness and revocation privacy); the
  PABs (requiring unforgeability, and [weak] privacy); and the commitment
  scheme (requiring blinding and hiding, and opening extractability).
\item[\cite{dmm+18}:] The authors give a game-based security model, covering
  properties of unforgeability, anonymity, and policy-hiding. The latter
  means that, for delegated verification, the verifier is oblivious to the
  policy being verified.
\item[\cite{fhs19}:] Gives a game-based security model, with anonymity and
  unforgeability being the main security properties (besides correctness).
  Unforgeability resembles the traceability property of group signatures,
  which are recognised as an influence in their modelling -- more specifically,
  \cite{bmw03} (the model -- and the oracles therein -- definitely resembles
  the BMW model for GSs). The authors claim to be the firsts to give a well defined and
  formal model for ACs, besides \cite{ckl+15} (which gives game-based properties
  in the simulation paradigm) and \cite{cdhk15} (which follows the UC framework).
  However, \cite{fhs19} claims that adopting non-simulation based definitions
  allows them to reach a more efficient construction. They also support malicious
  issuer key generation, as issuers have to prove knowledge of their secret keys
  (which allows extraction in the games).
  An interesting aspect of their model is that all attributes in their
  credentials are learned by the issuer -- i.e., there are no blindly signed
  attributes. This implies that it is not possible to sign secret keys owned
  by the users (as in the conventional CL credentials). Then, for showing a
  credential, all the user has to do is reveal the attributes s/he is required
  to show. This has the disadvantage that, depending on the use case, this may
  be trivial to attack. For instance, assume a COVID-like pandemic situation,
  in which citizens have to prove being vaccinated. An AC system for this use
  could be naively defined as: { Type of vaccine, Date of vaccination }. But
  this is very susceptible to dictionary attacks, and leaves much to the
  engineering/implementation level. E.g., if it is easy for an eavesdroper
  to get hold of a credential (not even the plaintext attributes), s/he can
  just iterate through the small list of possible values, and then take
  advantage of the ``stolen'' credential. This is certainly secure under the
  model defined in \cite{fhs19}, but it is definitely easy for implementations
  following this model to go wrong...
  They also model only interactive show processes. Furthermore, to prevent
  replay attacks, the user has to prove knowledge of some randomness included
  in the credential. However (\textbf{cross-check this!}), since signatures
  are randomizable without knowledge of any secret value, any adversary that
  learns a credential (not necessarily any plaintext attribute associated to
  it), would be capable of authenticating through a dictionary attack in case
  of small attribute ranges.
  Note: this may just be a modelling issue (or not even that, cross-check later
  in case I misunderstood something). In their construction, they include
  a secret key of the user in the credentials, of which the user has to prove
  knowledge during the issue protocol. They do this by ``reusing'' the user
  secret key as the randomness employed for the set commitment scheme, and
  requiring the user to prove knowledge of the corresponding public key.
  However, I am not sure of whether or not this somehow thwarts the previously
  mentioned ``attack'', which affects the show protocol (\todo{i.e., the user does
    not need to prove knowledge of the usk during show, no?} \note{Actually, it
    should not, as it would probably prevent anonymity}) In any case, even if
  the specific construction and instantiation thwarts this attack, I think their
  model allows it.
\item[\cite{sms+19}:] No security model is given nor adopted.
\item[\cite{halp20}:] No formal definition is given. The following properties
  are claimed: undetectability (this seems similar to traceability in group sigs),
  unlinkability, selective attributes disclosure, unforgeability, persistent
  accounts, sybil resistance, extensibility, availability, decentralization.
\item[\cite{hs21}:] Security is given through game-based definitions of
  unforgeability, anonymity, and dependability. The first two are the usual ones,
  while the latter roughly requires that an adversarial helper cannot produce
  successful showings without cooperation from the core. It would be interesting
  to make a detailed comparison with security models of DAA (usually in the UC
  framework, although some game-based (seemingly flawed) definitions exist).
  The main argument that the authors give when comparing to DAA, is that the
  latter does not support attributes, and it is essentially a group signature
  scheme without revocation and optional linkability (this frontier is also
  blurry, see \cite{dl21}).
\end{enumerate}

\paragraph{Variations in functionality.}

\begin{enumerate}
\item[\cite{cl01}:] Extensions provided to the main functionality are:
  \begin{itemize}
  \item PKI-based and all-or-nothing non-transferability: Both options
    involve sharing a verifiable encryption of the user secrets. In the
    case of PKI-based non-transferability, the CA receives a verifiable
    encryption, using as key the user's master secret, of some valuable
    piece of information belonging to the user (e.g., the user's secret
    key associated to his public key with that CA). In the all-or-nothing
    case, the user shares with each organization (with which it has a
    pseudonym), a verifiable encryption of his pseudonym secret data, encrypted
    with his master secret. In both cases, if the user shares his master
    secret with someone else, then he automatically gives access to
    other relevant pieces of information (as the verifiable encryptions are
    made public by the corresponding entity).    
  \item One-show credentials: The validating tags over pseudonyms are extended
    to include an extra value in their commitment. Then, users have to reveal
    that value (or, rather, a value deterministically derived from that) in the
    zero-knowledge proofs they create at show time. Since the value is always
    the same, verifiers can check whether they have seen it before, or not.
  \item Revocation: Both local (i.e., within an organization) and global (i.e.,
    referring to an identity of the user in an external CA) is possible by
    extending the pseudonyms of the user, and the credential showing protocols,
    with values that allow subsequent decryption by a trusted party of their
    pseudonym or external identity, given a transcript of the credential
    showing.    
  \item Attributes: The paper also succinctly describes how multiple attributes
    per credential could be included. This would be done by dividing the inverval
    within the exponent used to compute the pseudonym in sub-intervals, and then
    proving that the exponent lies within a sub-interval.
  \end{itemize}
\item[\cite{cmz14}:] The paper mentions ``Credential translation'', which
  basically means being able to prove that two attributes in different
  credentials are actually the same. This is not formalized or detailed,
  though.
\item[\cite{cdhk15}:] The same as in the previous works (\cite{cl01,cl02,cl04}).
\item[\cite{dmm+18}:] The scheme supports delegated verification. Basically,
  to implement delegated verification, the delegatee receives an encryption of
  the policy it has to verify, plus a signature of the encryption. If the
  signature is valid, it re-randomizes the ciphertext for the policy, and
  sends it to the user, who follows the same protocol as in the normal case.
  The paper also mentions how to add more functionality, like delegation of
  credentials and revocation, but they are not modelled. Also interestingly,
  they claim somewhat informally (in Section 2.3) that no scheme satisfying
  anonymity in the presence of an untrustworthy issuer can support delegation
  of verification with policy hiding.
\item[\cite{sms+19}:] It is essentially the same as previous works, although it
  supports threshold issuance. Interestingly, \cite{sms+19} does not make use
  of pseudonyms. I.e., credentials are just signed (blocks of) attributes. This
  probably lets them achieve better efficiency. \doubt{This makes the scheme
    to be practically like a group signature scheme (without revocation...)}
\item[\cite{hs21}:] The main variation here is the application of the core-helper
  division (typical of DAA) to the AC domain.
\end{enumerate}

\paragraph{Efficiency.}

\cite{sms+19} seems to be the most efficient one (note: relies on the
Fiat-Shamir heuristic). Credentials are only composed by two G1 elements,
independently on the number of attributes. \todo{Cross-check this after
  review.}

\cite{cdhk15} and \cite{fhs19} achieve credential showings of size independent
of the number of attributes.

\paragraph{Main challenges.}

\subsection{Related Primitives}

Most schemes \cite{cl01,cl02,cl04,cdhk15,sms+19} are based on randomizable
signatures, with the possibility of signing committed values, that also support
efficient proofs of knowledge.

The most recurrent signature schemes are Camenisch-Lysyanskaya \cite{cl02,cl04},
BBS/BBS+ \cite{asm06,cdl16b}, and Pointcheval Sanders \cite{ps16}.
\cite{cdhk15} leverages their own signature scheme, called Unlinkable
Randomizable Signatures, which lets users obtain a signature on a set of
(committed) messages and then prove knowledge of (a subset of) these
messages, introducing rerandomization to prevent linking. \cite{fhs19}
follows a similar approach, using vector commitments. Both schemes
\cite{cdhk15,fhs19} seem to be the most efficient AC instantiations to date.
\cite{cdhk15} follows the UC framework; \cite{fhs19} a game-based definition.

The ``family'' of AC systems known in the industry as BBS+ originates from
\cite{asm06} (or, probably, from the BBS+ variant in \cite{cdl16b}, which
introduces several improvements). \cite{asm06} presents a GS-like system that
lets users obtain a credential from a Group Manager. This credential is a
BBS+ signature. BBS+ signatures are a digital signature scheme, based on
\cite{bbs04} and following the modification guidelines put forward in
\cite{cl04}\footnote{Essentially, the authors of \cite{asm06} acknowledge
  \cite{cl04} as those who proposed the BBS+ scheme -- but without security
  proofs, which are provided in (the full version of) \cite{asm06}.}. On top
of the signature scheme, they add efficient protocols for obtaining a signature
over a set of committed messages, and proving -- in zero knowledge -- knowledge
of a signature. Users can then be authorized by Application Providers (APs) to
access their services: for this, APs maintain a list of the users they have
granted/revoked access to, using a dynamic accumulator. APs also allow users to
authenticate anonymously up to $k$ times. In order to access the services of APs,
users have to derive a value using a VRF, and further prove that the value used
to seed the VRF is a number within $[1,k]$ -- in addition to, of course, proving
knowledge of a valid BBS+ signature (their credential). APs keep a record of the
accesses. Thus, if a user accesses more than $k$ times a given AP, anyone with
access to the record will be able to learn that a user has accessed more than
$k$ times, as there will be more than one matching VRF-derived values.
Furthermore, these two (or more) values let anyone extract a ``pseudonymous''
identity of the user. The model (only informally presented, formal definitions
appear in \cite{ts06} and proofs are deferred to a full version) seems to be
used in $k$-TAA systems ($k$ Times Anonymous Authentication), and defines
properties for:

\begin{itemize}
\item D-Detectability: Adversaries authenticating more than $k$ times are
  detected.  
\item D-Anonymity: Honest users authenticating less than $k$ times remain
  anonymous.
\item D-Exculpability: No honest user can be framed as having authenticated
  more than $k$ times. The GM cannot be framed as dishonest if that is not
  the case.
\end{itemize}

The system in \cite{aks12} also uses BBS+ signatures as a main building block
to design a system with accountable anonymity, although it is not ``strictly''
an anonymous credentials system: users receive a credential that lets them
authenticate anonymously against service providers, but this credential does
not include attributes, which seems to be the common case in ACs (however,
with BBS+ it is trivial to add attributes, although this should be modelled).
The main contribution of \cite{aks12} is the design and incorporation of
protocols to block misbehaving users depending on their behaviour, without
relying on trusted third parties. This is based on ``reputation lists'',
maintained by each service provider. Uses that have previously registered
with a Group Manager can authenticate against service providers by proving
knowledge of a valid credential. Depending on how these users behave during
the session with the service provider, the latter adds the interaction to
either a \emph{good behaviour} list, or a \emph{misbehaviour list}. For
subsequent authentications, users prove (in an ``almost unlinkable'' way
w.r.t. previous authentications) that they have a certain reputation,
computed as a weighted sum of their presence in the good/bad behaviour
lists. Moreover, each service provider can maintain multiple thematic lists,
and then require users to satisfy arbitrary boolean policies based on these
lists (e.g., reputation $\ge X_1$ on $list_1$ and reputation $\le X_2$ on
$list_2$ or reputation $= X_3$ on $list_3$) so that only users meeting
the defined policy can access their system. These proves have though a linear
dependency on the number of authentications that each user has done in
the past. To reduce the impact of this, the authors create epochs. When a
user authenticates successfully against a service provider, at a given epoch
$i$, the service provider gives the user a token that authenticates the
reputation that the user had at epoch $i$. Then, when this user tries to
authenticate again, he only needs to give that token, and prove his reputation
since epoch $i$ (this is why authentication is ``almost unlinkable'', as
service providers may learn when the user authenticated for the last time).
Note that no trusted party is involved in the blacklisting process, as
correctness of all tokens exchanged between users and service providers to prove
meeting certain policy, is ensured via the corresponing proof protocols.
Concerning security, the system is proven secure in the simulation paradigm,
claiming to meet (informally defined) properties of:

\begin{itemize}
\item Anonymity: All the SPs can infer from authenticating users is that they
  meet the predefined policy, and that they authenticated at a certain epoch.
\item Authenticity: Only users meeting the defined policies can authenticate.
\item Mis-authentication resistance: Un-registered users cannot authenticate
  successfully.
\item Non-frameability: No coalition of users can cause that another user
  meeting the required policy, cannot authenticate successfully.
\end{itemize}

Somehow, these properties resemble those in the group signatures model in
\cite{kty04}. Probably, the addition of non-frameability is a consequence
of adding a variant of revocation.

\paragraph{Primitives for credentials with generic policies.} \cite{cgm16}
describes techniques to be able to create efficient proofs that combine both
algebraic and non-algebraic relations. For instance, proving knowledge of an
ECDSA or RSA-FDH signature (both of which include computation of a hash
function); but also being able to prove any arbitrary (non-algebraic) funcion
$f$. This may be very useful to achieve compatibility with traditional
credentials/signature schemes, that mostly do not follow the algorithms
used in anonymous credentials systems. In fact, that is one of the main
motivations in the paper. The approach in \cite{cgm16} builds on committing
Oblivious Transfer and Garbled Circuits (not necessarily with privacy).
A closely related work seems to be \cite{dfkp16},
which achieves the same goal but using SNARKS, which results in different
tradeoffs -- hence, it may be interesting, as the SNARK domain has improved
significantly since 2016. \cite{kkl+16} also allows building anonymous
credentials for arbitrary policies, but using a different approach too.

\subsubsection{Pseudonym Systems}
\label{sssec:pseudonyms}

\doubt{Not sure if this fits here.}

\subsection{Applications}
\label{ssec:acapplication}

\subsubsection{Theory}
\label{sssec:actheoryapp}

Main approaches to build ACs (from ToPS paper I reviewed):

\begin{itemize}
\item Re-randomizable signatures on commitments (also as in DAA):
  \cite{cl02,cl04,lmpy16,ps16}.
\item Equivalence class signatures: \cite{fhs19,hs14}.
\item Redactable signatures \cite{cdhk15,sand20} and malleable signatures
  \cite{ckl14}.
\item Predicate encryption: \cite{dmm+18}.
\end{itemize}

Also:

\begin{itemize}
  \item ACs based on aggregate signatures: \cite{cl11}.
\end{itemize}

PS: Tokenization.

\subsubsection{Real World Deployments}
\label{sssec:acrwdeploy}

PS: Tokenization.
Signal's AC \cite{cpz20}.

\begin{itemize}
\item U-Prove (see ref. 65 in \cite{hs21}):  This is largely based on Brands
  thesis. There was a deployment by Microsoft, mostly concurrent to Idemix.
  They are one-show credentials, but seemingly there is recent real-world
  deployments based on them (references in p.1 of \cite{hs21}):
  
  \begin{itemize}
  \item Privacy-Pass by Cloudflare.
  \item Trust Tokens API by Google.
  \item PrivateStats proposal by Facebook.
  \end{itemize}
  
\item Idemix \cite{ch02}: This is largely based on CL signatures. It has been
  deployed (or is being deployed) in Hyperledger Fabric, and I think there is
  some of it in Indy too.
\end{itemize}

\cite{hs21} contains also many references to ``practical'' applications of
ACs. However, these are mostly references to papers in cryptographic venues,
so I suspect that they are more academic than real-world systems.

\subsection{TODO}

\begin{itemize}
\item : Delegation (\cite{cklm14} relies on \cite{cklm12}).
\item \cite{cgm16} An AC system that builds ACs from existing (conventional,
  such as ECDSA) signatures. \cite{kkl+16} achieves similar results with a
  different approach?
\end{itemize}

[ToPS paper I reviewed] refers to a strategy to do threshold issuing in \cite{bbh06}.

Concepts in ACs: multishow, one-show (Brands et al); interactive vs non-interactive
showings (differentiation made in \cite[p.6]{fhs19} but, why would one want interactive,
in the real world?)

Which schemes support comibning credentials from multiple issuers? I think this
is directly obtained in the schemes following the CL approach; but others?

A recurrent claim in schemes avoiding using zero-knowledge proofs of knowledge
of some attributes being included in a credential is that these proofs are costly
(like, linear in the number of attributes). Yet, they resort to other primitives
that may also be costly (e.g., predicate encryption \cite{dmm+18}) or somehow move
the cost to some other place (e.g., \cite{fhs19} uses a vector commitment scheme
that requires a public key with size that grows with the number of attributes
that \emph{can} be committed). It would be nice to have a detailed and complete
(e.g., not skipping the data that makes one scheme look bad, like the size of
the public keys) comparison. And, ideally, with concrete instantiations (not only
theoretical analysis)?

\cite{fhs19} has a nice comparison table.

How to compare efficiency of AC systems?


%%% Local Variables:
%%% mode: latex
%%% TeX-master: "sok-privsig"
%%% End:
