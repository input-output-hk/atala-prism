\section{Anonymous Credentials}
\label{sec:ac}

First proposed in \cite{chau85}.
CL signatures (idemix) \cite{cl01,cl04}.
UC formalisation in \cite{cdhk15}.
Brands works (U-prove).
Coconut paper, referenced in \cite[Section 1]{cdl+20}: anonymous credentials with
threshold opening.

\begin{enumerate}
\item Main intuition.
\item Evolution of main security models.
\item Variations concerning functionality.
\item Evolution in efficiency.
\item Main issues.
\item Approx. number of related works in main crypto+security venues: IACR,
  Esorics, CCS, AsiaCCS, SP, EuroSP, Usenix Sec and Privacy, PETs...
\end{enumerate}

Anonymous Credentials (ACs) were first proposed by Chaum \cite{chau85}, and
later formally treated and optimized in a series of papers by Camenisch and
Lysyanskaya \cite{cl01,cl02,cl04}. Roughly speaking, users receive credentials
from an issuing organization, which can then use to authenticate against some
verifier. The main property is that this authentication process can be done
in a privacy-preserving way. This typically means that the verifier cannot
link different authentication processes by that same user. And, usually, in
credentials composed by multiple attributes, that the verifier only learns
the attributes that the user is willing to share (or a function of them).

\subsection{Approaches, Models, and Variations}

\paragraph{Main approaches.} %

\begin{enumerate}
\item[\cite{cl01}:]
  \begin{itemize}
  \item Functionality: FormNym, GrantCred, VerifyCred, VerifyCredOnNym.
  \item Approach: Users create a pseudonym with each organization. For this,
    they have to register first with the organization (associating a key to
    a user name). Whenever the user wants to get a credential from a organization
    in which he registered, he has to authenticate, and then the organization
    decides whether or not to grant the credential. Users can show their
    credentials to organizations in two ways: (1) just proving that they own a
    credential issued by some organization; or (2), proving that they own a
    credential issued by some organization \emph{and} this credential is over
    the same pseudonym that the user holds with the verifying organization.
    Essentially, pseudonyms are the concatenation of two random numbers (one
    provided by the user, and the other by the organization). They get
    associated an authenticating tag, which is a statistically hiding commitment
    to the user secret (computed using randomness provided by both user and
    organization). Organizations keep a database of the generated pseudonyms.
    To get a credential from an organization where the user registered
    previously, the user proves knowledge of the secret involved in the
    pseudonym. The organization then creates a ``random signature'' (by using a
    the inverse of random exponent, with the help of the factorization of an RSA
    modulus chosen by the organization at setup time) over the pseudonym, and
    gives that to the user. Users can show the credentials (in any of the
    previously stated two ways) by creating non-interactive zero-knowledge
    proofs of knowledge of the appropriate secret. Note that, in case (2),
    this means that the proof is over two pseudonyms, which must share the
    same secret.    
  \end{itemize}
\item[\cite{cl02}:]
  \begin{itemize}
  \item Functionality:
  \item Approach: Te paper aims at improving efficiency of AC systems, as
    well as still ensuring that the user can retrieve their credentials in
    a privacy-preserving manner. For this purpose, the core of the paper
    describes a signature scheme that allows signing blocks of messages,
    and later prove, in zero-knowledge, knowledge of such a signature. To
    enable private retrieval of credentials, the authors describe a protocol
    for their signature scheme, whereby the user specifices a commitment to
    a message (or a block of messages, although this is not actually
    formalized), and get a signature of the message in return. While the
    authors do not strictly define an AC system, the application of their
    signature system and related protocols to the AC domain seems direct.
    Indeed, the same functionality as in \cite{cl01} seems straight
    forward given the cryptosystems in \cite{cl02}.
  \end{itemize}
\item[\cite{cl04}:] Follows a similar approach to \cite{cl02}, although
  changing the security assumptions to be based only on problems related
  to the discrete logarithm, and on bilinear maps. Basically, defines a
  signature scheme suitable for signing multiple messages, a protocol to
  obtain a signature con a committed value, and a protocol to prove knowledge
  of a signature. Concerning commitments, they rely on Pedersen commitment
  scheme, for which protocols are known to prove knowledge and equality of
  committed values. This lets them follow the same approach to \cite{cl02},
  which is in turn the one in \cite{cl01}.
\item[\cite{cdhk15}:]
\end{enumerate}

\paragraph{Main security models.}

\begin{enumerate}
\item[\cite{cl01}:] Security in \cite{cl01} is proven in an ad-hoc way,
  using the ideal-world/real-world paradigm. The authors describe a
  simulator for the operations supported by the system (Setup, FormNym,
  GrantCred, VerifyCred and VerifyCredOnNym), and prove that the output of this
  simulator is indistinguishable from that of the real protocol, under the
  strong RSA and Diffie-Hellman assumptions.  
\item[\cite{cl02}:] No security model is given for the anonymous credential
  application of the signature scheme (reasonably, as it is not the main focus
  of the paper). The signature scheme is proven secure (no specific mention
  to a security model, although I suppose it would be quite straight forward)
  based on the Strong RSA and Diffie-Hellman assumptions.
\item[\cite{cl04}:] Same as in \cite{cl02} (no actual security model for the
  AC system). The custom proofs are based on the DDH and LRSW assumptions.
\item[\cite{cdhk15}:]
\end{enumerate}

\paragraph{Variations in functionality.}

\begin{enumerate}
\item[\cite{cl01}:] Extensions provided to the main functionality are:
  \begin{itemize}
  \item PKI-based and all-or-nothing non-transferability: Both options
    involve sharing a verifiable encryption of the user secrets. In the
    case of PKI-based non-transferability, the CA receives a verifiable
    encryption, using as key the user's master secret, of some valuable
    piece of information belonging to the user (e.g., the user's secret
    key associated to his public key with that CA). In the all-or-nothing
    case, the user shares with each organization (with which it has a
    pseudonym), a verifiable encryption of his pseudonym secret data, encrypted
    with his master secret. In both cases, if the user shares his master
    secret with someone else, then he automatically gives access to
    other relevant pieces of information (as the verifiable encryptions are
    made public by the corresponding entity).    
  \item One-show credentials: The validating tags over pseudonyms are extended
    to include an extra value in their commitment. Then, users have to reveal
    that value (or, rather, a value deterministically derived from that) in the
    zero-knowledge proofs they create at show time. Since the value is always
    the same, verifiers can check whether they have seen it before, or not.
  \item Revocation: Both local (i.e., within an organization) and global (i.e.,
    referring to an identity of the user in an external CA) is possible by
    extending the pseudonyms of the user, and the credential showing protocols,
    with values that allow subsequent decryption by a trusted party of their
    pseudonym or external identity, given a transcript of the credential
    showing.    
  \item Attributes: The paper also succinctly describes how multiple attributes
    per credential could be included. This would be done by dividing the inverval
    within the exponent used to compute the pseudonym in sub-intervals, and then
    proving that the exponent lies within a sub-interval.
  \end{itemize}
\item[\cite{cdhk15}:]
\end{enumerate}

\paragraph{Efficiency.}

\paragraph{Main challenges.}

\subsection{Related Primitives}

\subsubsection{Pseudonym Systems}
\label{sssec:pseudonyms}

\doubt{Not sure if this fits here.}

\subsection{Applications}
\label{ssec:acapplication}

\subsubsection{Theory}
\label{sssec:actheoryapp}

Main approaches to build ACs (from ToPS paper I reviewed):

\begin{itemize}
\item Re-randomizable signatures on commitments (also as in DAA):
  \cite{cl02,cl04,lmpy16,ps16}.
\item Equivalence class signatures: \cite{fhs19,hs14}.
\item Redactable signatures \cite{cdhk15,sand20} and malleable signatures
  \cite{ckl14}.
\item Predicate encryption: \cite{dmm+18}.
\end{itemize}

PS: Tokenization.

\subsubsection{Real World Deployments}
\label{sssec:acrwdeploy}

PS: Tokenization.
U-Prove.
Idemix.
Signal's AC \cite{}

\subsection{TODO}

\cite{cl04,ckl+15}.

Look also for the non-bilinear-based approach of \cite{cl04}.

[ToPS paper I reviewed] refers to a strategy to do threshold issuing in \cite{bbh06}.

Mental note: some of the early CL papers mentioned something about ACs
containing some secret key/value of the credential holder (I think, to avoid
sharing or something like that). Check which paper was it, and what happened
with this concept in subsequent refinements. If this has been removed, this may
be one key difference with respect to group signatures: i.e., without some
holder-specific secret value in the show-verify protocols, it is just not
possible to ensure traceability and/or non-frameability.

%%% Local Variables:
%%% mode: latex
%%% TeX-master: "sok-privsig"
%%% End:
