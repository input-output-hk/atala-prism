\section{Introduction}
\label{sec:introduction}

Privacy is a human right \needcite. It is also one of the most threatened ones
as a consequence of digitalization and hyperconnectivity \needcite. Maybe one
of the reasons behind this is that, on many occasions, it conflicts with another
human right: security \needcite. The most representative occasion in which this
happens is when we, as individuals%
\footnote{This, of course, also happens when authenticating machines -- as they
  carry to some extent information of some individual(s).}%
, want to gain access to some (possibly
digital) good or service. In that context, one has to prove having the legitimate
rights to access that good or service. However, in the process of doing so, the
entity verifying that we have the rights, typically learns something about us,
like our real name. This has been traditionally fine in physical systems, as the
impact was mostly local. However, with the advent of digitalization and
hyperconnectivity, this usually means that something related to our identity is
being logged in some server, who knows where and with what security controls,
and probably vulnerable to a number of criminal activities \needcite.

Of course, the cryptographic community has not been stranger to the need of
providing satisfactory solutions that enable both privacy and security. We
refer to all works trying to address to this need as \emph{privacy-preserving
  authentication} mechanisms. For decades, a huge amount of research, with many
variations, has been produced in the field. Sadly, still with limited impact in
the real world. Within this context, the goal of this review is manyfold: from a
practical perspective, serve
as a hub to security and privacy professionals that may find themselves in need
to balance privacy and security in some real world application; from a more
academic one, provide a (somehow detailed) overview of the vast amount of
literature in the field, which is already overwhelming enough to make it hard
to find the most appropriate variant of some property we may find ourselves
in need to provide; and finally, attempt to throw some insight into why such a
vast academic work has so far failed to gain more real world adoption.

In order to achieve this, %
we go over the main cryptographic primitives used to build privacy-preserving
authentication mechanisms. For all of them, we first provide a high-level
intuition of the approach to protect users' privacy. Then, we overview the
most prominent security models since their inception, and complement this
with some relevant variations concerning the provided functionality. We also
summarise the main results from an efficiency standpoint, and the main
challenges found so far both in theory and practice. To gain some initial
insight into the academic impact of each primitive, we provide (approximate)
numbers on publications in the main cryptography, security and privacy venues.
Finally, with a more practical mindset, we overview the applications that each
primitive has ``in theory'', and the extent to which they have been deployed
in the real world.

\note{Argue/mention that focus is on primitives that can be used for signing.
  This therefore excludes protocols that can only be used for anonymous
  authentication, but not anonymous singing per se (e.g., PAKE, PASTA, PESTO...
  and probably also Nymble, PEREA, BLAC...)}

\note{Create a graph of relationships between basic primitives and their
  applications. E.g., much of
  the work in SRP group signatures and DAA seems to be derived from randomizable
  signatures (which, afaik, are also a foundation for anonymous credentials).}

%%% Local Variables:
%%% mode: latex
%%% TeX-master: "sok-privsig"
%%% End:
