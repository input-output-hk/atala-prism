\section{Group Signatures}
\label{sec:gs}

\paragraph{Main intuition.}
%
Group signatures were first proposed by Chaum and van Heyst in 1991 \cite{ch91}
as a privacy-respectful extension to conventional digital signatures.
The intuition is simple: in order to avoid revealing the identity of the signer,
we make her signatures look indistinguishable to the signatures of all other
members of a group. Then, any entity with access to the group public key can
verify that a group signature comes from a member of the group, but cannot
tell which specific one actually created it.
%
Beyond this, group signature schemes have one main feature: a (set of) group
manager(s) control who can become a member of the group. Also, most frequently,
a (set of) group manager(s) decide when to expell a member from the group, or
when to re-identify the creator of a specific signature.
%
The initial idea put forward in \cite{ch91} had, as is to be expected, a number
of limitations. In the following years, the seminal work was extended, both
formally, functionally and efficiency-wise.

\subsection{Approaches, Models, and Variations}
\label{ssec:gsapproach}

\subsubsection{Main approaches.}
%
There are two main ways that are recurrently used to instantiate group signature
schemes: \emph{Sign-Encrypt-Prove} and \emph{Sign-Randomize-Prove}. In both
variants, when new members of the group join, they receive a credential (i.e.,
a signature by the issuer) derived from a secret they choose. Then, when
creating a group signature, they prove control a valid credential and the
associated secret. The difference lies on how this credential is kept hidden
from the verifier.
%
In the sign-encrypt-prove variant, proposed in \cite{bmw03}, signers
encrypt their credential with a \doubt{IND-CCA2? what type?} scheme and then
create a NIZK \needcite proof of knowledge of such a valid credential
over their secret (plus the message they want to sign). This typically results
in schemes with more efficient openings, as the opening authority just has
to decrypt the encrypted credential. However, group signatures require more
space.
%
In the sign-randomize-prove variant, proposed in \needcite \todo{Get Shorty?},
credentials are actually randomizable signatures by the issuer. Signers then
randomize the signature (credential), and create a NIZK proof of knowledge that
this credential is issued for a secret they know (again, plus the message
they want to sign). This usually results in schemes with less efficient openings,
as the opening authority needs to iterate over the registered credentials until
finding the one from which the randomized credential was derived. However, the
produced group signatures are also shorter.

When a new user wants to become a group member, a common habit is also to have
her prove ownership of a conventional digital identity. This identity is then
added to the de-identifying information leveraged by the opening authority, in
order to be able to associate (anonymous) group signatures to a real-world
identity. These conventional digital identities can be any type of PKI-based
(e.g. X.509 or equivalent) certificate. \todo{This is important for
  non-frameability! Check \cite[Section 3.2]{cdl+20} and re-state this.}
\todo{This is indeed necessary for non-repudiation (mentioned in
  \cite[Section 4]{lpy12}).}
\todo{Typically used in schemes with verifiable openings. See
  \cite[Section 1.3.3]{bsi12}.}

\subsubsection{Main security models.}
%
Security for group signatures is given in the form of game-based definitions
\needcite.
The first formal security model appeared in \cite{bmw03}. It captured security
for groups with a prefixed set of members, added by the group manager upon the
creation of the group. This static requirement was too restrictive to be
realistic, and was quickly extended in \cite{bsz05} and \cite{kty04,ky06}, to
model that new users can join the group after it has been created. The models
in \cite{bsz05,kty04,ky06} are usually referred to as dynamic or partially
dynamic. \cite{bcc+16} generalizes the (partially) dynamic setting into a fully
dynamic model, which considers extra desirable security properties when users
can be revoked (i.e., removed from the group).

Up to some subtleties, the main security properties in most works that appeared
after \cite{bmw03,bsz05,kty04,ky06} are as follows:

\begin{itemize}
\item {\bf Anonymity}. This is the core property concerning privacy. It
  ensures that given a signature produced by one out of two honest (challenge)
  users picked at random from the set of honest group members, it should be
  unfeasible to determine which one actually created it. 
  
\item {\bf Traceability}. This property (covered by the misidentification attack
  in \cite{kty04,ky06}, and spaning also non-frameability in \cite{bmw03})
  captures that it must be possible (by the proper authority) to unambiguously
  recover the identity of the signer of any valid group signature.

\item {\bf Non-frameability}. This property (considered in the framing attacks
  in \cite{kty04,ky06}) requires that no adversary should be able to create
  valid signatures that can be used to incriminate honest users by having them
  open to an honest user who did not create them.
 
\end{itemize}

\cite{bcc+16} also includes a trace soundness property, ensuring that signatures
can be associated (via opening proofs) to only one user \todo{why does this not
  appear in the other models? Maybe because they don't support verifiable
  openings? Check \cite{kss19} and, if so, move this to the variations in
  functionality.}

\subsubsection{Variations in functionality.}
%
Note that, while the security properties just described roughly have the same
purpose across models, there are still differences making some models stronger
or more general than others in some respects. This includes subtleties like who
generates the keys of the authorities (the environment vs the adversary), how
many queries can be made to the challenge oracles and whether or not the
signatures are fixed, etc. However, while important, there are usually generic
techniques that allow to migrate from one variant to another \todo{\needcite,
  ellaborate more, or rewrite}. Now we are interested in revisiting some of
the variations that have impact into the functionality that schemes
following each model can provide, or variations that are specific to group
signatures schemes and do not appear in other cryptosystems.

\paragraph{Separating issuer and opener.}
In some schemes \cite{bmw03,kty04} one single authority concentrates the
power to add new members to the group, and open signatures. This endows such authority with great
power -- and trust -- and is therefore one of the main concerns for privacy
advocates: not only this authority can prevent users from joining (in the case
of dynamic groups), but it can also reidentify group members. In order to reduce
this trust, and since it is a clear distinction between both functionalities
(issuing and opening), many schemes separate both roles into an issuer and an
opener. While both authorities need to be trusted for their respective tasks,
one can now make independent assumptions and even ensure certain properties
if only one of the two authorities is compromised or malicious. For instance,
non-frameability can be achieved even when both authorities are malicious;
traceability can be achieved with malicious openers; and anonymity can be
kept as long as the opener is honest. This is the case in \cite{bsz05} and
\cite{ky06} (which, although starts from a single authority, also considers
and formalises the case of separate authorities).

\paragraph{Protecting against corrupt authorities.}
Even when the issuance and opening functionalities are separated, as mentioned,
each entity can unilaterally break crucial properties in a group signature scheme.
While placing such kind of trust may be reasonable in some use cases, it may not
be so for others. Thus, a natural extension is to reduce this trust as much as
possible. In this direction, there have been works that have applied threshold
and similar techniques \needcite to require that a set of openers (as opposed
to a single opener) need to cooperate in order to open any group signature
\cite{bcly08} \needcite.
Recently, the role of the issuer was also subject to these techniques
\cite{cdl+20}, and it is now known how to produce group signatures such that no
single entity can prevent users from joining nor reidentify them.

\paragraph{Forward security.}
Forward security aims at restricting the impact of security compromises (e.g.,
key leakage) to the future: i.e., if an attacker gains control of a user's
private key, only signatures produced after the compromise should be considered
invalid. In schemes that do not ensure forward secrecy, such a compromise also
renders invalid signatures produced before the compromise. Note that, for
conventional signatures, only the compromised users is (directly) affected by
such event. However,  in group signatures, all group members are directly
affected, as the signer's identity is not revealed by the signature. Thus,
in order to learn whether or not a past signature was issued by a compromised
user, it is necessary to resort to the opener -- therefore, reidentifying
\emph{all} users in the system.
%
Forward security has been introduced in group signatures in works like
\cite{song01,ly12} \needcite.
%
Interestingly (but logically), the approach to incorporate this property
holds similarities with the functionality to implement fully dynamic group
signatures: for forward security, users are usually required to apply an
update procedure to their private keys \cite{ly12}; in fully dynamic group
signatures, it is the group manager (or the opener) who has to update some
key material \cite{bcc+16}. In both cases, time has to be discretized in epochs,
and conditions are introduced into the security properties that prevent an
adversary being able to deanonymize (produce) a valid signature generated
(using) a key in an epoch in which such key was not supposed to be active
(either because the user updated it -- in a forward secure scheme -- or the
group manager revoked it -- in a fully dynamic scheme).

\paragraph{Revocation mechanisms.}
Verifier Local Revocation \cite{bs04}, accumulators approach.

\paragraph{Opening controls.}
Even when separating the roles of the authorities or distributing them, the
fact that some entity (or set of entities) can open group signatures is still
a delicate matter that creates friction. A minimal control is to enable the
opener to prove that the opening process has been performed honestly. This is
precisely what is done by schemes that support \emph{verifiable opening}
\needcite. To further aleviate this issue, some schemes
introduce mechanisms to even prevent opening as long as some condition is
met. For instance, in \cite{seh+12,ehk+19}, the condition is determined by
the message (e.g., only signatures over offending messages can be opened).
To implement this, the authors introduce an extra authority, the
\emph{admitter}, such that the opener needs to get some information from the
admitter in order to be able to open signatures over some specific message.
In \cite{sbm10}, the authors propose \emph{contractual} anonymity: before a user
is able to anonymously access some service, both user and service sign a
contract defining the conditions upon which anonymity can be revoked. This
contract can only be enforced by a trusted entity, the \emph{accountability
  service} who runs a trusted execution environment (such as today's Intel
SGX \needcite) ensuring that the conditions are checked in a trustworthy manner.

\paragraph{Alternatives to opening.}
%
Even with separated authorities, with distributed openers, or with more generic
opening controls such as message-dependent opening, opening may not even be
the type of reduction of privacy that is necessary. In this vein, alternative
schemes have been proposed that do not include any kind of
opening at all.
%
In the schemes proposed in \cite{kty04,cpy06,bcly08}, there is an extra
authority that can produce \emph{tracing} trapdoors on a per-user basis.
Once the tracing trapdoor for a specific user is given to some entity, this
entity can link (or trace) all group signatures produced by that user.
These trapdoors are also an effective mechanism for implementing verifier-local
revocation.
%
In \cite{gl19}, the opener is replaced by a \emph{converter} who, rather than
opening any given group signature, can link sets of signatures. More
specifically, when queried with a set of signatures, the converter transforms
them so that every signature coming from the same user gets assigned a same
randomized pseudonym. Furthermore, to prevent the converter from gradually
relinking all the signatures, these conversions are non-transitive: i.e.,
signatures produced by user $A$ will receive a different randomized pseudonym in
different queries.
%
In \cite{dl21}, there is directly no trusted entity that can open or link group
signatures. Instead, every user can create proofs of several signatures having
been issued by herself. In this case, the authors also propose a scheme whereby
a user can prove sequentiality of a set of signatures: i.e., that all signatures
were produced by herself, and no signature has been omitted or inserted into the
sequence, which is also given in the order in which the original signatures were
produced.

\subsubsection{Efficiency.}

\begin{itemize}
\item Size of (group) keys.
\item Size of signatures.
\item Verification time (batching, including revocation checks...)
\item Accumulators.
\end{itemize}

\subsubsection{Main challenges.}

Efficiency is possibly good for many real-world cases. Still: utility
and unavailability of implementations.

\note{Concurrent joins -- avoid extraction \cite{ky05,klap20}.}

\subsection{Related Primitives}

\subsubsection{Ring Signatures}
\label{sssec:rs}

\cite{rst06}

\cite{bcc+15} uses ring signatures to build a fully dynamic group signature
scheme.

\subsection{Applications}
\label{ssec:gsapplication}

See \cite[Section 1.2.3]{bsi12}: it seems mostly theoretical applications.

\subsubsection{Theory}
\label{sssec:gstheoryapp}

GS-MDO imply IBE \cite{ehk+19}.
GS are implied (shouldn't his by imply?) by trapdoor permutations \cite{bmw03}.
GS imply one-way functions \cite{romp90}.
GS imply public key encryption \cite{aw04}.
Commitments from group signatures (group signatures imply commitments?)
(ref. 25 in \cite{bfg+11}).

\subsubsection{Real World Deployments}
\label{sssec:gsrwdeploy}

\cite{ehk+19} mentions anonymous auctions, anonymous bulletin boards and
anomaly detection systems.

Maybe make some comments on how to deploy gs schemes, or what aspects may be
relevant. For instnace, some schemes/models (e.g. \cite{ly12}) assume a
trusted party who generates the keys for the authorities. While this has
a big impact in security/trust model, in the real world it would probably
be circumventable with audited processes for most use cases. Also, the
issue with concurrent issuance and the approach we took for implementing
\cite{gl19} in ICT4Cart, or the limitation to logarithmic numbers of users, etc.

\subsection{TODOs}

\begin{itemize}
\item Fully anonymous dynamic GS from SPS-EQ \cite{ds16}
\item Multi-group group signature schemes. Compare with ACs with multiple
  issuers.
\end{itemize}

\subsubsection{Further Notes}
\note{Integrate this section or delete it.}

\todo{CCA2 type of anonymity is defined in \cite{bbs04}. Schemes with CPA type
  of anonymity are (according to \cite{ehk+19}): \cite{bw06,bw07}.}

\cite{ehk+19} has a nice related work mentioning the assumptions that are
typically used.

Signature highjacking attacks (referenced in \cite{ehk+19}).

If we want to have traceability, the issuer needs to be honest. See what
implications this has in practice (e.g., what architectures are compatible
with this?) This probably makes schemes with separate issuing and opening
authorities more attractive in practice.

Other minor variants: group blind signatures, democratic group signatures,
mediated group signatures. All referenced in \cite[Section 1.3.5]{bsi12}.

Backward linkability property of group signatures with Timed Verifier-Local
Revocation, proposed in \cite{nf05}. Mentioned in \cite[Section 8.1.2]{bsi12}.

Relation of forward security with proactive security \cite{oy91}?

References 1, 24, 32 and, probably, 35 in \cite{bfg+11} seem to be (traceable)
group signatures that use the nym approach (or similar) for linking. Check.

\paragraph{Academic indicators.}
IACR, Esorics, CCS, AsiaCCS, SP, EuroSP, Usenix Sec and Privacy, PETs... Not
sure where to fit this.

Main cryptographic building blocks:

\begin{itemize}
\item BB signatures \cite{bb04}.
  \begin{itemize}
  \item Used in: \cite{ky05}.
  \item Paillier encryption of discrete-logarithms? Used in \cite{ky05} and,
    I think, \cite{gl19}.    
  \end{itemize}
\item Model for reduction: list of group signature schemes based on RO and
  non-RO are given in \cite{bcc+16}.
\end{itemize}

%%% Local Variables:
%%% mode: latex
%%% TeX-master: "sok-privsig"
%%% End:
